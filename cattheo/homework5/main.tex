\documentclass{article}

\usepackage[utf8]{inputenc}
\usepackage{enumerate}
\usepackage{amsthm, amssymb, mathtools, amsmath, bbm, mathrsfs, stmaryrd}
\usepackage[margin=1in]{geometry}
\usepackage[parfill]{parskip}
\usepackage[hidelinks]{hyperref}
\usepackage{quiver}
\usepackage{float}
\usepackage{cleveref}

\newcommand{\N}{\mathbb{N}}
\newcommand{\Z}{\mathbb{Z}}
\newcommand{\NZ}{\mathbb{N}_{0}}
\newcommand{\Q}{\mathbb{Q}}
\newcommand{\R}{\mathbb{R}}
\newcommand{\C}{\mathbb{C}}

\newcommand{\F}{\mathbb{F}}
\newcommand{\incl}{\imath}

\newcommand{\tuple}[2]{\left\langle#1\colon #2\right\rangle}

\DeclareMathOperator{\order}{orde}
\DeclareMathOperator{\Id}{Id}
\DeclareMathOperator{\im}{im}
\DeclareMathOperator{\ggd}{ggd}
\DeclareMathOperator{\kgv}{kgv}
\DeclareMathOperator{\degree}{gr}
\DeclareMathOperator{\coker}{coker}

\DeclareMathOperator{\gl}{GL}

\DeclareMathOperator{\Aut}{Aut}
\DeclareMathOperator{\Hom}{Hom}
\DeclareMathOperator{\End}{End}
\newcommand{\isom}{\overset{\sim}{\longrightarrow}}

\newcommand{\Grp}{{\bf Grp}}
\newcommand{\Ab}{{\bf Ab}}
\newcommand{\cring}{{\bf CRing}}
\DeclareMathOperator{\modules}{{\bf Mod}}
\newcommand{\catset}{{\bf Set}}
\newcommand{\cat}{\mathcal{C}}
\newcommand{\catt}{\mathcal{D}}
\newcommand{\pcat}{\mathcal{P}}
\newcommand{\chains}{{\bf Ch}}
\newcommand{\homot}{{\bf Ho}}
\DeclareMathOperator{\objects}{Ob}
\newcommand{\gen}[1]{\left<#1\right>}
\DeclareMathOperator{\cone}{Cone} \newcommand{\set}[1]{\left\{#1\right\}}
\newcommand{\setwith}[2]{\left\{#1:#2\right\}} \newcommand{\setcat}{{\bf Set}}
\DeclareMathOperator{\Ext}{Ext} \DeclareMathOperator{\nil}{Nil}
\DeclareMathOperator{\idem}{Idem} \DeclareMathOperator{\rad}{Rad}
\newcommand{\alg}[1]{T\textnormal{-Alg}} \DeclareMathOperator{\ev}{ev}
\DeclareMathOperator{\psh}{Psh} \DeclareMathOperator{\famm}{Fam}
\DeclareMathOperator{\dom}{dom} \DeclareMathOperator{\codom}{codom}

\newenvironment{solution}{\begin{proof}[Solution]\renewcommand\qedsymbol{}}{\end{proof}}

\newtheorem{theorem}{Theorem}[section]
\newtheorem{lemma}[theorem]{Lemma}
\newtheorem{proposition}[theorem]{Proposition}


\theoremstyle{definition}
\newtheorem{question}{Exercise}
\newtheorem{definition}[theorem]{Definition}
\newtheorem{remark}[theorem]{Remark}
\newtheorem{example}[theorem]{Example}
\newtheorem{corollary}[theorem]{Corollary}

\title{Homework Category Theory}
\author{Jonas van der Schaaf}
\date{}

\begin{document}
\maketitle

\begin{question}
    There is a forgetful functor \(U:\psh(\cat)\to\famm(\cat_{0})\).

    \begin{enumerate}[(a)]
        \item Show that \(U\) has a right adjoint.

              \begin{proof}
                  Let \(R_{X}:\cat^{op}\to\psh(\cat)\) be
                  the presheaf defined by
                  \[
                      R(C)=\prod_{\alpha:E\to C}X_{E}.
                  \]
                  We define the projection maps \(\pi_{\alpha:E\to
                      C}:\prod_{\beta:E\to C}X_{E}\to X_{\codom(\alpha)}\) to
                  project onto the corresponding copy of the codomain in the
                  product.

                  For a morphism \((f:C\to D)^{op}\) take \(R(f):R_{X}\) the
                  unique morphism given by the universal property in the
                  following manner: for each \(\beta:E\to C\) we have a map
                  \(f\circ\beta:E\to D\). Therefore for each \(\beta:E\to C\)
                  there is an \(\alpha:E\to D\) such that
                  \(\beta=f\circ\alpha\). This means that for each \(\alpha:E\to
                  C\) there is a morphism \(\pi_{f\circ\beta}:\prod_{\gamma:E\to
                      D}X_{E}\to X_{\codom(\alpha)}\). This means by the universal
                  property there is a unique morphism which is the image of
                  \(f\) under \(R\): \(R(f):=\prod_{\alpha:E\to
                      D}X_{E}\to\prod_{\beta:E\to C}X_{E}\) such that the cone of
                  the latter product commutes.
                  \[
                      \begin{tikzcd}[ampersand replacement=\&]
                          \&\& {\prod_{\alpha:E\to F}X_{E}} \\
                          \\
                          {X_{\codom(\gamma)}} \&\& {\prod_{\alpha:E\to D}X_{E}} \\
                          \\
                          {X_{\codom(\varepsilon\circ\gamma)}} \&\& {\prod_{\alpha:E\to C}X_{E}}
                          \arrow["{\pi_{\gamma}}", from=1-3, to=3-1]
                          \arrow["{\pi_{f\circ\gamma}}"', from=3-3, to=3-1]
                          \arrow["{\exists!R_{X}(f)}", dashed, from=1-3, to=3-3]
                          \arrow["\Id", from=3-1, to=5-1]
                          \arrow["{\pi_{g\circ\varepsilon\circ\gamma}}"', from=5-3, to=5-1]
                          \arrow["{\exists!R_{X}(g)}", dashed, from=3-3, to=5-3]
                          \arrow["{\pi_{\varepsilon\circ\gamma}}", curve={height=-6pt}, from=1-3, to=5-1]
                          \arrow["{\pi_{\varepsilon}}", from=3-3, to=5-1]
                      \end{tikzcd}
                  \]

                  First of all the identity map is preserved because
                  \(\Id:\prod_{\alpha:E\to D}X_{E}\to\prod_{\alpha:E\to
                      D}X_{E}\) satisfies the universal propert as described above
                  because \(\pi_{\alpha}=\pi_{\alpha}\circ\Id\).

                  On top of that for two morphisms \(f\circ g\), it's clear that
                  in the below diagram \(R_{X}(g)\circ R_{X}(f)\) satisfies the
                  universal property of the maps from \(\prod_{\alpha:E\to
                      F}X_{E}\) to each \(X_{E}\). This makes \(R_{X}\) a
                  contravariant functor \(\cat^{op}\to\catset\) and thus a
                  presheaf.

                  We will use exercise 67 to show the existence of a left
                  adjoint.

                  Let \(F\) be a presheaf and \(f:UF\to X\) be a morphism. We
                  will show there is an \(\varepsilon:UR_{X}\to X\) such that
                  there is a unique morphism \(\tilde{f}:F\to R_{X}\) with
                  \(\varepsilon_{X}\circ U\tilde{f}=f\).

                  For the map \(\varepsilon_{X}:UR_{X}\to X\) we define
                  \[
                      \varepsilon_{D}=\left(\pi_{\Id_{C}}:\prod_{\alpha:E\to C}X_{E}\to X_{C}\right)_{C\in\cat_{0}}.
                  \]
                  Now we define a natural transformation \(\tilde{f}:F\to
                  R_{X}\) and prove its uniqueness to satisfy all conditions of
                  exercise 67 and thus obtain a left adjoint.

                  We define \(\tilde{f}\) componentwise in the following manner:
                  for each \(\alpha:E\to C\) there is a morphism \(f_{E}\circ
                  F(\alpha):F(C)\to X_{E}\). This means there is also a morphism
                  \[
                      \tilde{f}_{C}:F(C)\to\prod_{\alpha:E\to C}X_{E}=\prod_{\alpha:E\to C}f_{E}\circ F(\alpha).
                  \]
                  These are the components of a natural transformation, so we
                  need to prove this forms a commutative diagram. It suffices to
                  prove the diagram commutes for each component of the product.
                  Given an \((f:C\to D)^{op}\), we know that for each component
                  of the product the transformation is given by some
                  \(\alpha:FD\to X_{E}\) with \(\beta f=\alpha\) for some
                  \(\beta:FC\to X_{e}\). This means the following diagram
                  commutes and thus that this is a natural transformation:
                  \[
                      \begin{tikzcd}[ampersand replacement=\&]
                          FD \&\& FC \\
                          \\
                          {X_{E}} \&\& {X_{E}}
                          \arrow["Ff"', from=1-1, to=1-3]
                          \arrow["F\beta", from=1-1, to=3-1]
                          \arrow["\Id", from=3-1, to=3-3]
                          \arrow["F\alpha"', from=1-3, to=3-3]
                      \end{tikzcd}
                  \]
                  For any map natural transformation \(\tilde{f}'\) such that
                  the diagram commutes we must have that \(\varepsilon_{X}\circ
                  U\tilde{f}'=f\). However because the maps
                  \(\varepsilon_{D,C}\) is the projection map \(\pi_{\Id_{C}}\)
                  and thus an epimorphism, this makes this natural
                  transformation unique with this property.

                  This means that \(\tilde{f}\) is the unique morphism with this
                  property and so \(U\) must have a right-adjoint.
              \end{proof}

        \item Show \(U\) has a left-adjoint.

              \begin{proof}
                  We will use exercise 68 to demonstrate \(U\) has a left
                  adjoint.

                  For any family of sets \(X\) we will demonstrate that the
                  presheaf with
                  \[
                      L_{X}(C)=\bigsqcup_{\alpha:C\to E}X_{E}
                  \]
                  is an initial object in the category \(X\downarrow U\)
                  together with a map \(g\).

                  First we will show it is a presheaf. To do this for a morphism
                  \((f:C\to D)^{op}\) we define \(R(f)\) a unique morphism
                  \(L_{X}(D)\to L_{X}(D)\) in a similar way as in part a.

                  For each \(\beta:D\to E\) we have \(\beta\circ
                  f\in\Hom(C,E)\). This means for each \(\beta:D\to E\) there is
                  a morphism \(X_{\codom(\beta)}\to\bigsqcup_{\alpha:C\to
                      E}X_{E}\) and so a unique morphism \(\bigsqcup_{\beta:D\to
                      E}X_{E}\to \bigsqcup_{\alpha:C\to E}X_{E}\). By the same
                  reasoning as before this map preserves identities and function
                  composition and this is therefore a presheaf.

                  The map \(g:X_{C}\to U L_{X}(C)\) is given componentwise as
                  \(g_{C}=\imath_{\Id_{C}}\) where \(\imath\) denotes coproduct
                  inclusion.

                  To show this presheaf is an initial object we will construct a
                  morphism to every object and show it is unique. Let \((F,g')\)
                  be an object of the category. Then we have a morphism
                  \(X_{E}\to F(C)\) given by \(F(\alpha)\circ g'_{C}\) for all
                  \(\alpha:C\to E\). This means we have maps \(f_{C}:L_{X}(C)\to
                  F(C)\) given by the universal property of the coproduct. These
                  form a morphism of presheafs \(f\), we show the following
                  diagram commutes:
                  \[
                      \begin{tikzcd}[ampersand replacement=\&]
                          {\bigsqcup_{\beta:D\to E}X_{E}} \&\& {\bigsqcup_{\alpha:C\to E}X_{E}} \\
                          \\
                          FD \&\& FC
                          \arrow["{f_{D}}", from=1-1, to=3-1]
                          \arrow["{f_{C}}", from=1-3, to=3-3]
                          \arrow["{L_{X}f}", from=1-1, to=1-3]
                          \arrow["Ff"', from=3-1, to=3-3]
                      \end{tikzcd}
                  \]
                  We do this by proving commutativity for each component of the
                  coproduct Because by definition we have
                  \(f_{C}\circ\imath_{\alpha}=F\alpha\circ g_{E}'\) and
                  \(L_{X}h\circ\imath_{\alpha}=\imath_{\alpha\circ h}\) we can
                  simply rewrite as below to obtain commutativity:
                  \begin{align*}
                      f_{D}\circ L_{X}h\circ\imath_{\alpha} & =f_{D}\imath_{\alpha\circ h}         \\
                                                            & =F(\alpha\circ h)\circ\ g'_{E}       \\
                                                            & =Fh\circ F\alpha\circ g'_{E}         \\
                                                            & =Fh\circ f_{C}\circ \imath_{\alpha}.
                  \end{align*}
                  This proves there is a morphism of sheafs. As we defined it
                  using the universal product it makes the desired triangle
                  commute:
                  \begin{align*}
                      f_{C}\circ g_{C} & =f_{C}\circ\imath_{\Id} \\
                                       & =g'_{C}.
                  \end{align*}
                  The uniqueness of this map follows immediately from the
                  uniqueness condition of the universal product.

                  This means that every category \(X\downarrow U\) has an
                  initial object \(L_{X}\) and so \(U\) has a right-adjoint.
              \end{proof}
    \end{enumerate}
\end{question}

\end{document}