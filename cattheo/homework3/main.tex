\documentclass{article}

\usepackage[utf8]{inputenc}
\usepackage{enumerate}
\usepackage{amsthm, amssymb, mathtools, amsmath, bbm, mathrsfs, stmaryrd}
\usepackage[margin=1in]{geometry}
\usepackage[parfill]{parskip}
\usepackage[hidelinks]{hyperref}
\usepackage{quiver}
\usepackage{float}
\usepackage{cleveref}

\newcommand{\N}{\mathbb{N}}
\newcommand{\Z}{\mathbb{Z}}
\newcommand{\NZ}{\mathbb{N}_{0}}
\newcommand{\Q}{\mathbb{Q}}
\newcommand{\R}{\mathbb{R}}
\newcommand{\C}{\mathbb{C}}

\newcommand{\F}{\mathbb{F}}
\newcommand{\incl}{\imath}

\newcommand{\tuple}[2]{\left\langle#1\colon #2\right\rangle}

\DeclareMathOperator{\order}{orde}
\DeclareMathOperator{\Id}{Id}
\DeclareMathOperator{\im}{im}
\DeclareMathOperator{\ggd}{ggd}
\DeclareMathOperator{\kgv}{kgv}
\DeclareMathOperator{\degree}{gr}
\DeclareMathOperator{\coker}{coker}

\DeclareMathOperator{\gl}{GL}

\DeclareMathOperator{\Aut}{Aut}
\DeclareMathOperator{\Hom}{Hom}
\DeclareMathOperator{\End}{End}
\newcommand{\isom}{\overset{\sim}{\longrightarrow}}

\newcommand{\Grp}{{\bf Grp}}
\newcommand{\Ab}{{\bf Ab}}
\newcommand{\cring}{{\bf CRing}}
\DeclareMathOperator{\modules}{{\bf Mod}}
\newcommand{\catset}{{\bf Set}}
\newcommand{\cat}{\mathcal{C}}
\newcommand{\catt}{\mathcal{D}}
\newcommand{\pcat}{\mathcal{P}}
\newcommand{\chains}{{\bf Ch}}
\newcommand{\homot}{{\bf Ho}}
\DeclareMathOperator{\objects}{Ob}
\newcommand{\gen}[1]{\left<#1\right>}
\DeclareMathOperator{\cone}{Cone}
\newcommand{\set}[1]{\left\{#1\right\}}
\newcommand{\setwith}[2]{\left\{#1:#2\right\}}
\DeclareMathOperator{\Ext}{Ext}
\DeclareMathOperator{\nil}{Nil}
\DeclareMathOperator{\idem}{Idem}
\DeclareMathOperator{\rad}{Rad}
\newcommand{\alg}[1]{T\textnormal{-Alg}}

\newenvironment{question}[1][]{\begin{paragraph}{Question #1}}{\end{paragraph}}
\newenvironment{solution}{\begin{proof}[Solution]\renewcommand\qedsymbol{}}{\end{proof}}

\newtheorem{theorem}{Theorem}[section]
\newtheorem{lemma}[theorem]{Lemma}
\newtheorem{proposition}[theorem]{Proposition}


\theoremstyle{definition}
\newtheorem{definition}[theorem]{Definition}
\newtheorem{remark}[theorem]{Remark}
\newtheorem{example}[theorem]{Example}
\newtheorem{corollary}[theorem]{Corollary}

\title{Homework Category Theory}
\author{Jonas van der Schaaf}
\date{October 10, 2022}

\begin{document}
\maketitle

\begin{question}[1]
    Let \(\cat\) be a category and \(T:\cat\Rightarrow\cat\) an endofunctor.

    \begin{enumerate}[(a)]
        \item Show that if \((X,a)\) is an initial \(T\)-algebra, then \(a\) is
              an isomorphism.

              \begin{proof}
                  If \((X,a)\) is initial, there is a unique \(T\)-algebra
                  morphism \((X,a)\to(Y,b)\) for all \(T\)-algebras \((Y,b)\).
                  Taking \(f:(X,a)\to(TX,Ta)\) we obtain the commutative diagram
                  in \Cref{fig:init-obj-iso}.
                  \begin{figure}[H]
                      \[
                          \begin{tikzcd}[ampersand replacement=\&]
                              TX \&\& TTX \&\& TX \\
                              \\
                              X \&\& TX \&\& X
                              \arrow["Ta", from=1-3, to=3-3]
                              \arrow["a", from=1-5, to=3-5]
                              \arrow["a"{description}, from=3-3, to=3-5]
                              \arrow["Ta"{description}, from=1-3, to=1-5]
                              \arrow["Tf"{description}, from=1-1, to=1-3]
                              \arrow["a", from=1-1, to=3-1]
                              \arrow["f"{description}, from=3-1, to=3-3]
                          \end{tikzcd}
                      \]
                      \caption{A commutative diagram.}
                      \label{fig:init-obj-iso}
                  \end{figure}
                  Because the two inner squares in \Cref{fig:init-obj-iso} are
                  commutative the outer square is commutative as well. Therefore
                  \(f\circ a\) is the unique endomorphism \((X,a)\to(X,a)\) as
                  given by the universal property. However \(\Id\) is another
                  such map. Therefore \(f\circ a=\Id\) so \(a\) is split mono.

                  We know \(fa=\Id\) so \(Taf=T\Id=\Id\). By commutativity of
                  the left square of \Cref{fig:init-obj-iso} this means
                  \(af=\Id\). Therefore \(f\) is a left and right inverse of
                  \(a\) so \(a\) is an isomorphism.
              \end{proof}

        \item Show that if \(\cat\) is complete, then so is \(\alg{T}\) and the
              forgetful functor \(U:\alg{T}\to\cat\) sending \((X,a)\) to \(X\)
              preserves limits.

              \begin{proof}
                  We will show that for a complete category \(\alg{T}\) has
                  equalizers and products. This is equivalent to \(\alg{T}\)
                  being complete.

                  Consider two morphisms \(f,g:(X,a)\to
                  (Y,b)\in\Hom_{\alg{T}}((X,a),(Y,b))\). Let \(E\) be their
                  equalizer with map \(e:E\to X\) in the category \(\cat\).
                  Then we obtain the diagram in \Cref{fig:dia-1}.
                  \begin{figure}[H]
                      \[
                          \begin{tikzcd}[ampersand replacement=\&]
                              TE \&\& TX \&\& TY \\
                              \\
                              E \&\& X \&\& Y
                              \arrow["f"', shift right=1, from=3-3, to=3-5]
                              \arrow["g", shift left=1, from=3-3, to=3-5]
                              \arrow["a", from=1-3, to=3-3]
                              \arrow["b", from=1-5, to=3-5]
                              \arrow["Tf"', shift right=1, from=1-3, to=1-5]
                              \arrow["Tg", shift left=1, from=1-3, to=1-5]
                              \arrow["e", from=3-1, to=3-3]
                              \arrow["Te", from=1-1, to=1-3]
                          \end{tikzcd}
                      \]
                      \caption{A diagram.}
                      \label{fig:dia-1}
                  \end{figure}
                  Because \(fe=ge\) we get
                  \begin{align*}
                      gaTe & =bTgTe \\
                           & =bTfTe \\
                           & =faTe,
                  \end{align*}
                  so \(faTe=gaTe\). By the universal property of the equalizer
                  there is then a unique map \(c:TE\to E\) such that the left
                  square commutes. This gives a \(T\)-algebra \((E,c)\) which we
                  will show has satisfies the universal property for the
                  equalizer.

                  Let \(h:(Z,d)\to (X,a)\) be a morphism of \(T\)-algebras such
                  that \(fh=gh\). Then we obtain \Cref{fig:dia-2}.
                  \begin{figure}[H]
                      \[
                          \begin{tikzcd}[ampersand replacement=\&]
                              TZ \\
                              \&\& TE \&\& TX \&\& TY \\
                              \\
                              \&\& E \&\& X \&\& Y \\
                              Z
                              \arrow["f"', shift right=1, from=4-5, to=4-7]
                              \arrow["g", shift left=1, from=4-5, to=4-7]
                              \arrow["a", from=2-5, to=4-5]
                              \arrow["b", from=2-7, to=4-7]
                              \arrow["Tf"', shift right=1, from=2-5, to=2-7]
                              \arrow["Tg", shift left=1, from=2-5, to=2-7]
                              \arrow["e", from=4-3, to=4-5]
                              \arrow["Te", from=2-3, to=2-5]
                              \arrow["c", from=2-3, to=4-3]
                              \arrow["Th", curve={height=-12pt}, from=1-1, to=2-5]
                              \arrow["h", curve={height=12pt}, from=5-1, to=4-5]
                              \arrow["d", from=1-1, to=5-1]
                              \arrow["{\exists!\tilde{h}}", dashed, from=5-1, to=4-3]
                              \arrow["{T\tilde{h}}"', from=1-1, to=2-3]
                          \end{tikzcd}
                      \]
                      \caption{Another diagram.}
                      \label{fig:dia-2}
                  \end{figure}
                  I will show \(\tilde{h}\) is a morphism \((Z,d)\to(E,c)\),
                  i.e. the left square commutes. The reasoning goes as follows:
                  \begin{align*}
                      ecT\tilde{h} & =aTeT\tilde{h} \\
                                   & =aTh           \\
                                   & =hd            \\
                                   & =e\tilde{h}d.
                  \end{align*}
                  Because \(e\) is mono this gives \(\tilde{h}d=cT\tilde{h}\) so
                  \(\tilde{h}\) is a morphism of \(T\)-algebras. It is the
                  unique morphism such that the diagram commutes simply because
                  it is the only morphism which makes the triangles commute.
                  Therefore \(E\) is the equalizer.

                  To show \(\alg{T}\) contains small products let
                  \(\set{(X_{i},a_{i})}_{i}\) be a set of \(T\)-algebras. Then
                  because \(\cat\) is complete the product \(\prod_{i}X_{i}\)
                  exists in \(\cat\). As diagram in \Cref{fig:dia-3} below
                  indicates because \(a_{i}T\pi_{i}:T\prod_{i}X_{i}\to X_{i}\)
                  is a morphism for all \(i\) there is a unique
                  \(\prod_{i}a_{i}:T\prod_{i}X_{i}\to\prod_{i}X_{i}\) such that
                  \Cref{fig:dia-3} commutes
                  \begin{figure}[H]
                      \[
                          \begin{tikzcd}[ampersand replacement=\&]
                              {T\prod_{i}X_{i}} \&\& {TX_{i}} \\
                              \\
                              {\prod_{i}X_{i}} \&\& {X_{i}}
                              \arrow["{a_{i}}", from=1-3, to=3-3]
                              \arrow["{\pi_{i}}"', from=3-1, to=3-3]
                              \arrow["{T\pi_{i}}", from=1-1, to=1-3]
                              \arrow["{\exists!\prod_{i}a_{i}}"', dashed, from=1-1, to=3-1]
                          \end{tikzcd}
                      \]
                      \caption{Yet another diagram.}
                      \label{fig:dia-3}
                  \end{figure}
                  Now suppose there is a \(T\)-algebra \((Z,b)\) such that there
                  are \(T\)-algebra morphisms \(h_{i}:Z\to X_{i}\). Then this
                  map lifts in \(\cat\) to a unique map
                  \(\prod_{i}h_{i}:Z\to\prod X_{i}\) by the universal property
                  of the product. Therefore we obtain the diagram of
                  \Cref{fig:dia-3}.
                  \begin{figure}[H]
                      \[
                          \begin{tikzcd}[ampersand replacement=\&]
                              TY \\
                              \&\& {T\prod_{i}X_{i}} \&\& {TX_{i}} \\
                              \\
                              \&\& {\prod_{i}X_{i}} \&\& {X_{i}} \\
                              Y
                              \arrow["{a_{i}}", from=2-5, to=4-5]
                              \arrow["{\pi_{i}}"', from=4-3, to=4-5]
                              \arrow["{T\pi_{i}}"', from=2-3, to=2-5]
                              \arrow["{\prod_{i}a_{i}}"', from=2-3, to=4-3]
                              \arrow["b"', from=1-1, to=5-1]
                              \arrow["Th_{i}", curve={height=-12pt}, from=1-1, to=2-5]
                              \arrow["{h_{i}}"', curve={height=6pt}, from=5-1, to=4-5]
                              \arrow["{\exists!\prod_{i}h_{i}}", dashed, from=5-1, to=4-3]
                              \arrow["{T\prod_{i}h_{i}}"', from=1-1, to=2-3]
                          \end{tikzcd}
                      \]
                      \caption{Yet another diagram\('\).}
                      \label{fig:dia-4}
                  \end{figure}
                  If the left square commutes then \(\prod_{i}h_{i}\) is a
                  morphism of \(T\)-algebras. To see the square commutes notice
                  that \(a_{i}Th_{i}=h_{i}b\) so these maps
                  \(a_{i}Th_{i},h_{i}b:Y\to X_{i}\) induce the same unique map
                  \(g:TY\to\prod_{i}X_{i}\) by the universal property. However
                  \(a_{i}Th_{i}=\pi_{i}\prod_{i}a_{i}T\prod_{i}h_{i}\) so the
                  induces map for all these morphisms must be the same because
                  they are the same map.

                  \begin{figure}[H]
                      \[
                          \begin{tikzcd}[ampersand replacement=\&]
                              TY \&\& {\prod_{i}X_{i}} \\
                              \\
                              Y \&\& {X_{i}}
                              \arrow["b", from=1-1, to=3-1]
                              \arrow["{\exists!g}"', dashed, from=1-1, to=1-3]
                              \arrow["{a_{i}}"', from=1-3, to=3-3]
                              \arrow["{h_{i}}", from=3-1, to=3-3]
                              \arrow["{\prod_{i}h_{i}}"{description}, from=3-1, to=1-3]
                          \end{tikzcd}
                      \]
                      \[
                          \begin{tikzcd}[ampersand replacement=\&]
                              TY \&\& {\prod_{i}X_{i}} \\
                              \& {T\prod_{i}X_{i}}
                              \arrow["{\exists!g}"', dashed, from=1-1, to=1-3]
                              \arrow["{T\prod_{i}h_{i}}"', from=1-1, to=2-2]
                              \arrow["{\prod_{i}a_{i}}"', from=2-2, to=1-3]
                          \end{tikzcd}
                      \]
                      \caption{Two diagrams commutative by the universal property of products.}
                  \end{figure}
                  Because the maps \(g\) in both diagrams are the same this
                  means that \(\prod_{i}a_{i}T\prod_{i}h_{i}=\prod_{i}h_{i}b\).
                  Therefore the left square in \Cref{fig:dia-4} is commutative
                  so \(\prod_{i}h_{i}\) is a morphism of \(T\)-algebras.

                  Therefore \(\alg{T}\) has all small products and equalizers
                  and is complete.

                  To show all limits are preserved by \(U\), by exercise 43 it's
                  sufficient to show \(\alg{T}\) is complete and \(U\) preserves
                  small products and equalizers. The category \(U\) has just
                  been shown to be complete so just the preservation needs to be
                  demonstrated.

                  As demonstrated above, the equalizer for two maps
                  \(f,g:(X,a)\to(Y,b)\) is the equalizer \((E,c)\) for some
                  morphism \(c:TE\to E\). Similarly the small product of
                  \(\set{(X_{i},a_{i})}_{i}\) is
                  \((\prod_{i}X_{i},\prod_{i}a_{i})\) for some morphism
                  \(\prod_{i}a_{i}:T\prod_{i}X_{i}\to\prod_{i}X_{i}\). This
                  means the forgetful functor \(U\) preserves small products and
                  equalizers so it preserves all limits.
              \end{proof}
    \end{enumerate}
\end{question}
\end{document}