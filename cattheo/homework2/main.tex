\documentclass{article}

\usepackage[utf8]{inputenc}
\usepackage{enumerate}
\usepackage{amsthm, amssymb, mathtools, amsmath, bbm, mathrsfs, stmaryrd}
\usepackage[margin=1in]{geometry}
\usepackage[parfill]{parskip}
\usepackage[hidelinks]{hyperref}
\usepackage{quiver}
\usepackage{float}
\usepackage{cleveref}

\newcommand{\N}{\mathbb{N}}
\newcommand{\Z}{\mathbb{Z}}
\newcommand{\NZ}{\mathbb{N}_{0}}
\newcommand{\Q}{\mathbb{Q}}
\newcommand{\R}{\mathbb{R}}
\newcommand{\C}{\mathbb{C}}

\newcommand{\F}{\mathbb{F}}
\newcommand{\incl}{\imath}

\newcommand{\tuple}[2]{\left\langle#1\colon #2\right\rangle}

\DeclareMathOperator{\order}{orde}
\DeclareMathOperator{\Id}{Id}
\DeclareMathOperator{\im}{im}
\DeclareMathOperator{\ggd}{ggd}
\DeclareMathOperator{\kgv}{kgv}
\DeclareMathOperator{\degree}{gr}
\DeclareMathOperator{\coker}{coker}

\DeclareMathOperator{\gl}{GL}

\DeclareMathOperator{\Aut}{Aut}
\DeclareMathOperator{\Hom}{Hom}
\DeclareMathOperator{\End}{End}
\newcommand{\isom}{\overset{\sim}{\longrightarrow}}

\newcommand{\Grp}{{\bf Grp}}
\newcommand{\Ab}{{\bf Ab}}
\newcommand{\cring}{{\bf CRing}}
\DeclareMathOperator{\modules}{{\bf Mod}}
\newcommand{\catset}{{\bf Set}}
\newcommand{\cat}{\mathcal{C}}
\newcommand{\catt}{\mathcal{D}}
\newcommand{\pcat}{\mathcal{P}}
\newcommand{\chains}{{\bf Ch}}
\newcommand{\homot}{{\bf Ho}}
\DeclareMathOperator{\objects}{Ob}
\newcommand{\gen}[1]{\left<#1\right>}
\DeclareMathOperator{\cone}{Cone}
\newcommand{\set}[1]{\left\{#1\right\}}
\newcommand{\setwith}[2]{\left\{#1:#2\right\}}
\DeclareMathOperator{\Ext}{Ext}
\DeclareMathOperator{\nil}{Nil}
\DeclareMathOperator{\idem}{Idem}
\DeclareMathOperator{\rad}{Rad}

\newenvironment{question}[1][]{\begin{paragraph}{Question #1}}{\end{paragraph}}
\newenvironment{solution}{\begin{proof}[Solution]\renewcommand\qedsymbol{}}{\end{proof}}

\newtheorem{theorem}{Theorem}[section]
\newtheorem{lemma}[theorem]{Lemma}
\newtheorem{proposition}[theorem]{Proposition}


\theoremstyle{definition}
\newtheorem{definition}[theorem]{Definition}
\newtheorem{remark}[theorem]{Remark}
\newtheorem{example}[theorem]{Example}
\newtheorem{corollary}[theorem]{Corollary}

\title{Homework Category Theory}
\author{Jonas van der Schaaf}
\date{October 10, 2022}

\begin{document}
\maketitle

\begin{question}[1]
    Let \(\cat\) be a category with binary products. Show that for any triples
    \(X,Y,Z\in\objects(\cat)\) we have the isomorphism
    \[
        X\times(Y\times Z)\cong (X\times Y)\times Z.
    \]

    \begin{proof}
        To prove the products are isomorphic, I will show \((X\times Y)\times
        Z\) satisfies the universal property of \(X\times(Y\times Z)\). Let
        \(A\) be an object with maps \(A\to X,A\to Y,A\to Z\), then there are
        unique maps \(A\to X\times Y\) and \(A\to Y\times Z\) such that
        \Cref{fig:comm-dia-1} commutes.
        \begin{figure}[H]
            \[
                \begin{tikzcd}[ampersand replacement=\&]
                    \&\&\&\& A \\
                    \\
                    \\
                    X \&\& {X\times Y} \&\& Y \&\& {Y\times Z} \&\& Z \\
                    \arrow["f"',from=1-5, to=4-1]
                    \arrow["g",from=1-5, to=4-5]
                    \arrow["h",from=1-5, to=4-9]
                    \arrow["\pi_{Z}",from=4-7, to=4-9]
                    \arrow["\pi_{Y}'"',from=4-7, to=4-5]
                    \arrow["\pi_{Y}", from=4-3, to=4-5]
                    \arrow["\pi_{X}"',from=4-3, to=4-1]
                    \arrow["{\exists!}f\times g", dashed, from=1-5, to=4-3]
                    \arrow["{\exists!}g\times h"', dashed, from=1-5, to=4-7]
                \end{tikzcd}
            \]
            \caption{The commutative diagram indicating the universal property of the products.}
            \label{fig:comm-dia-1}
        \end{figure}
        This means that there are also unique maps \(A\to (X\times Y)\times Z\)
        and \(A\to X\times(Y\times Z)\) such that the diagram in
        \Cref{fig:comm-dia-2} commutes.
        \begin{figure}[H]
            \[
                \begin{tikzcd}[ampersand replacement=\&]
                    \&\&\&\& A \\
                    \\
                    \\
                    X \&\& {X\times Y} \&\& Y \&\& {Y\times Z} \&\& Z \\
                    \&\&\& {(X\times Y)\times Z} \&\& {X\times(Y\times Z)}
                    \arrow["f"',from=1-5, to=4-1]
                    \arrow["g",from=1-5, to=4-5]
                    \arrow["h",from=1-5, to=4-9]
                    \arrow["\pi_{Z}",from=4-7, to=4-9]
                    \arrow["\pi_{Y}'"',from=4-7, to=4-5]
                    \arrow["\pi_{Y}", from=4-3, to=4-5]
                    \arrow["\pi_{X}"',from=4-3, to=4-1]
                    \arrow["{f\times g}", from=1-5, to=4-3]
                    \arrow["{g\times h}"', from=1-5, to=4-7]
                    \arrow["\pi_{Y\times Z}"',from=5-6, to=4-7]
                    \arrow["\pi_{X}'", , curve={height=-35pt}, from=5-6, to=4-1]
                    \arrow["\pi_{X\times Y}", from=5-4, to=4-3]
                    \arrow["\pi_{Z}'"', curve={height=35pt}, from=5-4, to=4-9]
                    \arrow["{\exists!(f\times g)\times h}"', dashed, from=1-5, to=5-4]
                    \arrow["{\exists!f\times(g\times h)}", dashed, from=1-5, to=5-6]
                \end{tikzcd}
            \]
            \caption{The commutative diagram indicating the universal properties of \((X\times Y)\times Z\) and \(X\times (Y\times Z)\).}
            \label{fig:comm-dia-2}
        \end{figure}
        On top of that because there are maps \(\pi_{Y}\circ\pi_{X\times
            Y}:(X\times Y)\times Z\to Y\) and \(\pi_{Z}':(X\times Y)\times Z\to Z\)
        there is a unique map \(\pi_{Y\times Z}':(X\times Y)\times Z\to (Y\times
        Z)\) such that the diagram commutes. Note that this map is not dependent
        on \(f,g,h\).

        Now suppose there is a map \(k:A\to X\times Y\) and \(h:A\to Z\). Then
        by composition of \(k\) with \(\pi_{X},\pi_{Y}\) we obtain maps
        \(f=\pi_{X}k\) and \(g=\pi_{Y}k\). By the above reasoning there are
        unique maps \((f\times g)\times h\) and \(f\times(g\times h)\) such that
        \Cref{fig:comm-dia-3} commutes.
        \begin{figure}[H]
            \[
                \begin{tikzcd}[ampersand replacement=\&]
                    \&\&\&\& A \\
                    \\
                    \\
                    X \&\& {X\times Y} \&\& Y \&\& {Y\times Z} \&\& Z \\
                    \&\&\& {(X\times Y)\times Z} \&\& {X\times(Y\times Z)}
                    \arrow["f"',from=1-5, to=4-1]
                    \arrow["g",from=1-5, to=4-5]
                    \arrow["h",from=1-5, to=4-9]
                    \arrow["\pi_{Z}",from=4-7, to=4-9]
                    \arrow["\pi_{Y}'"',from=4-7, to=4-5]
                    \arrow["\pi_{Y}", from=4-3, to=4-5]
                    \arrow["\pi_{X}"',from=4-3, to=4-1]
                    \arrow["{f\times g}", from=1-5, to=4-3]
                    \arrow["{g\times h}"', from=1-5, to=4-7]
                    \arrow["\pi_{Y\times Z}"',from=5-6, to=4-7]
                    \arrow["\pi_{X}'", , curve={height=-35pt}, from=5-6, to=4-1]
                    \arrow["\pi_{X\times Y}", from=5-4, to=4-3]
                    \arrow["\pi_{Z}'"', curve={height=35pt}, from=5-4, to=4-9]
                    \arrow["{(f\times g)\times h}"', from=1-5, to=5-4]
                    \arrow["{f\times(g\times h)}", from=1-5, to=5-6]
                    \arrow["\pi_{Y\times Z}'", from=5-4, to=4-7]
                \end{tikzcd}
            \]
            \caption{The commutative diagram indicating the universal properties of \(X\times(Y\times Z)\) is satisfied by \((X\times Y)\times Z\).}
            \label{fig:comm-dia-3}
        \end{figure}
        But then this means that \((X\times Y)\times Z\) satisfies the universal
        property of \(X\times(Y\times Z)\) so it must be isomorphic.
    \end{proof}
\end{question}

\begin{question}
    Let \(\cat\) be a category which has equalizers. Furthermore let \(\pcat\)
    be the category generated by the quiver
    \[
        \begin{tikzcd}[ampersand replacement=\&]
            1 \&\& 2
            \arrow["f", shift left=1, from=1-1, to=1-3]
            \arrow["g"',shift right=1, from=1-1, to=1-3]
        \end{tikzcd}
    \]

    Show that there is a functor \(F:[\pcat,\cat]\to\cat\) which maps a functor
    \(G\) to the equalizer of \(G(f)\) and \(G(g)\).

    \begin{proof}
        Define \(F\) as above on the functors \(\pcat\to\cat\). I will show that
        each natural transformation \(\eta:G\to G'\) for \(G,G':\pcat\to\cat\)
        induces a unique morphism \(F(G)\to F(G')\).

        For any two functors \(G,G':\pcat\to\cat\) and \(\eta:G\to G'\) a
        natural transformation we have the following (in general not
        commutative because of the parallel arrows in the square) diagram:
        \[
            \begin{tikzcd}[ampersand replacement=\&]
                E \&\& G1 \&\& G2 \\
                \\
                {E'} \&\& {G'1} \&\& {G'2}
                \arrow["Gg"', shift right=1, from=1-3, to=1-5]
                \arrow["Gf", shift left=1, from=1-3, to=1-5]
                \arrow["{G'f}", shift left=1, from=3-3, to=3-5]
                \arrow["{G'g}"', shift right=1, from=3-3, to=3-5]
                \arrow["{\eta_{1}}"{description}, from=1-3, to=3-3]
                \arrow["{\eta_{2}}"{description}, from=1-5, to=3-5]
                \arrow["e"', from=1-1, to=1-3]
                \arrow["{e'}"', from=3-1, to=3-3]
            \end{tikzcd}
        \]
        Using diagram chasing we obtain
        \begin{align*}
            (G'f)\eta_{1}e & =\eta_{2}(Gf)e  \\
                           & =\eta_{2}(Gg)e  \\
                           & =(Gg)\eta_{1}e,
        \end{align*}
        which by the universal property of the equalizer means that
        \(\eta_{1}e\) factorizes through the equalizer \(E'\):
        \[
            \begin{tikzcd}[ampersand replacement=\&]
                E \&\& G1 \&\& G2 \\
                \\
                {E'} \&\& {G'1} \&\& {G'2}
                \arrow["Gg"', shift right=1, from=1-3, to=1-5]
                \arrow["Gf", shift left=1, from=1-3, to=1-5]
                \arrow["{G'f}", shift left=1, from=3-3, to=3-5]
                \arrow["{G'g}"', shift right=1, from=3-3, to=3-5]
                \arrow["{\eta_{1}}"{description}, from=1-3, to=3-3]
                \arrow["{\eta_{2}}"{description}, from=1-5, to=3-5]
                \arrow["e"', from=1-1, to=1-3]
                \arrow["{e'}"', from=3-1, to=3-3]
                \arrow["{\exists!h}"{description}, dashed, from=1-1, to=3-1]
            \end{tikzcd}
        \]
        We define \(F\eta=h\).

        It is easy to see that for \(G'=G\) and \(\eta=\Id\) the map \(\Id:E\to
        E'=E\) makes the diagram commute so \(F\Id_{G}=\Id_{FG}\).

        To see this map is multiplicative consider the diagram
        \[
            \begin{tikzcd}[ampersand replacement=\&]
                E \&\& G1 \&\& G2 \\
                \\
                {E'} \&\& {G'1} \&\& {G'2} \\
                \\
                {E''} \&\& {G''1} \&\& {G''2}
                \arrow["Gg"', shift right=1, from=1-3, to=1-5]
                \arrow["Gf", shift left=1, from=1-3, to=1-5]
                \arrow["{G'f}", shift left=1, from=3-3, to=3-5]
                \arrow["{G'g}"', shift right=1, from=3-3, to=3-5]
                \arrow["{\eta_{1}}"{description}, from=1-3, to=3-3]
                \arrow["{\eta_{2}}"{description}, from=1-5, to=3-5]
                \arrow["e"', from=1-1, to=1-3]
                \arrow["{e'}"', from=3-1, to=3-3]
                \arrow["{\exists!h}"{description}, dashed, from=1-1, to=3-1]
                \arrow["{\eta'_{1}}"{description}, from=3-3, to=5-3]
                \arrow["{\eta'_{2}}"{description}, from=3-5, to=5-5]
                \arrow["{G''f}", shift left=1, from=5-3, to=5-5]
                \arrow["{G''g}"', shift right=1, from=5-3, to=5-5]
                \arrow["{\exists!h'}"', dashed, from=3-1, to=5-1]
                \arrow["{e''}"', from=5-1, to=5-3]
            \end{tikzcd}
        \]
        Because adjacent commutative squares commute we get that
        \(\eta'_{1}\circ\eta_{1}\circ e=e''h'h\) must factor through the
        equalizer \(E''\) with a unique map \(h''\). This map must be \(h'h\)
        because the left squares are commutative and therefore \(h'h\) satisfies
        the requirements for the unique map.
    \end{proof}
\end{question}
\end{document}