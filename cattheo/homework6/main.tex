\documentclass{article}

\usepackage[utf8]{inputenc}
\usepackage{enumerate}
\usepackage{amsthm, amssymb, mathtools, amsmath, bbm, mathrsfs, stmaryrd}
\usepackage[margin=1in]{geometry}
\usepackage[parfill]{parskip}
\usepackage[hidelinks]{hyperref}
\usepackage{quiver}
\usepackage{float}
\usepackage{cleveref}

\newcommand{\N}{\mathbb{N}}
\newcommand{\Z}{\mathbb{Z}}
\newcommand{\NZ}{\mathbb{N}_{0}}
\newcommand{\Q}{\mathbb{Q}}
\newcommand{\R}{\mathbb{R}}
\newcommand{\C}{\mathbb{C}}

\newcommand{\F}{\mathbb{F}}
\newcommand{\incl}{\imath}

\newcommand{\tuple}[2]{\left\langle#1\colon #2\right\rangle}

\DeclareMathOperator{\order}{orde}
\DeclareMathOperator{\Id}{Id}
\DeclareMathOperator{\im}{im}
\DeclareMathOperator{\ggd}{ggd}
\DeclareMathOperator{\kgv}{kgv}
\DeclareMathOperator{\degree}{gr}
\DeclareMathOperator{\coker}{coker}

\DeclareMathOperator{\gl}{GL}

\DeclareMathOperator{\Aut}{Aut}
\DeclareMathOperator{\Hom}{Hom}
\DeclareMathOperator{\End}{End}
\newcommand{\isom}{\overset{\sim}{\longrightarrow}}

\newcommand{\Grp}{{\bf Grp}}
\newcommand{\Ab}{{\bf Ab}}
\newcommand{\cring}{{\bf CRing}}
\newcommand{\catcat}{{\bf Cat}}
\DeclareMathOperator{\modules}{{\bf Mod}}
\newcommand{\catset}{{\bf Set}}
\newcommand{\cat}{\mathcal{C}}
\newcommand{\catt}{\mathcal{D}}
\newcommand{\cattt}{\mathcal{E}}
\newcommand{\pcat}{\mathcal{P}}
\newcommand{\chains}{{\bf Ch}}
\newcommand{\homot}{{\bf Ho}}
\DeclareMathOperator{\objects}{Ob}
\newcommand{\gen}[1]{\left<#1\right>}
\DeclareMathOperator{\cone}{Cone}
\newcommand{\set}[1]{\left\{#1\right\}}
\newcommand{\setwith}[2]{\left\{#1:#2\right\}}
\newcommand{\setcat}{{\bf Set}}
\DeclareMathOperator{\Ext}{Ext}
\DeclareMathOperator{\nil}{Nil}
\DeclareMathOperator{\idem}{Idem}
\DeclareMathOperator{\rad}{Rad}
\newcommand{\alg}[1]{T\textnormal{-Alg}}
\DeclareMathOperator{\ev}{ev}
\DeclareMathOperator{\psh}{Psh}
\DeclareMathOperator{\famm}{Fam}
\DeclareMathOperator{\dfib}{DFib}

\newenvironment{solution}{\begin{proof}[Solution]\renewcommand\qedsymbol{}}{\end{proof}}

\newtheorem{theorem}{Theorem}[section]
\newtheorem{lemma}[theorem]{Lemma}
\newtheorem{proposition}[theorem]{Proposition}


\theoremstyle{definition}
\newtheorem{question}{Exercise}
\newtheorem{definition}[theorem]{Definition}
\newtheorem{remark}[theorem]{Remark}
\newtheorem{example}[theorem]{Example}
\newtheorem{corollary}[theorem]{Corollary}

\title{Homework Category Theory}
\author{Jonas van der Schaaf}
\date{}

\begin{document}
\maketitle

\begin{question}
    \begin{enumerate}[(a)]
        \item Show that if \(X\) is a presheaf over \(\cat\), then the forgetful
              functor \(U:y\downarrow X\to \cat\) is a discrete fibration.

              \begin{proof}
                  Let \((C',\nu)\) be an object of the comma category
                  \(y\downarrow X\) and \(f\in\Hom(C,UyC')=\Hom(C,C')\) a
                  morphism of \(\cat\). Take \(\eta=yf:yC\to yC'\) the natural
                  transformation and consider the object \((yC,\nu\circ\eta)\)
                  of the comma category. Then by definition the following
                  diagram commutes:
                  \begin{figure}[H]
                      \[
                          \begin{tikzcd}[ampersand replacement=\&]
                              {yC} \&\& yC' \\
                              \& X
                              \arrow["yf", from=1-1, to=1-3]
                              \arrow["{\nu\circ yf}"', from=1-1, to=2-2]
                              \arrow["\nu", from=1-3, to=2-2]
                          \end{tikzcd}
                      \]
                      \caption{A commutative diagram showing that \(yf\) is a morphism in the comma category \(y\downarrow X\).}
                      \label{fig:comm-dia-2}
                  \end{figure}
                  From this we see that \(yf:yC\to yC'\) is a morphism in the
                  comma category. We have by definition of the forgetful functor
                  \(Uyf=f\) and \(yf\) is unique in this regard because this
                  forgetful functor \(U\) is faithful.

                  This shows \(U\) is a discrete fibration.
              \end{proof}

        \item Show there is a functor \(\psh(\cat)\to\dfib(\cat)\) which is an
              equivalence of categories.

              \begin{proof}
                  We define a functor \(F:\psh(\cat)\to\dfib(\cat)\) which acts
                  on objects in the following manner \(FX=(y\downarrow
                  X,U_{X}:y\downarrow X\to\cat)\) mapping \(X\) to the forgetful
                  functor from part a. For a function \(f:X\to Y\) it maps
                  \((yC,\nu)\) to \((yC,f\circ\nu)\). We now show this is a
                  functor \(\psh(\cat)\to\dfib(\cat)\):

                  The identity morphism results in the identity functor because
                  postcomposition with identity is the identity. The map is
                  multiplicative because for \(f:X\to Y\) and \(g:Y\to Z\) we
                  have \(g\circ(f\circ\nu)=(g\circ f)\circ\nu\) by asociativity
                  of function composition.

                  To show \(F\) is an equivalence we show it's fully faithful
                  and essentially surjective.

                  To show faithfulness we take any \(f,g:X\to Y\) and show that
                  if \(f\circ\nu=g\circ\nu\) for all \(C\in\objects(\cat)\) and
                  morphisms \(\nu:yC\to X\) then \(f=g\). Suppose \(f\) and
                  \(g\) are as above. Take any object \(C\) of \(\cat\) and any
                  \(x\in X(C)\). By the Yoneda lemma there is a unique
                  \(\nu:yC\to X\) such that \(\nu_{C}(\Id_{C})=x\). This means
                  in particular that
                  \begin{align*}
                      f_{C}(x) & =f_{C}(\nu_{C}(\Id_{C}))    \\
                               & =f_{C}\circ\nu_{C}(\Id_{C}) \\
                               & =g_{C}\circ\nu_{C}(\Id_{C}) \\
                               & =g_{C}(x).
                  \end{align*}
                  This means \(f_{C}=g_{C}\) so \(f=g\). This proves
                  faithfulness of \(F\).

                  To show fullness take any \(f:y\downarrow X\to y\downarrow Y\)
                  for two presheaves \(X,Y\). We will construct a \(g:X\to Y\)
                  such that \(Fg=f\). For any object \(C\) of \(\cat\) define
                  \(g_{C}(x)=(F\nu^{x})(\Id_{C})\) where \(\nu^{x}:yC\to X\) is
                  the unique natural transformation associated to \(x\) by the
                  Yoneda lemma. We will show this is a morphism of presheaves
                  such that \(Fg=f\).

                  By definition of \(g\) we must have that
                  \(g_{C}\circ\nu_{C}=(F\nu)_{C}\).

                  Now we will show that this is a morphism of presheaves. Let
                  \(C,D\) be any two objects of \(\cat\) and \(f:C\to D\) a
                  morphism. Then take any \(x\in XC\) and let \(\nu^{x}\) be the
                  associated natural transformation \(\nu^{x}:yC\to XC\) given
                  by the Yoneda lemma. We then get the diagram of
                  \Cref{fig:comm-dia-3}, where the top square is commutative and
                  the outer square is as well because \(g\circ\nu=F\nu\) which
                  is a morphism of presheaves.
                  \begin{figure}[H]
                      \[
                          \begin{tikzcd}[ampersand replacement=\&]
                              yC \&\& yD \\
                              \\
                              XC \&\& XD \\
                              \\
                              YC \&\& YD
                              \arrow["yf"', from=1-1, to=1-3]
                              \arrow["{\nu_{C}}", from=1-1, to=3-1]
                              \arrow["{\nu_{D}}"', from=1-3, to=3-3]
                              \arrow["Xf", from=3-1, to=3-3]
                              \arrow["{g_{C}}", from=3-1, to=5-1]
                              \arrow["{g_{D}}", from=3-3, to=5-3]
                              \arrow["Yf"{description}, from=5-1, to=5-3]
                          \end{tikzcd}
                      \]
                      \caption{A diagram where the outer and upper square commute.}
                      \label{fig:comm-dia-3}
                  \end{figure}
                  We then have
                  \begin{align*}
                      Yf\circ g_{C}(x) & =Yf\circ g_{C}\circ\nu^{x}_{C}(\Id_{C}) \\
                                       & =g_{D}\circ\nu_{D}\circ yf(\Id_{C})     \\
                                       & =g_{D}\circ X_{f}\circ\nu_{C}(\Id_{C})  \\
                                       & =g_{D}\circ X_{f}(x).
                  \end{align*}
                  From this we conclude that \(Yf\circ g_{C}=g_{D}\circ X_{f}\)
                  so \(g\) is a morphism of presheaves with \(Fg=f\). This
                  proves
                  \(F\) is full as well.

                  To show essential surjectivity let \((\catt,G:\catt\to\cat)\)
                  be an object of \(\dfib(\cat)\), in particular \(G\) is a full
                  fibration. We will define a presheaf \(X\) such that
                  \((y\downarrow X,U_{X}:y\downarrow X\to \cat)\) is isomorphic
                  to \((\catt,G)\).

                  We define the set \(X(C)\) to be the set
                  \(\setwith{E\in\catt}{FE=C}\). For any map
                  \(f\in\Hom_{\cat}(C,D)\) and \(E\in X(D)\) we define \(Xf(E)\)
                  as the unique \(E'\in X(C)\) such that there is a \(g:E'\to
                  E\) such that \(GE'=C\) and \(Gg=f\) (we get well-definedness
                  because \(G\) is a discrete fibration). We show that this
                  defines a presheaf.

                  If \(f:C\to C\) is the identity then for \(E\in X(C)\) we see
                  that \(E\) and \(\Id_{E}:E\to E\) satisfy the desired
                  property. This means \(Xf(E)=E\) for all \(E\in X(C)\) so
                  \(Xf\) is the identity morphism.

                  To show multiplicativity take any \(f_{1}:C_{1}\to C_{2}\) and
                  \(f_{2}:C_{2}\to C_{3}\) and \(E\in X(C_{3})\) if it exists.
                  We want to show that \(X(f_{2}\circ f_{1})=X(f_{1})\circ
                  X(f_{2})\). Since \(F\) is a discrete fibration there are
                  unique \(g_{1}:D_{1}\to D_{2}\) and \(g_{2}:D_{2}\to D_{3}\)
                  with \(D_{3}=E\) such that \(Fg_{1}=f_{1}\) and
                  \(Fg_{2}=f_{2}\). We then have that \(X(f_{1})\circ
                  X(f_{2})(E)=D_{1}\). On the other hand we have \(F(g_{2}\circ
                  g_{1})=f_{2}\circ f_{1}\). So \(g_{2}\circ g_{1}\) is the
                  unique map as given by the fibration so \(X(f_{2}\circ
                  f_{1})(E)=D_{1}\). This proves that \(X\) is multiplicative
                  and so that \(X\) is a functor.

                  We will now show that \((\catt,F)\cong (y\downarrow
                  X,U_{X})\). We can define for each object
                  \(E\in\objects(\catt)\) a presheaf \(yFE\) and natural
                  transformation \(\nu^{E}:yFE\to X\) such that
                  \(\nu^{E}_{FE}(\Id_{FE})=E\in X(FE)\). Define the functor
                  \(G:\catt\to y\downarrow X\) as given by \(G(E)=(FE,\nu^{E})\)
                  on objects. Given a morphism \(f:E\to E'\) define
                  \(Gf:(FE,\nu^{E})\to (FE',\nu^{E'})\) as the functor
                  \(Gf=yFf\). We will show this is a morphism in the comma
                  category. This boils down to showing commutativity of the
                  diagram in \Cref{fig:comm-dia-4}.
                  \begin{figure}[H]
                      \[
                          \begin{tikzcd}[ampersand replacement=\&]
                              yFE \&\& {yFE'} \\
                              \& X
                              \arrow["{yFf}", from=1-1, to=1-3]
                              \arrow["{\nu^{E}}"', from=1-1, to=2-2]
                              \arrow["{\nu^{E'}}", from=1-3, to=2-2]
                          \end{tikzcd}
                      \]
                      \caption{A diagram indicating the defintion of the functor \(G\).}
                      \label{fig:comm-dia-4}
                  \end{figure}
                  By the Yoneda lemma it suffices to show that the above maps
                  agree on the image of \(\Id_{FE}\). By definition we have
                  \(\nu^{E}_{FE}(\Id_{FE})=E\). We want to show that
                  \[
                      \nu^{E'}_{FE'}\circ yFf(\Id_{FE})=\nu^{E'}_{FE'}(Ff)=E.
                  \]
                  By naturality we have the diagram of \Cref{fig:comm-dia-5}.
                  In this diagram the image of \(\Id_{FE'}\) is mapped by
                  \((yFE')(Ff)\) onto \(Ff\). On the other hand we have
                  \begin{align*}
                      X(Ff)\circ\nu^{E'}_{FE'}(\Id_{FE'}) & =X(Ff)(E') \\
                                                          & =E.
                  \end{align*}
                  This proves that we must have \(\nu^{E'}_{FE'}(Ff)=E\) as well
                  as the diagram commutes.
                  \begin{figure}[H]
                      \[
                          \begin{tikzcd}[ampersand replacement=\&]
                              {yFE'(FE')} \&\& {yFE'(FE)} \\
                              \\
                              {X(FE')} \&\& {X(FE)}
                              \arrow["{(yFE')(Ff)}", from=1-1, to=1-3]
                              \arrow["{\nu^{E'}_{FE}}", from=1-3, to=3-3]
                              \arrow["{\nu^{E'}_{FE'}}"', from=1-1, to=3-1]
                              \arrow["{X(Ff)}"', from=3-1, to=3-3]
                          \end{tikzcd}
                      \]
                      \caption{A diagram with a natural transformation.}
                      \label{fig:comm-dia-5}
                  \end{figure}
                  This means that \(G\) maps morphisms to morphisms.

                  It's easy to see \(G\) preserves identity because
                  by functoriality of \(F\) and \(y\)
                  \begin{align*}
                      G\Id_{E} & =yF\Id_{E}  \\
                               & =y\Id_{FE}  \\
                               & =\Id_{yFE}.
                  \end{align*}
                  Multiplicativity goes very similar:
                  \begin{align*}
                      G(f\circ g) & =yF(f\circ g)     \\
                                  & =yF(f)\circ yF(g) \\
                                  & =Gf\circ Gg.
                  \end{align*}
                  Now we show this functor is an isomorphism of categories by
                  constructing an inverse. We construct an inverse map on
                  objects and images. This suffices because this is the inverse
                  of a functor and therefore automatically a functor.

                  For any object \(E\in\objects(\catt)\) we can easily recover
                  it from \(GE=(FE,\nu^{E})\) by computing
                  \[
                      \nu^{E}_{FE}(\Id_{yFE})=E.
                  \]

                  Similarly for any function \(f,f':E\to E'\) we must have that
                  if \(yFf=Gf=Gf'=yFf'\) then \(f=f'\) because \(F\) is a
                  discrete fibration and therefore faithful and \(y\) is
                  faithful by the Yoneda lemma.

                  Now take any morphism \(f:(yFE,\nu^{E})\to (yFE',\nu^{E'})\)
                  in the comma category. By the Yoneda lemma there is some
                  \(f':FE\to FE'\) such that \(yf'=f\). Because \(F\) is a
                  discrete fibration this gives a \(g:E\to E'\) with \(Fg=f'\)
                  so \(Gg=yFg=yf'=f\).

                  This shows that \(G\) is a bijective functor and therefore an
                  isomorphism.

                  This shows that \(F:\catt\to \dfib(\cat)\) is a fully faithful
                  and essentially surjective functor and therefore an
                  equivalence.
              \end{proof}

        \item Suppose we have a commuting triangle in \(\catcat\) of the form
              \[
                  \begin{tikzcd}[ampersand replacement=\&]
                      \cattt \&\& \catt \\
                      \& \cat
                      \arrow["F", from=1-1, to=1-3]
                      \arrow["H"', from=1-1, to=2-2]
                      \arrow["G", from=1-3, to=2-2]
                  \end{tikzcd}
              \]
              Show that if this \(G\) is a discrete fibration then \(H\) is if
              and only if \(F\) is. Use this to do show \(\psh(\cat)/X\cong
              \psh(y\downarrow X)\) for any presheaf \(X\) over \(\cat\).

              \begin{proof}
                  Suppose \(F\) is a discrete fibration and let
                  \(C\in\objects(\cat)\) and \(E\in\objects(\catt)\) such that
                  there is a morphism \(f:C\to HE\). By assumption \(G\) is a
                  discrete fibration and \(H=G\circ F\). This means that by
                  rewriting \(f\) as \(f:C\to GFE\) there must be a unique
                  \(C'\in\objects\catt\) and \(f':C'\to FE\) such that \(GC'=C\)
                  and \(Gf'=f\). Because \(F\) is also a discrete fibration this
                  means there is a unique \(C''\in\objects(\cattt)\) and
                  \(f'':C''\to E\) such that \(FC''=C'\) and \(Ff''=f'\). This
                  means that \(C''\) and \(f''\) must also be a unique
                  object-morphism pair such that \(HC''=C\) and \(Hf''=f\)
                  because the triange commutes.

                  Suppose \(H\) is a discrete fibration,
                  \(D\in\objects(\catt)\), and \(f:D\to FE\) for some object
                  \(E\in\objects(\cattt)\). Applying the functor \(G\) to both
                  gives \(GD\) and \(Gf:GD\to GFE=HE\). Because \(H\) is a
                  discrete fibration there is a unique \(E'\in\objects(\cattt)\)
                  and \(f':E'\to E\) such that \(Hf:HE'\to HE=Gf:GD\to GFE\).
                  Applying the functor \(F\) to \(f'\) and \(E'\) gives two
                  objects \(FE'\) and \(Ff'\) such that \(GFE=FC\) and
                  \(GFf'=f\). Because \(G\) is a discrete fibration by the
                  uniqueness condition this means that \(FE'=D\) and \(Ff'=f\).
                  These \(E'\) and \(f'\) were the unique objects with this
                  condition and so \(F\) is a discrete fibration.

                  Now let \(X\) be a presheaf over a category \(\cat\). Applying
                  the equivalence \(\psh(\cat)\cong\dfib(\cat)\) this means
                  there is also an induced equivalence of categories
                  \(\psh(\cat)/X\to\dfib(\cat)/U_{X}\). This slice category has
                  objects \((\catt,H)\) with \(H\) a discrete fibration and a
                  functor \(F:\catt\to y\downarrow X\) such that \(F\circ
                  U_{X}=Y\). From this we know that \(F\) is a discrete
                  fibration as well. This gives an equivalence
                  \[
                      \dfib(\cat)/U_{X}\cong \dfib(y\downarrow X)\cong\psh(y\downarrow X).
                  \]
              \end{proof}
    \end{enumerate}
\end{question}
\end{document}