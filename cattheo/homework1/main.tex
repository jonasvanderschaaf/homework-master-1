\documentclass{article}

\usepackage[utf8]{inputenc}
\usepackage{enumerate}
\usepackage{amsthm, amssymb, mathtools, amsmath, bbm, mathrsfs, stmaryrd}
\usepackage[margin=1in]{geometry}
\usepackage[parfill]{parskip}
\usepackage[hidelinks]{hyperref}
\usepackage{quiver}
\usepackage{float}
\usepackage{cleveref}

\newcommand{\N}{\mathbb{N}}
\newcommand{\Z}{\mathbb{Z}}
\newcommand{\NZ}{\mathbb{N}_{0}}
\newcommand{\Q}{\mathbb{Q}}
\newcommand{\R}{\mathbb{R}}
\newcommand{\C}{\mathbb{C}}

\newcommand{\F}{\mathbb{F}}
\newcommand{\incl}{\imath}

\newcommand{\tuple}[2]{\left\langle#1\colon #2\right\rangle}

\DeclareMathOperator{\order}{orde}
\DeclareMathOperator{\Id}{Id}
\DeclareMathOperator{\im}{im}
\DeclareMathOperator{\ggd}{ggd}
\DeclareMathOperator{\kgv}{kgv}
\DeclareMathOperator{\degree}{gr}
\DeclareMathOperator{\coker}{coker}

\DeclareMathOperator{\gl}{GL}

\DeclareMathOperator{\Aut}{Aut}
\DeclareMathOperator{\Hom}{Hom}
\DeclareMathOperator{\End}{End}
\newcommand{\isom}{\overset{\sim}{\longrightarrow}}

\newcommand{\Grp}{{\bf Grp}}
\newcommand{\Ab}{{\bf Ab}}
\newcommand{\cring}{{\bf CRing}}
\DeclareMathOperator{\modules}{{\bf Mod}}
\newcommand{\catset}{{\bf Set}}
\newcommand{\cat}{\mathcal{C}}
\newcommand{\catt}{\mathcal{D}}
\newcommand{\chains}{{\bf Ch}}
\newcommand{\homot}{{\bf Ho}}
\DeclareMathOperator{\objects}{Ob}
\newcommand{\gen}[1]{\left<#1\right>}
\DeclareMathOperator{\cone}{Cone}
\newcommand{\set}[1]{\left\{#1\right\}}
\newcommand{\setwith}[2]{\left\{#1:#2\right\}}
\DeclareMathOperator{\Ext}{Ext}
\DeclareMathOperator{\nil}{Nil}
\DeclareMathOperator{\idem}{Idem}
\DeclareMathOperator{\rad}{Rad}

\newenvironment{question}[1][]{\begin{paragraph}{Question #1}}{\end{paragraph}}
\newenvironment{solution}{\begin{proof}[Solution]\renewcommand\qedsymbol{}}{\end{proof}}

\newtheorem{theorem}{Theorem}[section]
\newtheorem{lemma}[theorem]{Lemma}
\newtheorem{proposition}[theorem]{Proposition}


\theoremstyle{definition}
\newtheorem{definition}[theorem]{Definition}
\newtheorem{remark}[theorem]{Remark}
\newtheorem{example}[theorem]{Example}
\newtheorem{corollary}[theorem]{Corollary}

\title{Homework Category Theory}
\author{Jonas van der Schaaf}
\date{September 22, 2022}

\begin{document}
\maketitle

\begin{question}[1]
    Throughout this exercise \(\cat\) is a locally small category.

    \begin{enumerate}[(a)]
        \item Let \(C\) be an object in \(\cat\). Show that there is a functor
              \(G:\cat\to\catset\) which sends an object \(X\) to \(\Hom(C,X)\)
              and whose action on morphisms in \(\cat\) is given by
              postcomposition.

              \begin{proof}
                  Consider the given map \(G\). We will show that it is
                  functorial: both compatible with function composition and it
                  preserves the unit map.

                  Let \(X,Y,Z\) be objects of \(\cat\) with \(f\in\Hom(X,Y)\)
                  and \(g\in\Hom(Y,Z)\). Then for all \(h\)
                  \begin{align*}
                      G(gf)h & =gfh        \\
                             & =Gg(fh)     \\
                             & =(Gg)(Gf)h.
                  \end{align*}
                  This means that \(Ggf=(Gg)(Gf)\) for all compatible maps
                  \(g,f\).
                  On top of that for all \(f\in\Hom(X,Y)\)
                  \begin{align*}
                      (G\Id)f & =\Id f \\
                              & =f     \\
                  \end{align*}
                  so this map preserves identity. It is therefore a functor.
              \end{proof}

        \item Show that a morphism is an absolute epi iff it is split epi.

              \begin{proof}
                  If a morphism \(f:X\to Y\) is split epi and \(\catt\) any
                  category, then it is trivially absolute epi. This is because
                  there is a map \(g:Y\to X\) with \(fg=\Id\). This means that
                  all functors \(F:\cat\to\catt\) must satisfy
                  \(F(f)F(g)=F(fg)=\Id\). This means that \(Ff\) is split epi
                  and therefore also epi.

                  Suppose \(f:X\to Y\) is absolute epi, then \(\Hom(Y,f)\) must
                  be an epimorphism in \(\catset\), so
                  \(\Hom(Y,f):\Hom(Y,X)\to\Hom(Y,Y)\) is a surjective map. In
                  particular this means that there is a \(g\in\Hom(Y,X)\) such
                  that \(fg=\Id\) which is equivalent to being a split epi in
                  \(\cat\).
              \end{proof}

        \item Show that a morphism is absolute mono iff it is split mono.

              \begin{proof}
                  This is a proof by duality. It should be obvious that a
                  morphism is absolute mono iff it is absolute epi in
                  \(\cat^{op}\). By part b, this absolute epi in \(\cat^{op}\)
                  is split in \(\cat^{op}\). Therefore it is split mono in
                  \(\cat\).
              \end{proof}
    \end{enumerate}
\end{question}
\end{document}