\documentclass{article}

\usepackage[utf8]{inputenc}
\usepackage{enumerate}
\usepackage{amsthm, amssymb, mathtools, amsmath, bbm, mathrsfs, stmaryrd}
\usepackage[margin=1in]{geometry}
\usepackage[parfill]{parskip}
\usepackage[hidelinks]{hyperref}
\usepackage{quiver}
\usepackage{float}
\usepackage{cleveref}

\newcommand{\N}{\mathbb{N}}
\newcommand{\Z}{\mathbb{Z}}
\newcommand{\NZ}{\mathbb{N}_{0}}
\newcommand{\Q}{\mathbb{Q}}
\newcommand{\R}{\mathbb{R}}
\newcommand{\C}{\mathbb{C}}

\newcommand{\F}{\mathbb{F}}
\newcommand{\incl}{\imath}

\newcommand{\tuple}[2]{\left\langle#1\colon #2\right\rangle}

\DeclareMathOperator{\order}{orde}
\DeclareMathOperator{\Id}{Id}
\DeclareMathOperator{\im}{im}
\DeclareMathOperator{\ggd}{ggd}
\DeclareMathOperator{\kgv}{kgv}
\DeclareMathOperator{\degree}{gr}
\DeclareMathOperator{\coker}{coker}

\DeclareMathOperator{\gl}{GL}

\DeclareMathOperator{\Aut}{Aut}
\DeclareMathOperator{\Hom}{Hom}
\DeclareMathOperator{\End}{End}
\newcommand{\isom}{\overset{\sim}{\longrightarrow}}

\newcommand{\Grp}{{\bf Grp}}
\newcommand{\Ab}{{\bf Ab}}
\newcommand{\cring}{{\bf CRing}}
\DeclareMathOperator{\modules}{{\bf Mod}}
\newcommand{\catset}{{\bf Set}}
\newcommand{\cat}{\mathcal{C}}
\newcommand{\catt}{\mathcal{D}}
\newcommand{\pcat}{\mathcal{P}}
\newcommand{\chains}{{\bf Ch}}
\newcommand{\homot}{{\bf Ho}}
\DeclareMathOperator{\objects}{Ob}
\newcommand{\gen}[1]{\left<#1\right>}
\DeclareMathOperator{\cone}{Cone}
\newcommand{\set}[1]{\left\{#1\right\}}
\newcommand{\setwith}[2]{\left\{#1:#2\right\}}
\DeclareMathOperator{\Ext}{Ext}
\DeclareMathOperator{\nil}{Nil}
\DeclareMathOperator{\idem}{Idem}
\DeclareMathOperator{\rad}{Rad}
\newcommand{\alg}[1]{T\textnormal{-Alg}}
\DeclareMathOperator{\ev}{ev}

\newenvironment{question}[1][]{\begin{paragraph}{Question #1}}{\end{paragraph}}
\newenvironment{solution}{\begin{proof}[Solution]\renewcommand\qedsymbol{}}{\end{proof}}

\newtheorem{theorem}{Theorem}[section]
\newtheorem{lemma}[theorem]{Lemma}
\newtheorem{proposition}[theorem]{Proposition}


\theoremstyle{definition}
\newtheorem{definition}[theorem]{Definition}
\newtheorem{remark}[theorem]{Remark}
\newtheorem{example}[theorem]{Example}
\newtheorem{corollary}[theorem]{Corollary}

\title{Homework Category Theory}
\author{Jonas van der Schaaf}
\date{}

\begin{document}
\maketitle
\begin{question}[1]
    Suppose \(X,Y\) are \(G\)-sets. Show that you can equip the set of functions
    \(\Hom_{\catset}(X,Y)\) with a \(G\)-action in such a way that it becomes
    the exponential \(Y^{X}\) in the category of \(G\)-sets.

    \begin{proof}
        The category of right \(G\)-actions has binary products because for
        \(f:Z\to X,g:Z\to Y\) morphisms compatible with the respective
        \(G\)-actions taking \(X\times Y\) as a set with \(G\)-action
        \((x,y)a=(xa,ya)\) and projection morphisms \(\pi_{X}:X\times Y\to
        X:(x,y)\mapsto x\) and \(\pi_{Y}:X\times Y\to Y:(x,y)\to y\) satisfies
        the universal property. We clearly have the map \(Z\to X\times
        Y:z\mapsto (f(z),g(z))\) which is a \(G\)-compatible function and it is
        unique because any map with \(\pi_{X}(f'(z),g'(z))=f(z)\) must have
        \(f'(z)=f(z)\) for all \(z\in Z\), the same argument holds for \(g'\).

        We define a \(G\)-action \(f\cdot a:=(x\mapsto
        f(xa^{-1})a)\in\Hom_{\catset}(X,Y)\). This is a \(G\)-action because
        \begin{align*}
            fe & =(x\mapsto f(xe^{-1})e) \\
               & =(x\mapsto f(x))
        \end{align*}
        for the unit \(e\) of \(G\). The mapping is also compatible because
        \begin{align*}
            (fa)b & =(x\mapsto fa(xb^{-1})b)       \\
                  & =(x\mapsto f(xb^{-1}a^{-1})ab) \\
                  & =(x\mapsto f(x(ab)^{-1})ab)    \\
                  & =f\cdot(ab)
        \end{align*}
        by the compatibility of \(G\) on \(X\) and \(Y\).

        To show that \(Y^{X}:=\Hom_{\catset}(X,Y)\) is the exponential in the
        category of \(G\)-sets we first define the evaluation map
        \(\ev:Y^{X}\times X\to Y\) as \(\ev(f,x)=f(x)\). This is
        \(G\)-compatible because
        \begin{align*}
            \ev((f,x)a) & =\ev(fa,xa)   \\
                        & =fa(xa)       \\
                        & =f(xaa^{-1})a \\
                        & =f(x)a        \\
                        & =\ev(f,x)a.
        \end{align*}
        To see this satisfies the universal property take any morphism
        \(h:A\times X\to Y\) and define \(H:A\to Y{^X}\) as \(H(a)=(x\mapsto
        h(a,x))\in\Hom_{\catset}(X,Y)\). Then clearly we have \(\ev\circ H\times
        \Id_{X}(a,x)=\ev((z\mapsto h(a,z)),x)=h(a,x)\). To show \(H\) is the
        unique map with this property consider an \(H'\) with the desired
        property. Then \(\ev(H'(a),x)=H'(a)(x)=h(a,x)=H(a)(x)\) for all \(a\in
        A,x\in X\). This clearly means \(H'(a)=H(a)\) so \(H'=H\). This proves
        the universal property of \(Y^{X}\) and so this category of \(G\)-sets
        is cartesian closed.
    \end{proof}
\end{question}
\end{document}