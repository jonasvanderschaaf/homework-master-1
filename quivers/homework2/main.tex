\documentclass{article}

\usepackage[utf8]{inputenc}
\usepackage{enumerate}
\usepackage{amsthm, amssymb, mathtools, amsmath, bbm, mathrsfs, stmaryrd}
\usepackage[margin=1in]{geometry}
\usepackage[parfill]{parskip}
\usepackage[hidelinks]{hyperref}
\usepackage{quiver}
\usepackage{float}
\usepackage{cleveref}

\newcommand{\N}{\mathbb{N}}
\newcommand{\Z}{\mathbb{Z}}
\newcommand{\NZ}{\mathbb{N}_{0}}
\newcommand{\Q}{\mathbb{Q}}
\newcommand{\R}{\mathbb{R}}
\newcommand{\C}{\mathbb{C}}

\newcommand{\F}{\mathbb{F}}
\newcommand{\incl}{\imath}

\newcommand{\tuple}[2]{\left\langle#1\colon #2\right\rangle}

\DeclareMathOperator{\order}{orde}
\DeclareMathOperator{\Id}{Id}
\DeclareMathOperator{\im}{im}
\DeclareMathOperator{\ggd}{ggd}
\DeclareMathOperator{\kgv}{kgv}
\DeclareMathOperator{\degree}{gr}
\DeclareMathOperator{\coker}{coker}

\DeclareMathOperator{\gl}{GL}

\DeclareMathOperator{\Aut}{Aut}
\DeclareMathOperator{\Hom}{Hom}
\DeclareMathOperator{\End}{End}
\newcommand{\isom}{\overset{\sim}{\longrightarrow}}

\newcommand{\Grp}{{\bf Grp}}
\newcommand{\Ab}{{\bf Ab}}
\newcommand{\cring}{{\bf CRing}}
\DeclareMathOperator{\modules}{{\bf Mod}}
\newcommand{\catset}{{\bf Set}}
\newcommand{\cat}{\mathcal{C}}
\newcommand{\chains}{{\bf Ch}}
\newcommand{\homot}{{\bf Ho}}
\DeclareMathOperator{\objects}{Ob}
\newcommand{\gen}[1]{\left<#1\right>}
\DeclareMathOperator{\cone}{Cone}
\newcommand{\set}[1]{\left\{#1\right\}}
\newcommand{\setwith}[2]{\left\{#1:#2\right\}}
\DeclareMathOperator{\Ext}{Ext}
\DeclareMathOperator{\nil}{Nil}
\DeclareMathOperator{\idem}{Idem}
\DeclareMathOperator{\rad}{Rad}

\newenvironment{question}[1][]{\begin{paragraph}{Question #1}}{\end{paragraph}}
\newenvironment{solution}{\begin{proof}[Solution]\renewcommand\qedsymbol{}}{\end{proof}}

\newtheorem{theorem}{Theorem}[section]
\newtheorem{lemma}[theorem]{Lemma}
\newtheorem{proposition}[theorem]{Proposition}


\theoremstyle{definition}
\newtheorem{definition}[theorem]{Definition}
\newtheorem{remark}[theorem]{Remark}
\newtheorem{example}[theorem]{Example}
\newtheorem{corollary}[theorem]{Corollary}

\title{Homework Quivers}
\author{Jonas van der Schaaf}
\date{October 3, 2022}

\begin{document}
\maketitle

\begin{question}[3.19]
    \begin{enumerate}[(i)]
        \item Show that the quiver representation
              \[
                  M=\begin{tikzcd}[ampersand replacement=\&]
                      {k^{2}} \&\& {k^{2}} \&\& k
                      \arrow["{\begin{pmatrix}1&1\\0&1\end{pmatrix}}", from=1-1, to=1-3]
                      \arrow["{\begin{pmatrix}1\\1\end{pmatrix}}"', from=1-5, to=1-3]
                  \end{tikzcd}
              \]
              is decomposable.

              \begin{proof}
                  We know that \(k^{2}=k\cdot(1,0)\oplus k\cdot(1,1)\).

                  Therefore the following is a decomposition of \(M\):
                  \[
                      \begin{tikzcd}[ampersand replacement=\&]
                          {k\oplus k} \&\& {k\cdot(1,0)\oplus k\cdot(1,1)} \&\& 0\oplus k
                          \arrow["{\begin{pmatrix}1\\0\end{pmatrix}\oplus\begin{pmatrix}1\\1\end{pmatrix}}", from=1-1, to=1-3]
                          \arrow["{\begin{pmatrix}0\\0\end{pmatrix}\oplus\begin{pmatrix}1\\1\end{pmatrix}}"', from=1-5, to=1-3]
                      \end{tikzcd}
                  \]
                  This means that \(M\) is decomposable.
              \end{proof}

        \item What is the number of indecomposable factors of \(M\)?

              \begin{solution}
                  The above decomposition consists of indecomposables. I will
                  show this by demonstrating that each part of the decomposition
                  is indecomposable.

                  The first indecomposable part is
                  \[
                      N=\begin{tikzcd}[ampersand replacement=\&]
                          k \&\& {k\cdot\begin{pmatrix}1\\0\end{pmatrix}} \&\& 0
                          \arrow["{\begin{pmatrix}1\\0\end{pmatrix}}", from=1-1, to=1-3]
                          \arrow["0"', from=1-5, to=1-3]
                      \end{tikzcd}
                  \]
                  This is indecomposable because each individual vector space of
                  the representation has dimension 1 or 0. Therefore the only
                  possible decomposition would be one where only a single point
                  has a non-zero vector space. This is impossible because the
                  map \(k\to k\cdot\begin{pmatrix}1\\0\end{pmatrix}\) is not the
                  zero map.

                  The representation
                  \[
                      N=\begin{tikzcd}[ampersand replacement=\&]
                          k \&\& {k\cdot\begin{pmatrix}1\\1\end{pmatrix}} \&\& k
                          \arrow["{\begin{pmatrix}1\\1\end{pmatrix}}", from=1-1, to=1-3]
                          \arrow["{\begin{pmatrix}1\\1\end{pmatrix}}"', from=1-5, to=1-3]
                      \end{tikzcd}
                  \]
                  is indecomposable because it is clearly isomorphic to the
                  injective representation \(I(2)\) (if we consider the points
                  to be ordered from left to right). This is indecomposable by
                  exercise 6.

                  Therefore \(M\) has two indecomposable factors.
              \end{solution}
    \end{enumerate}
\end{question}
\end{document}