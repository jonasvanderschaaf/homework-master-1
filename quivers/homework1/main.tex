\documentclass{article}

\usepackage[utf8]{inputenc}
\usepackage{enumerate}
\usepackage{amsthm, amssymb, mathtools, amsmath, bbm, mathrsfs, stmaryrd}
\usepackage[margin=1in]{geometry}
\usepackage[parfill]{parskip}
\usepackage[hidelinks]{hyperref}
\usepackage{quiver}
\usepackage{float}
\usepackage{cleveref}

\newcommand{\N}{\mathbb{N}}
\newcommand{\Z}{\mathbb{Z}}
\newcommand{\NZ}{\mathbb{N}_{0}}
\newcommand{\Q}{\mathbb{Q}}
\newcommand{\R}{\mathbb{R}}
\newcommand{\C}{\mathbb{C}}

\newcommand{\F}{\mathbb{F}}
\newcommand{\incl}{\imath}

\newcommand{\tuple}[2]{\left\langle#1\colon #2\right\rangle}

\DeclareMathOperator{\order}{orde}
\DeclareMathOperator{\Id}{Id}
\DeclareMathOperator{\im}{im}
\DeclareMathOperator{\ggd}{ggd}
\DeclareMathOperator{\kgv}{kgv}
\DeclareMathOperator{\degree}{gr}
\DeclareMathOperator{\coker}{coker}

\DeclareMathOperator{\gl}{GL}

\DeclareMathOperator{\Aut}{Aut}
\DeclareMathOperator{\Hom}{Hom}
\DeclareMathOperator{\End}{End}
\newcommand{\isom}{\overset{\sim}{\longrightarrow}}

\newcommand{\Grp}{{\bf Grp}}
\newcommand{\Ab}{{\bf Ab}}
\newcommand{\cring}{{\bf CRing}}
\DeclareMathOperator{\modules}{{\bf Mod}}
\newcommand{\catset}{{\bf Set}}
\newcommand{\cat}{\mathcal{C}}
\newcommand{\chains}{{\bf Ch}}
\newcommand{\homot}{{\bf Ho}}
\DeclareMathOperator{\objects}{Ob}
\newcommand{\gen}[1]{\left<#1\right>}
\DeclareMathOperator{\cone}{Cone}
\newcommand{\set}[1]{\left\{#1\right\}}
\newcommand{\setwith}[2]{\left\{#1:#2\right\}}
\DeclareMathOperator{\Ext}{Ext}
\DeclareMathOperator{\nil}{Nil}
\DeclareMathOperator{\idem}{Idem}
\DeclareMathOperator{\rad}{Rad}

\newenvironment{question}[1][]{\begin{paragraph}{Question #1}}{\end{paragraph}}
\newenvironment{solution}{\begin{proof}[Solution]\renewcommand\qedsymbol{}}{\end{proof}}

\newtheorem{theorem}{Theorem}[section]
\newtheorem{lemma}[theorem]{Lemma}
\newtheorem{proposition}[theorem]{Proposition}


\theoremstyle{definition}
\newtheorem{definition}[theorem]{Definition}
\newtheorem{remark}[theorem]{Remark}
\newtheorem{example}[theorem]{Example}
\newtheorem{corollary}[theorem]{Corollary}

\title{Homework Quivers}
\author{Jonas van der Schaaf}
\date{September 22, 2022}

\begin{document}
\maketitle

\begin{question}[2.9]
    Let \(\cat\) be a preabelian category and \(f\in\Hom(X,Y)\) a morphism.

    \begin{enumerate}[a)]
        \item Show that \(\incl:\ker f\to X\) is a monomorphism and
              \(\pi:Y\to\coker f\) is an epimorphism.

              \begin{proof}
                  Let \(g,h:Z\to\ker f\) be morphisms such that
                  \(g\circ\incl=h\circ\incl\). Then consider the commuting
                  diagram given in \Cref{fig:kernel-universal}.
                  \begin{figure}[H]
                      \[
                          \begin{tikzcd}
                              && X && Y \\
                              \\
                              Z && {\ker f} \\
                              &&& {}
                              \arrow["f", from=1-3, to=1-5]
                              \arrow["\imath", from=3-3, to=1-3]
                              \arrow["0"', from=3-3, to=1-5]
                              \arrow["{g\imath=h\imath}", from=3-1, to=1-3]
                              \arrow["h"', shift right=1, from=3-1, to=3-3]
                              \arrow["g", shift left=1, from=3-1, to=3-3]
                          \end{tikzcd}
                      \]
                      \caption{A commutative diagram.}
                      \label{fig:kernel-universal}
                  \end{figure}
                  By the universal property of the kernel, there is a unique map
                  \(Z\to\ker f\) such that the diagram commutes with
                  \(g\incl=h\incl\), so this map must be equal to \(g\) and
                  \(h\). Therefore \(g\) and \(h\) must be equal and \(\incl\)
                  is a monomorphism.

                  Now consider two morphisms \(g,h:\coker\pi\to Z\) such that
                  \(\pi\circ g=\pi\circ h\) and consider the commutative diagram
                  in \Cref{fig:cokernel-universal}.
                  \begin{figure}[H]
                      \[
                          \begin{tikzcd}
                              X && Y && \coker\pi \\
                              \\
                              &&&& Z
                              \arrow["f", from=1-1, to=1-3]
                              \arrow["\pi", from=1-3, to=1-5]
                              \arrow["g"', shift right=1, from=1-5, to=3-5]
                              \arrow["h", shift left=1, from=1-5, to=3-5]
                              \arrow["{g\pi=h\pi}"', from=1-3, to=3-5]
                          \end{tikzcd}
                      \]
                      \caption{Another commutative diagram.}
                      \label{fig:cokernel-universal}
                  \end{figure}
                  Then by the universal property there is a unique map
                  \(\coker\pi\to Z\) such that the diagram commutes. This map
                  must be equal to both \(g\) and \(h\), so \(g=h\) and
                  therefore \(\pi\) is an epimorphism.
              \end{proof}

        \item Prove that \(f\) is a monomorphism exactly if \(\ker f\cong0\).

              \begin{proof}
                  Suppose \(f\) is a monomorphism and \(\ell\in\End(\ker f)\).
                  Then \(f\circ\incl\circ\ell=0\) and also \(f\circ 0=0\). By
                  monomorphicity of \(f\)  this means that
                  \(\incl\circ\ell=0=\incl\circ0\) for all \(\ell\in\End(\ker
                  f)\). Because \(\incl\) is also a monomorphism this means that
                  \(\ell=0\) so the group \(\End(\ker f)\) is trivial so \(\ker
                  f=0\).

                  Now suppose the kernel is trivial and there are two maps
                  \(g,h:Z\to X\) such that \(fg=fh\). Because \(\cat\) is
                  preabelian we have \(f(g-h)=0\). This means that there is some
                  map \(\overline{g-h}:Z\to\ker f=0\) such that the following
                  diagram commutes:
                  \begin{figure}[H]
                      \[
                          \begin{tikzcd}
                              Z && X && Y \\
                              \\
                              && 0
                              \arrow["0"', from=3-3, to=1-5]
                              \arrow["0"', hook, from=3-3, to=1-3]
                              \arrow["f"', from=1-3, to=1-5]
                              \arrow["{g-h}"', from=1-1, to=1-3]
                              \arrow["{\overline{g-h}}"', from=1-1, to=3-3]
                          \end{tikzcd}
                      \]
                      \caption{Yet another commutative diagram.}
                      \label{fig:ker-zero-f-mono}
                  \end{figure}
                  The map \(\overline{g-h}\) must be the \(0\) map because \(0\)
                  is terminal. Therefore \(g-h=0\circ0=0\) so \(g=h\) and \(f\)
                  is monomorphic.
              \end{proof}
        \item Prove that \(f:X\to Y\) is an epimorphism exactly if \(\coker(f)\cong0\).

              \begin{proof}
                  Suppose \(f\) is epi and take \(g\in\End(\coker f)\). Then
                  \(f\circ\pi\circ g=0=f\circ 0\). Because \(f\) is epi we get
                  that \(\pi\circ g=0=\pi\circ 0\). Because \(\pi\) is also epi
                  we get \(g=0\). This means that \(\End(\coker f)=0\).

                  Suppose \(\coker f\) is the \(0\) object and there are maps
                  \(g,h:Y\to Z\) such that \(fg=fh\). Then \(f(g-h)=0\), so
                  there is a map \(\overline{g-h}:\coker f\to Z\) such that the
                  diagram below commutes
                  \begin{figure}[H]
                      \[
                          \begin{tikzcd}
                              X && Y && 0 \\
                              \\
                              &&&& \bullet
                              \arrow["f", from=1-1, to=1-3]
                              \arrow["{g-h}"', from=1-3, to=3-5]
                              \arrow["{\overline{g-h}}"', from=1-5, to=3-5]
                              \arrow["\pi", two heads, from=1-3, to=1-5]
                          \end{tikzcd}
                      \]
                      \caption{Yet another commutative diagram\('\).}
                      \label{fig:cokernel-0-f-epi}
                  \end{figure}
                  Then because \(\coker f\) is trivial \(g-h=0\circ0=0\) so
                  \(g=h\) and \(f\) is epi.
              \end{proof}
    \end{enumerate}
\end{question}
\end{document}