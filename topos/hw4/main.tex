\documentclass{article}

\usepackage[utf8]{inputenc}
\usepackage{enumerate}
\usepackage{svg}
\usepackage{amsthm, amssymb, mathtools, amsmath, bbm, mathrsfs, stmaryrd}
\usepackage[margin=1in]{geometry}
\usepackage[parfill]{parskip}
\usepackage[hidelinks]{hyperref}
\usepackage{quiver}
\usepackage{float}
\usepackage{cleveref}

\newcommand{\N}{\mathbb{N}}
\newcommand{\Z}{\mathbb{Z}}
\newcommand{\NZ}{\mathbb{N}_{0}}
\newcommand{\Q}{\mathbb{Q}}
\newcommand{\R}{\mathbb{R}}
\newcommand{\C}{\mathbb{C}}

\newcommand{\F}{\mathbb{F}}
\newcommand{\incl}{\imath}

\newcommand{\tuple}[2]{\left\langle#1\colon #2\right\rangle}

\DeclareMathOperator{\order}{orde}
\DeclareMathOperator{\Id}{Id}
\DeclareMathOperator{\im}{im}
\DeclareMathOperator{\ggd}{ggd}
\DeclareMathOperator{\kgv}{kgv}
\DeclareMathOperator{\degree}{gr}
\DeclareMathOperator{\coker}{coker}

\DeclareMathOperator{\gl}{GL}

\DeclareMathOperator{\Aut}{Aut}
\DeclareMathOperator{\Hom}{Hom}
\DeclareMathOperator{\End}{End}
\newcommand{\isom}{\overset{\sim}{\longrightarrow}}

\newcommand{\Grp}{{\bf Grp}}
\newcommand{\Ab}{{\bf Ab}}
\newcommand{\cring}{{\bf CRing}}
\newcommand{\catcat}{{\bf Cat}}
\newcommand{\modules}{{\bf Mod}}
\newcommand{\catset}{{\bf Set}}
\newcommand{\cat}{\mathcal{C}}
\newcommand{\catt}{\mathcal{D}}
\newcommand{\cattt}{\mathcal{E}}
\newcommand{\topos}{\mathcal{E}}
\newcommand{\pcat}{\mathcal{P}}
\newcommand{\chains}{{\bf Ch}}
\newcommand{\homot}{{\bf Ho}}
\DeclareMathOperator{\objects}{Ob}
\newcommand{\gen}[1]{\left<#1\right>}
\DeclareMathOperator{\cone}{Cone}
\newcommand{\set}[1]{\left\{#1\right\}}
\newcommand{\setwith}[2]{\left\{#1:#2\right\}}
\newcommand{\setcat}{{\bf Set}}
\DeclareMathOperator{\Ext}{Ext}
\DeclareMathOperator{\nil}{Nil}
\DeclareMathOperator{\idem}{Idem}
\DeclareMathOperator{\rad}{Rad}
\newcommand{\alg}[1]{T\textnormal{-Alg}}
\DeclareMathOperator{\ev}{ev}
\newcommand{\psh}[1]{\widehat{#1}}
\DeclareMathOperator{\famm}{Fam}
\DeclareMathOperator{\dfib}{DFib}
\newcommand{\restrict}{\upharpoonright}
\DeclareMathOperator{\subobj}{Sub}

\tikzset{
    rot/.style={anchor=south, rotate=90, inner sep=.5mm}
}


\newenvironment{solution}{\begin{proof}[Solution]\renewcommand\qedsymbol{}}{\end{proof}}
\renewcommand{\qedsymbol}{\raisebox{-0.5cm}{\includesvg[width=0.5cm]{../../qedboy.svg}}}

\newtheorem{theorem}{Theorem}[section]
\newtheorem{lemma}[theorem]{Lemma}
\newtheorem{proposition}[theorem]{Proposition}


\theoremstyle{definition}
\newtheorem{question}{Exercise}
\newtheorem{definition}[theorem]{Definition}
\newtheorem{remark}[theorem]{Remark}
\newtheorem{example}[theorem]{Example}
\newtheorem{corollary}[theorem]{Corollary}

\title{Homework Topos Theory}
\author{Jonas van der Schaaf}
\date{}

\begin{document}
\maketitle

\begin{question}
    Let \(\topos\) be a topos with Lawvere-Tierney topology \(j\). Let
    \(X\in\topos\).

    \begin{enumerate}[(a)]
        \item Show that the diagram

              \[
                  \begin{tikzcd}[ampersand replacement=\&]
                      \Omega\times X \&\&\Omega\times X\\
                      \& X
                      \arrow["j\times \Id", from=1-1, to=1-3]
                      \arrow["\pi_{1}", from=1-1, to=2-2]
                      \arrow["\pi_{1}", from=2-2, to=1-3]
                  \end{tikzcd}
              \]
              is a Lawvere-Tierney topology in \(\topos/X\).

              \begin{proof}
                  The map \(j_{X}:\Omega\times X=j\times\)
                  is idempotent:
                  \begin{align*}
                      (j\times\Id)\circ(j\times\Id) & =j^{2}\times\Id^{2} \\
                                                    & =j\times\Id.
                  \end{align*}
                  Precomposing with the classifying mono
                  \(\gen{t,\Id}:X\to\Omega\times X\) also gives back \(t\):
                  \begin{align*}
                      (j\times\Id)\circ\gen{t,\Id} & =\gen{g\circ j,\Id\circ\Id} \\
                                                   & =\gen{g,\Id}.
                  \end{align*}
                  For the final property the function \(\wedge\) is given by the
                  pullback
                  \[
                      \begin{tikzcd}[ampersand replacement=\&]
                          X \&\& X \\
                          \\
                          {\Omega^{2}\times_{X}X} \&\& {\Omega\times_{X}X}
                          \arrow["\Id", from=1-1, to=1-3]
                          \arrow["{\gen{\gen{t,t},\Id}}"', from=1-1, to=3-1]
                          \arrow["\wedge\times\Id"', from=3-1, to=3-3]
                          \arrow["{\gen{t,\Id}}", from=1-3, to=3-3]
                          \arrow["\lrcorner"{anchor=center, pos=0}, draw=none, from=1-1, to=3-3]
                      \end{tikzcd}
                  \]
                  because \(X^{*}\) is a logical functor it preserves products
                  and the subobject classifier. This means that
                  \(\gen{\gen{t,t},\Id}\) is the mono classified by the
                  \(\wedge\) map in \(\topos/X\). In particular this means that
                  \(\wedge\times\Id\) is this map in \(\topos/X\). Now we can
                  easily see that
                  \begin{align*}
                      (j\times\Id)\circ(\wedge\times\Id) & =(j\circ\wedge)\times\Id                           \\
                                                         & =(\wedge\circ(j\times j))\times\Id                 \\
                                                         & =(\wedge\times\Id)\circ((j\times j)\times\Id_{X}).
                  \end{align*}
                  The map \((j\times j)\times\Id_{X}\) is again the product map
                  \((j\times\Id)\times(j\times\Id)\) in \(\topos/X\) because
                  \(X^{*}\) is a logical functor.

                  This demonstrates that \(j\times\Id\) is a Lawvere-Tierney
                  topology.
              \end{proof}

        \item Let
              \[
                  \begin{tikzcd}[ampersand replacement=\&]
                      M\&\& A\\
                      \&X
                      \arrow["f", from=1-1, to=1-3]
                      \arrow["m", from=1-1, to=2-2]
                      \arrow["a", from=1-3, to=2-2]
                  \end{tikzcd}
              \]
              be a mono in \(\topos/X\). Show that this mono is closed for
              \(j_{X}\) if and only if the map \(\varphi\) factors through
              \(\Omega_{j}\), where \(\varphi:A\to\Omega\) is such that the
              square
              \[
                  \begin{tikzcd}[ampersand replacement=\&]
                      M\&\& A\\
                      \\
                      X\&\& \Omega\times X
                      \arrow["f", from=1-1, to=1-3]
                      \arrow["m", from=1-1, to=3-1]
                      \arrow["{\gen{t,\Id}}", from=3-1, to=3-3]
                      \arrow["\gen{\varphi,a}", from=1-3, to=3-3]
                      \arrow["\lrcorner"{anchor=center, pos=0}, draw=none, from=1-1, to=3-3]
                  \end{tikzcd}
              \]
              is a pullback.

              \begin{proof}
                  It is easy to see that map \(\gen{\varphi,a}\) classifies the
                  mono \(f\) in \(\topos/X\). The given square is after all
                  exactly the classifying pullback in \(\topos/X\).

                  If \(f\) is a closed mono this means that
                  \begin{align*}
                      J_{X}\circ\gen{\varphi,a} & =j\times\Id\circ\gen{\varphi,a} \\
                                                & =\gen{j\circ\varphi,a}.
                  \end{align*}
                  Therefore \(\varphi=j\circ\varphi\) and so \(\varphi\)
                  factorizes through the equalizer \(\Omega_{j}\) as shown in the
                  diagram below
                  \[
                      \begin{tikzcd}[ampersand replacement=\&]
                          A \\
                          \\
                          {\Omega_{j}} \&\& \Omega \&\& \Omega
                          \arrow["e", from=3-1, to=3-3]
                          \arrow["\Id"', shift right=1, from=3-3, to=3-5]
                          \arrow["j", shift left=1, from=3-3, to=3-5]
                          \arrow["\varphi", from=1-1, to=3-3]
                          \arrow["{\tilde{\varphi}}"', dashed, from=1-1, to=3-1]
                      \end{tikzcd}
                  \]
                  Conversely if \(\varphi\) factors through the equalizer
                  \[
                      \begin{tikzcd}[ampersand replacement=\&]
                          A \\
                          \\
                          {\Omega_{j}} \&\& \Omega \&\& \Omega
                          \arrow["e", from=3-1, to=3-3]
                          \arrow["\Id"', shift right=1, from=3-3, to=3-5]
                          \arrow["j", shift left=1, from=3-3, to=3-5]
                          \arrow["\varphi", from=1-1, to=3-3]
                          \arrow["{\tilde{\varphi}}"', from=1-1, to=3-1]
                      \end{tikzcd}
                  \]
                  then because \(j\circ e=e\) we see that
                  \begin{align*}
                      j\circ\varphi & =j\circ e\circ\tilde{\varphi} \\
                                    & =e\circ\tilde{\varphi}        \\
                                    & =\varphi.
                  \end{align*}
                  This means that
                  \(j\times\Id\circ\gen{\varphi,a}=\gen{\varphi,a}\) and so
                  \(j\) is a closed subobject.
              \end{proof}

        \item Let \(\alpha:X\to Y\) be an arrow in \(\topos\). Show that if
              \(g:B\to X\) is a sheaf for \(j_{X}\), then \(\Pi_{\alpha}(g)\) is
              a sheaf for \(j_{Y}\).

              \begin{proof}
                  Suppose we have a diagram
                  \[
                      \begin{tikzcd}[ampersand replacement=\&]
                          M \&\& {M'} \\
                          \\
                          {\Pi_{a}B} \&\& Y
                          \arrow[hook, from=1-1, to=1-3]
                          \arrow[from=1-1, to=3-1]
                          \arrow["{\Pi_{a}g}", from=3-1, to=3-3]
                          \arrow["h", from=1-3, to=3-3]
                      \end{tikzcd}
                  \]
                  where \(M\to M'\) is a dense mono then transposing gives a
                  diagram
                  \[
                      \begin{tikzcd}[ampersand replacement=\&]
                          {\alpha^{*}M} \&\& {\alpha^{*}M'} \\
                          \\
                          B \&\& X
                          \arrow["{\alpha^{*}m}", from=1-1, to=1-3]
                          \arrow[from=1-1, to=3-1]
                          \arrow["g"', from=3-1, to=3-3]
                          \arrow["{\alpha^{*}h}", from=1-3, to=3-3]
                      \end{tikzcd}
                  \]
                  We show that \(\alpha^{*}m\) is a dense mono for \(J_{X}\),
                  meaning that there is a unique filler \(\alpha^{*}M'\to B\).
                  Transposing again gives that a unique filler \(M'\to
                  \Pi_{a}B\).

                  We know that
                  \begin{align*}
                      c_{X}(\alpha^{*}M) & =\alpha^{*}c_{Y}(M) \\
                                         & =\alpha^{*}M'.
                  \end{align*}
                  Therefore \(\alpha^{*}m\) is a dense mono. This shows there is
                  a unique filler which transposes to a unique filler in the
                  original diagram.
              \end{proof}
    \end{enumerate}
\end{question}
\end{document}