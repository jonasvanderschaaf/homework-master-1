\documentclass{article}

\usepackage[utf8]{inputenc}
\usepackage{enumerate}
\usepackage{svg}
\usepackage{amsthm, amssymb, mathtools, amsmath, bbm, mathrsfs, stmaryrd}
\usepackage[margin=1in]{geometry}
\usepackage[parfill]{parskip}
\usepackage[hidelinks]{hyperref}
\usepackage{quiver}
\usepackage{float}
\usepackage{cleveref}

\newcommand{\N}{\mathbb{N}}
\newcommand{\Z}{\mathbb{Z}}
\newcommand{\NZ}{\mathbb{N}_{0}}
\newcommand{\Q}{\mathbb{Q}}
\newcommand{\R}{\mathbb{R}}
\newcommand{\C}{\mathbb{C}}

\newcommand{\F}{\mathbb{F}}
\newcommand{\incl}{\imath}

\newcommand{\tuple}[2]{\left\langle#1\colon #2\right\rangle}

\DeclareMathOperator{\order}{orde}
\DeclareMathOperator{\Id}{Id}
\DeclareMathOperator{\im}{im}
\DeclareMathOperator{\ggd}{ggd}
\DeclareMathOperator{\kgv}{kgv}
\DeclareMathOperator{\degree}{gr}
\DeclareMathOperator{\coker}{coker}

\DeclareMathOperator{\gl}{GL}

\DeclareMathOperator{\Aut}{Aut}
\DeclareMathOperator{\Hom}{Hom}
\DeclareMathOperator{\End}{End}
\newcommand{\isom}{\overset{\sim}{\longrightarrow}}

\newcommand{\Grp}{{\bf Grp}}
\newcommand{\Ab}{{\bf Ab}}
\newcommand{\cring}{{\bf CRing}}
\newcommand{\catcat}{{\bf Cat}}
\DeclareMathOperator{\modules}{{\bf Mod}}
\newcommand{\catset}{{\bf Set}}
\newcommand{\cat}{\mathcal{C}}
\newcommand{\catt}{\mathcal{D}}
\newcommand{\cattt}{\mathcal{E}}
\newcommand{\pcat}{\mathcal{P}}
\newcommand{\chains}{{\bf Ch}}
\newcommand{\homot}{{\bf Ho}}
\DeclareMathOperator{\objects}{Ob}
\newcommand{\gen}[1]{\left<#1\right>}
\DeclareMathOperator{\cone}{Cone}
\newcommand{\set}[1]{\left\{#1\right\}}
\newcommand{\setwith}[2]{\left\{#1:#2\right\}}
\newcommand{\setcat}{{\bf Set}}
\DeclareMathOperator{\Ext}{Ext}
\DeclareMathOperator{\nil}{Nil}
\DeclareMathOperator{\idem}{Idem}
\DeclareMathOperator{\rad}{Rad}
\newcommand{\alg}[1]{T\textnormal{-Alg}}
\DeclareMathOperator{\ev}{ev}
\newcommand{\psh}[1]{\widehat{#1}}
\DeclareMathOperator{\famm}{Fam}
\DeclareMathOperator{\dfib}{DFib}
\newcommand{\restrict}{\upharpoonright}
\DeclareMathOperator{\subobj}{Sub}

\newenvironment{solution}{\begin{proof}[Solution]\renewcommand\qedsymbol{}}{\end{proof}}
\renewcommand{\qedsymbol}{\raisebox{-0.5cm}{\includesvg[width=0.5cm]{../../qedboy.svg}}}

\newtheorem{theorem}{Theorem}[section]
\newtheorem{lemma}[theorem]{Lemma}
\newtheorem{proposition}[theorem]{Proposition}


\theoremstyle{definition}
\newtheorem{question}{Exercise}
\newtheorem{definition}[theorem]{Definition}
\newtheorem{remark}[theorem]{Remark}
\newtheorem{example}[theorem]{Example}
\newtheorem{corollary}[theorem]{Corollary}

\title{Homework Topos Theory}
\author{Jonas van der Schaaf}
\date{}

\begin{document}
\maketitle

\begin{question}
    Given a monomorphism \(g:X\to Y\) in \(\cattt\), consider the map
    \(g^{*}:\subobj(Y)\to\subobj(X)\) on subobjects given by pullback along
    \(g\).

    \begin{enumerate}[a)]
        \item Show that \(g^{*}\) has a left adjoint given as follows:
              if \(B\in\subobj(X)\) is represented by a mono \(n:B\to X\) then
              \(L_{g}(B)\in\subobj(Y)\) is represented by the mono
              \(gn:B\to Y\).

              \begin{proof}
                  We show that \(L_{g}\) is a left adjoint by showing there is a
                  universal morphism \(\eta_{A}:A\to g^{*}L_{g}A\) for each
                  \(A\in\subobj(Y)\). It is given by the universal property of
                  the pullback as indicated in the following commutative
                  diagram:
                  \[
                      \begin{tikzcd}[ampersand replacement=\&]
                          A \\
                          \& {Y\times_{X}A} \&\& A \\
                          \\
                          \&\&\& Y \\
                          \\
                          \& Y \&\& X
                          \arrow["g", from=4-4, to=6-4]
                          \arrow["m", from=2-4, to=4-4]
                          \arrow["{g^{*}(gm)}"', from=2-2, to=2-4]
                          \arrow["g"', from=6-2, to=6-4]
                          \arrow[from=2-2, to=6-2]
                          \arrow["\lrcorner"{anchor=center, pos=0.125}, draw=none, from=2-2, to=6-4]
                          \arrow["\Id", from=1-1, to=2-4]
                          \arrow["m"', from=1-1, to=6-2]
                          \arrow["{\eta_{A}}"', dashed, from=1-1, to=2-2]
                      \end{tikzcd}
                  \]

                  Now let \(B:n\to X\) be a monomorphism and \(L_{g}B\) the
                  domain of a morphism between subobjects \(f:A\to L_{g}B\).
                  Then we get the following diagram:
                  \[\begin{tikzcd}[ampersand replacement=\&]
                          A \\
                          \&\& {Y\times_{X}A} \&\& A \\
                          \\
                          \&\&\&\& Y \\
                          \\
                          \&\& Y \&\& X \\
                          \\
                          \&\& {Y\times_{X}B} \&\& B
                          \arrow["{(gm)^{*}g}"', from=2-3, to=2-5]
                          \arrow["m"', from=2-5, to=4-5]
                          \arrow["g"', from=4-5, to=6-5]
                          \arrow["{g^{*}(mg)}", from=2-3, to=6-3]
                          \arrow["g"', from=6-3, to=6-5]
                          \arrow["{\eta_{A}}"', dashed, from=1-1, to=2-3]
                          \arrow["{\Id_{A}}", curve={height=-12pt}, from=1-1, to=2-5]
                          \arrow["m", from=1-1, to=6-3]
                          \arrow["{g^{*}n}"', from=8-3, to=6-3]
                          \arrow["{n^{*}g}"', from=8-3, to=8-5]
                          \arrow["n"', from=8-5, to=6-5]
                          \arrow["f"', curve={height=18pt}, from=1-1, to=8-3]
                          \arrow["{g^{*}n\circ f}", curve={height=-18pt}, from=2-5, to=8-5]
                          \arrow["\lrcorner"{anchor=center, pos=0}, draw=none, from=2-3, to=6-5]
                          \arrow["\lrcorner"{anchor=center, pos=0, rotate=90}, draw=none, from=8-3, to=6-5]
                      \end{tikzcd}\]
                  We claim that \(n^{*}g\circ f\) is the desired map of
                  subobjects \(A\to B\) for the universal morphism. First of all
                  we see that
                  \begin{align*}
                      n\circ n^{*}g\circ f & =g\circ g^{*}n\circ f                 \\
                                           & =gm                                   \\
                                           & =g\circ g^{*}(mg)\circ\eta_{A}        \\
                                           & =g\circ m\circ (gm)^{*}g\circ\eta_{A} \\
                                           & =\Id_{A}gm                            \\
                                           & =gm
                  \end{align*}
                  and so this is a morphism of subobjects. The induced morphism
                  \(Y\times_{X}A\to Y\times_{X}B\) is then given by the
                  universal property of the pullback: we have a map
                  \(n^{*}g\circ f\circ (gm)^{*}g:Y\times_{X}A\to B\) and a
                  morphism \(Y\times_{X}A\to Y:g^{*}(mg)\). We know that
                  \(g\circ g^{*}(mg)=n\circ n^{*}g\circ f\circ (gm)^{*}g\) and
                  thus there is a unique induced morphism of subobjects between
                  pullbacks \(Y\times_{X}A\to Y\times_{X}B\) making the above
                  diagram commute, this is also by definition the map
                  \(g^{*}(g^{*}n\circ f)\). The map \(g^{*}n\circ f\) is also
                  the unique map with this property because \(n\) is a
                  monomorphism and we know that for any map \(h\) with the desired
                  property
                  \[
                      n\circ h=n\circ g^{*}n\circ g^{*}(h)
                  \]
                  and \(n\) is mono so \(h=n^{*}g\circ f\).

                  This proves that \(\eta_{A}\) is indeed such a unique morphism
                  and thus that \(g^{*}\) is a right adjoint of the functor
                  \(L_{g}\).
              \end{proof}

        \item Show that the map \(L_{g}\) can also be constructed
              as the composite
              \[
                  \subobj(X)\cong\cattt(1,\Omega^{X})\overset{\cattt(1,\exists g)}{\to}\cattt(1,\Omega^{Y})\cong\subobj(Y).
              \]

              \begin{proof}
                  In this proof, I will identify \(X\) and \(X\times 1\) through
                  the canonical isomorphism given by the projection map
                  \(\pi_{X}:X\times 1\to 1\).

                  Let \(n:B\to X\) be a subobject of \(X\) classified by
                  \(\varphi:X\to\Omega\) and let
                  \(\tilde{\varphi}:1\to\Omega^{X}\) denote the exponential
                  transpose.
                  \begin{equation}
                      \begin{tikzcd}[ampersand replacement=\&]
                          B \&\& 1 \\
                          \\
                          X \&\& \Omega
                          \arrow["m", hook, from=1-1, to=3-1]
                          \arrow[from=1-1, to=1-3]
                          \arrow[from=1-3, to=3-3]
                          \arrow["\varphi", from=3-1, to=3-3]
                          \arrow["\lrcorner"{anchor=center, pos=0}, draw=none, from=1-1, to=3-3]
                      \end{tikzcd}
                  \end{equation}
                  Composing \(\tilde{\varphi}\) with \(\exists g\) gives the
                  following map with exponential transpose:

                  \begin{equation}
                      \begin{tikzcd}[ampersand replacement=\&]
                          {\Omega^{Y}\times Y} \&\& \Omega \\
                          \\
                          {\Omega^{X}\times Y} \\
                          \\
                          {1\times Y}
                          \arrow["{\tilde{\varphi}\times\Id_{Y}}", from=5-1, to=3-1]
                          \arrow["{\exists g\times\Id_{Y}}", from=3-1, to=1-1]
                          \arrow["{\ev_{Y}}"', from=1-1, to=1-3]
                          \arrow["{\widetilde{\exists g\circ\tilde{\varphi}}=\widetilde{\exists g}\circ (\tilde{\varphi}\times\Id)}"', from=5-1, to=1-3]
                      \end{tikzcd}
                      \label{fig:exp-transpose}
                  \end{equation}

                  We then get the commutative diagram as seen below. The
                  bottom-left and right squares are pullbacks. If we show the
                  top-left square is a pullback then applying the pullback lemma
                  twice, we get that the whole square is a pullback.

                  \begin{equation}
                      \begin{tikzcd}[ampersand replacement=\&]
                          B \&\& {\in_{X}} \&\& 1 \\
                          \\
                          {1\times X} \&\& {\Omega^{X}\times X} \\
                          \\
                          {1\times Y} \&\& {\Omega^{X}\times Y} \&\& \Omega
                          \arrow[from=1-3, to=1-5]
                          \arrow[from=1-5, to=5-5]
                          \arrow["{\widetilde{\exists g}}"', from=5-3, to=5-5]
                          \arrow[from=1-3, to=3-3]
                          \arrow["{\Id\times g}"', from=3-3, to=5-3]
                          \arrow["\lrcorner"{anchor=center, pos=0}, draw=none, from=1-3, to=5-5]
                          \arrow["n"', from=1-1, to=3-1]
                          \arrow["{\Id\times g}"', from=3-1, to=5-1]
                          \arrow["{\tilde{\varphi}\times\Id}", from=5-1, to=5-3]
                          \arrow["\tilde{\varphi}\times\Id"{description}, from=3-1, to=3-3]
                          \arrow["\lrcorner"{anchor=center, pos=0}, draw=none, from=3-1, to=5-3]
                          \arrow[from=1-1, to=1-3]
                      \end{tikzcd}
                      \label{fig:pullback}
                  \end{equation}

                  Consider the square
                  \begin{equation}
                      \begin{tikzcd}[ampersand replacement=\&]
                          B \&\& {\in_{X}} \&\& 1 \\
                          \\
                          {1\times X} \&\& {\Omega^{X}\times X} \&\& \Omega
                          \arrow["n"', from=1-1, to=3-1]
                          \arrow["{\tilde{\varphi}\times\Id}"', from=3-1, to=3-3]
                          \arrow[from=1-1, to=1-3]
                          \arrow[from=1-3, to=3-3]
                          \arrow[from=1-3, to=1-5]
                          \arrow[from=1-5, to=3-5]
                          \arrow["{\exists g\circ(\Id\times g)}", from=3-3, to=3-5]
                          \arrow["\lrcorner"{anchor=center, pos=0}, draw=none, from=1-3, to=3-5]
                      \end{tikzcd}
                      \label{fig:pullback2}
                  \end{equation}
                  The right square is a pullback because \(\Id\times g\) is mono
                  and the right square of \Cref{fig:pullback} is a pullback. The
                  entire square is also a pullback because \(\exists
                  g\circ(\Id\circ g)=\ev_{X}\) and therefore
                  \(\ev_{X}\circ(\tilde{\varphi}\times)\Id=\varphi\) which
                  classifies \(n\) by assumption. The left square is then also a
                  pullback. From this we deduce that the large square in
                  \Cref{fig:pullback} is also a pullback diagram.

                  This means that the monomorphism \((\Id\times g)\circ n\) is
                  classified by \(\widetilde{\exists
                      g}\circ(\tilde{\varphi}\times\Id)\) and thus the functor
                  \(L_{g}(n:B\to X)=g\circ n:B\to Y\) is the same functor as the
                  composition as described in the question.
              \end{proof}

        \item Show that for any subobject \(m:A\hookrightarrow Y\) we have
              \(L_{g}(f^{*}A)=k^{*}L_{h}(A)\).

              \begin{proof}
                  We have a commutative diagram of pullbacks
                  \begin{equation}
                      \begin{tikzcd}[ampersand replacement=\&]
                          {X\times_{Y}A} \&\& A \&\& 1 \\
                          \\
                          X \&\& Y \&\& \Omega \\
                          \\
                          Z \&\& T
                          \arrow["f", hook, from=3-1, to=3-3]
                          \arrow["g"', hook, from=3-1, to=5-1]
                          \arrow["k"', hook, from=5-1, to=5-3]
                          \arrow["h"', hook, from=3-3, to=5-3]
                          \arrow["{f^{*}m}"', hook, from=1-1, to=3-1]
                          \arrow[from=1-1, hook, to=1-3]
                          \arrow["m", hook, from=1-3, to=3-3]
                          \arrow["\lrcorner"{anchor=center, pos=0}, draw=none, from=1-1, to=3-3]
                          \arrow["\lrcorner"{anchor=center, pos=0}, draw=none, from=3-1, to=5-3]
                          \arrow["\varphi", from=3-3, to=3-5]
                          \arrow[from=1-3, to=1-5]
                          \arrow["t"', hook, from=1-5, to=3-5]
                          \arrow["\lrcorner"{anchor=center, pos=0}, draw=none, from=1-3, to=3-5]
                          \arrow["{\vartheta}", from=5-3, to=3-5]
                      \end{tikzcd}
                      \label{eq:pullback-galore}
                  \end{equation}
                  where \(\varphi\) classifies the subobject \(A\) and
                  \(\vartheta\) classifies \(h\circ m\). Note that
                  \(\vartheta\circ h=\varphi\) because \(m\) is mono and
                  \begin{align*}
                      \vartheta\circ h\circ m & =t\circ !_{A}    \\
                                              & =\varphi\circ m.
                  \end{align*}
                  If we show that \(g\circ f^{*}m\) is also the subobject given
                  by \(k^{*}L_{h}(m:A\to Y)\), then we are done.

                  We know that the square
                  \begin{equation}
                      \begin{tikzcd}[ampersand replacement=\&]
                          {X\times_{Y}A} \&\& A \\
                          \\
                          Z \&\& T
                          \arrow[hook, from=1-1, to=1-3]
                          \arrow["{k\circ f^{*}m}"', hook, from=1-1, to=3-1]
                          \arrow["k"', hook, from=3-1, to=3-3]
                          \arrow["{h\circ m}", hook, from=1-3, to=3-3]
                          \arrow["\lrcorner"{anchor=center, pos=0}, draw=none, from=1-1, to=3-3]
                      \end{tikzcd}
                  \end{equation}
                  is a pullback by the pullback lemma. However we also have that
                  \begin{equation}
                      \begin{tikzcd}[ampersand replacement=\&]
                          {Z\times_{T}A} \&\& A \\
                          \\
                          Z \&\& T
                          \arrow[hook, from=1-1, to=1-3]
                          \arrow["{k^{*}(h\circ m)}"', hook, from=1-1, to=3-1]
                          \arrow["k"', hook, from=3-1, to=3-3]
                          \arrow["{h\circ m}", hook, from=1-3, to=3-3]
                          \arrow["\lrcorner"{anchor=center, pos=0}, draw=none, from=1-1, to=3-3]
                      \end{tikzcd}
                  \end{equation}
                  is a pullback. This means there is an isomorphism as indicated
                  in \Cref{eq:pullback-iso} and therefore the two subobjects are
                  isomorphic in \(\subobj(Z)\).
                  \begin{equation}
                      \begin{tikzcd}[ampersand replacement=\&]
                          {X\times_{Y}A} \\
                          \& {Z\times_{T}A} \&\& A \\
                          \\
                          \& Z \&\& T
                          \arrow[hook, from=2-2, to=2-4]
                          \arrow["{k^{*}(h\circ m)}", hook, from=2-2, to=4-2]
                          \arrow["k"', hook, from=4-2, to=4-4]
                          \arrow["{h\circ m}", hook, from=2-4, to=4-4]
                          \arrow[curve={height=-12pt}, from=1-1, to=2-4]
                          \arrow[curve={height=12pt}, from=1-1, to=4-2]
                          \arrow["\sim"{description}, dashed, from=1-1, to=2-2]
                      \end{tikzcd}
                      \label{eq:pullback-iso}
                  \end{equation}
              \end{proof}
    \end{enumerate}
\end{question}
\end{document}