\documentclass{article}

\usepackage[utf8]{inputenc}
\usepackage{enumerate}
\usepackage{svg}
\usepackage{amsthm, amssymb, mathtools, amsmath, bbm, mathrsfs, stmaryrd}
\usepackage[margin=1in]{geometry}
\usepackage[parfill]{parskip}
\usepackage[hidelinks]{hyperref}
\usepackage{quiver}
\usepackage{float}
\usepackage{cleveref}

\newcommand{\N}{\mathbb{N}}
\newcommand{\Z}{\mathbb{Z}}
\newcommand{\NZ}{\mathbb{N}_{0}}
\newcommand{\Q}{\mathbb{Q}}
\newcommand{\R}{\mathbb{R}}
\newcommand{\C}{\mathbb{C}}

\newcommand{\F}{\mathbb{F}}
\newcommand{\incl}{\imath}

\newcommand{\tuple}[2]{\left\langle#1\colon #2\right\rangle}

\DeclareMathOperator{\order}{orde}
\DeclareMathOperator{\Id}{Id}
\DeclareMathOperator{\im}{im}
\DeclareMathOperator{\ggd}{ggd}
\DeclareMathOperator{\kgv}{kgv}
\DeclareMathOperator{\degree}{gr}
\DeclareMathOperator{\coker}{coker}

\DeclareMathOperator{\gl}{GL}

\DeclareMathOperator{\Aut}{Aut}
\DeclareMathOperator{\Hom}{Hom}
\DeclareMathOperator{\End}{End}
\newcommand{\isom}{\overset{\sim}{\longrightarrow}}

\newcommand{\Grp}{{\bf Grp}}
\newcommand{\Ab}{{\bf Ab}}
\newcommand{\cring}{{\bf CRing}}
\newcommand{\catcat}{{\bf Cat}}
\newcommand{\modules}{{\bf Mod}}
\newcommand{\catset}{{\bf Set}}
\newcommand{\cat}{\mathcal{C}}
\newcommand{\catt}{\mathcal{D}}
\newcommand{\cattt}{\mathcal{E}}
\newcommand{\topos}{\mathcal{E}}
\newcommand{\pcat}{\mathcal{P}}
\newcommand{\chains}{{\bf Ch}}
\newcommand{\homot}{{\bf Ho}}
\DeclareMathOperator{\objects}{Ob}
\newcommand{\gen}[1]{\left<#1\right>}
\DeclareMathOperator{\cone}{Cone}
\newcommand{\set}[1]{\left\{#1\right\}}
\newcommand{\setwith}[2]{\left\{#1:#2\right\}}
\newcommand{\setcat}{{\bf Set}}
\DeclareMathOperator{\Ext}{Ext}
\DeclareMathOperator{\nil}{Nil}
\DeclareMathOperator{\idem}{Idem}
\DeclareMathOperator{\rad}{Rad}
\newcommand{\alg}[1]{T\textnormal{-Alg}}
\DeclareMathOperator{\ev}{ev}
\newcommand{\psh}[1]{\widehat{#1}}
\DeclareMathOperator{\famm}{Fam}
\DeclareMathOperator{\dfib}{DFib}
\newcommand{\restrict}{\upharpoonright}
\DeclareMathOperator{\subobj}{Sub}
\DeclareMathOperator{\elements}{Elts}

\tikzset{
    rot/.style={anchor=south, rotate=90, inner sep=.5mm}
}


\newenvironment{solution}{\begin{proof}[Solution]\renewcommand\qedsymbol{}}{\end{proof}}
\renewcommand{\qedsymbol}{\raisebox{-0.5cm}{\includesvg[width=0.5cm]{../../qedboy.svg}}}

\newtheorem{theorem}{Theorem}[section]
\newtheorem{lemma}[theorem]{Lemma}
\newtheorem{proposition}[theorem]{Proposition}


\theoremstyle{definition}
\newtheorem{question}{Exercise}
\newtheorem{definition}[theorem]{Definition}
\newtheorem{remark}[theorem]{Remark}
\newtheorem{example}[theorem]{Example}
\newtheorem{corollary}[theorem]{Corollary}

\title{Homework Topos Theory}
\author{Jonas van der Schaaf}
\date{}

\begin{document}
\maketitle

\begin{question}
    Call a functor \(F:\mathcal{A}\to\mathcal{B}\) flat if for each
    \(B\in\mathcal{B}\) the functor
    \(\mathcal{B}(B,F(-)):\mathcal{A}\to\setcat\) is flat.

    \begin{enumerate}[(a)]
        \item Prove if \(F:\mathcal{A}\to\mathcal{B}\) is flat and
              \(G:\mathcal{B}\to\mathcal{C}\) is flat then
              \(GF:\mathcal{A}\to\mathcal{C}\) is flat.

              \begin{proof}
                  We demonstrate that \(\elements\mathcal{C}(C,GF-)\) is a
                  filtering category for each \(C\in\mathcal{C}\). We know that
                  \(\elements\mathcal{B}(B,F-)\) and
                  \(\elements\mathcal{C}(C,G-)\) are filtering categories for
                  arbitrary \(B\in\mathcal{B}\) and \(C\in\mathcal{C}\). We fix
                  an arbitrary \(C\in\mathcal{C}\).

                  We first show the category is non-empty. There is a
                  \(B\in\mathcal{B}\) such that there is an \(g:C\to GB\)
                  because \(\elements\mathcal{C}(C,G-)\) is filtering. Similarly
                  there is an \(f:B\to FA\) for some \(A\in\mathcal{A}\). Then
                  \(Gf\circ g:C\to GFA\in\Hom(C,GFA)\) so \((Gf\circ
                  g,A)\in\elements\mathcal{C}(C,GF-)\) and it is therefore
                  non-empty for each \(C\in\mathcal{C}\).

                  Take \(x=(f:C\to GFA,A),x'=(f':C\to
                  GFA',A')\in\elements\mathcal{C}(C,GF-)\). Then because
                  \(\elements\mathcal{C}(C,G-)\) is a filtering category there
                  is a \(y=(g:C\to GB,B)\) such that there are morphisms
                  \(h:y\to x\) and \(h':y\to x'\), these are morphisms \(h:B\to
                  FA\) and \(h:B\to FA'\) such that \(Gh\circ g=f\) and
                  \(Gh'\circ g=f'\). These are elements
                  \(z=(h,A),z'=(h',A')\in\elements\mathcal{B}(B,F-)\). Because
                  this is also a filtered category there is a \(v=(k:B\to
                  FA'',A'')\) and \(l:v\to z,l':v\to z'\). Then \(u=(Gk\circ
                  g,A'')\) has \(l:A''\to A\) and \(l':A''\to A'\) such that
                  \begin{align*}
                      GFl\circ Gk\circ g  & =G(Fl\circ k)\circ g  \\
                                          & =Gh\circ g            \\
                                          & =f,                   \\
                      GFl'\circ Gk\circ g & =G(Fl'\circ k)\circ g \\
                                          & =Gh'\circ g           \\
                                          & =f'.
                  \end{align*}
                  Therefore \(l:u\to x,l':u\to x'\) are two morphisms in
                  \(\elements\mathcal{C}(C,GF-)\).

                  Now we show the category has equalizers. Let \(l,l':a\to b\)
                  be morphisms from \(a=(f:C\to GFA,A)\) to \(b=(g:C\to GFA')\)
                  i.e. we have a commutative diagram
                  \[
                      \begin{tikzcd}[ampersand replacement=\&]
                          C\&\&GFA\\
                          \\
                          \&\&GFA'
                          \arrow["GFl"', shift right=1, from=1-3, to=3-3]
                          \arrow["GFl'", shift left=1, from=1-3, to=3-3]
                          \arrow["f", from=1-1, to=1-3]
                          \arrow["g"', from=1-1, to=3-3]
                      \end{tikzcd}
                  \]
                  by assumption this is equalized by some \((h:C\to
                  GB,B)\in\elements\mathcal{C}(C,G-)\). There is then also an
                  \(E\in\mathcal{A}\) and \(k:B\to FE\) such that
                  \((k,A)\in\elements\mathcal{B}(B,F-)\) equalizes \(l,l'\) with
                  an arrow \(n:E\to A\):
                  \[
                      \begin{tikzcd}[ampersand replacement=\&]
                          \&\&FE\\
                          \\
                          B\&\& FA\\
                          \\
                          \&\&FA'
                          \arrow["Fl"', shift right=1, from=3-3, to=5-3]
                          \arrow["Fl'", shift left=1, from=3-3, to=5-3]
                          \arrow["k", from=3-1, to=1-3]
                          \arrow["n", from=1-3, to=3-3]
                          \arrow[from=3-1, to=3-3]
                          \arrow[from=3-1, to=5-3]
                      \end{tikzcd}
                  \]
                  Therefore we have an object \((Gk\circ h:C\to GFE, E)\) such
                  that \(h\) equalizes the parallel arrows \(a,b\).
              \end{proof}

        \item For every category \(\mathcal{A}\) and object \(A\in\mathcal{A}\),
              show that the functor \(\Hom(A,-)=y_{A}\) is flat.

              \begin{proof}
                  Then Kan extension of \(\Hom(A,-)\) is the evaluation functor
                  \(\ev_{A}:\hat{\mathcal{A}}\to\setcat:\ev_{A}(X)=X(A)\). This
                  is because \(y_{C}(A)=\Hom(A,C)\) so the required
                  commutativity demand is satisfied. It preserves (co)limits
                  because (co)limits are computed pointwise.
              \end{proof}

        \item Suppose \(G:\mathcal{A}\to\mathcal{B}\) has a left adjoint. Then
              \(G\) is flat.

              \begin{proof}
                  If \(G\vdash F\), then \(\Hom(B,G-)\cong\Hom(FB,-)\) which is
                  flat by part b. We know Kan extensions are unique up to
                  functor isomorphism so isomorphic functors have isomorphic
                  Kan-extensions and therefore flatness is preserved under
                  isomorphism.
              \end{proof}
    \end{enumerate}
\end{question}
\end{document}