\documentclass{article}

\usepackage[utf8]{inputenc}
\usepackage{enumerate}
\usepackage{svg}
\usepackage{amsthm, amssymb, mathtools, amsmath, bbm, mathrsfs, stmaryrd}
\usepackage[margin=1in]{geometry}
\usepackage[parfill]{parskip}
\usepackage[hidelinks]{hyperref}
\usepackage{quiver}
\usepackage{float}
\usepackage{cleveref}

\newcommand{\N}{\mathbb{N}}
\newcommand{\Z}{\mathbb{Z}}
\newcommand{\NZ}{\mathbb{N}_{0}}
\newcommand{\Q}{\mathbb{Q}}
\newcommand{\R}{\mathbb{R}}
\newcommand{\C}{\mathbb{C}}

\newcommand{\F}{\mathbb{F}}
\newcommand{\incl}{\imath}

\newcommand{\tuple}[2]{\left\langle#1\colon #2\right\rangle}

\DeclareMathOperator{\order}{orde}
\DeclareMathOperator{\Id}{Id}
\DeclareMathOperator{\im}{im}
\DeclareMathOperator{\ggd}{ggd}
\DeclareMathOperator{\kgv}{kgv}
\DeclareMathOperator{\degree}{gr}
\DeclareMathOperator{\coker}{coker}

\DeclareMathOperator{\gl}{GL}

\DeclareMathOperator{\Aut}{Aut}
\DeclareMathOperator{\Hom}{Hom}
\DeclareMathOperator{\End}{End}
\newcommand{\isom}{\overset{\sim}{\longrightarrow}}

\newcommand{\Grp}{{\bf Grp}}
\newcommand{\Ab}{{\bf Ab}}
\newcommand{\cring}{{\bf CRing}}
\newcommand{\catcat}{{\bf Cat}}
\DeclareMathOperator{\modules}{{\bf Mod}}
\newcommand{\catset}{{\bf Set}}
\newcommand{\cat}{\mathcal{C}}
\newcommand{\catt}{\mathcal{D}}
\newcommand{\cattt}{\mathcal{E}}
\newcommand{\pcat}{\mathcal{P}}
\newcommand{\chains}{{\bf Ch}}
\newcommand{\homot}{{\bf Ho}}
\DeclareMathOperator{\objects}{Ob}
\newcommand{\gen}[1]{\left<#1\right>}
\DeclareMathOperator{\cone}{Cone}
\newcommand{\set}[1]{\left\{#1\right\}}
\newcommand{\setwith}[2]{\left\{#1:#2\right\}}
\newcommand{\setcat}{{\bf Set}}
\DeclareMathOperator{\Ext}{Ext}
\DeclareMathOperator{\nil}{Nil}
\DeclareMathOperator{\idem}{Idem}
\DeclareMathOperator{\rad}{Rad}
\newcommand{\alg}[1]{T\textnormal{-Alg}}
\DeclareMathOperator{\ev}{ev}
\newcommand{\psh}[1]{\widehat{#1}}
\DeclareMathOperator{\famm}{Fam}
\DeclareMathOperator{\dfib}{DFib}
\newcommand{\restrict}{\upharpoonright}

\newenvironment{solution}{\begin{proof}[Solution]\renewcommand\qedsymbol{}}{\end{proof}}
\renewcommand{\qedsymbol}{\raisebox{-0.5cm}{\includesvg[width=0.5cm]{../../qedboy.svg}}}

\newtheorem{theorem}{Theorem}[section]
\newtheorem{lemma}[theorem]{Lemma}
\newtheorem{proposition}[theorem]{Proposition}


\theoremstyle{definition}
\newtheorem{question}{Exercise}
\newtheorem{definition}[theorem]{Definition}
\newtheorem{remark}[theorem]{Remark}
\newtheorem{example}[theorem]{Example}
\newtheorem{corollary}[theorem]{Corollary}

\title{Homework Topos Theory}
\author{Jonas van der Schaaf}
\date{}

\begin{document}
\maketitle

\begin{question}
    We consider the poset \(\N\) of natural numbers and the category
    \(\psh{\N}\) of presheaves on \(\N\).

    \begin{enumerate}[a)]
        \item Show that in \(\psh{\N}\), the terminal object is not projective.

              \begin{proof}
                  Consider the presheaf
                  \[
                      F=\coprod_{n\in\omega}y_{n}
                  \]
                  where \(y_{n}\) denotes the Yoneda embedding which has
                  \[
                      y_{n}(m)=\begin{cases}
                          \set{*}     & m\leq n,           \\
                          \varnothing & \textnormal{else}.
                      \end{cases}
                  \]
                  This maps surjectively onto the terminal object because for
                  every \(n\in\omega\) we have that \(y_{n}(n)\neq\varnothing\)
                  so the coproduct cannot be empty here either. Therefore we
                  have the commutative\footnote{Technically speaking
                      commutativity of this diagram is vacuously true.} diagram
                  \[
                      \begin{tikzcd}[ampersand replacement=\&]
                          \& 1 \\
                          {F} \& 1
                          \arrow["\Id_{1}", from=1-2, to=2-2]
                          \arrow[two heads, from=2-1, to=2-2]
                      \end{tikzcd}
                  \]
                  There is however no morphism \(1\to F\) at all. This is
                  because whatever the image of the single element of \(1(0)\)
                  is, must lift to \(F(n)\) for each \(n\in\omega\). Suppose
                  there is a morphism \(f:1\to F\) then for \(*_{i}\in1(i)\)
                  \(f(*_{i})\in y_{n}\) for some fixed \(n\in\omega\). This
                  means that \(f(*_{n+1})\in y_{n}(n+1)=\varnothing\): a contradiction
                  because the empty set is empty.


                  Therefore the terminal object \(1\) cannot be projective.
              \end{proof}

        \item Show that if an object is projective in \(\psh(\N)\) then every
              restriction \(F(n+1)\to F(n)\) is injective.

              \begin{proof}
                  Let \(F\) be a projective presheaf. Consider the following
                  presheaf:
                  \[
                      G=\coprod_{n\in\omega}\coprod_{x\in Fn}X_{n,x}
                  \]
                  with \(X_{n,x}=y_{n}\) and an epimorphism \(G\to F\) induced
                  by \(X_{n,x}\to F\) corresponding to \(x\in F(n)\) by the
                  Yoneda lemma. This morphism is componentwise surjective
                  because for each \(x\in Fn\) the single element \(X_{n,x}(n)\)
                  maps onto it. Therefore this is an epimorphism. This means we
                  have a commutative diagram as such:
                  \[
                      \begin{tikzcd}[ampersand replacement=\&]
                          \&\& F \\
                          \\
                          {\coprod_{n\in\omega}\coprod_{x\in Fn}} \&\& F
                          \arrow["\eta", dashed, from=1-3, to=3-1]
                          \arrow["\pi", two heads, from=3-1, to=3-3]
                          \arrow["\Id"', hook, two heads, from=1-3, to=3-3]
                      \end{tikzcd}
                  \]
                  The map \(\eta\) must be a monomorphism because
                  \(\Id=\pi\circ\eta\) is as well. This means that each
                  \(\eta_{n}:Fn\to Gn\) is a monomorphism as well. Let
                  \(f_{n+1}:n\to n+1\) be the unique map in \(\N\), then we have
                  the following commutative diagram:
                  \[
                      \begin{tikzcd}[ampersand replacement=\&]
                          {F(n+1)} \&\& {\coprod_{m\in\omega}\coprod_{x\in Fm}y_{m}(n+1)} \\
                          \\
                          {F_{n}} \&\& {\coprod_{m\in\omega}\coprod_{x\in Fm}y_{m}(n)}
                          \arrow["{\eta_{n+1}}"', hook, from=1-1, to=1-3]
                          \arrow["{Gf_{n+1}}"', hook, from=1-3, to=3-3]
                          \arrow["{Ff_{n+1}}", from=1-1, to=3-1]
                          \arrow["{\eta_{n}}", hook, from=3-1, to=3-3]
                      \end{tikzcd}
                  \]
                  The counterclockwise composition can only be injective if the
                  map \(Ff_{n+1}\) must then also be injective and so the
                  restrictions of \(F\) must be injective.
              \end{proof}

        \item Show: An object in \(\psh{\N}\) is projective if and only if it is
              the coproduct of representables.

              \begin{proof}
                  It is clear that representables are indecomposable
                  projectives: they are trivially retracts of representables.
                  The coproduct of projectives is clearly projective as well,
                  so coproducts of representables are projectives.

                  Now suppose \(F\) is a projective sheaf with injective
                  restrictions \(Ff_{n+1}:F(n+1)\to Fn\). Then for each
                  \(n\in\omega\) define \(X_{n}=Fn\setminus Ff_{n+1}[F(n+1)]\).
                  Then there is a unique morphism \(y_{n}\to F\) given by
                  \(y_{n}(*)=x\in F(n)\). If we define the presheaf
                  \[
                      G=\coprod_{n\in\omega}\coprod_{x\in X_{n}}y_{n}
                  \]
                  and the map \(G\to F\) by
                  \[
                      \eta=\coprod_{n\in\omega}\coprod_{x\in X_{n}}x
                  \]
                  where \(x\in X_{n}\) denotes the morphism \(y_{n}\to F\) given
                  by the Yoneda lemma then \(\eta_{n}\) maps \(Gn\) bijectively
                  onto the elements of \(Fn\) which do not have arbitrarily high
                  amalgamations. By the reasoning of part a, no element of a
                  projective presheaf can have arbitrarily high amalgamations.
                  Therefore this morphism must be bijective on components and
                  therefore an isomorphism .
              \end{proof}
    \end{enumerate}
\end{question}

\end{document}