\documentclass{article}

\usepackage[utf8]{inputenc}
\usepackage{enumerate}
\usepackage{svg}
\usepackage{amsthm, amssymb, mathtools, amsmath, bbm, mathrsfs, stmaryrd}
\usepackage[margin=1in]{geometry}
\usepackage[parfill]{parskip}
\usepackage[hidelinks]{hyperref}
\usepackage{quiver}
\usepackage{float}
\usepackage{cleveref}

\newcommand{\N}{\mathbb{N}}
\newcommand{\Z}{\mathbb{Z}}
\newcommand{\NZ}{\mathbb{N}_{0}}
\newcommand{\Q}{\mathbb{Q}}
\newcommand{\R}{\mathbb{R}}
\newcommand{\C}{\mathbb{C}}

\newcommand{\F}{\mathbb{F}}
\newcommand{\incl}{\imath}

\newcommand{\tuple}[2]{\left\langle#1\colon #2\right\rangle}

\DeclareMathOperator{\order}{orde}
\DeclareMathOperator{\Id}{Id}
\DeclareMathOperator{\im}{im}
\DeclareMathOperator{\ggd}{ggd}
\DeclareMathOperator{\kgv}{kgv}
\DeclareMathOperator{\degree}{gr}
\DeclareMathOperator{\coker}{coker}

\DeclareMathOperator{\gl}{GL}

\DeclareMathOperator{\Aut}{Aut}
\DeclareMathOperator{\Hom}{Hom}
\DeclareMathOperator{\End}{End}
\newcommand{\isom}{\overset{\sim}{\longrightarrow}}

\newcommand{\Grp}{{\bf Grp}}
\newcommand{\Ab}{{\bf Ab}}
\newcommand{\cring}{{\bf CRing}}
\newcommand{\catcat}{{\bf Cat}}
\DeclareMathOperator{\modules}{{\bf Mod}}
\newcommand{\catset}{{\bf Set}}
\newcommand{\cat}{\mathcal{C}}
\newcommand{\catt}{\mathcal{D}}
\newcommand{\cattt}{\mathcal{E}}
\newcommand{\topos}{\mathcal{E}}
\newcommand{\pcat}{\mathcal{P}}
\newcommand{\chains}{{\bf Ch}}
\newcommand{\homot}{{\bf Ho}}
\DeclareMathOperator{\objects}{Ob}
\newcommand{\gen}[1]{\left<#1\right>}
\DeclareMathOperator{\cone}{Cone}
\newcommand{\set}[1]{\left\{#1\right\}}
\newcommand{\setwith}[2]{\left\{#1:#2\right\}}
\newcommand{\setcat}{{\bf Set}}
\DeclareMathOperator{\Ext}{Ext}
\DeclareMathOperator{\nil}{Nil}
\DeclareMathOperator{\idem}{Idem}
\DeclareMathOperator{\rad}{Rad}
\newcommand{\alg}[1]{T\textnormal{-Alg}}
\DeclareMathOperator{\ev}{ev}
\newcommand{\psh}[1]{\widehat{#1}}
\DeclareMathOperator{\famm}{Fam}
\DeclareMathOperator{\dfib}{DFib}
\newcommand{\restrict}{\upharpoonright}
\DeclareMathOperator{\subobj}{Sub}

\tikzset{
    rot/.style={anchor=south, rotate=90, inner sep=.5mm}
}


\newenvironment{solution}{\begin{proof}[Solution]\renewcommand\qedsymbol{}}{\end{proof}}
\renewcommand{\qedsymbol}{\raisebox{-0.5cm}{\includesvg[width=0.5cm]{../../qedboy.svg}}}

\newtheorem{theorem}{Theorem}[section]
\newtheorem{lemma}[theorem]{Lemma}
\newtheorem{proposition}[theorem]{Proposition}


\theoremstyle{definition}
\newtheorem{question}{Exercise}
\newtheorem{definition}[theorem]{Definition}
\newtheorem{remark}[theorem]{Remark}
\newtheorem{example}[theorem]{Example}
\newtheorem{corollary}[theorem]{Corollary}

\title{Homework Topos Theory}
\author{Jonas van der Schaaf}
\date{}

\begin{document}
\maketitle

\begin{question}
    As usual \(\topos\) is a topos.
    \begin{enumerate}[(a)]
        \item Deduce from proposition 1.28 that every arrow \(f:0\to X\) in
              \(\topos\) is monic.

              \begin{proof}
                  Suppose \(g,h:Y\to 0\) are two arrows into \(0\), then
                  \(g=h\). This is clearly a strictly stronger property than
                  \(f\) being mono. We know \(g,h\) must be isomorphisms and so
                  their inverses \(g^{-1},h^{-1}:0\to Y\) must be the unique map
                  given by the universal property. Therefore \(g\) and \(h\)
                  have the same inverse and are the same map.
              \end{proof}

        \item Use propositions 1.28 and 1.29 to show that if \(\topos^{op}\) is
              also a topos then \(\topos\) is trivial.

              \begin{proof}
                  Suppose \(\topos^{op}\) is a topos, then \(0\) is the terminal
                  object \(1'\) in \(\topos^{op}\). Every \(0\to X\) in
                  \(\topos\) is mono so every \(X\to 1'\) in \(\topos^{op}\) is
                  epi. In particular this means that the map \(0'\to
                  1'=!_{0}\in\Hom_{\topos}(0,1)\) is epi, but \(\topos\) is a
                  topos so it is both epi and mono. Toposes are balanced and
                  thus this map has an inverse in \(\topos^{op}\) and therefore
                  there is a map \(\imath:1\to 0\) in \(\topos\). This means
                  that all objects \(X\) of the topos have a map \(\imath\circ
                  !_{X}:X\to 0\) which is an isomorphism because \(\topos\) is a
                  topos. Therefore all objects of \(\topos\) are isomorphic to
                  \(0\) so \(\topos\) is equivalent its skeleton: the one arrow
                  category.
              \end{proof}

        \item Prove that a map \(\alpha:X\to Y\) is monic in \(B/\topos\) iff
              \(\alpha\) is monic in \(\topos\).

              \begin{proof}
                  If \(\alpha\) is monic in \(\topos\) and there are two
                  \(h_{1},h_{2}:Z\to Y\) morphisms in \(B/\topos\) with \(\alpha
                  h_{1}=\alpha h_{2}\), then \(h_{1}=h_{2}\) so \(\alpha\) is a
                  mono in \(B/\topos\).

                  For the converse we show that the forgetful functor
                  \(B/\topos\to\topos\) has left adjoint \(B_{*}:\topos\to
                  B/\topos\) given by \(X\mapsto (B\to X+B)\). The morphisms
                  \(f\) as indicated in the commutative diagram below correspond
                  to the universal property of the coproduct correspond to pairs
                  \((h:X\to Y,k:B\to Y)\) such that \(k=h+k\circ\imath=g\).
                  \[
                      \begin{tikzcd}[ampersand replacement=\&]
                          \&B\&\\
                          \\
                          X+B\&\&Y
                          \arrow["\imath"', from=1-2, to=3-1]
                          \arrow["g", from=1-2, to=3-3]
                          \arrow["f", from=3-1, to=3-3]
                      \end{tikzcd}
                  \]
                  This means that there is a natural bijection
                  \(B/\topos(B_{*}X,g:B\to Y)\cong\topos(X,Y)=\topos(X,U(g:B\to
                  Y))\) where \(U\) is the forgetful functor and thus \(B_{*}\)
                  is a right adjoint.

                  We now claim right adjoints preserve monos, and thus that
                  \(U\) does as well. This means that any mono \(\alpha\) in
                  \(B/\topos\) is mono in \(\topos\). Let
                  \(F:\cat\to\cattt\dashv G\) be two functors and \(f:X\to Y\)
                  in \(\cat\) a mono. Then for all \(Z\in\catt\) the map
                  \(\cat(1,f):\cat(FZ,X)\to\cat(FZ,Y)\) is injective. We have a
                  commutative diagram
                  \[
                      \begin{tikzcd}[ampersand replacement=\&]
                          \cat(FZ,X)\&\&\cat(FZ,Y)\\
                          \\
                          \cat(Z,GX)\&\&\cat(Z,GY)\\
                          \arrow["{\topos(1,f)}", from=1-1, to=1-3]
                          \arrow["{\topos(1,Gf)}", from=3-1, to=3-3]
                          \arrow["\sim" rot, from=1-1, to=3-1]
                          \arrow["\sim" rot, from=1-3, to=3-3]
                      \end{tikzcd}
                  \]
                  The top map is monic and the vertical maps are isomorphisms.
                  Therefore \(E(1,Gf)\) is monic which is the definition of
                  \(Gf\) monic.
              \end{proof}
        \item Use parts a and c to conclude that if the unique map \(B\to 1\) is
              not monic, then \(B/\topos\) is not a topos.

              \begin{proof}
                  The map \(\Id:B\to B\) is the initial object in \(B/\topos\):
                  for every \(f:B\to X\) there is a unique map \(g:B\to X\) such
                  that \(g\circ\Id =f\) which is \(g=f\).

                  If \(B\to 1\) is not monic, then the map \(f:\Id\to !_{B}\) in
                  \(B/\topos\) is not monic either. If \(B/\topos\) were a topos
                  then \(f\) would be monic because \(\Id\) is the initial
                  object in \(B/\topos\). Therefore \(B/\topos\) is not a topos.
              \end{proof}
    \end{enumerate}
\end{question}
\end{document}