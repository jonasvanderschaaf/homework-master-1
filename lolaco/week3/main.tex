\documentclass{article}

\usepackage[margin=1in]{geometry}

\title{Summary: What is music?}
\date{September 19, 2022}
\author{Jonas van der Schaaf}

\begin{document}
\maketitle

In her talk Makiko Sadakata explained the phenomenon of the Speech to Song
illusion first discovered by Dana Deutsch. It is the transformation of the
perception of a repeated phrase into a melody. In particular the phenomenon
occurred when a recording of a phrase was played. There are many factors which
contribute or detract from the illusion, many being closely related to qualities
of music. The most important factor for the transformation is the repetition but
qualities of the speech sample such as stable pitch, changes in pitch being
close to the twelve tone scale, regular rhythm and vowel quality also play an
important role.

The effect is also dependent on the specific person listening: the native
language has a big effect on the strength of the illusion. In many languages
pitch plays a role in the meaning of words and sentences and native speakers of
such languages experienced the phenomenon to a lesser extent.

Sadataka's research has also looked at constructing machine learning models
which recognizes and categorizes specific features of speech samples to predict
the occurrence of the illusion.
\end{document}