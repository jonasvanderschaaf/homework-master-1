\documentclass{article}

\usepackage[margin=1in]{geometry}

\title{Summary: The Real Number Continuum and Foundations of Mathematics}
\author{Jonas van der Schaaf}
\date{}

\begin{document}
\maketitle

In his seminar, Yuri explained the basis of independence proofs for set theory.
The standard axioms of set theory are called ZF and are usually extended with
the axiom of choice: the assumption that any (infinite) product is non-empty.
This is then called ZFC. It turns out that both ZF with the axiom of choice and
without are valid theories to do mathematics with: from ZF there is no proof of
choice or its converse. The standard method to demonstrate independence of some
statement from ZF is to construct two models of ZF: one where the statement
holds and one where it does not.

For the axiom of choice consistency can be proven using Gödels constructible
universe. Given a model of ZF one can define an inner model defined in such a
way that the axiom of choice holds. Analogously there is a method of adding new
sets to a model of set theory called forcing. Using this technique one can
construct models of set theory where the axiom of choice does not hold. From
Many independence proofs are given using this proof method.
\end{document}