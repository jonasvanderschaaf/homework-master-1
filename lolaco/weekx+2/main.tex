\documentclass{article}

\usepackage[margin=1in]{geometry}
\usepackage[parfill]{parskip}

\title{Ronald de Haan: Algorithms for Logical Reasoning}
\author{Jonas van der Schaaf}
\date{}

\begin{document}
\maketitle

Ronald's talk started with positive logic programs: a collection of rules and
facts (logical formulas) without negation. These are sets of formulas for which
it is easy to find models. In fact there exists a unique minimal model for this
set of formulas which can easily be computed. This minimal model is an
assignment of proposition letters such that the amount of positive assignments
is minimal.

These kinds of programs can be extended with stratified negations: an extension
of the language where negation is allowed within certain strict conditions.
These programs still have a unique minimal model.

For full propositional logic finding assignments satisfying formulas is a
problem called satisfiability or SAT for short. It is unknown if it is possible
to efficiently find solutions but all currently known algorithms take
exponential time to finish.

Two ``efficient'' algorithms for SAT were discussed in the lecture. One is DPLL,
an algorithm based on unit propagation which will cleverly simplify formulas to
find a solution. The other one is called CDCL which is an algorithm based on a
structure called a conflict graph to determine how to satisfy a formula.
\end{document}