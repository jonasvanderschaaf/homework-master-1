\documentclass{article}

\usepackage[margin=1in]{geometry}

\title{Summary: The quest for truth in the information age}
\author{Jonas van der Schaaf}
\date{}

\begin{document}
\maketitle

The main topic of Sonja's talk was describing if and how knowledge spreads
through a network of connected people. She did this using two opposite concepts:
\emph{wisdom of a crowd} and \emph{madness of a crowd}.

Wisdom of a crowd is the term for the phenomenon where a group of people share
their knowledge in such a manner that the knowledge of members grows by sharing
information. It's the idea that the potential knowledge of a group is typically
much higher than that of all individuals of a group.

The opposite phenomenon is also possible. There are cases where knowledge does
not spread efficiently through a group of people. This happens for example when
knowledge of different people does not interest others. More sociological and
psychological reasons for this phenomenon is that groups tend to focus on common
knowledge of the group instead of their private evidence. Another cause is when
people choose where they get their information from, which can lead to echo
chambers and polarisation.

Echo chambers and madness of a crowd are phenomenon which are very applicable in
modern society and preventing them is key to preserving a notion of truth in
society.

\hfill{\emph{199 words}}
\end{document}