\documentclass{article}

\usepackage[margin=1in]{geometry}

\title{Summary: Fixpoints and Circularity}
\date{September 19, 2022}
\author{Jonas van der Schaaf}

\begin{document}
\maketitle

The key concept in Yde's talk was the idea of circularity. Given some modal
language we can look for a formula which is equivalent to a larger formula in
which the original formula also occurs, similar to a recursive definition. These
formulas are rarely expressible in the original language, so the language can be
extended with a modal operator which will have the desired recursive properties.

Syntactically this circularity is expressed as a directed graph similar to
syntax trees with one major difference: these graphs are allowed to contain
cycles. It is these cycles which correspond to circular formulas. This new
definition of formulas also requires a new semantic interpretation. This is
given by game semantics similar to focus games introduced by Martin Lange and
Colin Stirling: two players play a game where they traverse the formula tree.
The player Eloise attempts to reach a contradiction and the player Abelard tries
to avoid one. A formula is satisfiable using this notion if there is a strategy
such that Eloise always wins.
\end{document}