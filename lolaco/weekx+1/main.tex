\documentclass{article}

\usepackage[margin=1in]{geometry}
\usepackage[parfill]{parskip}

\title{Ule Endriss: Automated Reasoning for Social Choice Theory}
\author{Jonas van der Schaaf}
\date{}

\begin{document}
\maketitle

The goal of social choice theory is to find good solutions for the problem of
preference aggregation. This is the problem where a group of people need to
decide on some issue and all have certain preferences. Social choice theory
examines voting rules, rules to decide the outcome of an election based on the
votes, and their properties. Some examples of voting rules are plurality (choose
the most popular option), plurality with runoff (choose the 2 most popular
options and hold another election) and borda (give points based on votes).

One of the main theorems in computational social choice is that of
Gibbard-Satterwaithe: any voting rule for more than three candidates in which
every candidate has a possibility of winning and a winner is chosen must have at
least one of two flaws: either voting differently to your preferences leads to a
more desirable result or the preference of one fixed voters preference is always
chosen.

The final part of his talk was dedicated to computational aspects. One example
of an application of computation theory is the proof of the aforementioned
Gibbard-Satterwaithe theorem. This is reducible to a satisfiability problem.

\hfill\emph{190 words}
\end{document}