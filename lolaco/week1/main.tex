\documentclass{article}
\usepackage[margin=1in]{geometry}

\title{Summary: Inquisitive Epistemic Logic}
\author{Jonas van der Schaaf}
\date{September 12, 2022}

\begin{document}
\maketitle

Floris Roelofsen introduced a logical language to model \emph{information
    exchange}. The goal of this language is to describe how different agents can
pose questions and learn new information. The language he described is an
extension of epistemic logic: a formal system which describes knowledge of
multiple different agents using a unique modal operator \(K_{\alpha}\) for
every agent describing they know something. The way this logic was extended
is with a concept called \emph{issues}. An issue is a question an agent
wants answered: something an agent wants to know.

In every possible world, every agent sees a set of other possible worlds, these
sets are called information states. Using these information states we can
describe issues: an issue is characterized by those information states where the
issue is resolved. These information states are also downward closed: if some
set of worlds answers a question then any subset of that world clearly also
does. An agent knows whether an issue is true if one of their information states
is contained within that issue.

\hfill 177 words

\end{document}
