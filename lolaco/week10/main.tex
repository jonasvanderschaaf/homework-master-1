\documentclass{article}

\usepackage[margin=1in]{geometry}

\title{Summary: Complexity Theory and Quantum Computing}
\author{Jonas van der Schaaf}
\date{}

\begin{document}
\maketitle

Ronald's talk discussed advantages of quantum computers over classical ones. He
started his talk by discussing computability. A function is computable if there
is a Turing machine which computes the output of that function in finitely many
steps. Being computable or non-computable is not the only property a function
can have with respect to Turing machines: often we care about how fast a Turing
machine can calculate the answer.

The measure of how fast a Turing machine can compute a function is called the
complexity. Those machines which take a polynomial amount of steps to compute an
output are in the complexity class P, while those which can be computed in
polynomial time on a non-deterministic Turing machine are in the complexity
class NP.

A different model of computation from Turing machines are quantum computers.
These are systems where bits can be in superposition: a bit can be both 0 and 1
until measured. There are a few problems which can be computed efficiently on a
quantum computer, this is done by performing computations on superpositions and
then collapsing them, allowing for doing ``more computations at once''.

\hfill\emph{194 words}
\end{document}