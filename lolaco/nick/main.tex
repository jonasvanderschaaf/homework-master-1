\documentclass{article}

\usepackage[margin=1in]{geometry}
\usepackage[parfill]{parskip}

\title{Report meeting researcher}
\author{Jonas van der Schaaf}
\date{\today}

\begin{document}
\maketitle

My meeting with Nick Bezhanishvili was about a paper he wrote about choice-free
Stone duality. Stone duality is a dual equivalence between the category of
boolean algebras and the category of Stone spaces: topological spaces which are
compact, Hausdorff, and totally disconnected. Unfortunately this duality
is only a duality when the ultrafilter lemma is true, which is a weaker version
of the axiom of choice. This means the proof is not constructive.

The paper by Nick and Wesley Holliday gives an alternative duality between
boolean algebras and the space of compact regular opens (CORO) of spaces called
UV-spaces. This is given by taking all proper filters of the boolean algebra
into a space as a functor to a topology, and constructing a boolean algebra from
the CORO sets of the topology by taking intersections
\(\textnormal{int}(X\setminus\cdot)\) and
\(\textnormal{int}(\textnormal{cl}(U\cup V))\). This also induces a bijection
between morphisms of the respective spaces such that this is a dual equivalence
of categories.

My talk with Nick mostly involved discussing the paper itself. He also shed a
lot of context onto possible applications. In particular he was very interested
in seeing whether there is some proof of a theorem that is easier to prove using
this duality. Because any choice-free proof using this duality is choice-free,
whereas no choice-free proof using Stone duality exists, this would mean that
the proof is true in models of set theory without choice.

He also highlighted applications in modal logic. In particular one can apply a
similar construction to the standard canonical frame. He mentioned a
construction of a different canonical frame, which leads to a different way of
proving completeness.

We also quickly talked about PhD defenses, because Nick had been part of a PhD
defense the day before. This was really interesting to hear about because I
haven't oriented myself too well in this area.
\end{document}