\documentclass{article}

\usepackage[margin=1in]{geometry}

\title{Summary: Nothing is logical}
\date{September 19, 2022}
\author{Jonas van der Schaaf}

\begin{document}
\maketitle

In her talk Maria explored particular difficulties in translation of speech to a
logical formula. In particular she expressed difficulties in finding a logic to
translate to such that common inferences people make when speaking are valid in
the logical language. Her main interest were deontic and epistemic inference:
consider for example the sentence ``Mr. X might be in Victoria or Brixton''.
From this one can infer that ``Mr. X might be in Victoria and Mr. X might be in
Brixton''. This pattern is an example of epistemic inference, but naive attempts
to capture this inference swiftly lead to contradictions.

Maria and her research group suggest that this pattern in reasoning can be
modelled using the neglect-zero hypothesis. This is the idea that when reasoning
about a statement people ignore configurations where it is vacuously true. This
is implemented in the logic by considering enriched formulas: these have extra
terms expressing that the model satisfying the formula is non-empty. This idea
is very promising for modelling speech, conversation, and inference.
\end{document}