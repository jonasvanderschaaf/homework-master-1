\documentclass{article}

\usepackage[margin=1in]{geometry}
\usepackage[parfill]{parskip}

\title{Report meeting PhD student}
\author{Jonas van der Schaaf}
\date{\today}

\begin{document}
\maketitle

For my meeting with a PhD student I met with Rodrigo Almeida: who is doing a PhD
in mathematical logic: in particular he is currently working on Stone type
dualities, which is what we talked about for the first part of our meeting.

Stone duality is an equivalence of categories between the opposite category of
boolean lattices and the category of Haussdorf, compact, and totally
disconnected topological spaces. This equivalence is given by mapping a boolean
lattice to the Zariski topology on the prime ideals of the boolean ring
associated with the lattice. The surprising part of this result is that any
boolean ring can be uniquely recovered (up to isomorphism) from its prime
ideals, which is really not clear a priori.

We also discussed interplay between different parts of logic and mathematics in
Rodrigo's research and research in general. In particularly for Stone duality we
talked about the combination of logic, topology and (universal) algebra.

Afterwards we discussed what it's like to do a PhD. One of the main questions I
asked Rodrigo about doing research was how to find questions to ask in the first
place. He started of by talking about how you usually start out answering
questions of more experienced researchers. Another great source of questions is
recognizing similarities between current interests and already obtained
knowledge and trying to extend those similarities. His main advice however was
to not worry about coming up with questions it because ``it just happens''.

We also discussed how to obtain a PhD position. Rodrigo's main advice was to
keep in touch with people because they are the key to start a PhD. Going to
conferences and seminars and talking to people is a great way to get your name
out there.
\end{document}