\section{Yde Venema: Fixpoint \& Circularity in Modal Logic}

Introducing recursion in semantics of modal logic.

\subsection{Common knowledge}
Byzantine generals: when are you certain a message has been received?

Infinite conjunction of all combinations of \(K_{1},K_{2}\) before formula
\(\varphi\) is not expressible in standard modal logic/epistemic.

Circularity allows expression of common knowledge modal operator \(C\):
\[
    Ca\equiv a\wedge K_{1}Ca\wedge K_{2}Ca.
\]

\subsection{Gödels theorem}
There is a true sentence in arithmetic which is not provable in Peano
arithmetic.

\begin{proof}
    Let \(\mathbb{N}\) be a model of the natural numbers and \(PA\) the
    axiomatisation of Peano arithmetic. Then \(PA\vdash\varphi\vee
    PA\nvdash\varphi\).

    Define \(\square_{PA}\varphi\) as ``There is a derivation in \(PA\) of
    \(\varphi\)''. Then there is a formula \(\gamma\) with \(\gamma\equiv
    \neg\square_{PA}\gamma\).

    Assume for contradiction that \(\mathbb{N}\models\neg\gamma\).
    Then \(\mathbb{N}\models\square_{PA}\gamma\), which means that
    \(\mathbb{N}\models\gamma\) by soundness. This means that
    \(\mathbb{N}\models\gamma\).

    However \(PA\nvdash\gamma\). Therefore there is a non-provable true
    statement in Peano arithmetic.
\end{proof}

There formulas can be expressed as fixed points of the ``equation'' \(x\equiv
a\wedge K_{1}x\wedge K_{2}x\) and \(x\equiv\neg\square_{PA}x\).

These are examples of a poor language where we want to express that what we want
to express is a fixed point of that language.

\subsection{Temporal logic}
Until operator: \(pUq\) (\(p\) until \(q\)).

What happens on the natural numbers? Then \(pUq\equiv q\vee (p\wedge
\lozenge(pUq))\) where \(\lozenge\) is the successor.

Fixpoint: \(x\equiv q\vee(p\vee\lozenge x)\).

Fixpoint logic (generally) causes failure of compactness.

\subsection{Game semantics of the until operator}
Formula tree of a formula

Formula satisfiable at point \(\Leftrightarrow\) there is a winning strategy.

Unfold \(pUq\) by going back to top of formula tree.

For intinite repetition: it depends on the logic who wins

We use game semantics because they correspond naturally to human intuition.

\subsection{Parity formulas}
Let \(\mathbb{G}=(G,E)\) be a finite directed graph with a labelling
\[
    L:G\to\{\vee,\wedge,X\}\cup \text{Prop} \cup \overline{\text{Prop}}.
\]
which is compatible with obvious stuff.

Interpret these as formulas. Who wins the infinite matches?

We add a colouring \(\gamma:G\to\{\textnormal{magenta},\textnormal{navy}\}\) and
a starting point \(v_{I}\in G\). Cycles can only be coloured with one colour.

We say a game is won by \(\exists\) if we end up in a navy cycle, in a magenta
cycle \(\exists\) wins.

Game theoretic approach is good for complexity theory.

They're currently thinking about Craig interpolation.

Potential future work: finite model theory