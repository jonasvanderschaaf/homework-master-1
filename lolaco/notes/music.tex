\section{What is music?}
Special type of sound:
\begin{itemize}
    \item Regular
    \item Stable
\end{itemize}

Repeated speech transforms to music. Students repeated back the melody.

Transposition and shuffles cancel effect \(\to\) exact repetition is needed.

Not all phrases were transformed.

\subsection{When does the illusion occur}
\begin{itemize}
    \item Repetition
    \item Stable pitch
    \item Stable pitch close to musical interval
    \item Regular rhythm
    \item Vowel quality
\end{itemize}

No musical training/English speaking necessary

Unfamiliar language has stronger effect, musical aptitude more correlated than
training.

Non speech sounds also transform to music and it transforms much easier than
speech: no exact repetition needed

\subsection{Their research}
\begin{itemize}
    \item Can machines distinguish transforming sounds

          \begin{itemize}
              \item Can AI distinguish
              \item Feature extraction
              \item Presence of major 3rd and fifth helps with transformation
              \item Degree of dissonance
              \item \(\%\) pitched
          \end{itemize}

    \item Role of exact repetition
    \item
\end{itemize}

\subsection{Temporal dynamics}
Hard to unhear the music: long temporal persistence.

We are more likely to hear a speech as song when we know an implied melody.

Autotune helps (but decreasing autotune amount helps)

\subsubsection{Why is it irreversible}
Sensory systems don't want to give up a plausible interpretation

Why is musical interpretation preferred

\subsection{The role of repetition}
Exact repetition needed for speech sounds, not for other?

Mean fundamental frequency hypothesis: pitch signal for music tends to be
higher. Rejected

Perceptual shift happens when focus shifts from linguistical processing.

Verbal satiation \(\to\) another repetition becomes active \(\to\) musical
representation.