\section{Floris Roelofsen: Inquisitive Epistemic Logic}

Developing framework to analyze information exchange through communication

Model facts and whether agents know facts

Information exchange is based on questions asked, which gives direction to
information exchange. Model should describe what agents want to know.

Declarative sentences provide information, interrogative sentences request
information.

Model should be enriched with issues, language needs to be enriched with
interrogatices. Meaning needs to be enriched with inquisitive content.

\subsection{Epistemic logic}
A model is
\begin{align*}
    M          & =(\mathcal{W},V,\setwith{\sigma_{a}}{a\in A})  \\
    W          & \>\>\text{Set of worlds}                       \\
    V          & :W\to\mathcal{P}(P) \text{valuation}           \\
    \sigma_{a} & :W\to \mathcal{P}(W)\,\text{Information state}
\end{align*}

\subsection{Knowledge modalities}
\(K_{a}\varphi\) agent \(a\) knows \(\varphi\)

Defined normally with modality:
\[
    M,w\models K_{a}\varphi\Leftrightarrow\forall v\in\sigma_{a}(w):M,v\models\varphi
\]
We define
\[
    \abs{\varphi}_{M}=\setwith{w\in W}{M,w\models\varphi}.
\]
Also
\[
    M,w\models K_{a}\varphi\Leftrightarrow \sigma_{a}(w)\subseteq\abs{\varphi}_{M}.
\]

Agents cannot believe false things if \(\forall w\in W:\forall a:
w\in\sigma_{a}(w)\)

\subsection{Information exchange}
An issue can be modeled as the set of information states \(I\): those where the issue
is resolved.

Issue states:
\begin{enumerate}
    \item Downward closed. If an issue is resolved in \(s\) and \(t\subseteq s\)
          then the issue is resolved in \(t\). \(s\in I\) and \(t\subseteq s\)
          then \(t\in I\)

    \item Issues contain the inconsistent information state: \(\emptyset\in I\).
          Or equivalently issues are non-empty.
\end{enumerate}

\[
    ?p=\setwith{s}{\text{\(s\) is information state} \wedge \forall w,v\in s w\in V(p)\leftrightarrow v\in V(p)}
\]

Issues are depicted as its maximal elements.