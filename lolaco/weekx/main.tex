\documentclass{article}

\usepackage[margin=1in]{geometry}

\title{Semantic conditioning and regularit}
\author{Jonas van der Schaaf}
\date{}

\begin{document}
\maketitle

Mariekes talk discussed the relation between naturality and regularity between
word order and their relation to a sign language which emerged at a Nicaraguan
school for deaf children. The most important parts of a sentence are the
subject, object, and verb. While these occur in almost every language, the order
in which they occur can vary wildly, but there are only six possible
permutations.

The order can be dependent on the meaning of a sentence. If this happens often a
language is called natural, and contrary if the word order is rigid the language
is called regular. This is of course a spectrum where languages can be any
combination of regular and natural.

This naturalness shows up a lot in improvisation. In particular for extensional
sentences the natural word order happens to be subject-object-verb, while for
intensional sentences the natural word order is subject-verb-object.

The sign language of the Nicaraguan school is quite young, and therefore
researchers expected the language to have a very natural word order. Because
there were students over multiple generations, the researchers could also
research the change in language that happened over the years. The language was
strongly sov and little change happened over time.
\end{document}