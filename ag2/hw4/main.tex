\documentclass{article}

\usepackage[utf8]{inputenc}
\usepackage[shortlabels]{enumitem}
\usepackage{amsthm, amssymb, mathtools, amsmath, bbm, mathrsfs, stmaryrd, xcolor}
\usepackage{nicefrac}
\usepackage[margin=1in]{geometry}
\usepackage[parfill]{parskip}
\usepackage[hidelinks]{hyperref}
\usepackage{float}
\usepackage{cleveref}
\usepackage{svg}
\usepackage{tikz-cd}

\usepackage{quiver}

\renewcommand{\qedsymbol}{\raisebox{-0.5cm}{\includesvg[width=0.5cm]{../../qedboy.svg}}}


\newcommand{\N}{\mathbb{N}}
\newcommand{\Z}{\mathbb{Z}}
\newcommand{\NZ}{\mathbb{N}_{0}}
\newcommand{\Q}{\mathbb{Q}}
\newcommand{\R}{\mathbb{R}}
\newcommand{\C}{\mathbb{C}}
\newcommand{\A}{\mathbb{A}}
\newcommand{\proj}{\mathbb{P}}
\newcommand{\sheaf}{\mathcal{O}}
\newcommand{\FF}{\mathcal{F}}
\newcommand{\G}{\mathcal{G}}

\newcommand{\zproj}{Z_{\textnormal{proj}}}

\newcommand{\maxid}{\mathfrak{m}}
\newcommand{\primeid}{\mathfrak{p}}
\newcommand{\primeidd}{\mathfrak{q}}

\newcommand{\F}{\mathbb{F}}
\newcommand{\incl}{\imath}

\newcommand{\tuple}[2]{\left\langle#1\colon #2\right\rangle}

\DeclareMathOperator{\order}{order}
\DeclareMathOperator{\Id}{Id}
\DeclareMathOperator{\im}{im}
\DeclareMathOperator{\ggd}{ggd}
\DeclareMathOperator{\kgv}{kgv}
\DeclareMathOperator{\degree}{gr}
\DeclareMathOperator{\coker}{coker}
\DeclareMathOperator{\matrices}{Mat}

\DeclareMathOperator{\gl}{GL}

\DeclareMathOperator{\Aut}{Aut}
\DeclareMathOperator{\Hom}{Hom}
\DeclareMathOperator{\End}{End}
\DeclareMathOperator{\colim}{colim}
\newcommand{\isom}{\overset{\sim}{\longrightarrow}}

\newcommand{\schemes}{{\bf Sch}}
\newcommand{\aff}{{\bf Aff}}
\newcommand{\Grp}{{\bf Grp}}
\newcommand{\Ab}{{\bf Ab}}
\newcommand{\cring}{{\bf CRing}}
\DeclareMathOperator{\modules}{{\bf Mod}}
\newcommand{\catset}{{\bf Set}}
\newcommand{\cat}{\mathcal{C}}
\newcommand{\chains}{{\bf Ch}}
\newcommand{\homot}{{\bf Ho}}
\DeclareMathOperator{\objects}{Ob}
\newcommand{\gen}[1]{\left<#1\right>}
\DeclareMathOperator{\cone}{Cone}
\newcommand{\set}[1]{\left\{#1\right\}}
\newcommand{\setwith}[2]{\left\{#1:#2\right\}}
\DeclareMathOperator{\Ext}{Ext}
\DeclareMathOperator{\Nil}{Nil}
\DeclareMathOperator{\idem}{Idem}
\DeclareMathOperator{\rad}{Rad}
\DeclareMathOperator{\divisor}{div}
\DeclareMathOperator{\Pic}{Pic}
\DeclareMathOperator{\spec}{Spec}
\DeclareMathOperator{\supp}{Supp}
\newcommand{\ideal}{\triangleleft}
\newcommand{\ff}{\mathcal{K}}

\newenvironment{solution}{\begin{proof}[Solution]\renewcommand\qedsymbol{}}{\end{proof}}

\newtheorem{lemma}{Lemma}
\newtheorem{proposition}{Proposition}

\theoremstyle{definition}
\newtheorem{question}{Exercise}
\newtheorem{definition}{Definition}

\title{Homework Algebraic Geometry 2}
\author{Jonas van der Schaaf}
\date{}

\begin{document}
\maketitle

\begin{question}
    I will use a bunch of results which are exercises in the lecture notes or
    just stuff I need. These are proven at the end of the exercise.

    Let \(\incl:Z\to X\) be a closed immersion and \(\FF\) a quasi-coherent
    \(\sheaf_{X}\)-module.
    \begin{enumerate}[(i)]
        \item Show that there is a natural isomorphism
              \(\incl_{*}\incl^{*}\FF\cong\FF\otimes_{\sheaf_{X}}\incl_{*}\sheaf_{Z}\).

              \begin{proof}
                  We exhibit a morphism
                  \(\FF\otimes_{\sheaf_{X}}\incl_{*}\sheaf_{Z}\to\incl_{*}\incl^{*}\FF\)
                  which is an isomorphism on stalks and therefore an
                  isomorphism.

                  The morphism is induced by a bilinear map
                  \(\FF\times\incl_{*}\sheaf_{Z}\to\incl_{*}\incl^{*}\FF\)
                  given by the following composition
                  \[
                      \FF\times\incl_{*}\sheaf_{Z}\to\incl_{*}\incl^{*}\FF\times_{\sheaf_{X}}\incl_{*}\sheaf_{Z}\cong\incl_{*}(\incl^{*}\FF\times\sheaf_{Z})\to\incl_{*}\incl^{*}\FF.
                  \]
                  Where \(\FF\to\incl_{*}\incl^{*}\FF\) is given by the co-unit
                  of the adjunction and
                  \(\incl^{*}\FF\times\sheaf_{Z}\to\incl^{*}\FF\) is the
                  multiplication map. Because multiplication is bilinear, this
                  is clearly a bilinear map and induces a map from the tensor
                  product. Now we consider how this map acts on stalks. The
                  product of two sheafs is also the coproduct and stalks
                  preserve coproducts so on stalks we have a map
                  \[
                      (\FF\times\incl_{*}\sheaf_{Z})_{x}=\FF_{x}\oplus(\incl_{*}\sheaf_{Z})_{x}\to(\incl_{*}\incl^{*}\FF)_{x}\cong\begin{cases}
                          \FF_{x}\otimes_{\sheaf_{X,x}}\incl_{*}\sheaf_{Z}\cong\FF_{x}\otimes_{\sheaf_{X,x}}\sheaf_{Z,x} & x\in Z,    \\
                          0                                                                                              & x\notin Z.
                      \end{cases}
                  \]
                  This map is given for \(x\in Z\) by \((f,g)\mapsto fg\otimes
                  1=f\otimes g\).

                  We know that
                  \[
                      \FF_{x}\otimes_{\sheaf_{X}}\incl_{*}\sheaf_{Z}\cong\begin{cases}
                          \FF_{x}\otimes_{\sheaf_{X,x}}\sheaf_{Z,x} & x\in Z,    \\
                          \FF_{x}\otimes 0\cong0                    & x\notin Z.
                      \end{cases}
                  \]
                  The map on stalks
                  \(\FF_{x}\otimes_{\sheaf_{X}}\incl_{*}\sheaf_{Z}\to\incl_{*}\incl^{*}\FF\)
                  for \(x\in Z\) is then given by
                  \[
                      f\otimes g\mapsto fg\otimes 1=f\otimes g.
                  \]
                  Therefore it is clearly an isomorphism on all stalks.
              \end{proof}

        \item Show that there is a short exact sequence
              \[
                  0\to\sheaf_{X}(-d)\to\sheaf_{X}\to\incl_{*}\sheaf_{Z}\to0.
              \]

              \begin{proof}
                  We prove that \(I\cong\sheaf_{X}(-d)\) where \(I\) is the
                  ideal sheaf such that
                  \(\incl_{*}\sheaf_{Z}\cong\sheaf_{X}/I\).

                  We give an isomorphism of graded modules \(S(-d)\to (f)\). We
                  know that \(\widetilde{(f)}=I\) and
                  \(\widetilde{S(-d)}=\sheaf_{X}(-d)\) and so these two are
                  isomorphic as wel.

                  We exhibit an isomorphism \(g:S(-d)\to(f)\) by \(g(a)=af\).
                  This is clearly a morphism of graded modules because \(f\) is
                  homogeneous of degree \(d\). It is injective because the ring
                  \(S\) is an integral domain. It is surjective because for
                  \(af\in(f)\) we have that \(g(a)=af\). Therefore this is an
                  isomorphism.
              \end{proof}

        \item Show that
              \(H^{i}(X,\sheaf_{X}(n)\otimes_{\sheaf_{X}}\incl_{*}\sheaf_{Z})\)
              is isomorphic to \(H^{i}(Z,\incl^{*}\sheaf_{X}(n))\).

              \begin{proof}
                  We know that
                  \(\incl_{*}\incl^{*}\FF\cong\FF\otimes_{\sheaf_{X}}\incl_{*}\sheaf_{Z}\)
                  and so we have
                  \begin{align*}
                      H^{i}(X,\sheaf_{X}(n)\otimes_{\sheaf_{X}}\incl_{*}\sheaf_{Z}) & \cong H^{i}(X,\incl_{*}\incl^{*}\FF) \\
                                                                                    & \cong H^{i}(Z,\incl^{*}\FF)
                  \end{align*}
                  by Exercise 12.9 below.
              \end{proof}

        \item Show that \(H^{i}(Z,\incl^{*}\sheaf_{X}(n))=0\) for all
              \(0<i<N-1\).

              \begin{proof}
                  We know that
                  \(H^{i}(X,\sheaf_{X}(n)\otimes_{\sheaf_{X}}\incl_{*}\sheaf_{Z})\cong
                  H^{i}(Z,\incl^{*}\sheaf_{X}(n))\) and so we will compute the
                  former instead.

                  We know that taking the tensor product with a (locally) free
                  module is exact. Therefore we have an exact sequence:
                  \[
                      0\to\sheaf_{X}(-d)\otimes_{\sheaf_{X}}\sheaf_{X}(n)\cong\sheaf_{X}(n-d)\to\sheaf_{X}\otimes_{\sheaf_{X}}\sheaf_{X}(n)\cong\sheaf_{X}(n)\to\incl_{*}\sheaf_{Z}\otimes\sheaf_{X}(n)\to 0.
                  \]
                  This means we obtain a long sequence of homology groups. In
                  particular we have an exact sequence:
                  \[
                      H^{i}(X,\sheaf_{X}(n))\to H^{i}(X,\sheaf_{X}(n)\otimes_{\sheaf_{X}}\incl_{*}\sheaf_{Z})\to H^{i+1}(X,\sheaf_{X}(n-d)).
                  \]
                  By theorem 13.4.1 we know that for \(0<j<N\) we have
                  \(H^{j}(X,\sheaf_{X}(m))=0\) for all \(m\in\Z\). This means
                  that
                  \(H^{i}(X,\sheaf_{X}(n)\otimes_{\sheaf_{X}}\incl_{*}\sheaf_{Z})=0\)
                  for \(0<i<N-1\).
              \end{proof}

        \item Give a formula for the dimension of \(H^{0}(Z,\sheaf_{Z})\) and
              \(H^{N-1}(Z,\sheaf_{Z})\).

              \begin{solution}
                  We know that for any coherent sheaf \(\FF\) on \(Z\) we have
                  \(H^{i}(Z,\FF)\cong H^{i}(X,\incl_{*}\FF)\). We compute the
                  corresponding cohomology groups of \(\incl_{*}\FF\). When
                  writing \(H^{i}(X,\FF)\) we will omit the \(X\) and write
                  \(H^{i}(\FF)\).

                  First we consider \(H^{0}(Z,\sheaf_{Z})\). We have an exact
                  sequence
                  \[
                      0\to H^{0}(\sheaf_{X}(-d))\to H^{0}(\sheaf_{X})\to H^{0}(\incl_{*}\sheaf_{Z})\to H^{1}(\sheaf_{X}(-d)).
                  \]
                  We know that \(H^{1}(\sheaf_{X}(-d))=0\) and so the above is
                  in fact a short exact sequence. Therefore the dimensions of
                  \(H^{0}(\sheaf_{X}(-d))\to H^{0}(\sheaf_{X})\) and
                  \(H^{0}(\incl_{*}\sheaf_{Z})\) add up to the dimension of
                  \(H^{0}(\sheaf_{X})\).

                  We know \(H^{0}(\sheaf_{X})=S_{0}=k\) and
                  \(H^{0}(\sheaf_{X}(-d))=k[X_{0},\ldots,X_{n}]_{-d}=0\).
                  Therefore \(\dim H^{0}(\sheaf_{X})=1-0=1\).

                  Similarly for \(H^{N-1}(\incl_{*}\sheaf_{Z})\) we have the
                  exact sequence
                  \[
                      .H^{N-1}(\sheaf_{X})\to H^{N-1}(\incl_{*}\sheaf_{Z})\to H^{N}(\sheaf_{X}(-d))\to H^{N}(\sheaf_{X}).
                  \]
                  We know
                  \[
                      H^{N}(\sheaf_{X})=\frac{1}{X_{0}\cdots X_{N}}\cdot k\left[\frac{1}{X_{0},\ldots,\frac{1}{X_{n}}}\right]_{0}=0
                  \]
                  and
                  \[
                      H^{N-1}(\sheaf_{X})=0.
                  \]
                  Therefore
                  \[
                      H^{N-1}(\incl_{*}\sheaf_{Z})\cong H^{N}(\sheaf_{X}(-d))=\left(\frac{1}{X_{0}\cdots X_{N}}\cdot k\left[\frac{1}{X_{0}},\ldots,\frac{1}{X_{N}}\right]\right)_{-d}\cong \left(k\left[\frac{1}{X_{0}},\ldots,\frac{1}{X_{N}}\right]\right)_{-d+N+1}.
                  \]
                  This space is generated as \(k\)-vector space by the inverses
                  of monomials of degree \(d-N+1\). If \(d-N+1<0\) i.e.
                  \(d-1<N\) these don't exist so the dimension is \(0\). Else it
                  is given by a combinatorial problem called stars and stripes
                  which gives
                  \[
                      \dim_{k}H^{N-1}(\incl_{*}\sheaf_{Z})=\begin{cases}
                          0              & d-1<N,             \\
                          \binom{d-1}{N} & \textnormal{else}.
                      \end{cases}
                  \]
              \end{solution}
    \end{enumerate}

    \paragraph{Exercise 12.9.}Let \(X\) be a topological space. Let \(K\) be a
    closed subset of \(X\), and denote \(\imath:K\to X\) the inclusion. Let
    \(\FF\) be a sheaf on \(K\). Denote by \(\imath_{*}\FF\) the pushforward of
    \(\FF\) on \(X\).

    \begin{enumerate}[(i)]
        \item Show that
              \[
                  (\incl_{*}\FF)_{x}=\begin{cases}
                      \FF_{x} & x\in Z,    \\
                      0       & x\notin Z.
                  \end{cases}
              \]

              \begin{proof}
                  We know that \((\incl_{*}\FF)_{x}=\colim_{X\supseteq
                      U,x\in U}\FF(\incl^{-1}U)\) and
                  \(\FF_{x}=\colim_{X\supseteq U\ni x}\FF(U)\).

                  If \(x\notin Z\) then \(x\in X\setminus Z\). We have
                  \(\incl_{*}\FF(X\setminus Z)=\FF(\incl^{-1}X\setminus Z)=0\).
                  This means that \((\incl_{*}\FF)_{x}=0\) as well.

                  If \(x\in Z\), because every open of \(X\) is of the form
                  \(U\cap Z\) with \(U\) open in \(X\), the colimits
                  \(\colim_{X\supseteq U\ni x}\FF(U)\) and \(\colim_{X\supseteq
                      U,x\in U}\FF(\incl^{-1}U)\) are exactly the same and so the
                  stalks are isomorphic.
              \end{proof}

        \item Show that \(\FF\mapsto\incl_{*}\FF\) is an exact functor.

              \begin{proof}
                  By the previous part the functor \(\incl_{*}\) preserves
                  stalks on \(Z\) and stalks on \(X\setminus Z\) are \(0\). An
                  exact sequence of sheafs is exact if it is exact on stalks.

                  For \(x\notin Z\) the sequence of stalks is mapped to the
                  short exact sequence of \(0\)-modules. If \(x\in Z\) the short
                  exact sequence is preserved. Therefore \(\incl_{*}\) preserves
                  short exact sequences.
              \end{proof}

        \item Show that \(\incl_{*}\) sends flasque sheafs to flasque sheafs.

              \begin{proof}
                  We take \(\FF\) to be a flasque sheaf. Then for \(U\subseteq
                  V\subseteq X\) the restriction
                  \[
                      \incl_{*}\FF(V)=\FF(\incl^{-1}V)\to\FF(\incl^{-1}U)=\incl_{*}\FF(V)
                  \]
                  is surjective because \(\FF\) is flasque.
              \end{proof}

        \item Show that there are natural isomorphisms
              \(H^{j}(X,\incl_{*}\FF)\cong H^{j}(K,\FF)\).

              \begin{proof}
                  Let \(\FF\to C^{\bullet}\) be a flasque resolution of \(\FF\).
                  Because \(\incl_{*}\) is exact
                  \(\incl_{*}\FF\to\incl_{*}C^{\bullet}\) is a flasque
                  resolution as well.Then \(H^{i}(Z,\FF)\cong
                  h^{i}(C^{\bullet})\) and \(H^{i}(X,\incl_{*}\FF)\cong
                  h^{i}(\incl_{*}C^{\bullet})\). Now we have
                  that
                  \[
                      \incl_{*}C^{i}(X)=C^{i}(\imath^{-1}X)=C^{i}(Z).
                  \]
                  The functor \(\incl_{*}\) preserves images/kernels/cokernels
                  by exactness, so the two quotients
                  \(h^{i}(\incl_{*}C^{\bullet})\) and \(h^{i}(C^{\bullet})\) are
                  naturally isomorphic.
              \end{proof}
    \end{enumerate}
\end{question}

\begin{question}
    Let \(k\) be a field. Let \(X\) be an integral scheme of finite type over
    \(k\). We call \(X\) a curve over \(k\) if \(\dim(X)=1\). Assume \(X\) is a
    curve over \(k\) and let \(|X|\) denote the closed points of \(X\). Let
    \(\eta\) denote the generic point.

    \begin{enumerate}[(i)]
        \item Show that we have a decomposition \(X=|X|\sqcup\set{\eta}\) as
              point sets.

              \begin{proof}
                  Let \(x\neq\eta\) be a point of \(X\) and suppose
                  \(\overline{\set{x}}\neq\set{x}\). Then \(x\) has an open
                  affine neighbourhood \(U=\spec R\), where \(x=\primeid\ideal
                  U\). Then there is a maximal ideal
                  \(R\triangleright\maxid\supseteq\primeid\) and
                  \(\set{\maxid}\) is a closed point. Then \(\maxid\) is
                  contained in every open neighbourhood of \(x\) so it is a
                  limit point of \(\set{x}\). This means that
                  \(\set{\maxid}\subsetneq\bar{\set{\primeid}}\subsetneq X\) is
                  a chain of irreducible closed subsets of length \(3\) so
                  \(\dim(X)\geq 3\). Therefore no such non-closed \(x\) can
                  exist.
              \end{proof}
    \end{enumerate}

    Let \(K(X)\) be the function field of \(X\). For each \(x\in X\) we view the
    local ring \(\sheaf_{X,x}\) as a subring of \(K(X)\).

    \begin{enumerate}[(i), resume]
        \item Show that the assignment \(f\mapsto(f\mod \sheaf_{X,x})_{x\in|X|}\)
              determines a map of \(k\)-vector spaces
              \(\alpha:K(X)\to\bigoplus_{x\in|X|}K(X)/\sheaf_{X,x}\).

              \begin{proof}
                  We first show the map is well-defined: take \(f\in K(X)\),
                  then there is an open neighbourhood \(U\subseteq X\) of
                  \(\eta\) with \(f\in\Gamma(\sheaf_{X},U)\). This \(U\) is a
                  dense open subset and so its complement \(Z=X\setminus U\) is
                  a closed subset of the irreducible space \(X\). It is then the
                  union of finitely many irreducible closed subsets
                  \(Z_{1},\ldots,Z_{n}\) which have generic points
                  \(x_{1},\ldots,x_{n}\) which are different from \(\eta\).
                  Therefore \(Z_{i}=\overline{\set{x_{i}}}=\set{x_{i}}\) so
                  \(U=X\setminus\set{x_{1},\ldots,x_{n}}\). For all \(x\in U\)
                  we have \(f\in\sheaf_{X,x}\) and \(U\) is cofinite in \(X\),
                  proving the map is well-defined.

                  The map is componentwise a morphism so it is a morphism.
              \end{proof}

        \item Exhibit an isomorphism
              \(K_{X}/\sheaf_{X}\isom\bigoplus_{x\in|X|}i_{x,*}(K(X)/\sheaf_{X,x})\).

              \begin{solution}
                  The constant presheaf associated to \(\ff(X)\) is already a
                  sheaf on \(X\) because it is an irreducible topological space.
                  For \(x\in X\) we know that
                  \[
                      i_{x,*}(K(X)/\sheaf_{X,x})(U)=\begin{cases}
                          K(X)/\sheaf_{X,x} & x\in X,            \\
                          0                 & \textnormal{else}.
                      \end{cases}
                  \]
                  Therefore we have a natural map \(\ff_{X}\to
                  i_{x,*}(K(X)/\sheaf_{X,x})\), where for each \(f\in K(X)\)
                  there are finitely many \(x\in |X|\) such that \(f\neq 0\in
                  K(X)/\sheaf_{X,x}\). This means there is an induced morphism
                  of sheaves
                  \[
                      \beta:\ff_{X}\to\bigoplus_{x\in|X|}i_{x,*}(K(X)/\sheaf_{X,x}).
                  \]
                  We show its kernel is given by \(\sheaf_{X}\) and then that
                  the quotient map is surjective on stalks. This is then
                  automatically injective because its kernel has already been
                  quotiented out.

                  Take \(f\in\ker(\beta)(U)\). Then for all \(x\in |X|\cap U\)
                  we have that \(f_{x}\in \ker(K(X)\to
                  K(X)/\sheaf_{X,x})=\sheaf_{X,x}\). We also trivially have
                  \(f_{\eta}\in \sheaf_{X,\eta}=K(X)\) This means it glues to a
                  section \(f\in\Gamma(\sheaf_{X},U)\). Therefore
                  \(\ker(\beta)\) is contained in \(\sheaf_{X}\).

                  Conversely take \(f\in\sheaf_{X}(U)\). Then
                  \(f_{x}\in\sheaf_{X,x}\) for all \(x\in U\) and so it is in
                  the kernel of all projections \(K(X)\to K(X)/\sheaf_{X,x}\).
                  For \(x\notin U\) we have \((K(X)/\sheaf_{X,x})(U)=0\) so it
                  is trivially in the kernel of these maps too. This means that
                  \(\beta(f)=0\) for all \(f\in\sheaf_{X}(U)\) and so
                  \(\sheaf_{X}\) is contained in \(\ker(\beta)\).

                  Now we examine the stalk maps:
                  \[
                      \beta_{x}:\ff_{X}\to\left(\bigoplus_{x\in|X|}i_{x,*}(K(X)/\sheaf_{X,x})\right)_{x}.
                  \]
                  For \(y\neq x\) we can easily see that the stalk of
                  \(i_{x,*}(K(X)/\sheaf_{X,x})\) at \(y\) is zero: there is an
                  open neighbourhood of \(y\) separating \(x\) from \(y\)
                  because both are closed points. Taking the stalks commutes
                  with direct sums so \(\beta_{x}\) is the map
                  \[
                      \beta_{x}:\ff_{X,x}=K(X)\to\left(\bigoplus_{x\in|X|}i_{x,*}(K(X)/\sheaf_{X,x})\right)_{x}=(i_{x,*}(K(X)/\sheaf_{X,x}))_{x}=K(X)/\sheaf_{X,x}
                  \]
                  which is clearly surjective. Therefore the induced map
                  \[
                      \ff_{X}/\sheaf_{X}\to\bigoplus_{x\in|X|}i_{x,*}(K(X)/\sheaf_{X,x})
                  \]
                  is both injective and surjective: an isomorphism.
              \end{solution}

        \item Show that the above exact sequence is a flasque resolution.

              \begin{proof}
                  The sheaf \(\ff_{X}\) is a flasque sheaf: for every non-empty
                  open \(U\subseteq X\) we have \(\ff_{X}(U)=K(X)\) because
                  \(X\) is an irreducible space. The restriction maps to a
                  non-empty set are the identity, which is clearly surjective.
                  The empty set is the trivial module, any map to which is also
                  clearly surjective.

                  For the second sheaf, we show all direct summands are flasque
                  as this property is clearly preserved when taking direct sums.

                  For \(x\in|X|\) consider \(i_{x,*}(K(X)/\sheaf_{X,x})\). Take
                  \(U\subseteq V\subseteq X\) open, we show the restriction is
                  surjective. We separate by two cases: \(x\in U\) or \(x\notin
                  U\). If \(x\in U\), then \(x\in V\), so
                  \[
                      i_{x,*}(K(X)/\sheaf_{X,x})(U)=i_{x,*}(K(X)/\sheaf_{X,x})(V)=K(X)/\sheaf_{X,x}
                  \]
                  with identity as the restriction map which is surjective.

                  If \(x\notin U\) then \(i_{x,*}(K(X)/\sheaf_{X,x})(U)=0\) and
                  any morphism to this module is surjective.

                  This shows this sheaf is flasque.

                  We easily see it is a resolution because it is an exact
                  sequence.
              \end{proof}

        \item Deduce from (iv) that
              \[
                  H^{0}(X,\sheaf_{X})=\bigcap_{x\in|X|}O_{X,x}
              \]
              and
              \[
                  H^{1}(X,\sheaf_{X})\cong\coker(K(X)\to\bigoplus_{x\in|X|}K(X)/\sheaf_{X,x}).
              \]

              \begin{proof}
                  We know that \(H^{0}(X,\sheaf_{X})\) is the module given by
                  \(\ker
                  (\Gamma(X,\ff_{X})\to\Gamma(X,\bigoplus_{x\in|X|}K(X)/\sheaf_{X,x}))\)
                  which because \(\Gamma\) is left exact is \(\Gamma\) of the
                  kernel of \(\ff_{X}\to\bigoplus_{x\in|X|}K(X)/\sheaf_{X,x}\).
                  We know \(f\in\Gamma(X,\ff(X))\) is in the kernel if it is
                  \(0\) in \(\Gamma(X,K(X)/\sheaf_{X,x})=K(X)/\sheaf_{X,x}\) for
                  each \(x\in |X|\). Therefore \(f\in H^{0}(X,\sheaf_{X})\) iff
                  \(f\in\bigcap_{x\in |X|}\sheaf_{X,x}\).

                  Similarly \(H^{1}(X,\sheaf_{X})\) is given by
                  \[
                      \ker\left(\Gamma\left(X,\bigoplus_{x\in|X|}K(X)/\sheaf_{X,x}\to 0\right)\right)/\im \left(K(X)\to\bigoplus_{x\in|X|}K(X)/\sheaf_{X,x}\right)
                  \]
                  Because this kernel is just
                  \(\Gamma(X,\bigoplus_{x\in|X|}K(X)/\sheaf_{X,x})=\bigoplus_{x\in|X|}K(X)/\sheaf_{X,x}\)
                  and in this case the image is just the image, this quotient is
                  by definition the desired cokernel.
              \end{proof}

        \item Deduce from (iv) that \(H^{i}(X,\sheaf_{X})=(0)\) for \(i>1\).

              \begin{proof}
                  For \(i>1\) we know that \(d^{i}\) is given by \(0\to 0\). We
                  know that \(H^{i}(X,\sheaf_{X})\) is a quotient of a submodule
                  of the domain of \(d^{i}\), which is \(0\). Therefore
                  \(H^{i}(X,\sheaf_{X})=0\) for all \(i>1\).
              \end{proof}
    \end{enumerate}
\end{question}
\end{document}