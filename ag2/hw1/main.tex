\documentclass{article}

\usepackage[utf8]{inputenc}
\usepackage{enumerate}
\usepackage{amsthm, amssymb, mathtools, amsmath, bbm, mathrsfs, stmaryrd, xcolor}
\usepackage{nicefrac}
\usepackage[margin=1in]{geometry}
\usepackage[parfill]{parskip}
\usepackage[hidelinks]{hyperref}
\usepackage{float}
\usepackage{cleveref}
\usepackage{svg}
\usepackage{tikz-cd}

\renewcommand{\qedsymbol}{\raisebox{-0.5cm}{\includesvg[width=0.5cm]{../../qedboy.svg}}}


\newcommand{\N}{\mathbb{N}}
\newcommand{\Z}{\mathbb{Z}}
\newcommand{\NZ}{\mathbb{N}_{0}}
\newcommand{\Q}{\mathbb{Q}}
\newcommand{\R}{\mathbb{R}}
\newcommand{\C}{\mathbb{C}}
\newcommand{\A}{\mathbb{A}}
\newcommand{\proj}{\mathbb{P}}
\newcommand{\sheaf}{\mathcal{O}}
\newcommand{\FF}{\mathcal{F}}
\newcommand{\G}{\mathcal{G}}

\newcommand{\zproj}{Z_{\textnormal{proj}}}

\newcommand{\maxid}{\mathfrak{m}}
\newcommand{\primeid}{\mathfrak{p}}

\newcommand{\F}{\mathbb{F}}
\newcommand{\incl}{\imath}

\newcommand{\tuple}[2]{\left\langle#1\colon #2\right\rangle}

\DeclareMathOperator{\order}{order}
\DeclareMathOperator{\Id}{Id}
\DeclareMathOperator{\im}{im}
\DeclareMathOperator{\ggd}{ggd}
\DeclareMathOperator{\kgv}{kgv}
\DeclareMathOperator{\degree}{gr}
\DeclareMathOperator{\coker}{coker}

\DeclareMathOperator{\gl}{GL}

\DeclareMathOperator{\Aut}{Aut}
\DeclareMathOperator{\Hom}{Hom}
\DeclareMathOperator{\End}{End}
\DeclareMathOperator{\colim}{colim}
\newcommand{\isom}{\overset{\sim}{\longrightarrow}}

\newcommand{\schemes}{{\bf Sch}}
\newcommand{\aff}{{\bf Aff}}
\newcommand{\Grp}{{\bf Grp}}
\newcommand{\Ab}{{\bf Ab}}
\newcommand{\cring}{{\bf CRing}}
\DeclareMathOperator{\modules}{{\bf Mod}}
\newcommand{\catset}{{\bf Set}}
\newcommand{\cat}{\mathcal{C}}
\newcommand{\chains}{{\bf Ch}}
\newcommand{\homot}{{\bf Ho}}
\DeclareMathOperator{\objects}{Ob}
\newcommand{\gen}[1]{\left<#1\right>}
\DeclareMathOperator{\cone}{Cone}
\newcommand{\set}[1]{\left\{#1\right\}}
\newcommand{\setwith}[2]{\left\{#1:#2\right\}}
\DeclareMathOperator{\Ext}{Ext}
\DeclareMathOperator{\Nil}{Nil}
\DeclareMathOperator{\idem}{Idem}
\DeclareMathOperator{\rad}{Rad}
\DeclareMathOperator{\divisor}{div}
\DeclareMathOperator{\Pic}{Pic}
\DeclareMathOperator{\spec}{Spec}
\newcommand{\ideal}{\triangleleft}

\newenvironment{solution}{\begin{proof}[Solution]\renewcommand\qedsymbol{}}{\end{proof}}

\newtheorem{theorem}{Theorem}[section]
\newtheorem{lemma}[theorem]{Lemma}
\newtheorem{proposition}[theorem]{Proposition}

\theoremstyle{definition}
\newtheorem{question}{Exercise}
\newtheorem{definition}[theorem]{Definition}
\newtheorem{remark}[theorem]{Remark}
\newtheorem{example}[theorem]{Example}
\newtheorem{corollary}[theorem]{Corollary}

\title{Homework Algebraic Geometry 2}
\author{Jonas van der Schaaf}
\date{}

\begin{document}
\maketitle

\begin{question}
    Show that the following statements are equivalent for a scheme \(X\):

    \begin{enumerate}[(a)]
        \item \(X\) is reduced,
        \item all the stalks \(O_{X,x}\) have no nonzero nilpotent element,
        \item \(X\) has a covering by open affines \(U_{i}\) such that
              \(\Gamma(U_{i},O_{X})\) has no non-zero nilpotent elements.
    \end{enumerate}

    \begin{proof}
        \((a)\Rightarrow(c)\) is trivially true because an affine open is
        open.

        For \((c)\Rightarrow(b)\) take \(x\in U_{i}\subseteq X\). Then
        \(O_{X,x}=(O_{X}|_{U_{i}})_{x}\) the stalk at \(x\) of the restricted
        sheaf. This is then the ring
        \(O_{\spec(O_{x}(U_{I}))}(U_{i})_{\primeid}\) where \(\primeid\ideal
        O_{\spec(O_{x}(U_{I}))}(U_{i})=O_{X}(U_{i})\) is the prime ideal onto
        which \(x\) is mapped by the isomorphism witnessing affineness of
        \(U_{i}\). Because \(O_{X}(U_{i})\) has no non-zero nilpotents this
        means the localisation to the prime ideal
        \(O_{X}(U_{i})_{\primeid}=O_{X,x}\) does not either.

        For \((b)\Rightarrow(a)\) we know\footnote{This is exercise 4.7.22 from
            the algebraic geometry 1 course or alternatively lemma 6.11.1 from
            stacks.} that there is an injective morphism
        \(O_{X}(U)\hookrightarrow \prod_{x\in U}O_{X,x}\). Therefore
        \(O_{X}(U)\) is a subring of a ring with no non-zero nilpotents,
        so it has no non-zero nilpotents either.
    \end{proof}
\end{question}

\begin{question}
    Let \(R\) be a discrete valuation ring with fraction field \(K\). Let
    \(A:R\to K\) denote the inclusion map. Write \(X=\spec(K)\) and
    \(Y=\spec(R)\) and let \((f,f^{\sharp}):X\to Y\) be the morphism of schemes
    associated with \(A\).

    \begin{enumerate}[(i)]
        \item Describe the underlying spaces of \(X\) and \(Y\), the continuous
              map \(f:X\to Y\) and for all opens \(V\subseteq Y\) the map
              \(f^{\sharp}_{V}:Y\to X\).

              The ring \(K\) is a field so \(\spec(K)\) has just the one point.

              The spectrum of \(R\) is a two point set: \(\set{(0),\maxid}\)
              where \(\maxid\) is the maximal ideal of \(R\). This has 3 opens:
              \(Y_{0}=\varnothing\subseteq X_{t}=\set{(0)}\subseteq X_{1}=\spec
              R\) where \(t\in R\) is a uniformizer.

              Therefore we want the following diagram to commute:
              \[
                  \begin{tikzcd}[ampersand replacement=\&]
                      {\Gamma(Y_{1},O_{Y})=R} \&\& {K=\Gamma(X_{1},O_{X})} \\
                      \\
                      {\Gamma(Y_{t},O_{Y})=R\left[\frac{1}{t}\right]} \&\& {K=\Gamma(X_{1},O_{X})} \\
                      \\
                      {\Gamma(Y_{0},O_{Y})=0} \&\& {0=\Gamma(X_{0},O_{X})}
                      \arrow["0", from=5-1, to=5-3]
                      \arrow[from=3-1, to=5-1]
                      \arrow[from=1-1, to=3-1]
                      \arrow[from=1-3, to=3-3]
                      \arrow[from=3-3, to=5-3]
                      \arrow["{f^{\sharp}_{Y_{1}}}", from=1-1, to=1-3]
                      \arrow["{f^{\sharp}_{Y_{t}}}", from=3-1, to=3-3]
                  \end{tikzcd}
              \]
              The vertical maps are the natural inclusions given by sheaf
              restriction.

              The map \(f^{\sharp}_{Y_{1}}\) is given by \(A\) because of the
              duality between affine schemes and commutative rings.

              The bottom two rings are the terminal object in \(\cring\) so
              the maps to (and from) those are uniquely determined by the
              universal property.

              For the open \(Y_{t}\), first notice that there is a natural
              isomorphism \(R[\nicefrac{1}{t}]\cong K\) because every element of
              \(R\) is of the form \(ut^{n}\) for \(u\in R^{*}\) and \(n\in\N\).
              Therefore \(f^{\sharp}_{Y_{t}}\) should be a map such that the
              following diagram commutes
              \[
                  \begin{tikzcd}[ampersand replacement=\&]
                      R \\
                      \\
                      {R\left[\frac{1}{t}\right]} \&\& K
                      \arrow[from=1-1, to=3-1]
                      \arrow["{f^{\sharp}_{Y_{t}}}", from=3-1, to=3-3]
                      \arrow["{A}", from=1-1, to=3-3]
                  \end{tikzcd}
              \]
              The image of \(t\) in \(K\) is invertible so there is a unique map
              \(R[\nicefrac{1}{t}]\to K\) such that this map commutes. Because
              \(R[\nicefrac{1}{t}]\cong K\), this is just the witnessing
              natural isomorphism.

        \item Let \(g:X\to Y\) be the unique continuous map that has image the
              closed point of \(Y\). Show that there exists a unique morphism of
              sheaves determined by setting \(g^{\sharp}_{Y}=f^{\sharp}_{Y}\).

              \begin{proof}
                  We first consider all opens in \(Y\) and look at their
                  preimages, the opens are \(\varnothing,\set{(0)},Y\). Their
                  inverse images under this map are \(\varnothing,\varnothing\)
                  and \(X\) respectively. Therefore we are looking for morphisms
                  \(g^{\sharp}\) such that the following diagram commutes:
                  \[
                      \begin{tikzcd}[ampersand replacement=\&]
                          {\Gamma(Y,O_{Y})=R} \&\& {K=\Gamma(X,O_{X})} \\
                          \\
                          {\Gamma(\set{(0)},O_{Y})=K} \&\& {0=\Gamma(\varnothing,O_{X})} \\
                          \\
                          {0=\Gamma(\varnothing,O_{Y})} \&\& {0=\Gamma(\varnothing,O_{X})}
                          \arrow["{f^{\sharp}_{Y}}", from=1-1, to=1-3]
                          \arrow[from=1-1, to=3-1]
                          \arrow[from=1-3, to=3-3]
                          \arrow[from=3-3, to=5-3]
                          \arrow[from=3-1, to=5-1]
                          \arrow["{g^{\sharp}_{\varnothing}}", from=5-1, to=5-3]
                          \arrow["{g^{\sharp}_{\set{(0)}}}", from=3-1, to=3-3]
                      \end{tikzcd}
                  \]
                  However because \(0\) is the terminal ring,
                  \(g^{\sharp}_{\set{(0)}}\) and \(g^{\sharp}_{\varnothing}\)
                  are the unique morphisms to this ring and these squares
                  commute trivially.
              \end{proof}

        \item Is \((g,g^{\sharp})\) a morphism of schemes?

              \begin{solution}
                  This is not a morphism of schemes because the morphism of
                  stalks is not a local morphism for each stalk.

                  At the point \(\maxid\in Y\), the stalk is given by
                  \(R_{\maxid}=R\) because \(R\) is already a local ring with
                  maximal ideal \(\maxid\). The stalk at \((0)\) is given by
                  \(K_{(0)}=K\). The ring morphism is the induced one making the
                  diagram
                  \[
                      \begin{tikzcd}[ampersand replacement=\&]
                          {O_{Y}(Y)=R} \&\& {K=O_{X}(X)} \\
                          \\
                          {\colim_{U\ni\maxid} O_{Y}(U)=R_{\maxid}=R} \&\& K={\colim_{U\ni(0)} O_{X}(U)}
                          \arrow[from=3-1, to=3-3]
                          \arrow["\Id", from=1-1, to=3-1]
                          \arrow["{f^{\sharp}_{Y}}"', from=1-1, to=1-3]
                          \arrow["\Id"', from=1-3, to=3-3]
                      \end{tikzcd}
                  \]
                  By the universal property of \(R_{\maxid}\) this must just be
                  the inclusion \(R\to K\). This is not a local morphism because
                  \(\maxid\ideal R\subseteq K\) is not a subset of the zero
                  ideal \((0)\ideal K\).
              \end{solution}
    \end{enumerate}
\end{question}

\begin{question}
    Consider the rings \(A=\Z\times\Z\), \(B=\Z[X]/(X^{2}-1)\),
    \(C=\Z/(X^{2}+1)\) and set \(X=\spec A\), \(Y=\spec B\) and \(Z=\spec C\).
    \begin{enumerate}[(i)]
        \item For each of the following rings determine whether it is: (a)
              irreducible, (b) connected, (c) reduced:
              \[
                  X,Y,Z,Y\times_{\spec\Z}\F_{2}
              \]

              \begin{solution}
                  We show our main results in this table and give the
                  justification afterwards:

                  \begin{figure}[ht]
                      \centering
                      \begin{tabular}{l|c|c|c}
                                                      & Irreducible & Connected & Reduced \\\hline
                          \(X\)                       & No          & No        & Yes     \\
                          \(Y\)                       & No          & Yes       & Yes     \\
                          \(Z\)                       & Yes         & Yes       & Yes     \\
                          \(Y\times_{\spec\Z}\F_{2}\) & Yes         & Yes       & No
                      \end{tabular}
                      \caption{A table showing all results concisely.}
                  \end{figure}

                  The scheme \(X\) is not connected and therefore not
                  irreducible because it has non-trivial idempotents: \((0,1)\)
                  and \((1,0)\). It is reduced because \(\spec A\) itself is an
                  affine cover with no nilpotent elements.

                  The ring \(B\) has no non-trivial idempotents: all elements of
                  this ring have linear representatives \(aX+b\). Suppose
                  one of these is idempotent, then
                  \begin{align*}
                      aX+b & =(aX+b)^{2}                      \\
                           & =a^{2}X+2abX+b^{2}               \\
                           & =2abX+b^{2}-a^{2}\mod (X^{2}+1).
                  \end{align*}
                  In particular this would mean that
                  \(2a(b-1)X+a^{2}-b^{2}\in(X^{2}+1)\ideal\Z[x]\). As this is
                  clearly impossible for \(a,b\in\Z\) different from \(a,b=0,1\)
                  or \(a,b=0,0\) we conclude \(Y\) is connected.

                  The scheme \(Y\) is however not irreducible: there is no
                  generic point the closure of which is the entire ring. The
                  prime ideals of \(B\) are given by
                  \((X-1),(X+1),(X-1,p),(X+1,p)\) for arbitrary primes \(p\)
                  (note that \((X+1,X-1)=(X+1,2)\)). The lattice produced by
                  these primes does not have a minimal element (there is no
                  prime contained in both \((X+1)\) and \((X-1)\)) and therefore
                  \(Y\) is not irreducible. This scheme is once again reduced
                  because \(\Gamma(Z,O_{Y})\) does not have nilpotents and \(Y\)
                  is an affine cover of \(Y\).

                  The scheme \(Z\) is irreducible because
                  \(\Z/(X^{2}+1)\cong\Z[i]\) which is a domain and therefore
                  \((0)\) is a prime ideal contained in all others. It is
                  therefore the generic point of the closed set \(Z\). This
                  means it is also connected. This scheme is reduced because
                  it is affine and the open \(Z\) has as associated ring
                  \(O_{Z}(Z)=C\) which has no nilpotent elements.

                  The scheme \(Y\times_{\spec Z}Y\) is the spectrum of
                  \(B\otimes_{\Z}\F_{2}\) which is isomorphic to
                  \begin{align*}
                      B\otimes_{\Z}\F_{2} & \cong\Z[X]/(X^{2}-1)\otimes\Z/2\Z \\
                                          & \cong(\Z[X]/(X^{2}-1))/(2)        \\
                                          & \cong\F_{2}[X]/(X^{2}-1)          \\
                                          & \cong \F_{2}[X]/(X+1)^{2}         \\
                                          & \cong \F_{2}[X]/(X^{2}).
                  \end{align*}
                  This is the ring of dual numbers, which has a unique prime
                  ideal so this scheme has only one point and is therefore
                  irreducible and connected. This affine scheme is not reduced
                  because the ring of dual numbers has a nilpotent element
                  \(\varepsilon\).
              \end{solution}

        \item Using (1), show that \(A\), \(B\) and \(C\) are not mutually
              isomorphic rings.

              \begin{proof}
                  Isomorphic rings must have isomorphic spectra. Since
                  irreducibility, connectedness and being reduced are preserved
                  under isomorphism, this means that none of these rings can be
                  isomorphic to each other because no two spectra agree on all
                  three of these properties.
              \end{proof}
    \end{enumerate}
\end{question}

\begin{question}
    Let \(X=\proj_{\Z}^{1}\) with open affine sets \(U_{0}=\spec\Z[X_{01}]\)
    and \(U_{1}=\spec\Z[X_{10}]\).

    \begin{enumerate}[(i)]
        \item Show that the unique \(\proj_{\Z}^{1}\to\spec\Z\) is not an
              isomorphism.

              \begin{proof}
                  We show that \(\proj_{\Z}^{1}\) is not the terminal object. We
                  know that \(U_{0}\cong U_{1}\cong\spec\Z[X]\) as schemes so
                  there are two scheme morphisms
                  \(\spec\Z[X]\to\proj_{\Z}^{1}\): both onto \(U_{0}\) and
                  \(U_{1}\). Therefore \(\proj_{\Z}^{1}\) cannot be terminal
                  so it is not isomorphic to \(\spec\Z\).
              \end{proof}

        \item Show that \(\sheaf_{X}(X)=\Z\).

              \begin{proof}
                  As stated in the hint there is an exact sequence
                  \[
                      0\to O_{X}(X)\to O_{X}(U_{0})\times O_{X}(U_{1})\to O_{X}(U_{0}\cap U_{1})
                  \]
                  where filling in the known rings we get
                  \[
                      0\to O_{X}(X)\to \Z[t]\times \Z[t]\to \Z[t,t^{-1}]
                  \]
                  with \(O_{X}(X)\to \Z[t]\times \Z[t]\) given by the
                  restriction maps and \(\Z[t]\times \Z[t]\to \Z[t,t^{-1}]\)
                  given by the map \((f,g)\mapsto f(t)-g(\nicefrac{1}{t})\).

                  It is clear that no non-constant polynomials \(f,g\) have the
                  property that \(f(t)-g(\nicefrac{1}{t})=0\), so the kernel of
                  this map is exactly the diagonal of \(\Z\times\Z\) which is
                  isomorphic to \(\Z\). By exactness this means that
                  \(O_{X}(X)\cong\Z\) as well.
              \end{proof}

        \item Show that \(\proj_{\Z}^{1}\) is not affine.

              \begin{proof}
                  We show it is not affine by contradiction. Suppose it were
                  affine, then its image under the equivalence
                  \(\aff\simeq\cring^{op}\) would be the ring \(\Z\). This means
                  it is would be isomorphic to \(\spec\Z\). We know this is not
                  the case so this scheme is not affine.
              \end{proof}
    \end{enumerate}
\end{question}
\end{document}