\documentclass{article}

\usepackage[utf8]{inputenc}
\usepackage{enumerate}
\usepackage{amsthm, amssymb, mathtools, amsmath, bbm, mathrsfs, stmaryrd, xcolor}
\usepackage{nicefrac}
\usepackage[margin=1in]{geometry}
\usepackage[parfill]{parskip}
\usepackage[hidelinks]{hyperref}
\usepackage{float}
\usepackage{cleveref}
\usepackage{svg}
\usepackage{tikz-cd}

\usepackage{quiver}

\renewcommand{\qedsymbol}{\raisebox{-0.5cm}{\includesvg[width=0.5cm]{../../qedboy.svg}}}


\newcommand{\N}{\mathbb{N}}
\newcommand{\Z}{\mathbb{Z}}
\newcommand{\NZ}{\mathbb{N}_{0}}
\newcommand{\Q}{\mathbb{Q}}
\newcommand{\R}{\mathbb{R}}
\newcommand{\C}{\mathbb{C}}
\newcommand{\A}{\mathbb{A}}
\newcommand{\proj}{\mathbb{P}}
\newcommand{\sheaf}{\mathcal{O}}
\newcommand{\FF}{\mathcal{F}}
\newcommand{\G}{\mathcal{G}}

\newcommand{\zproj}{Z_{\textnormal{proj}}}

\newcommand{\maxid}{\mathfrak{m}}
\newcommand{\primeid}{\mathfrak{p}}
\newcommand{\primeidd}{\mathfrak{q}}

\newcommand{\F}{\mathbb{F}}
\newcommand{\incl}{\imath}

\newcommand{\tuple}[2]{\left\langle#1\colon #2\right\rangle}

\DeclareMathOperator{\order}{order}
\DeclareMathOperator{\Id}{Id}
\DeclareMathOperator{\im}{im}
\DeclareMathOperator{\ggd}{ggd}
\DeclareMathOperator{\kgv}{kgv}
\DeclareMathOperator{\degree}{gr}
\DeclareMathOperator{\coker}{coker}
\DeclareMathOperator{\matrices}{Mat}

\DeclareMathOperator{\gl}{GL}

\DeclareMathOperator{\Aut}{Aut}
\DeclareMathOperator{\Hom}{Hom}
\DeclareMathOperator{\End}{End}
\DeclareMathOperator{\colim}{colim}
\newcommand{\isom}{\overset{\sim}{\longrightarrow}}

\newcommand{\schemes}{{\bf Sch}}
\newcommand{\aff}{{\bf Aff}}
\newcommand{\Grp}{{\bf Grp}}
\newcommand{\Ab}{{\bf Ab}}
\newcommand{\cring}{{\bf CRing}}
\DeclareMathOperator{\modules}{{\bf Mod}}
\newcommand{\catset}{{\bf Set}}
\newcommand{\cat}{\mathcal{C}}
\newcommand{\chains}{{\bf Ch}}
\newcommand{\homot}{{\bf Ho}}
\DeclareMathOperator{\objects}{Ob}
\newcommand{\gen}[1]{\left<#1\right>}
\DeclareMathOperator{\cone}{Cone}
\newcommand{\set}[1]{\left\{#1\right\}}
\newcommand{\setwith}[2]{\left\{#1:#2\right\}}
\DeclareMathOperator{\Ext}{Ext}
\DeclareMathOperator{\Nil}{Nil}
\DeclareMathOperator{\idem}{Idem}
\DeclareMathOperator{\rad}{Rad}
\DeclareMathOperator{\divisor}{div}
\DeclareMathOperator{\Pic}{Pic}
\DeclareMathOperator{\spec}{Spec}
\newcommand{\ideal}{\triangleleft}

\newenvironment{solution}{\begin{proof}[Solution]\renewcommand\qedsymbol{}}{\end{proof}}

\newtheorem{lemma}{Lemma}
\newtheorem{proposition}{Proposition}

\theoremstyle{definition}
\newtheorem{question}{Exercise}
\newtheorem{definition}{Definition}

\title{Homework Algebraic Geometry 2}
\author{Jonas van der Schaaf}
\date{}

\begin{document}
\maketitle

\begin{question}
    Let \(X\) and \(K\) be schemes and \(f,g:K\to X\) be morphisms. Assume that
    for all \(x\in K\) we have that \(f(x)\equiv g(x)\).

    \begin{enumerate}[(i)]
        \item Show that if \(K\) is reduced, then \(f=g\).

              \begin{proof}
                  We first show the affine case and then reduce to the affine
                  case for non-affine schemes.

                  \begin{lemma}
                      The statement is true for affine schemes \(K,X\).

                      \begin{proof}
                          Let \(R,S\) be rings such that \(R\) is nilpotent free
                          \(K=\spec R,X=\spec S\), let \(f,g:\spec R\to\spec S\)
                          be as described above, and let
                          \(f^{\sharp},g^{\sharp}:S\to R\) be the corresponding
                          maps of rings. Let \(s\in S\) be any element and
                          assume \(f(s)\neq g(x)\), then because
                          \(\bigcap_{\primeid\ideal R\\
                              \text{\(\primeid\) prime}}\primeid=\Nil(R)=(0)\) we know that
                          there is a prime ideal \(\primeid\ideal R\) such that
                          \(f(s)-g(s)\notin\primeid\).

                          Taking
                          \(\primeidd=f^{-1}[\primeid]=g^{-1}[\primeid]\ideal
                          S\) we know that the corresponding maps
                          \(g^{\sharp}_{\primeid}:S_{\primeidd}/\primeidd
                          S_{\primeidd}\to R_{\primeid}/\primeid R_{\primeid}\)
                          and \(f^{\sharp}_{\primeid}:S_{\primeidd}/\primeidd
                          S_{\primeidd}\to R_{\primeid}/\primeid R_{\primeid}\)
                          are equal. In particular this means that for
                          \(\tilde{f},\tilde{g}:S\to R\) the corresponding ring
                          morphisms for \(f,g\)
                          \begin{align*}
                              0 & =f^{\sharp}_{\primeid}(s)-g^{\sharp}_{\primeid}(s)                   \\
                                & =\overline{\frac{\tilde{f}(s)}{1}}-\overline{\frac{\tilde{g}(s)}{1}} \\
                                & =\overline{\frac{\tilde{f}(s)-\tilde{g}(s)}{1}}                      \\
                                & \in R_{\primeid}/\primeid R_{\primeid}.
                          \end{align*}
                          This means that either
                          \(\tilde{f}(s)-\tilde{g}(s)\in\ker(\imath:R\to
                          R_{\primeid})\) or
                          \(\frac{\tilde{f}(s)-\tilde{g}(s)}{1}\in\primeid
                          R_{\primeid}\) where \(\imath\) is the natural map
                          \(R\to R_{\primeid}\). We know that
                          \[
                              \ker\imath=\setwith{a\in R\setminus\primeid}{\exists x\in R\setminus\primeid:ax=0}.
                          \]
                          Because \(\tilde{f}(s)-\tilde{g}(s)\notin\primeid\)
                          this means that it cannot be in this kernel either.
                          This would mean that
                          \(\tilde{f}(s)-\tilde{g}(s)\in\primeid R_{\primeid}\)
                          however it is a unit and thus cannot be in the kernel
                          of a morphism to a field. This is a contradiction and
                          therefore \(\tilde{f}(s)-\tilde{g}(s)=0\) for all
                          \(s\in S\) so \(\tilde{f}=\tilde{g}\) and therefore
                          also \(f=g\).
                      \end{proof}
                  \end{lemma}

                  Now for the non-affine case. We show that for each
                  \(\primeid\in K\) there is an open neighbourhood such that
                  \(f\) and \(g\) agree on this neighbourhood. This immediately
                  gives that \(f\) and \(g\) agree on the entirety of \(K\).

                  Take \(\primeid\in K\). Then \(f(\primeid)\) has an affine
                  open neighbourhood \(U\subseteq X\) which naturally has the
                  structure of an open subscheme of \(X\). The inverse image
                  \(V=f^{-1}[X]\) is then an open subscheme of \(K\) with
                  \(\primeid\in V\) and restrictions \(f|_{V},g|_{V}:V\to U\).
                  Now there is also an affine neighbourhood \(W\subseteq V\) of
                  \(\primeid\). Taking \(f|_{W},g|_{W}:W\to U\) now two
                  morphisms between affine schemes, we see immediately that
                  \(f(\primeid)\equiv g(\primeid)\) and \(W\) is reduced because
                  it is an open subscheme of a reduced scheme. The lemma then
                  gives that \(f|_{W}=g|_{W}\).

                  This means that \(f,g\) agree everywhere and therefore that
                  \(f=g\).
              \end{proof}

        \item Give an example where \(f\neq g\).

              \begin{solution}
                  Consider the two rings \(S=\Q[\varepsilon]\) and
                  \(R=\Q[\varepsilon]\). We have two maps \(f,g:S\to R\) given
                  by \(f(\varepsilon)=0\) and \(g(\varepsilon)=\varepsilon\).

                  The map on stalks \(f^{\sharp}_{x},g^{\sharp}_{x}:\Q\to\Q\)
                  for \(x\) the unique prime ideal agree because there is only
                  one map \(\Q\to\Q\) but \(f\neq g\).
              \end{solution}
    \end{enumerate}
\end{question}

\begin{question}
    Let \(X,Y,Z\) be separated schemes. Assume that \(f:X\to Y\) is surjective,
    \(g:Y\to Z\) is of finite type, and that \(g\circ f\) is proper. Show that
    \(g\) is proper.

    \begin{proof}
        A map is proper if it is separated, of finite type, and universally
        closed. We show \(g\) has all three properties.

        For separatedness, we have a commutative diagram
        \[
            \begin{tikzcd}[ampersand replacement=\&]
                Y \&\& Z \\
                \\
                \&\& \spec\Z
                \arrow["g", from=1-1, to=1-3]
                \arrow["{!_{Z}}", from=1-3, to=3-3]
                \arrow["{!_{Y}}"{description}, from=1-1, to=3-3]
            \end{tikzcd}
        \]
        Because \(Y\) and \(Z\) are separated we know that \(!_{Z}\circ g\) is
        separated and therefore that \(g\) is separated as well.

        We know \(g\) is of finite type because that is an assumption we made.

        Now for universal closedness, we have a diagram of two pullbacks
        \[
            \begin{tikzcd}[ampersand replacement=\&]
                {X\times_{Y}(Y\times_{Z}T)} \&\& {Y\times_{Z}T} \&\& T \\
                \\
                X \&\& Y \&\& Z
                \arrow["g", from=3-3, to=3-5]
                \arrow["f", from=3-1, to=3-3]
                \arrow["h", from=1-5, to=3-5]
                \arrow["h^{*}g", from=1-3, to=1-5]
                \arrow["\pi_{Y}^{*}f", from=1-1, to=1-3]
                \arrow[from=1-1, to=3-1]
                \arrow["\pi_{Y}", from=1-3, to=3-3]
                \arrow["\lrcorner"{anchor=center, pos=0}, draw=none, from=1-3, to=3-5]
                \arrow["\lrcorner"{anchor=center, pos=0}, draw=none, from=1-1, to=3-3]
            \end{tikzcd}
        \]
        where the outer square is also a pullback by the pullback lemma. Because
        the map \(g\circ f\) is universally closed the top row of the pullback
        \(h^{*}g\circ\pi_{Y}^{*}f\) is a closed map. Being surjective is also
        closed under base change so \(\pi_{Y}^{*}f\) is a surjective map. This
        is sufficient for \(h^{*}g\) being a closed map:

        Take \(U\subseteq Y\times_{\Z}T\) a closed subset. Then
        \(V=(\pi_{Y}^{*}f)^{-1}[U]\) is a closed subset of
        \(X\times_{Y}(Y\times_{Z}T)\) and \(h^{*}g\circ \pi_{Y}^{*}f[V]\subseteq
        T\) is closed. Because \(\pi_{Y}^{*}f\) is surjective we have that
        \(\pi_{Y}^{*}f[V]=U\) so \(h^{*}g[U]=h^{*}g\circ \pi_{Y}^{*}f[V]\) is
        closed. This shows that any base change of \(g\) is closed, so \(g\) is
        universally closed.
    \end{proof}
\end{question}

\begin{question}
    I don't want to copy this introduction.

    \begin{enumerate}[(i)]
        \item Let \(a\in X(k)\). We define \(T_{X}(a)=p^{-1}(a)\). Exhibit a
              natural bijection
              \[
                  T_{X}(A)\to\setwith{v\in k^{n}}{\nabla f_{i}\cdot v=0}.
              \]

              \begin{proof}
                  A morphism \(a'\in A\to k[\varepsilon]\) is uniquely
                  determined by \(z=(a'(x_{j}))_{j}\in k[\varepsilon]^{n}\) such
                  that \(f_{i}(z)=0\) for all \(1\leq i\leq m\) where
                  \(z_{j}=b_{j}+c_{j}\varepsilon\).

                  Now if \(a':A\to k[\varepsilon]\) is a ring morphism then for
                  \(g:k[\varepsilon]\to k\) the composition \(g\circ a'\) is
                  also a morphism and \(z=(g\circ a'(x_{j}))_{j}\) has
                  \(f_{i}(b)=0\) for all \(b\). This means that \(a'\) must have
                  the particular property that for
                  \(z=(a'(x_{j}))_{j}=(b_{j}+c_{j}\varepsilon)\) both
                  \(f_{i}(b_{1},\ldots,b_{n})=0\) for all \(i\) and
                  \(f_{i}(b_{1}+c_{1}\varepsilon,\ldots,b_{n}+c_{n}\varepsilon)=0\).

                  This means that \(p^{-1}(a)\) for \(a\in X(k)\) corresponds
                  uniquely to the set of tuples
                  \[
                      \setwith{z=(b_{j}+c_{j}\varepsilon)_{j}\in k[\varepsilon]^{n}}{\forall 1\leq i\leq m:f_{i}(z)=0}
                  \]
                  such that \((b_{j})_{j}\in k^{n}\) corresponds to \(a\)
                  through the natural bijection.

                  Now we consider what \(f_{i}(z)\) looks like for such a tuple
                  \(z\). For a monomial \(x_{1}^{l_{1}}\cdots x_{n}^{l_{n}}\)
                  filling in \((b_{j}+c_{j}\varepsilon)_{j}\) gives by Newtons
                  binomium and the fact that \(\varepsilon^{2}=0\)
                  \begin{align*}
                      \prod_{j}(b_{j}+c_{j}\varepsilon)^{l_{j}} & =\prod_{j}\sum_{t=0}^{l_{i}}\binom{l_{i}}{t}b_{j}^{t}c_{j}^{l_{i}-t}\varepsilon^{l_{i}-t} \\
                                                                & =\prod_{j}\left(b_{j}^{l_{i}}+l_{j}b_{j}^{l_{i}-1}c_{j}\varepsilon\right)                 \\
                                                                & =\prod_{j}b^{l_{i}}+\sum_{j}l_{j}b_{j}^{l_{i}-1}c_{j}\varepsilon.
                  \end{align*}
                  By linearity of polynomial evaluation\footnote{Linearity in
                      the polynomial, not in the argument obviously.} we see that
                  defining for \(z=(b_{j}+c_{j}\varepsilon)_{j}\),
                  \(b=(b_{j})_{j}\in k^{n}\), \(c=(c_{j})_{j}\in k^{n}\) and
                  \(f\) an arbitrary polynomial
                  \[
                      f(z)=f(b)+(\nabla f(b)\cdot c)\varepsilon.
                  \]

                  Combining all of this if \(a'\in T_{X}(a)\) for some \(a\in
                  X(k)\), let \(z=(b_{j}+c_{j})_{j}\in k[\varepsilon]^{n}\) be
                  the unique tuple corresponding to \(a'\) then \(f_{i}(z)=0\)
                  and also \(f_{i}(b)=0\) for all \(i\). This means that each
                  \(a'\in T_{X}(a)\) corresponds uniquely to a vector \(c\in
                  k^{n}\) such that
                  \[
                      0=f(z)=(\nabla(f)(b)\cdot c)\varepsilon
                  \]
                  where \(b\in k^{n}\) corresponds to \(a\). This above equality
                  holds iff \((\nabla f(b)\cdot c)=0\) showing that each \(a'\in
                  T_{X}(a)\) corresponds uniquely to such a vector \(c\).
              \end{proof}

        \item Exhibit a natural bijection
              \[
                  T_{O_{n}}(I_{n})\to\setwith{m\in\matrices(n\times n,k)}{m^{\top}+m=0}.
              \]

              \begin{solution}
                  The orthogonal matrices are given by the equations
                  \[
                      f_{ij}=\sum_{k=1}^{n}m_{ik}m_{jk}-\delta_{ij}=0
                  \]
                  where \(\delta_{ij}\) is the Kronecker \(\delta\). Taking the
                  the partial derivative of \(f_{ij}\) with respect to
                  \(m_{pq}\) is given by
                  \[
                      \frac{\partial f_{ij}}{\partial m_{pq}}=\begin{cases}
                          m_{jq}  & p=i\wedge p\neq j, \\
                          m_{iq}  & p=j\wedge p\neq i, \\
                          2m_{iq} & p=i=j,             \\
                          0       & \text{else}.
                      \end{cases}
                  \]
                  We are then looking for \(c\in k^{n\times n}\) such that
                  \[
                      \frac{\partial f_{ij}}{\partial m_{pq}}(I_{n})_{kl}c_{kl}=0
                  \]
                  for all \(i,j\). Writing this out gives
                  \begin{align*}
                      0 & =\sum_{p,q=1}^{n}\frac{\partial f_{ij}}{\partial m_{pq}}(I_{n})c_{pq} \\
                        & =\begin{cases}
                          \sum_{q}\delta_{jq}c_{iq}+\delta_{iq}c_{jq} & i\neq j, \\
                          \sum_{q}2\delta_{iq}c_{ii}                  & i=j.
                      \end{cases}                                            \\
                        & =c_{ij}+c_{ji}                                                        \\
                        & =(c+c^{\top})_{ij}
                  \end{align*}
                  This means that \(T_{O_{n}}(I_{n})\) corresponds to the
                  matrices \(c\) with \(c+c^{T}=0\).
              \end{solution}
    \end{enumerate}
\end{question}

\begin{question}
    Let \(f,g:Y\to X\) be morphisms of schemes. When \(T\) is a scheme and
    \(y\in Y(T)\) we write \(f(y)\) for the element \(f\circ y\) of \(X(T)\). We
    have a natural functor \(E(f,g):\schemes\to\catset\) given by sending a
    scheme \(T\) to the set \(\setwith{y\in Y(T)}{f(y)=g(y)}\).

    \begin{enumerate}[(i)]
        \item Show that \(E(f,g)\) is naturally represented by a scheme, which
              naturally fits in a cartesian diagram
              \[
                  \begin{tikzcd}[ampersand replacement=\&]
                      {E(f,g)} \&\& Y \\
                      \\
                      X \&\& {X\times X}
                      \arrow[from=1-1, to=1-3]
                      \arrow["{(f,g)}", from=1-3, to=3-3]
                      \arrow["\Delta"', from=3-1, to=3-3]
                      \arrow["\pi_{X}", from=1-1, to=3-1]
                      \arrow["\lrcorner"{anchor=center, pos=0}, draw=none, from=1-1, to=3-3]
                  \end{tikzcd}
              \]

              \begin{proof}
                  By the universal property of the pullback we have the
                  following sequence of natural isomorphisms. Take \(Z\) an object
                  \begin{align*}
                      \Hom(Z,Y\times_{X\times X}X) & =\set{a:Z\to Y\times_{X\times X}X}                                \\
                                                   & \cong\setwith{(a:Z\to X,b:Z\to Y)}{(\Id,\Id)\circ a=(f,g)\circ b} \\
                                                   & =\setwith{(a:Z\to X,b:Z\to Y)}{(a,a)=(f\circ b,g\circ b)}         \\
                                                   & \cong\setwith{b:Z\to Y}{f\circ b=g\circ b}                        \\
                                                   & =E(f,g).
                  \end{align*}
                  Therefore pullback represents the equaliser \(E(f,g)\).
              \end{proof}

        \item Show that \(f=g\) iff the natural map \(E(f,g)\to Y\) is an
              isomorphism.

              \begin{proof}
                  If \(\pi_{Y}:E(f,g)\to Y\) is an isomorphism, it is in
                  particular an epimorphism. We know that
                  \begin{align*}
                      (f\circ\pi_{Y},g\circ\pi_{Y}) & =(f,g)\circ\pi_{Y}     \\
                                                    & =(\Id,\Id)\circ\pi_{X} \\
                                                    & =(\pi_{X},\pi_{X})     \\
                  \end{align*}
                  Therefore \(f\circ\pi_{Y}=g\circ\pi_{Y}\) and by \(\pi_{Y}\)
                  being epi this gives \(f=g\).

                  If \(f=g\), then the following diagram commutes:
                  \[
                      \begin{tikzcd}[ampersand replacement=\&]
                          Y \\
                          \& {E(f,g)} \&\& Y \\
                          \\
                          \& X \&\& {X\times X}
                          \arrow["{\pi_{Y}}", from=2-2, to=2-4]
                          \arrow["{(f,g)}", from=2-4, to=4-4]
                          \arrow["\Delta"', from=4-2, to=4-4]
                          \arrow["{\pi_{X}}"', from=2-2, to=4-2]
                          \arrow["\lrcorner"{anchor=center, pos=0}, draw=none, from=2-2, to=4-4]
                          \arrow["\Id", curve={height=-12pt}, from=1-1, to=2-4]
                          \arrow["f", curve={height=12pt}, from=1-1, to=4-2]
                          \arrow["\imath_{Y}", dashed, from=1-1, to=2-2]
                      \end{tikzcd}
                  \]
                  This means that the map \(\pi_{Y}\) is a split epimorphism
                  with right inverse \(\imath_{Y}\). The map \(\Delta\) is a
                  monomorphism and therefore \(\pi_{Y}\) is also mono. This
                  means that
                  \begin{align*}
                      \pi_{Y}\circ\imath_{Y}\circ\pi_{Y} & =\Id\circ\pi_{Y}  \\
                                                         & =\pi_{Y}\circ\Id.
                  \end{align*}
                  Because \(\pi_{Y}\) is mono this means that
                  \(\imath_{Y}\circ\pi_{Y}=\Id\) as well so \(\pi_{Y}\) has a
                  left and right inverse and is therefore an isomorphism.
              \end{proof}

        \item Let \(X\) be a separated scheme, and let \(f:R_{1}\to R_{2}\) be
              an injective ring homomorphism. Show that the induced map
              \(\tilde{f}^{*}:X(R_{1})\to X(R_{2})\) is injective.

              \begin{proof}
                  Let \(g,h:\spec R_{1}\to X\) be two scheme morphisms such that
                  \(\tilde{f}^{*}(g)=\tilde{f}^{*}(h)\) and consider the
                  following commutative diagram:
                  \[
                      \begin{tikzcd}[ampersand replacement=\&]
                          {\spec R_{2}} \\
                          \& {E(f,g)} \&\& {\spec R_{1}} \\
                          \\
                          \& X \&\& {X\times X}
                          \arrow["{\pi_{Y}}"', from=2-2, to=2-4]
                          \arrow["{(g,h)}"', from=2-4, to=4-4]
                          \arrow["\Delta", from=4-2, to=4-4]
                          \arrow["{\pi_{X}}", from=2-2, to=4-2]
                          \arrow["{\tilde{f}}", curve={height=-12pt}, from=1-1, to=2-4]
                          \arrow["{g\circ\tilde{f}}"', curve={height=12pt}, from=1-1, to=4-2]
                          \arrow[dashed, from=1-1, to=2-2]
                          \arrow["\lrcorner"{anchor=center, pos=0}, draw=none, from=2-2, to=4-4]
                      \end{tikzcd}
                  \]
                  Because \(X\) is separated so \(\Delta\) is a closed immersion
                  and therefore \(\pi_{Y}\) is as well because being a closed
                  immersion is stable under base change. Because \(\spec R_{1}\)
                  is affine and \(\pi_{Y}\) is a closed immersion this means
                  that \(E(f,g)\cong \spec R_{1}/I\) for some ideal \(I\ideal
                  R_{1}\) and \(\pi_{Y}\) is the transpose of the projection map
                  \(R_{1}\to R_{1}/I\).

                  All schemes in the top triangle are affine so we can transpose
                  to get the diagram
                  \[
                      \begin{tikzcd}[ampersand replacement=\&]
                          R_{2}\\
                          \\
                          R_{1}/I \&\& R_{1}
                          \arrow["f"', from=3-3, to=1-1]
                          \arrow["\widetilde{\pi_{Y}}", from=3-3, to=3-1]
                          \arrow[from=3-1, to=1-1]
                      \end{tikzcd}
                  \]
                  Because \(f\) is injective, \(\widetilde{\pi_{Y}}\) must be
                  injective. This means that \(I=\ker(\widetilde{\pi_{Y}})=(0)\)
                  so \(\widetilde{\pi_{Y}}\) is an isomorphism and therefore so
                  is \(\pi_{Y}\) demonstrating that \(g=h\) by part b and thus
                  that \(\tilde{f}^{*}:X(R_{1})\to X(R_{2})\) is injective.
              \end{proof}
    \end{enumerate}
\end{question}
\end{document}