\documentclass{article}

\usepackage[utf8]{inputenc}
\usepackage[shortlabels]{enumitem}
\usepackage{amsthm, amssymb, mathtools, amsmath, bbm, mathrsfs, stmaryrd, xcolor}
\usepackage{nicefrac}
\usepackage[margin=1in]{geometry}
\usepackage[parfill]{parskip}
\usepackage[hidelinks]{hyperref}
\usepackage{float}
\usepackage{cleveref}
\usepackage{svg}
\usepackage{tikz-cd}

\usepackage{quiver}

\renewcommand{\qedsymbol}{\raisebox{-0.5cm}{\includesvg[width=0.5cm]{../../qedboy.svg}}}


\newcommand{\N}{\mathbb{N}}
\newcommand{\Z}{\mathbb{Z}}
\newcommand{\NZ}{\mathbb{N}_{0}}
\newcommand{\Q}{\mathbb{Q}}
\newcommand{\R}{\mathbb{R}}
\newcommand{\C}{\mathbb{C}}
\newcommand{\A}{\mathbb{A}}
\newcommand{\proj}{\mathbb{P}}
\newcommand{\sheaf}{\mathcal{O}}
\newcommand{\FF}{\mathcal{F}}
\newcommand{\G}{\mathcal{G}}

\newcommand{\zproj}{Z_{\textnormal{proj}}}

\newcommand{\maxid}{\mathfrak{m}}
\newcommand{\primeid}{\mathfrak{p}}
\newcommand{\primeidd}{\mathfrak{q}}

\newcommand{\F}{\mathbb{F}}
\newcommand{\incl}{\imath}

\newcommand{\tuple}[2]{\left\langle#1\colon #2\right\rangle}

\DeclareMathOperator{\order}{order}
\DeclareMathOperator{\Id}{Id}
\DeclareMathOperator{\im}{im}
\DeclareMathOperator{\ggd}{ggd}
\DeclareMathOperator{\kgv}{kgv}
\DeclareMathOperator{\degree}{gr}
\DeclareMathOperator{\coker}{coker}
\DeclareMathOperator{\matrices}{Mat}

\DeclareMathOperator{\gl}{GL}

\DeclareMathOperator{\Aut}{Aut}
\DeclareMathOperator{\Hom}{Hom}
\DeclareMathOperator{\End}{End}
\DeclareMathOperator{\colim}{colim}
\newcommand{\isom}{\overset{\sim}{\longrightarrow}}

\newcommand{\schemes}{{\bf Sch}}
\newcommand{\aff}{{\bf Aff}}
\newcommand{\Grp}{{\bf Grp}}
\newcommand{\Ab}{{\bf Ab}}
\newcommand{\cring}{{\bf CRing}}
\DeclareMathOperator{\modules}{{\bf Mod}}
\newcommand{\catset}{{\bf Set}}
\newcommand{\cat}{\mathcal{C}}
\newcommand{\chains}{{\bf Ch}}
\newcommand{\homot}{{\bf Ho}}
\DeclareMathOperator{\objects}{Ob}
\newcommand{\gen}[1]{\left<#1\right>}
\DeclareMathOperator{\cone}{Cone}
\newcommand{\set}[1]{\left\{#1\right\}}
\newcommand{\setwith}[2]{\left\{#1:#2\right\}}
\DeclareMathOperator{\Ext}{Ext}
\DeclareMathOperator{\Nil}{Nil}
\DeclareMathOperator{\idem}{Idem}
\DeclareMathOperator{\rad}{Rad}
\DeclareMathOperator{\divisor}{div}
\DeclareMathOperator{\Pic}{Pic}
\DeclareMathOperator{\spec}{Spec}
\DeclareMathOperator{\supp}{Supp}
\newcommand{\ideal}{\triangleleft}

\newenvironment{solution}{\begin{proof}[Solution]\renewcommand\qedsymbol{}}{\end{proof}}

\newtheorem{lemma}{Lemma}
\newtheorem{proposition}{Proposition}

\theoremstyle{definition}
\newtheorem{question}{Exercise}
\newtheorem{definition}{Definition}

\title{Homework Algebraic Geometry 2}
\author{Jonas van der Schaaf}
\date{}

\begin{document}
\maketitle

\begin{question}
    Let \(k\) be a field, and let \(X,Y\) be schemes of finite type over \(k\).
    \begin{enumerate}[(a)]
        \item Show that \(X\) is Noetherian.

              \begin{proof}
                  We show that \(X\) has an open affine cover of Noetherian
                  affine spectra. Let \(\set{U_{i}}_{i}\) be an affine open
                  cover of \(X\). Then \(R_{i}=\Gamma(U_{i},\sheaf_{X})\) is a
                  finitely generated ring over the Noetherian ring \(k\). By
                  Hilberts basis theorem this means \(R_{i}\) is Noetherian for
                  all \(i\). This means that \(X\) is Noetherian.
              \end{proof}
    \end{enumerate}
    Let \(f:X\to Y\) be a morphism of \(k\)-schemes.
    \begin{enumerate}[(a), resume]
        \item Show \(f\) is a morphism of finite type.

              \begin{proof}
                  Let \(U\subseteq Y\) be an open and let
                  \(R=\Gamma(U,\sheaf_{Y})\) and
                  \(S=\Gamma(f^{-1}U,\sheaf_{X})\). Then \(R,S\) are both
                  finitely generated rings over \(k\) and \(k\) is a subring of
                  \(S\). By this latter fact it is immediate that \(S\) is
                  finitely generated over \(R\) through the \(k\)-algebra
                  morphism \(f^{\sharp}_{U}:R\to S\) unless \(S\) is the trivial
                  ring \(0\). In this case \(R\) must be trivial as well because
                  there is no morphism from \(S\) to a non-trivial ring because
                  we have \(0=1\).
              \end{proof}
    \end{enumerate}
\end{question}

\begin{question}
    Let \(X\) be a scheme. Let \(\mathcal{F}\) be an \(\sheaf_{X}\)-module with
    the property that there exists an open cover \(\set{U_{i}}_{i}\) of \(X\)
    with affine subsets such that for all \(i\) there exists an isomorphism
    \(\mathcal{F}|_{U_{i}}\cong\tilde{M_{i}}\) of \(\sheaf_{X}\)-modules with
    \(M_{i}\) a finitely generated \(\Gamma(U_{i},\sheaf_{X}|_{U_{i}})\)-module.
    Let \(\supp\mathcal{F}=\setwith{x\in X}{\mathcal{F}_{x}\neq 0}\).

    \begin{enumerate}[(a)]
        \item Show that \(\supp\mathcal{F}\) is a closed subset of \(X\).

              \begin{proof}
                  Take \(x\notin\supp\mathcal{F}\) and let \(U_{i}\) be the
                  affine open neighbourhood of \(x\) given by the question with
                  \(U_{i}=\spec R\) and \(\primeid\ideal R\) the prime ideal
                  corresponding to \(x\). By doing some computations with limits
                  we get that
                  \begin{align*}
                      \mathcal{F}_{x} & =\colim_{W\ni x}\Gamma(V,\mathcal{F})                                        \\
                                      & =\colim_{U_{i}\supseteq V\ni x}\Gamma(V,\mathcal{F}|_{U_{i}})                \\
                                      & =\colim_{U_{i}\supseteq V\ni x}M_{i}\otimes_{R}\Gamma(V,\sheaf_{X}|_{U_{i}}) \\
                                      & =M_{i}\otimes_{R}\colim_{U_{i}\supseteq V\ni x}\Gamma(V,\sheaf_{X}|_{U_{i}}) \\
                                      & =M_{i}\otimes_{R}R_{\primeid}                                                \\
                                      & =(M_{i})_{\primeid},
                  \end{align*}
                  where \(V\) denotes an open affine. We can commute the tensor
                  product and the colimit because the tensor product is a left
                  adjoint.

                  We have \((M_{i})_{\primeid}=\mathcal{F}_{x}=0\) and \(M_{i}\)
                  is a finitely generated \(R\)-module generated by
                  \(x_{1},\ldots,x_{n}\). This means that \((M_{i})_{\primeid}\)
                  is a finitely generated \(R_{\primeid}\)-module generated by
                  the \(x_{i}\). This means that each \(x_{i}\) is annihilated
                  by some \(r_{i}\in R\setminus\primeid\) hence
                  \(\prod_{i}r_{i}\notin\primeid\) annihilates each \(x_{i}\).
                  This means that taking \(s=\prod_{i}s_{i}\), the principal
                  open \(V_{s}\subseteq U\) must also have the property that
                  \(\Gamma(V_{s},\mathcal{F})=(0)\).
              \end{proof}

        \item Let \(\mathcal{L}\) be an invertible sheaf on \(X\), and let
              \(s\in\Gamma(X,\mathcal{L})\) be a global section of
              \(\mathcal{L}\). Show that \(X_{s}\) is an open subset of \(X\).

              \begin{proof}
                  We have an open cover \(\set{U_{i}}_{i}\) of \(X\) such that
                  \(\mathcal{L}|_{U_{i}}\cong\sheaf_{X}|_{U_{i}}=\widetilde{\sheaf_{X}(U_{i})}\)
                  where the latter is clearly a finitely generated
                  \(\sheaf_{X}(U_{i})\) module. The submodule of \(\mathcal{L}\)
                  generated by \(s\) written as \(\sheaf_{X}\cdot s\) is given
                  by \(\widetilde{\sheaf_{X}\cdot s}\). By exactness of the
                  tilde this means that \(\mathcal{L}/(\sheaf_{X}\cdot s)\) is
                  locally given by
                  \((\widetilde{\sheaf_{X}(U_{i})/(\sheaf_{X}\cdot s)})(U_{i})\)
                  which is therefore locally a finitely generated
                  \(\sheaf_{X}|_{U_{i}}\)-module. This means that
                  \(\mathcal{L}/(\sheaf_{X}\cdot s)\) satisfies the properties
                  of \(\mathcal{F}\) of part a.

                  We show that \(X_{s}\) is the complement of the closed set
                  \(\supp\mathcal{L}/(\sheaf_{X}\cdot s)\) which is sufficient
                  for showing it is open.

                  We have \(x\in\supp\mathcal{L}/(\sheaf_{X}\cdot s)\) iff
                  \((\mathcal{L}/(\sheaf_{X}\cdot s))_{x}\neq0\). Take a
                  \(U_{i}\) as above containing \(x\) and let
                  \(\primeid\ideal\Gamma(U_{i},\sheaf_{X})\) be the prime ideal
                  corresponding to \(x\). Then
                  \begin{align*}
                      0 & =(\mathcal{L}/(\sheaf_{X}\cdot s))_{x}                                       \\
                        & =(\sheaf_{X}|_{U_{i}}/(\sheaf_{X}|_{U_{i}}\cdot s))_\primeid                 \\
                        & \cong(\sheaf_{X}|_{U_{i}})_{\primeid}/(\sheaf_{X}|_{U_{i}}\cdot s)_\primeid.
                  \end{align*}
                  This means that \((\sheaf_{X}|_{U_{i}}\cdot
                  s)_\primeid=(\sheaf_{X}|_{U_{i}})_{\primeid}\cdot s\) so \(s_{x}\)
                  generates the stalk.

                  Conversely if \(s_{x}\) generates the stalk then
                  \((\sheaf_{X}|_{U_{i}})_{\primeid}/(\sheaf_{X}|_{U_{i}}\cdot
                  s)_\primeid\cong 0\), we can follow the chain of isomorphisms
                  above to get that \((\mathcal{L}/(\sheaf_{X}\cdot
                  s))_{x}\cong0\).

                  This means that \(x\in\supp\mathcal{L}/(\sheaf_{X}\cdot s)\)
                  iff \(s_{x}\) does not generate \(\mathcal{L}_{x}\): i.e.
                  \(X_{s}\) is the complement of the closed set
                  \(\supp\mathcal{L}/(\sheaf_{X}\cdot s)\).
              \end{proof}

        \item Let \(\mathcal{L}=\sheaf_{\proj^{r}}(1)\). Show that
              \(\proj^{r}_{X_{i}}=U_{i}\).

              \begin{proof}
                  We know that \(M_{i}=\Gamma(\sheaf_{\proj^{r}}(1),U_{i})=R_{i}\cdot
                  X_{i}\) where
                  \[
                      R_{i}=\Gamma(U_{i},\sheaf_{\proj^{r}})=\Z\left[\frac{X_{0}}{X_{i}},\ldots,\frac{X_{n}}{X_{i}}\right].
                  \]
                  This module \(M_{i}\) is generated by \(X_{i}\) as
                  \(\sheaf_{X}(U_{i})\)-module. This means the localisation
                  \((M_{i})_{x}\) is generated by \(X_{i}\) as
                  \(O_{\proj^{r},x}\)-module for all \(x\in U_{i}\) because
                  generators are preserved under localisation. Therefore
                  \(U_{i}\subseteq X_{X_{i}}\) for all \(i\).

                  We now show the opposite inclusion. Suppose \(X_{i}\)
                  generates the stalk at \(x\) and \(x\in U_{k}\) and \(x\notin
                  U_{i}\). Because \(U_{i}\cap
                  U_{k}=D\left(\frac{X_{i}}{X_{k}}\right)\subseteq U_{k}=\spec
                  R_{i}\) this means that \(\frac{X_{i}}{X_{k}}\in\primeidd\)
                  where \(\primeidd\ideal R_{k}\) corresponds to \(X\) in the
                  affine open \(U_{k}\).

                  Because \(\maxid_{X,x}=\primeidd\cdot (R_{i})_{\primeidd}\)
                  this means \(X_{ik}\in\maxid_{X,x}\). This means that
                  \(X_{i}\in\maxid_{X,x}\cdot O_{X}(1)\subsetneq O_{X}(1)\) by
                  Nakayama's lemma and therefore \(X_{i}\) cannot generate the
                  entire module.
              \end{proof}
    \end{enumerate}
\end{question}

\begin{question}
    Let \(k\) be a field. Let \(S=k[X_{0},X_{1}]\), and
    let \(I\ideal S\) be the homogenous ideal \((X_{0}X_{1})\). Let
    \(M=S/I\), equipped with its natural structure of graded \(S\)-module.

    Let \(Z\) be the closed subscheme of \(\proj^{1}_{k}\) determined by the
    homogenous ideal \(I\). In particular we have a natural isomorphism
    \(\tilde{M}\to i_{*}Z\).
    \begin{enumerate}[(a)]
        \item Show that \(Z\) is reduced, and that the underlying closed set is
              a disjoint union of two copies of \(\spec(k)\).

              \begin{proof}
                  We find the support of \(\sheaf_{X}/\tilde{I}\) on the two
                  affine opens \(U_{0},U_{1}\). We first compute
                  \(\tilde{I}|U_{i}\):
                  \begin{align*}
                      \tilde{I}|_{U_{i}} & =\widetilde{\left((X_{0}X_{1})\left[\frac{1}{X_{i}}\right]\right)_{0}} \\
                                         & =\widetilde{\left(\frac{X_{i-1}}{X_{i}}\right)}.
                  \end{align*}
                  The support of the quotient \(\sheaf_{X}/\tilde{I}\) is then
                  given on \(U_{i}\) by all prime ideals \(\primeid\ideal
                  R_{i}\) such that the localisation
                  \[
                      \left(k\left[\frac{X_{1-i}}{X_{i}}\right]/\left(\frac{X_{i-1}}{X_{i}}\right)\right)_{\primeid}\neq 0
                  \]
                  Because \(\left(\frac{X_{i-1}}{X_{i}}\right)\) is a maximal
                  ideal this localisation is non-zero only at
                  \(\primeid=\left(\frac{X_{i-1}}{X_{i}}\right)\). This point is
                  the only point of \(U_{i}\) disjoint from \(U_{i-1}\), so the
                  support of \(\sheaf_{X}/\tilde{I}\) is exactly these two
                  points of \(U_{0}\) and \(U_{1}\) respectively.

                  The closed subscheme \(Z\) then has an open affine cover of
                  reduced rings: both points are isolated and the ring
                  associated to this singleton open must be the quotient ring
                  \(k[X_{i(i-1)}]/(X_{i(i-1)})\cong k\) which is reduced.

                  This is a sheaf on two components, so the global sections are
                  given by the product of the sections on the two components of
                  \(Z\) so the ring of global sections is the reduced ring
                  \(k\times k=k^{2}\). We will use this in the next part.
              \end{proof}

        \item Show that the natural sequence of spaces of global sections
              \[
                  0\to\Gamma(\proj^{1}_{k},\tilde{I})\to\Gamma(\proj^{1}_{k},\sheaf_{\proj^{1}_{k}})\to\Gamma(\proj^{1}_{k},\tilde{M})\to 0
              \]
              is not exact.

              \begin{proof}
                  We know that
                  \(\Gamma(\proj^{1}_{k},\sheaf_{\proj^{1}_{k}})=(k[X_{0},X_{1}])_{0}=k\)
                  and
                  \begin{align*}
                      \Gamma(\proj^{1}_{k},\tilde{M}) & \cong\Gamma(\proj^{1}_{k},i_{*}Z)  \\
                                                      & =\sheaf_{Z}(i^{-1}(\proj^{1}_{k})) \\
                                                      & =\sheaf_{Z}(Z)                     \\
                                                      & =k^{2}.
                  \end{align*}
                  There is no surjective \(k\to k^{2}\) so the sequence is not
                  exact.
              \end{proof}

        \item Calculate \(\dim_{k}M_{d}\) and
              \(\dim_{k}\Gamma(\proj^{1}_{k},\tilde{M}\otimes\sheaf(d))\).

              \begin{solution}
                  The homogenous elements of degree \(d\) of \(S\) are linear
                  combinations of \(X_{0}^{i}X_{1}^{j}\) where \(i+j=d\). Taking
                  the quotient by \(I=(X_{0}X_{1})\) we see the only homogenous
                  elements of \(M\) of degree \(d\) are linear combinations of
                  \(X_{0}^{d}\) and \(X_{1}^{d}\) which are different for
                  \(d>0\). Therefore \(\dim_{k}M_{d}=2\) for \(d>0\) and
                  \(M_{0}=k\) which has dimension \(1\).

                  For the other sheaf we first note that because \(\sheaf(1)\)
                  is locally invertible
                  \begin{align*}
                      (\tilde{M}\otimes_{\sheaf_{X}}\sheaf(d))_{x} & \cong\tilde{M}_{x}\otimes_{\sheaf_{X,x}}\sheaf(d)_{x} \\
                                                                   & \cong\tilde{M}_{x}\otimes_{\sheaf_{X,x}}\sheaf_{X,x}  \\
                                                                   & =\tilde{M}_{x}.
                  \end{align*}
                  This stalk is \(k\) on two points and \(0\) elsewhere.
                  This means that the injective
                  \[
                      \Gamma(\proj^{1}_{k},\tilde{M}\otimes\sheaf(d))\hookrightarrow\prod_{x}(\tilde{M}\otimes_{\sheaf_{X}}\sheaf(d))_{x}\cong k^{2}
                  \]
                  gives an upper bound of \(2\) on the dimension of the global
                  sections vector space.

                  We also have an exact sequence
                  \[
                      0\to\Gamma(\proj^{1}_{k},\tilde{M}\otimes\sheaf(d))\to\Gamma(U_{0},\tilde{M}\otimes\sheaf(d))\oplus\Gamma(U_{1},\tilde{M}\otimes\sheaf(d))\to\Gamma(U_{0}\cap U_{1},\tilde{M}\otimes\sheaf(d))\to 0.
                  \]
                  We know \(\Gamma(U_{0}\cap
                  U_{1},\tilde{M}\otimes\sheaf(d))=0\) because all of the stalks
                  on the intersection \(U_{0}\cap U_{1}\) are \(0\). This means
                  that
                  \[
                      \Gamma(\proj^{1}_{k},\tilde{M}\otimes\sheaf(d))\cong\Gamma(U_{0},\tilde{M}\otimes\sheaf(d))\oplus\Gamma(U_{1},\tilde{M}\otimes\sheaf(d)).
                  \]
                  Simply showing the dimension of each of these direct summands
                  is non-zero is therefore sufficient for showing the dimension
                  is \(2\).

                  First note that the restriction of the tensor product is the
                  tensor product of the restrictions because the tensor product
                  is the sheafification of the naive tensor product presheaf
                  which are both trivially operations compatible with
                  restriction:
                  \[
                      (\mathcal{F}\otimes_{\sheaf_{X}}\mathcal{G})|_{U}\cong\mathcal{F}|_{U}\otimes_{\sheaf_{X}|_{U}}\mathcal{G}|_{U}.
                  \]
                  We know that
                  \[
                      \tilde{M}\otimes\sheaf(d)\otimes\sheaf(-d)\cong\tilde{M}.
                  \]
                  Therefore by exercise 8.2 and the previous remark we have that
                  \[
                      \tilde{M}(U_{i})\otimes\sheaf(d)(U_{i})\otimes\sheaf(-d)(U_{i})\cong\tilde{M}(U_{i}).
                  \]
                  The module \(\tilde{M}(U_{i})\) is clearly non-zero so the
                  tensor product \(\tilde{M}(U_{i})\otimes\sheaf(d)(U_{i})\)
                  cannot be either. This means
                  \(\Gamma(U_{i},\tilde{M}\otimes\sheaf(d)).\) is nonzero for
                  \(i\in\set{0,1}\) so the dimension as \(k\)-vector space is at
                  least \(1\). This was what we wanted to show.

                  This proves that
                  \(\dim_{k}\Gamma(X,\tilde{M}\otimes\sheaf(d))=2\).
              \end{solution}
    \end{enumerate}
\end{question}
\end{document}