\documentclass{article}

\usepackage[utf8]{inputenc}
\usepackage[english]{babel}
\usepackage[shortlabels]{enumitem}
\usepackage{amsthm, amssymb, mathtools, amsmath, bbm, mathrsfs,stmaryrd}
\usepackage[margin=1in]{geometry}
\usepackage[parfill]{parskip}
\usepackage[hidelinks]{hyperref}

\usepackage{caption}
\usepackage{subcaption}
\usepackage{tikz}

\newtheorem*{lemma}{Lemma}

\theoremstyle{definition}
\newtheorem{question}{Exercise}
\newtheorem*{corollary}{Corollary}

\newcommand{\set}[1]{\left\{#1\right\}}
\newcommand{\setwith}[2]{\set{#1\colon#2}}

\newcommand{\powset}{\mathcal{P}}
\DeclareMathOperator{\dom}{dom}
\DeclareMathOperator{\range}{ran}
\DeclareMathOperator{\field}{field}

\newcommand{\restrict}{\upharpoonright}

\DeclarePairedDelimiter{\abs}{|}{|}

\title{Homework Set Theory}
\date{October 11, 2022}
\author{Jonas van der Schaaf}

\begin{document}
\maketitle

\begin{question}
    This problem deals with the statement ``if \(A\) is an infinite set then
    \(A\approx A\times A\)''; let us call it \(\mathbb{P}\).
    \begin{enumerate}[a.]
        \item Prove the axiom of choice implies \(\mathbb{P}\).

              \begin{proof}
                  Let \(A\) be an infinite set and \(\kappa=\abs{A}\) which
                  exists by the axiom of choice. Then by
                  Hessenbergs\footnote{This guy is really hard to google since
                      he did ``useful'' linear algebra stuff as well.} theorem
                  \(\abs{A\times A}=\kappa\cdot\kappa=\kappa=\abs{A}\).
                  Therefore \(A\times A\approx A\).
              \end{proof}

        \item Prove \(\mathbb{P}\) implies the axiom of choice.

              \begin{proof}
                  Let \(f\) be a bijection from \(Y=X\sqcup\aleph(X)\) to
                  \(Y^{2}=(X\sqcup\aleph(X))^{2}\) where \(X\sqcup\aleph\)
                  denotes the disjoint union. By assumption there is now a
                  bijection \(\varphi:Y^{2}\to Y\).

                  For any \(x\in X\) there must be some \(\alpha,\beta\) such
                  that \(\varphi(x,\alpha)=\beta\). To see this consider the
                  fibres of the injective
                  \[
                      \varphi_{x}:\aleph(X)\to Y:\alpha\mapsto\varphi(x,\alpha).
                  \]
                  If the fibre \(\varphi_{x}^{-1}(\beta)\) is empty then
                  \(\varphi_{x}:\aleph(X)\to X\) is an injective map. This
                  cannot exist by definition of Hartogs number so such a
                  \(\beta\in\aleph(X)\) exists.

                  Now consider the set
                  \[
                      Z_{x}=\setwith{(\alpha,\beta)\in\aleph(X)^{2}}{\varphi(x,\alpha)=\beta}
                  \]
                  which when ordered co-lexicographically is well-ordered. For
                  each \(x\in X\) define
                  \[
                      \psi:X\to\aleph(X)^{2}:x\mapsto\min Z_{x}
                  \]
                  which exists by the replacement axiom. Then \(\psi\) is
                  injective because else \(\psi(x,\alpha)=\psi(y,\alpha)=\beta\)
                  for some \(x,y\in X\). Therefore \(\psi\) is an injection
                  from \(X\) into an ordinal \(\aleph(X)^{2}\) so \(X\) can be
                  well-ordered.

                  This means any set can be well-ordered and the axiom of choice
                  must therefore hold.
              \end{proof}
    \end{enumerate}
\end{question}

\begin{question}
    Prove: \(\forall(n\in\omega):n\not\approx n+1\).

    \begin{proof}
        This is a proof by induction.

        For \(n=0=\varnothing\) there is no map \(1\neq\varnothing\to0\).
        Therefore there cannot be an invertible map \(0\to 1\) so no bijection
        can exist.

        Suppose the theorem is true for \(n-1\) and consider \(n\). Suppose
        there is a bijection \(f:n\to n+1\). Then there is a bijection
        \(f\restrict_{n-1}:n-1\to n+1\setminus\set{f(n-1)}\). I will prove that
        \(n+1\setminus\set{f(n-1)}\approx n\) and therefore \(n-1\approx n\): a
        contradiction.

        Define \(g:n\to (n+1)\setminus\set{f(n-1)}\) as
        \[
            g(m)=\begin{cases}
                m   & m<f(n-1)      \\
                m+1 & m\geq f(n-1).
            \end{cases}
        \]
        Then this is a bijection. To see this take \(m,m'\in n\). Then if
        \(f(m)=f(m')\) it's clear that either we have \(g(m),g(m')>f(n-1)\) or
        \(g(m),g(m')<f(n-1)\). In the former case this means \(m+1=m'+1\) so
        \(m=m'\) and in the latter this directly gives \(m=m'\). Therefore \(g\)
        is injective.

        For any \(m\in (n+1)\setminus f(n-1)\) either \(m<f(n-1)\) or
        \(m>f(n-1)\). In the former case \(m\in n\) so \(g(m)=m\) and in the
        latter case (because \(m>f(n-1)\geq0\)) we have \(g(m-1)=m-1+1=m\).
        Therefore \(g\) is surjective.

        This means \(g:n\to (n+1)\setminus\set{f(n-1)}\) is a bijection so
        \(n\approx(n+1)\setminus\set{f(n-1)}\approx n-1\). This means \(n\approx
        n-1\) which by the induction hypothesis is impossible. Therefore the
        bijection \(f\) cannot exist.
    \end{proof}
\end{question}

\begin{question}
    \begin{enumerate}[a.]
        \item Prove every finite set is \(S\)-finite.

              \begin{lemma}
                  I will first prove every strict subset \(X\subsetneq
                  n\in\omega\) is equipotent to an \(m<n\).

                  \begin{proof}
                      The empty set has no strict subsets so the statement is
                      vacuous for \(n=0\).

                      Assume the statement is true for \(n\) and consider
                      \(X\subsetneq n+1\). Then \(Y=X\setminus\set{n}\subseteq n\).
                      If \(Y=n\) then \(X=n\) because \(X\) is a strict subset and
                      so cannot also contain \(n\).

                      If \(Y\neq n\) then \(Y\subsetneq n\) so \(Y\approx m\) for
                      some \(m<n\) by induction. If \(n\notin X\) then \(X=Y\) so
                      \(X\approx m\). If \(n\in X\) then
                      \begin{align*}
                          X & =Y\cup\set{n}        \\
                            & \approx m\cup\set{n} \\
                            & \approx m+1          \\
                            & \leq n<n+1.
                      \end{align*}
                      This proves the lemma.
                  \end{proof}
              \end{lemma}

              \begin{proof}
                  Because every finite set \(X\) is equipotent to a unique
                  ordinal \(n\) it suffices to show the statement is true for
                  all \(n\in\omega\).

                  Let \(f:n\to n\) be a surjective non-injective map. There is
                  some subset \(X\subsetneq n\) such that \(f[X]=n\) and
                  \(f\restrict_{X}\) is injective (choose one element from the
                  fibres of \(f\) using finite choice and stick them in a set).
                  Then \(X\approx m\) for some \(m<n\) and \(f\restrict_{X}\) is
                  an injective surjective (i.e. bijective) map to \(n\).
                  Therefore \(m\approx X\approx n\) which contradicts the
                  pidgeonhole principle so there is no surjective non-injective
                  morphism \(f:n\to n\).
              \end{proof}

        \item Prove that every \(S\)-finite set is Dedekind-finite.

              \begin{proof}
                  Let \(X\) be an \(S\)-finite set and \(f:X\to X\) be
                  injective. Then \(g=f^{-1}\restrict_{f[X]}:f[X]\to X\) is
                  surjective. Then \(g\) can be extended to a surjective map
                  \(\tilde{g}:X\to X\) by choosing a single \(a\in X\) and
                  defining
                  \[
                      \tilde{g}(x)=\begin{cases}
                          g(x) & x\in f[X],              \\
                          a    & \textnormal{otherwise}.
                      \end{cases}
                  \]
                  The map \(\tilde{g}\) is then still a surjective map, but this
                  time from \(X\) onto \(X\). By \(S\)-finiteness \(\tilde{g}\)
                  must therefore be injective. Because there is an \(x\in f[X]\)
                  with \(g(x)=a\) it must be the case that there is no element
                  \(y\in X\setminus f[X]\): \(f\) is surjective. Therefore
                  \(X\) is Dedekind-finite.
              \end{proof}

        \item Prove the following are equivalent:

              \begin{enumerate}[(1)]
                  \item \(X\) is finite,
                  \item \(X\) is \(T\)-finite,
                  \item \(X\) is \(K\)-finite.
              \end{enumerate}

              \begin{proof}
                  \((1)\Rightarrow(2)\) Let \(X\) be finite and consider a chain \(X_{1}\subsetneq
                  X_{2}\subsetneq\ldots\subsetneq X_{n}\) be a chain in
                  \(A\subseteq\powset(X)\). Because \(\powset(X)\approx
                  2^{|X|}\) it must be the case that every such chain has length
                  less than or equal to \(2^{\abs{X}}\). Therefore there is a
                  longest chain \(X_{1}\subsetneq\ldots\subsetneq X_{m}\) with
                  \(m\leq 2^{\abs{X}}\). Then \(X_{m}\) must be a maximal
                  element of \(A\) because the chain is of maximal length.

                  \((2)\Rightarrow(3)\) Let \(X\) be a \(T\)-finite set and let
                  \(Y\) be the smallest set containing the singletons
                  \(\set{x}\) for all \(x\in X\) which is closed under union.
                  Consider any \(Z\subseteq X\) and consider \(Y'=\powset(Z)\cap
                  Y\). By \(T\)-maximality this set has a maximal element
                  \(A\subseteq Z\). If \(A\neq Z\) then there is an \(a\in
                  Z\setminus A\) so \(A\subsetneq A\cup\set{a}\subseteq Z\) is a
                  larger subset of \(Z\) contained in \(Y\). Therefore \(A=Z\)
                  so each subset \(Z\subseteq X\) is an element of \(Y\).

                  \((3)\Rightarrow(1)\) This will be a proof by contrapositive.
                  Let \(X\) be an infinite set and consider
                  \[
                      Y=\setwith{A\in\powset(X)}{\textnormal{\(A\) is finite}}.
                  \]
                  Then \(Y\) clearly contains all singletons and is closed under
                  union because the union of countable sets is countable.
                  However \(X\subseteq X\) is infinite so \(X\notin Y\). This
                  means that \(Y\neq\powset(X)\).
              \end{proof}

        \item Assume the axiom of choice and prove that Dedekind sets are
              infinite.

              \begin{proof}
                  This will be a proof by contrapositive. Let \(X\) be infinite
                  and let \(f:X\to\abs{X}\) be a bijection inducing a well order
                  on \(X\) with \(\abs{X}>\omega\).

                  Then the map
                  \[
                      g:\abs{X}\to\abs{X}:x\mapsto\begin{cases}
                          x+1 & x<\omega               \\
                          x   & \textnormal{otherwise}
                      \end{cases}
                  \]
                  is clearly injective and non surjective because \(0\) is not
                  in its image. Because \(f\) is a bijection the map
                  \(f^{-1}gf\) is also injective non-surjective so \(X\) is
                  Dedekind infinite.

                  Therefore every infinite set is Dedekind-infinite. Therefore
                  by the contrapositive every Dedekind-finite set is infinite.
              \end{proof}
    \end{enumerate}
\end{question}
\end{document}