\documentclass{article}

\usepackage[utf8]{inputenc}
\usepackage[english]{babel}
\usepackage[shortlabels]{enumitem}
\usepackage{amsthm, amssymb, mathtools, amsmath, bbm, mathrsfs,stmaryrd}
\usepackage[margin=1in]{geometry}
\usepackage[parfill]{parskip}
\usepackage[hidelinks]{hyperref}
\usepackage{cleveref}

\usepackage{caption}
\usepackage{subcaption}
\usepackage{tikz}

\newtheorem*{lemma}{Lemma}

\theoremstyle{definition}
\newtheorem{question}{Exercise}
\newtheorem*{corollary}{Corollary}

\newcommand{\set}[1]{\left\{#1\right\}}
\newcommand{\setwith}[2]{\set{#1\colon#2}}
\newcommand{\sequence}[2]{\left<#1:#2\right>}

\newcommand{\powset}{\mathcal{P}}
\DeclareMathOperator{\dom}{dom}
\DeclareMathOperator{\range}{ran}
\DeclareMathOperator{\field}{field}
\DeclareMathOperator{\cof}{cf}

\newcommand{\restrict}{\upharpoonright}

\DeclarePairedDelimiter{\abs}{|}{|}

\newenvironment{solution}{\begin{proof}[Solution]}{\end{proof}}

\title{Homework Set Theory}
\date{Novenber 21, 2022}
\author{Jonas van der Schaaf}

\begin{document}
\maketitle

\begin{question}
    Let \(S\) be a stationary subset of \(\omega_{1}\). Prove that for every
    \(\alpha\in\omega_{1}\) there is a closed subset of any \(S\) of order type
    \(\alpha\).

    \begin{proof}
        This will be a proof by transfinite induction. For \(0\), the empty set
        \(\varnothing\subseteq S\) is closed because the empty set is closed in
        any topology.

        Suppose for \(\alpha\in\omega_{1}\) and \(S\) is stationary then there
        is a subset \(X\subseteq S\) with order type \(\alpha\). There is some
        element \(a\) such that \(x<a\) for all \(x\in X\). This is true because
        \(X\) cannot be cofinal in \(\omega_{1}\) so it must have a supremum
        smaller than \(\omega_{1}\). The set \(\set{a}\subseteq S\) is closed
        because it is the complement of the open set \([0,a)\cup(a,\omega_{1})\)
        which is open. Therefore \(X\cup\set{a}\) is closed. Because \(X\) has
        order type \(\alpha\) and \(a\) is larger than all elements of \(X\)
        \(X\cup\set{a}\) has order type \(\alpha+1\). Therefore \(S\) has a
        subset of order type \(\alpha+1\).

        Now let \(\lambda\) be a countable limit ordinal such that there is a
        closed subset \(X_{\alpha}\subseteq S\) for all stationary \(S\) such
        that the order type of \(X_{\alpha}\) is \(\alpha\). Let
        \(\sequence{\alpha_{n}}{n\in\omega}\) be a strictly increasing
        continuous cofinal sequence. This exists because \(\lambda\) is
        countable and a limit ordinal and thus must have at most countable
        cofinality and clearly at most countable cofinality.

        By recursion we will construct a sequence
        \(\sequence{C_{\gamma}}{\gamma\in\omega_{1}}\) of countable closed
        subsets such that \(C_{\gamma}\subseteq S\) for all \(\gamma\), \(\max
        C_{\gamma}<\min C_{\delta}\) for \(\gamma<\delta\) and if
        \(\gamma=\omega\cdot\delta+n\) then \(C_{\gamma}\) has order type
        \(\alpha_{n}+1\).

        Let \(C_{0}\) be a closed subset of order type \(\alpha_{0}+1\) in
        \(S\). This exists by induction. Also define \(\beta_{0}\) to be some
        ordinal strictly larger than \(\sup C_{0}\) which exists because nothing
        of order type \(\alpha_{0}+1\) can be cofinal in \(\omega_{1}\). Then
        the set \([\beta_{0},\omega_{1})\) is closed and unbounded so
        \(S_{1}=S\cap[\beta_{0},\omega_{1})\) must be stationary with \(\min
        S_{1}>\max C_{0}\).

        If \(\gamma=\delta+1\) is a successor ordinal we have a stationary set
        \(S_{\gamma}\) such that all \(C_{\delta}\) are below \(S_{\gamma}\).
        This must have a closed subset of order type \(\alpha_{n}+1\) where
        \(n\) is chosen such that \(\gamma=\omega\cdot\delta+n\) by assumption
        of the induction hypothesis. Defining \(\beta_{\gamma}=\max
        C_{\gamma}+1\) we have a new stationary set
        \(S_{\gamma}=S_{\delta}\cap[\beta_{\gamma},\omega_{1})=S\cap[\beta_{\gamma},\omega_{1})\).

        If \(\gamma\) is a limit ordinal then we have the sequence
        \(\sequence{\beta_{\delta}}{\delta<\gamma}\) where each
        \(\beta_{\delta}\) is a strict upper bound of \(C_{\delta}\). Because it
        is a countable sequence and \(\omega_{1}\) has uncountable cofinality it
        must have some supremum \(\eta=\sup_{\delta<\gamma}\beta_{\delta}\)
        which is an upper bound for all \(C_{\delta}\) with \(\delta<\gamma\).
        Then \(S\cap [\eta,\omega_{1})\) is a stationary set so it must have a
        closed subset \(C_{\gamma}\) of order type \(\alpha_{0}+1\). We have
        \(\gamma=\omega\cdot\delta+0\) for some \(\delta\) because it is a limit
        so this satisfies the required property. Define \(\beta_{\gamma}=\max
        C_{\gamma}+1\) and \(S_{\gamma}=S\cap[\beta_{\gamma},\omega_{1})\) to
        continue the recursion.

        This gives an uncountable sequence of closed sets which satisfies all
        the required definitions. Each \(C_{\gamma}\subseteq S\) because it is
        chosen to be in a subset of \(S\). They are also closed by construction.
        Because we define each \(\beta_{\gamma}\) to be a strict upper bound we
        get that \(\max C_{\gamma}<\min C_{\delta}\) for \(\gamma<\delta\). The
        order typed are also as required because we picked each \(C_{\gamma}\)
        such that it had the desired order type.

        I don't know how to proceed from here.
    \end{proof}
\end{question}

\begin{question}
    Some standard applications of Ramsey theory:
    \begin{enumerate}[a.]
        \item Let \(\left< L,<\right>\) be an infinite linearly
              ordered set. Prove that \(L\) has an infinite subset \(X\) that is
              well-ordered by \(<\) or an infinite subset \(Y\) that is
              well-ordered by \(>\).

              \begin{proof}
                  Let \(L'\) be a countable subset with bijection \(f:\omega\to
                  L'\) of \(L\) and consider the following colouring:
                  \[
                      F:[\omega]^{2}\to 2:X=\set{x,y}\mapsto\begin{cases}
                          0 & f(\min X)=\min\set{f(x),f(y)}, \\
                          1 & f(\min X)=\max\set{f(x),f(y)}.
                      \end{cases}
                  \]
                  By Ramsey's theorem there is a subset \(H\subseteq\omega\)
                  with \(\abs{H}=\aleph_{0}\) such that \(F\) is constant on
                  \([H]^{2}\subseteq[\omega]^{2}\). Then either \(<\) or \(>\)
                  well-orders \(H\). We will prove that if \(F\) is constantly
                  \(0\) then \(<\) well-orders \(H\). In the other case \(>\)
                  well-orders \(H\) but the proof is completely dual to the
                  below proof so we will not give it.

                  The order \(<\) restricted to \(f[H]\) is still ha linear
                  order, so if we prove that every \(X\subseteq f[H]\) has a
                  least lement \(<\) well-orders \(H\). Let \(X\neq\varnothing\)
                  be a subset of \(H\). Then taking \(i=\min f^{-1}[X]\) we know
                  that \(f(i)<f(j)\) for all \(j\in f^{-1}[X]\) different from
                  \(i\) by definition of \(F\). This means that \(f(i)\) is a
                  minimal element of \(X\subseteq H\).

                  This means every non-empty subset of \(H\) has a least element
                  so \(<\) well-orders \(H\).
              \end{proof}

        \item Prove that every bounded sequence of real numbers has a convergent
              subsequence.

              \begin{proof}
                  Let \(s\) be a bounded sequence and consider the following
                  function:
                  \[
                      F:[\omega]^{2}\to 2:X=\set{x,y}\mapsto\begin{cases}
                          0 & s_{\min X}=\min\set{s_{x},s_{y}}, \\
                          1 & s_{\min X}=\max\set{s_{x},s_{y}}.
                      \end{cases}
                  \]
                  Then by Ramsey's theorem there is a subset
                  \(H\subseteq\omega\) with \(\abs{H}=\aleph_{0}\) and \(F\)
                  constant on \([H]^{2}\subseteq[\omega]^{2}\). This means that
                  \(s\restrict_{H}\) must be monotone (weakly)
                  increasing/decreasing.

                  Bounded monotone sequences converge, and a subsequence of a
                  bounded sequence is bounded so \(s\restrict H\) must converge.
              \end{proof}

        \item Let \(\left<P,\leq\right>\) be an infinite partially ordered set.
              Prove \(P\) has an infinite subset \(C\) that is linearly ordered
              or an infinite subset \(U\) that is unordered.

              \begin{proof}
                  Let \(\left<P,\leq\right>\) be an infinite partially ordered
                  set and let \(P'\subseteq P\) be a subset of countable
                  cardinality and \(f:\omega\to P'\) a bijection. Define the
                  following colouring function
                  \[
                      F:[\omega]^{2}\to 3:X=\set{x,y}\mapsto\begin{cases}
                          0 & f(\min X)=\min\set{f(x),f(y)},                       \\
                          1 & f(\min X)=\max\set{f(x),f(y)},                       \\
                          2 & \textnormal{\(f(x)\) and \(f(y)\) are incomparable}.
                      \end{cases}
                  \]
                  By Ramsey's theorem there is an infinite \(H\subseteq\omega\)
                  such that \(F\) is constant on \([H]^{2}\). If
                  \(f[[H]^{2}]=\set{0}\) then \(\leq\) linearly orders \(H\) so
                  \(f\) is a monotone increasing function on \(f^{-1}[H]\) by
                  the same reasoning as in part a. If \(f[[H]^{2}]=\set{1}\)
                  then \(f\) is a monotone decreasing function on \(f^{-1}[H]\).
                  If \(F[[H]^{2}]=\set{0}\) then \(\sequence{f(i)}{i\in
                      f^{-1}[H]}\) is a monotone increasing sequence in \(P'\). In
                  either of these cases, it's an infinite chain.

                  If \(F([H]^{2})=\set{2}\) then all elements of \(f[H]\) must
                  be incomparable.

                  Therefore \(P'\) contains either an infinite chain
                  or an infinite set of incomparable elements and therefore so
                  does \(P\supseteq P'\).
              \end{proof}
    \end{enumerate}
\end{question}

\begin{question}
    Here are four well-behaved subfamilies of \(\omega\):
    \begin{enumerate}[(1)]
        \item \(\mathcal{A}=\setwith{\set{n}}{n\in\omega}\),
        \item \(\mathcal{B}=\setwith{n}{n\in\omega}\),
        \item \(\mathcal{C}=\setwith{\omega\setminus\set{n}}{n\in\omega}\),
        \item \(\mathcal{D}=\setwith{\omega\setminus n}{n\in\omega}\).
    \end{enumerate}
    Let \(X\) be an infinite set and \(\mathcal{S}\) an infinite family of
    subsets of \(X\). Prove that there is a sequence
    \(\sequence{x_{n}}{n\in\omega}\) of points in \(X\) and a sequence
    \(\sequence{S_{n}}{n\in\omega}\) such that
    \[
        \setwith{\setwith{m\in\omega}{x_{m}\in S_{n}}}{n\in\omega}
    \]
    is equal to one of the above sets.

    \begin{enumerate}[a.]
        \item Construct a sequence \(\sequence{x_{n}}{n\in\omega}\) of points in
              \(X\) and a sequence \(\sequence{\mathcal{S}_{n}}{n\in\omega}\) of
              infinite subfamilies of \(\mathcal{S}\) such that
              \(\mathcal{S}_{0}=\mathcal{S}\) and for every \(n\) the following
              hold: either
              \(\mathcal{S}_{n+1}=\setwith{S\in\mathcal{S}_{n}}{x_{n}\in S}\) or
              \(\mathcal{S}_{n+1}=\setwith{S\in\mathcal{S}_{n}}{x_{n}\notin
                  S}\).

              \begin{solution}
                  We will construct the sequence recursively. Given an infinite
                  \(\mathcal{S}_{n}\) we will give an \(x_{n}\) such that one of
                  the above properties hold.

                  We know that for each \(x\in X\) either the set
                  \(\setwith{S\in S_{n}}{x_{n}\in S}\) or \(\setwith{S\in
                      S_{n}}{x_{n}\in S}\) must be infinite. This is because they
                  are clearly disjoint and their union is \(\mathcal{S}_{n}\)
                  which is infinite. Because \(\mathcal{S}_{n}\) is infinite it
                  has two different elements, so in particular there is an
                  \(x\in X\) such that \(x\in S\in\mathcal{S}_{n}\) but
                  \(x\notin S'\in\mathcal{S}_{n}\). Defining \(x_{n}\) to be any
                  such \(n\) and defining \(\mathcal{S}_{n+1}\) to be the larger
                  of \(\setwith{S\in S_{n}}{x_{n}\in S}\) and \(\setwith{S\in
                      S_{n}}{x_{n}\in S}\).

                  This gives recursive sequences
                  \(\sequence{x_{n}}{n\in\omega}\) and
                  \(\sequence{\mathcal{S}_{n}}{n\in\omega}\) with the desired
                  properties.
              \end{solution}

        \item Choose \(S_{n}\in\mathcal{S}_{n}\setminus\mathcal{S}_{n+1}\) for
              every \(n\). Verify if \(x_{m}\in S_{m}\) then \(x_{m}\notin
              S_{n}\) for all \(n>m\) and conversely if \(x_{m}\notin S_{m}\)
              then \(x_{m}\in S_{n}\) for all \(n>m\).

              \begin{proof}
                  Suppose \(x_{m}\in S_{m}\). Then by the given property we know
                  that \(\mathcal{S}_{m+1}\) must be the set given by
                  \(\setwith{S\in\mathcal{S}_{m+1}}{x_{m}\notin S}\) because
                  else \(S_{m}\) would not be in
                  \(\mathcal{S}_{m}\setminus\mathcal{S}_{m+1}\). Because for
                  \(n>m\), \(\mathcal{S}_{n}\subsetneq\mathcal{S}_{m}\) we have
                  \(x_{m}\notin S\) for all \(S\in\mathcal{S}_{n}\) so in
                  particular \(x_{n}\notin S_{n}\).

                  Similarly if \(x_{m}\notin S_{m}\) then we must have
                  \(x_{m}\in S\) for all \(S\in\mathcal{S}_{m+1}\) because else
                  \(S_{m}\in\mathcal{S}_{m+1}\). This means by the same
                  reasoning as before that \(x_{m}\in S\) for all
                  \(S\in\mathcal{S}_{n}\) for all \(n>m\).
              \end{proof}

        \item Now consider the colouring \(F:[\omega]^{2}\to4\) given by: if
              \(i<j\) then
              \[
                  F(\set{i,j})=\begin{cases}
                      0 & x_{i}\notin S_{j}\wedge x_{j}\notin S_{i}, \\
                      1 & x_{i}\notin S_{j}\wedge x_{j}\in S_{i},    \\
                      2 & x_{i}\in S_{j}\wedge x_{j}\notin S_{i},    \\
                      3 & x_{i}\in S_{j}\wedge x_{j}\in S_{i}.
                  \end{cases}
              \]

              \begin{proof}
                  By Ramsey's theorem there is some subset \(H\subseteq\omega\)
                  such that \(F\) is constant on \([H]^{2}\).

                  We will define \(\imath:\omega\to H\) as the strictly
                  increasing enumeration of \(H\) which is countable. Then we
                  will define \(x'_{n}=x_{\imath(n)}\) and
                  \(S'_{n}=S_{\imath(n)}\) as subsequences of
                  \(\sequence{x_{n}}{n\in\omega}\) and
                  \(\sequence{S_{n}}{n\in\omega}\). These are the sequences with
                  the desired property\footnote{I am sorry for the abundance of
                      mixed up \(m\) and \(n\) indices that probably occur in the
                      following text. Bookkeeping is not my strong suit.}.

                  Suppose \(F\) is constantly \(0\) on \([H]^{2}\). Then for any
                  \(i,j\in \omega\) with we have \(x'_{i}\notin S'_{j}\) and
                  \(x'_{j}\notin S'_{i}\). In particular this means that the set
                  \(\setwith{m\in \omega}{x'_{m}\in S'_{n}}\) cannot contain any
                  other element than \(n\) for all \(n\in \omega\). By the
                  contrapositive of part b this means that \(x'_{n}\in S'_{n}\),
                  after all there is some \(m>n\) with \(m\in \omega\) such that
                  \(x'_{m}\notin S'_{n}\) because \(H\) is infinite. This means
                  \[
                      \setwith{\setwith{m\in\omega}{x'_{m}\in S'_{n}}}{n\in\omega}=\mathcal{A}.
                  \]

                  If \(F\) is constantly \(1\) on \([H]^{2}\) this means that
                  \(x_{i}\notin S_{j}\) and \(x_{j}\in S_{i}\) for all \(i,j\in
                  H\) with \(i<j\). This means that the set
                  \(\setwith{m\in\omega}{x'_{m}\in S'_{n}}\) contains all
                  elements greater than \(n\) and no elements smaller than \(n\)
                  for all \(n\in\omega\). It must also contain \(n\) by
                  contraposition: there is an \(m>n\) such that \(x'_{n}\notin
                  S'_{m}\) so not \(x'_{n}\notin S'_{n}\) which is equivalent to
                  \(x'_{n}\in S'_{n}\) if the intuitionists among us are
                  ignored. This means that \(\setwith{m\in\omega}{x'_{m}\in
                  S'_{n}}=\omega\setminus n\) so
                  \[
                      \setwith{\setwith{m\in\omega}{x'_{m}\in S'_{n}}}{n\in\omega}=\mathcal{D}.
                  \]

                  If \(F\) is constantly \(2\) on \([H]^{2}\) then for all
                  \(m<n\) we have \(x'_{m}\in S_{n}\) and for all \(m>n\) we
                  have \(x'_{m}\notin S_{n}\). By contraposition of part b this
                  gives that \(x'_{n}\notin S_{n}\) for all \(n\) so
                  \[
                      \setwith{\setwith{m\in\omega}{x'_{m}\in S'_{n}}}{n\in\omega}=\mathcal{B}.
                  \]

                  If \(F\) is constantly \(3\) on \([H]^{2}\) then for all
                  \(m<n\) and \(m>n\) we have \(x'_{m}\in S'_{j}\). By
                  contrapositive of part b this means \(x'_{n}\notin S'_{n}\).
                  \[
                      \setwith{\setwith{m\in\omega}{x'_{m}\in S'_{n}}}{n\in\omega}=\mathcal{C}.
                  \]
              \end{proof}
    \end{enumerate}
\end{question}
\end{document}