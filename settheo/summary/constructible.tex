\section{Gödels constructible universe}
The goal of this section is to build a model of ZF satisfying choice and the
coninuum hypothesis given an arbitrary model of ZF. We do this in a similar
fashion to constructing a well-founded model of set theory.

\begin{definition}
    Given a formula \(\varphi\) and set/class \(M\) we define \(\varphi^{M}\) to
    be \(\varphi\) restricted to the domain of \(M\) inductively\footnote{We
        only give the definition for one connective, the rest goes the same.}:
    \begin{align*}
        (x\in y)^{M}            & =x\in y                 \\
        (x=y)^{M}               & =x=y                    \\
        (\varphi\wedge\psi)^{M} & =\varphi\wedge\psi      \\
        (\forall x:\varphi)^{M} & =\forall x\in M:\varphi \\
        (\exists x:\varphi)^{M} & =\exists x\in M:\varphi
    \end{align*}
\end{definition}

\begin{definition}
    Given a set \(M\) we say \(A\subseteq M\) is definable if there is a formula
    \(\varphi(x,m_{1},\ldots,m_{n})\) such that
    \[
        A=\setwith{x\in M}{\varphi^{M}(x,m_{1},\ldots,m_{n})}
    \]
    where \(m_{1},\ldots,m_{n}\) are elements of \(M\). We then define
    \(\definable(M)\) to be the set of all definable subsets of \(M\).
\end{definition}

\begin{proposition}
    For a set \(M\) we have the following:
    \begin{enumerate}
        \item \(M\in\definable(M)\),
        \item \(M\subseteq\definable(M)\) if \(M\) is transitive,
        \item \(M\subseteq\powset(M)\).
    \end{enumerate}

    \begin{proof}
        The only not entirely trivial statement is 2. Let \(M\) be transitive
        and take \(x\in M\). Then
        \begin{align*}
            x & =\setwith{y\in M}{(y\in x)^{M}} \\
              & =\setwith{y\in M}{y\in x}       \\
              & =x.
        \end{align*}
    \end{proof}
\end{proposition}

\begin{definition}
    Now we define a class sequence
    \(\sequence{L_{\alpha}}{\alpha\in\ordinals}\). We define it inductively:
    \begin{align*}
        L_{0}        & =\varnothing,                        \\
        L_{\alpha+1} & =\definable(L_{\alpha}),             \\
        L_{\lambda}  & =\bigcup_{\alpha<\lambda}L_{\alpha}.
    \end{align*}
    Gödels constructible universe is then defined as
    \[
        \bigcup_{\alpha\in\ordinals}L_{\alpha}.
    \]
\end{definition}

\begin{proposition}
    These sets have the following property:
    \begin{enumerate}
        \item Each \(L_{\alpha}\) is transitive,
        \item \(\alpha=L_{\alpha}\cap\ordinals\),
        \item If \(\beta<\alpha\) then \(L_{\beta}\in L_{\alpha}\).
    \end{enumerate}

    \begin{proof}
        For \(\alpha=0\) and limit ordinals this is immediately clear. For the
        successor case this takes a bit more work.

        For transitivity
        \[
            L_{\alpha}\subseteq L_{\alpha+1}\subseteq\powset(L_{\alpha}).
        \]
        Therefore if \(x\in L_{\alpha+1}\) then \(x\in\powset(L_{\alpha})\) so
        \(x\subseteq L_{\alpha}\subseteq L_{\alpha+1}\).

        The ordinal thing is trivial.

        This follows immediately from transitivity and the fact that
        \(L_{\alpha}\in L_{\alpha+1}\).
    \end{proof}
\end{proposition}

\begin{definition}
    Similarly to the well-founded shizzle we can define a rank
    \(\rho(x)=\alpha\) as the smallest \(\alpha\) such that \(x\in
    L_{\alpha+1}\).
\end{definition}

\begin{proposition}
    We have the following properties of \(L_{\alpha}\):
    \begin{enumerate}
        \item \(L_{\alpha}\subseteq V_{\alpha}\),
        \item If \(F\subseteq L_{\alpha}\) is finite then \(F\in L_{\alpha+1}\),
        \item \(L_{n}=V_{n}\) for \(n\in\omega\),
        \item Given the axiom of choice \(\abs{L_{\alpha}}=\abs{\alpha}\) for
              \(\alpha\geq 0\),
    \end{enumerate}

    \begin{proof}
        This is because \(L_{\alpha+1}\subseteq\powset(L_{\alpha})\) so by
        induction this follows.

        This is a proof by induction on the cardinality of the finite set. Let
        \(f:n\to F\) be a bijection. Then \(f[\varnothing]=\varnothing\in
        L_{\alpha+1}\)

        Suppose \(f[i]\in L_{\alpha+1}\). Then there is some formula \(\varphi\)
        such that \(f[i]=\setwith{y\in
            L_{\alpha}}{\varphi^{L}}(y,m_{1},\ldots,m_{n})\). Then
        \((\varphi(x,m_{1},\ldots,m_{n})\vee(x=f(i)))^{L}\) gives the image
        of \(f[i+1]\).

        From the previous statement \(L_{n}=V_{n}\) follows immediately by a
        simple induction proof. Because there are countably many formulas which
        have finitely many arguments we see
        \[
            \abs{L_{\alpha+1}}\leq\aleph_{0}\cdot \abs{L_{\alpha}}
        \]
        for successor ordinals and
        \begin{align*}
            \abs{L_{\lambda}} & \leq\sum_{\alpha<\lambda}L_{\alpha}                      \\
                              & =\abs{\lambda}\cdot\sup_{\alpha<\lambda}\abs{L_{\alpha}} \\
                              & =\abs{\lambda}.
        \end{align*}
        Conversely we have \(\alpha\subseteq L_{\alpha}\) for all ordinals
        \(\alpha\) so the other inequality holds by definition.
    \end{proof}
\end{proposition}

\begin{proposition}
    \(L\) is a model of ZF.

    \begin{proof}
        We show this by demonstrating \(L\) satisfies all axioms of ZFC.

        Extensionality is satisfied because \(L\) is a transitive class.

        The empty set is contained in \(L_{1}\) and \(\omega\in L_{\omega+1}\).

        For union take \(x\in L_{\alpha}\). Then \(\bigcup x=\setwith{y\in
            L_{\alpha}}{\exists z\in x:y\in z}\). By transitivity
        \[
            (\exists z\in x:y\in z)\leftrightarrow(\exists z\in x:y\in z)^{L_{\alpha}}
        \]
        so \(\bigcup x\in L_{\alpha+1}\).

        Pair is pretty trivial as well.

        Powerset and replacement are a bit trickier we prove the weakened
        version which together with separation gives the full versions.

        For powerset take any \(x\in L\) and consider the set
        \[
            \setwith{\rho(y)}{y\in\powset(x)\cap L}.
        \]
        This must have a supremum \(\alpha\). Then all subsets of \(x\) in \(L\)
        are elements of \(L_{\alpha+1}\).

        For replacement take any \(A\in L\) and \(w_{1},\ldots,w_{n}\in L\) such
        that
        \[
            \forall x\in A:\exists!y\in L:\varphi(x,y,w_{1},\ldots,w_{n}).
        \]
        Once again take the set of all ranks of such \(y\):
        \[
            \setwith{\rho(y)}{\exists x\in A:\varphi(x,y,w_{1},\ldots,w_{n})}
        \]
        which has supremum \(\alpha\). Then \(L_{\alpha+1}\) contains all such
        \(y\).

        Now for the axiom schema of separation let
        \(\varphi(x,z,w_{1},\ldots,w_{n})\) be a formula and \(z\in L\) be any
        set. We want
        \[
            \setwith{x\in z}{\varphi(x,z,w_{1},\ldots,w_{n})}
        \]
        to be a set in \(L\). Take \(\alpha\) such that
        \(x,z,w_{1},\ldots,w_{n}\in L_{\alpha}\). Then not necessarily
        \[
            \varphi^{L}(x,z,w_{1},\ldots,w_{n})\leftrightarrow\varphi^{L_{\alpha}}(x,z,w_{1},\ldots,w_{n})
        \]
        but there is some \(\beta>\alpha\) such that
        \[
            \varphi^{L}(x,z,w_{1},\ldots,w_{n})\leftrightarrow\varphi^{L_{\beta}}(x,z,w_{1},\ldots,w_{n})
        \]
        We will show this later. Then the set
        \[
            \setwith{x\in L_{\beta}}{x\in z\wedge\varphi(x,z,w_{1},\ldots,w_{n})}
        \]
        is in \(L_{\beta+1}\).
    \end{proof}
\end{proposition}

\begin{definition}
    A formula \(\varphi\) is called \(\Delta_{0}\) if all of its quantifiers are
    bounded.
\end{definition}

\begin{definition}
    A formula \(\varphi\) is absolute for a set/class \(M\) iff
    \(\varphi(w_{1},\ldots,w_{n})\leftrightarrow\varphi^{M}(w_{1},\ldots,w_{n})\)
    for \(w_{1},\ldots,w_{n}\in M\).

    Given two sets/classes \(M\subseteq N\) a formula
    \(\varphi(x_{1},\ldots,x_{n})\) is upward absolute if
    \[
        \varphi^{M}(x_{1},\ldots,x_{n})\to\varphi^{N}(x_{1},\ldots,x_{n})
    \]
    and downward absolute if
    \[
        \varphi^{N}(x_{1},\ldots,x_{n})\to\varphi^{M}(x_{1},\ldots,x_{n}).
    \]
\end{definition}

\begin{proposition}
    If \(M\) is a set or class and \(\varphi\) is a \(\Delta_{0}\) formula then
    \(\varphi\) is absolute for \(M\).

    \begin{proof}
        This is a simple proof by induction on the complexity of \(\varphi\).
    \end{proof}
\end{proposition}

\begin{definition}[Gödel operations]
    The following class functions form the Gödel operations used to define
    definability rigorously:
    \begin{align*}
        G_{1}(x,y) & =\set{x,y}                            \\
        G_{2}(x,y) & =x\times y                            \\
        G_{3}(x,y) & =\setwith{(u,v)\in x\times y}{u\in v} \\
        G_{4}(x,y) & =x\setminus y                         \\
        G_{5}(x,y) & =x\cap y                              \\
        G_{6}(x)   & =\bigcup x                            \\
        G_{7}(x)   & =\dom x                               \\
        G_{8}(x)   & =\setwith{(u,v)}{(v,u)\in x}          \\
        G_{9}(x)   & =\setwith{(u,v,w)}{(u,w,v)\in x}      \\
        G_{10}(x)  & =\setwith{(u,v,w)}{(v,w,u)\in x}.
    \end{align*}
\end{definition}

\begin{proposition}
    Let \(\varphi_{1},\ldots,\varphi_{n}\) be a set of formulas closed under
    taking subformulas. Then these formulas are absolute for classses
    \(M\subseteq N\) iff whenever
    \[
        \varphi_{i}(w_{1},\ldots,w_{m})=\exists x:\varphi_{j}(x,w_{1},\ldots,w_{m})
    \]
    then
    \[
        \forall w_{1},\ldots,w_{n}\in M:(\exists x\in N:\varphi_{j}^{N}(x,w_{1},\ldots,w_{n}))\to\exists x\in M:\varphi_{j}^{N}(x,w_{1},\ldots,w_{n})
    \]

    \begin{proof}
        \(\Longrightarrow\) Take \(w_{1},\ldots,w_{m}\in M\) and assume
        \(\exists x\in N:\varphi^{M}(x,w_{1},\ldots,w_{m})\). Then
        \(\varphi_{i}^{N}\) holds and so by absoluteness down
        \(\varphi_{i}^{M}\) holds. By absoluteness up
        \(\varphi_{j}^{N}(x,w_{1},\ldots,w_{m})\) holds so
        \(\exists x\in M:\varphi_{j}^{N}(x,w_{1},\ldots,w_{m})\) holds as well.

        \(\Longleftarrow\) For the converse absoluteness is proven by induction
        on the complexity of formulas. The only interesting case is the
        existential quantifier:
        \begin{align*}
            \varphi_{i}^{M}(w_{1},\ldots,w_{m}) & \leftrightarrow\exists x\in M:\varphi_{j}^{M}(x,w_{1},\ldots,w_{m}) \\
                                                & \leftrightarrow\exists x\in M:\varphi_{j}^{M}(x,w_{1},\ldots,w_{m}) \\
                                                & \leftrightarrow\exists x\in N:\varphi_{j}^{N}(x,w_{1},\ldots,w_{m}) \\
                                                & \leftrightarrow\varphi_{i}^{N}(w_{1},\ldots,w_{m}).
        \end{align*}
    \end{proof}
\end{proposition}

\begin{proposition}
    If \(M_{0}\) is a set and \(\varphi(x,w_{1},\ldots,w_{n})\) a formula then
    there is a set \(M\supseteq M_{0}\) such that if \(\exists
    x:\varphi(x,w_{1},\ldots,w_{n})\) for \(w_{1},\ldots,w_{n}\in M_{0}\)
    then \(\exists x\in M:\varphi(x,w_{1},\ldots,w_{n})\).

    \begin{proof}
        Close under existence but only things with minimal rank to things remain
        sets.

        Given choice pick just one element.
    \end{proof}
\end{proposition}

\begin{corollary}
    The above theorem is also true for finitely many formulas.

    If \(M_{0}\subseteq L\) then \(M\) can be of the form \(L_{\alpha}\) for
    some limit ordinal \(\alpha\). This finishes the proof that \(L\) is a model
    of ZF.
\end{corollary}

\begin{definition}
    We define \(H(\kappa)=\setwith{x\in V}{\trcl(x)<\kappa}\). This is a model
    of ZF-P for regular cardinals (see homework).
\end{definition}

\begin{theorem}
    A formula \(\varphi(w_{1},\ldots,w_{n})\) is \(\Delta_{0}\) iff there is a
    composition \(G\) of Gödel operations such that
    \[
        G(x_{1},\ldots,x_{n})=\setwith{(u_{1},\ldots,u_{n})}{\bigwedge_{i}u_{i}\in x_{i}\wedge\varphi(u_{1},\ldots,u_{n})}.
    \]
\end{theorem}

\begin{corollary}
    If \(M\) is transitive and closed under the Gödel operations then \(M\)
    satisfies separation for \(\Delta_{0}\) formulas.
\end{corollary}

\begin{definition}
    Now we redefine \(\definable(M)\) as the closure of \(M\cup\set{M}\) under
    the Gödel operations intersected with the powerset.
\end{definition}

\begin{definition}
    A formula is \(\Sigma_{1}\) if it is of the type \(\exists x:\varphi\) where
    \(\varphi\) is a \(\Delta_{0}\) formula.

    A formula is \(\Pi_{1}\) if it of the type \(\forall x:\varphi\) where
    \(\varphi\) is a \(\Delta_{0}\) formula. These are downward absolute.

    A formula is \(\Delta_{1}\) if it is equivalent to a \(\Sigma_{1}\) and
    \(\Pi_{1}\) formula. These are then absolute.
\end{definition}

\begin{proposition}
    \(\sequence{L_{\alpha}}{\alpha\in\ordinals}\) is absolute for transitive
    models of ZF-P.

    \begin{proof}
        We want to show the map \(\alpha\mapsto L_{\alpha}\) is \(\Delta_{1}\)
        so absolute.

        It is \(\Sigma_{1}\) because definability it is given by:
        \begin{align*}
            y & =\definable(x)\equiv\exists W:\textnormal{\(W\) is a function}                             \\
              & \wedge(\dom W=\omega)\wedge W(0)=x                                                         \\
              & \wedge\forall n\in\omega: W(n+1)=W\cup\setwith{G_{i}(x,y)}{x,y\in W(n),i\in 11\setminus 1} \\
              & \wedge y=\bigcup\range W.
        \end{align*}
        Recursion applied to \(\Sigma_{1}\) function gives \(\Sigma_{1}\)
        function.

        It is also \(\Pi_{1}\).
    \end{proof}
\end{proposition}

\begin{corollary}
    The statement \(x\) is constructible is absolute.
\end{corollary}

\begin{corollary}
    If \(M\) is a transitive class satisfying the finitely many axioms used to
    prove the above let \(\varphi\) be the conjunction of these axioms. If
    \(\varphi^{M}\) is true then \(L\subseteq M\).

    If \(M\) is a transitive set then \(o(M):=M\cap\ordinals\) is the smallest
    ordinal not in \(M\). Conjunct \(\varphi\) with the axiom that there is no
    largest ordinal to het \(\psi\). Then if \(M^{\psi}\) the ordinal
    \(o(M)\) is a limit ordinal and \(L(M)=\bigcup_{\alpha<o(M)}L_{\alpha}\).
    This means
    \[
        L^{M}=\setwith{x\in M}{\exists\alpha:x\in L_{\alpha}}=\bigcup_{\alpha<o(M)}L_{\alpha}.
    \]
    Now define \(\chi=\psi\wedge (V=L)\). If \(M\) is a transitive proper class
    and \(M^{\chi}\) then \(M=L\). If \(M\) is a transitive set with
    \(M^{\chi}\) then \(M=L(o(M))\).
\end{corollary}

\subsection{Choice in \texorpdfstring{\(L\)}{L}}

\begin{theorem}
    In \(L\) the axiom of choice holds.

    \begin{proof}
        To prove the axiom of choice in \(L\) we construct a sequence of well
        orders \(\sequence{<_{\alpha}}{\alpha\in\ordinals}\) with the properties
        that
        \begin{enumerate}
            \item \(<_{\alpha}\subseteq<_{\beta}\) if \(\alpha<\beta\),
            \item \(x<_{\beta}y\) if \(\rho(x)<\rho(y)\).
        \end{enumerate}

        We construct this by recursion on ordinals.

        The empty set is the only well-order of \(L_{0}\). For limit ordinals
        \(\lambda\) we take \(<_{\lambda}=\bigcup_{\alpha<\lambda}<_{\alpha}\).

        Now for the successor case. Let \(\alpha\) be an ordinal such that
        \(<_{\alpha}\) is a well-order on \(L_{\alpha}\). We know
        \[
            L_{\alpha+1}=\powset(L_{\alpha})\cap\bigcup_{n<\omega}W^{\alpha}_{n}
        \]
        where \(W^{\alpha+1}_{0}=L_{\alpha}\bigcup\set{L_{\alpha}}\) and
        \[
            W^{\alpha+1}_{n+1}=W^{\alpha+1}_{n}\cup\setwith{G_{i}(x,y)}{x,y\in W^{\alpha+1}_{n},i\in11\setminus1}.
        \]
        We recursively define a well-order \(<^{\alpha+1}_{n}\) on each
        \(W^{\alpha+1}_{n}\). For \(n=0\) define
        \[
            <^{\alpha+1}_{0}=<^{\alpha}\cup\setwith{(x,L_{\alpha+1})}{x\in L_{\alpha}}.
        \]
        For \(n+1\) define \(x<^{\alpha+1}_{n+1}\) as follows
        \(x<^{\alpha+1}_{n+1}\) iff
        \begin{enumerate}
            \item \(x^{\alpha}_{n}y\) for \(x,y\in W^{\alpha+1}_{n}\),
            \item \(x\in W^{\alpha+1}_{n}\) and \(y\notin W^{\alpha+1}_{n}\),
            \item \(x,y\in W^{\alpha+1}_{n+1}\) and

                  \begin{enumerate}
                      \item \(x\) is produced by an earlier Gödel operation than
                            \(y\),
                      \item \(x=G_{i}(u,v)\) and \(y=G_{i}(u',v')\) and
                            \(u<^{\alpha+1}_{n}u'\),
                      \item \(x=G_{i}(u,v)\) and \(y=G_{i}(u',v')\) and
                            \(v<^{\alpha+1}_{n}v'\).
                  \end{enumerate}
        \end{enumerate}

        This is a well-order.
    \end{proof}
\end{theorem}

This well-order is in fact absolute for all \(M\) where the \(\chi\) from before
holds.

\subsection{Generalized Continuum Hypothesis in \texorpdfstring{\(L\)}{L}}

\begin{definition}
    A transitive set \(M\) is adequate if one of the following is true:
    \begin{enumerate}
        \item \(M\) is closed under the Gödel operations,
        \item if \(\alpha\in M\) then \(\sequence{L_{\beta}}{\beta<\alpha}\in
              M\).
    \end{enumerate}

    If \(\delta\) is a limit ordinal then \(L_{\delta}\) is adequate.

    There is a formula \(\sigma\) such that \(\sigma^{M}\) iff \(M\) is
    adequate, this can be improved to a \(\sigma\) such that \(M=L_{\delta}\)
    for some \(\delta\) if \(\sigma^{M}\).
\end{definition}

\begin{theorem}[Gödels condensation lemma]
    Every elementary substructructure \(M\prec L_{\delta}\) is isomorphic to an
    \(L_{\gamma}\) for \(\gamma<\delta\).

    \begin{proof}
        By Mostowski's collapse lemma \(M\) is isomorphic to a transitive
        \(N\subseteq L_{\delta}\). Because \(\sigma^{L_{\delta}}\) is true
        \(\sigma^{M}\) must also be true. Therefore \(\sigma^{N}\) is also true,
        which means \(N=L_{\delta}\) for some \(\gamma<\delta\).
    \end{proof}
\end{theorem}

\begin{theorem}
    The generalized continuum hypothesis is true in Gödels constructible
    universe.

    \begin{proof}
        We know that \(\abs{L_{\alpha}}=\abs{\alpha}\). The proof was given for
        models of set theory where choice holds. Choice holds in \(L\) and
        \(L_{\alpha}\) is absolute so \(\abs{L_{\alpha}}=\abs{\alpha}\) for
        infinite ordinals. In particular for infinite cardinals
        \(\abs{L_{\kappa^{+}}}=\kappa^{+}\). If we show that
        \(\powset(\kappa)\subseteq L_{\kappa^{+}}\) then we are done.

        Let \(X\subseteq\powset(\kappa)\) be a constructible subset and
        \(\delta>\kappa\) an infinite ordinal such that \(X\in L_{\delta}\).

        Let \(M\prec L_{\delta}\) be an elementary substructure containing
        \(L_{\kappa}\cup\set{X}\) and \(\abs{M}=\kappa\). Then there is an
        isomorphism \(\pi:M\to L_{\gamma}\) for some \(\gamma<M\) which must
        have \(\pi(X)=X\) because \(L_{\kappa}\cup X\) is transitive.

        The existence of such an \(M\) is left as an exercise.
    \end{proof}
\end{theorem}