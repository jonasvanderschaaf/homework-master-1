\section{Cardinals 'n shit}
\begin{definition}
    Let \(\kappa,\lambda\) be cardinal numbers. Then \(\kappa^{\lambda}\) is
    defined as the cardinality of \(\Hom(\lambda,\kappa)\).
\end{definition}

\begin{theorem}
    \label{thm:basic-exp}
    Let \(\kappa,\lambda,\mu\) be arbitrary cardinals, then the following
    (in)equalities hold:
    \begin{enumerate}
        \item \(\kappa^{\lambda+\mu}=\kappa^{\lambda}\cdot\kappa^{\mu}\),
        \item \(\kappa^{\lambda}\cdot\mu^{\lambda}=(\kappa\cdot\mu)^{\lambda}\),
        \item \((\kappa^{\lambda})^{\mu}=\kappa^{\lambda\cdot\mu}\)
        \item if \(\nu\) is a cardinal and \(\kappa\leq\lambda\) and
              \(\mu\leq\nu\) then \(\kappa^{\mu}\leq\lambda^{\nu}\).
    \end{enumerate}
\end{theorem}

\begin{proposition}
    Let \(\kappa,\lambda\) be two infinite cardinals such that
    \(2\leq\kappa\leq 2^{\lambda}\). Then
    \[
        \kappa^{\lambda}=2^{\lambda}.
    \]

    \begin{proof}
        We have \(2^{\lambda}\leq\kappa^{\lambda}\) by \Cref{thm:basic-exp}. If
        we show the opposite inequality and get equality by the
        Cantor-Schröder-Bernstein theorem:
        \begin{align*}
            \kappa^{\lambda} & \leq (2^{\lambda})^{\lambda} \\
                             & =2^{\lambda\cdot\lambda}     \\
                             & =2^{\lambda}.
        \end{align*}
    \end{proof}
\end{proposition}

Because \(x<\powset(x)\) for all sets \(x\), we deduce that
\(2^{\aleph_{\alpha}}\geq\aleph_{\alpha+1}\). It is however not necessarily true
that \(2^{\aleph_{\alpha}}=\aleph_{\alpha+1}\). This is known as the generalized
continuum hypothesis, which is independent of ZFC.

\begin{proposition}
    The cardinal \(2^{\aleph_{0}}\) cannot be equal to \(\aleph_{\omega}\).

    \begin{proof}
        Let \(\sequence{x_{n}}{n\in\omega}\) be a sequence with
        \(\abs{x}<2^{\aleph_{0}}\). Then
        \(\abs{\bigcup_{n}x_{n}}<2^{\aleph_{0}}\). To see this we notice that
        \begin{align*}
            \abs*{\bigcup_{n}x_{n}} & =\sup\abs{x_{n}}\cdot\aleph_{0} \\
                                    & \leq\aleph_{0}\cdot\aleph_{0}   \\
                                    & =\aleph_{0}
                                    & <2^{\aleph_{0}}.
        \end{align*}
        This means \(2^{\aleph_{0}}\) is not a countable union of countable
        sets, but \(\aleph_{0}\) is.
    \end{proof}
\end{proposition}