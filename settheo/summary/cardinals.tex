\section{Cardinals 'n shit}
\subsection{Exponentiation}
\begin{definition}
    Let \(\kappa,\lambda\) be cardinal numbers. Then \(\kappa^{\lambda}\) is
    defined as the cardinality of \(\Hom(\lambda,\kappa)\).
\end{definition}

\begin{theorem}
    \label{thm:basic-exp}
    Let \(\kappa,\lambda,\mu\) be arbitrary cardinals, then the following
    (in)equalities hold:
    \begin{enumerate}
        \item \(\kappa^{\lambda+\mu}=\kappa^{\lambda}\cdot\kappa^{\mu}\),
        \item \(\kappa^{\lambda}\cdot\mu^{\lambda}=(\kappa\cdot\mu)^{\lambda}\),
        \item \((\kappa^{\lambda})^{\mu}=\kappa^{\lambda\cdot\mu}\)
        \item if \(\nu\) is a cardinal and \(\kappa\leq\lambda\) and
              \(\mu\leq\nu\) then \(\kappa^{\mu}\leq\lambda^{\nu}\).
    \end{enumerate}
\end{theorem}

\begin{proposition}
    \label{thm:exp-smaller}
    Let \(\kappa,\lambda\) be two infinite cardinals such that
    \(2\leq\kappa\leq 2^{\lambda}\). Then
    \[
        \kappa^{\lambda}=2^{\lambda}.
    \]

    \begin{proof}
        We have \(2^{\lambda}\leq\kappa^{\lambda}\) by \Cref{thm:basic-exp}. If
        we show the opposite inequality and get equality by the
        Cantor-Schröder-Bernstein theorem:
        \begin{align*}
            \kappa^{\lambda} & \leq (2^{\lambda})^{\lambda} \\
                             & =2^{\lambda\cdot\lambda}     \\
                             & =2^{\lambda}.
        \end{align*}
    \end{proof}
\end{proposition}

Because \(x<\powset(x)\) for all sets \(x\), we deduce that
\(2^{\aleph_{\alpha}}\geq\aleph_{\alpha+1}\). It is however not necessarily true
that \(2^{\aleph_{\alpha}}=\aleph_{\alpha+1}\). This is known as the generalized
continuum hypothesis, which is independent of ZFC.

\begin{proposition}
    The cardinal \(2^{\aleph_{0}}\) cannot be equal to \(\aleph_{\omega}\).

    \begin{proof}
        Let \(\sequence{x_{n}}{n\in\omega}\) be a sequence with
        \(\abs{x}<2^{\aleph_{0}}\). Then
        \(\abs{\bigcup_{n}x_{n}}<2^{\aleph_{0}}\). To see this we notice that
        \begin{align*}
            \abs*{\bigcup_{n}x_{n}} & =\sup\abs{x_{n}}\cdot\aleph_{0} \\
                                    & \leq\aleph_{0}\cdot\aleph_{0}   \\
                                    & =\aleph_{0}
                                    & <2^{\aleph_{0}}.
        \end{align*}
        This means \(2^{\aleph_{0}}\) is not a countable union of countable
        sets, but \(\aleph_{0}\) is.
    \end{proof}
\end{proposition}

\subsection{Cofinality}
\begin{definition}
    Let \(\alpha,\beta\) be any two ordinals and \(f:\alpha\to\beta\) function.
    We say \(\alpha\) is cofinal in \(\beta\) if the image \(f[\alpha]\) is
    unbounded.

    For an ordinal \(\alpha\) we define \(\cf(\alpha)\) to be the smallest
    ordinal admitting a cofinal map into \(\alpha\).
\end{definition}

\begin{proposition}
    \label{prop:cof-basics}
    The cofinality \(\cf(\alpha)\) of any ordinal \(\alpha\) has the following
    properties:
    \begin{enumerate}
        \item \(\cf(\alpha)\leq\alpha\),
        \item If \(\alpha=\beta+1\) then \(\cf(\alpha)=1\),
        \item If \(\lambda\) is a limit ordinal \(\cf(\lambda)\geq\omega\),
        \item \(\cf(\omega)=\omega\).
    \end{enumerate}

    \begin{proof}
        \begin{enumerate}
            \item For any ordinal \(\alpha\) the identity
                  \(\id:\alpha\to\alpha\) is cofinal so the smallest ordinal cofinal
                  to \(\alpha\) is at most the ordinal \(\alpha\) itself.

            \item Consider the map \(f:1\to \alpha\) given by \(f(0)=\beta\).
                  This map has unbounded image (there is no larger element than
                  \(\beta\) in \(\alpha\)) so this map is cofinal. The empty set has
                  empty image under any function so \(1\) is the smallest ordinal with
                  cofinal map.

            \item For the the third statement we will prove that no finite
                  ordinal \(n\) can admit a cofinal \(f:n\to\lambda\). Let
                  \(f:n\to\lambda\) be any map, then because \(n\) is finite
                  \(f[n]\) has a maximum: \(\beta\). Because \(\lambda\) is a
                  limit ordinal this means \(\beta+1\in\lambda\) as well so
                  \(f[n]\) is not unbounded. This means that \(\cf(\lambda)\neq n\)
                  for all \(n\in\omega\) and thus \(\cf(\lambda)\geq\omega\).

            \item This is a simple corollary of part 1 and 3 of this
                  proposition.
        \end{enumerate}
    \end{proof}
\end{proposition}

\begin{proposition}
    For each ordinal \(\beta\) there is a continuous strictly increasing cofinal
    map \(\cf(\beta)\to\beta\).

    \begin{proof}
        We first construct a strictly increasing map and then make it
        continuous. Let \(f:\cf(\beta)\to\beta\) be a function witnessing the
        cofinality. We define a strictly increasing cofinal
        \(g:\cf(\beta)\to\beta\) using recursion as follows:
        \begin{align*}
            g(0)      & =0                                                         \\
            g(\alpha) & =\sup f[\alpha]\cup\setwith{g(\gamma)+1}{\gamma\leq\alpha} \\
        \end{align*}
        Because \(\alpha<\cf(\beta)\) we see that \(\sup g[\alpha]<\beta\) so
        this is a well-defined \(g:\cf(\beta)\to\beta\).

        We see that \(g\) is strictly increasing because \(g(\alpha)\geq
        g(\gamma)+1\) for all \(\gamma<\alpha\). To see \(g\) is cofinal we
        observe that \(g(\alpha)\geq f(\alpha)\) and \(f[\cf(\beta)]\) is
        unbounded so \(g[\cf(\beta)]\) must be as well. This means \(g\) is a
        strictly increasing cofinal map \(\cf(\beta)\to\beta\).

        From this \(g\) we now construct a continuous strictly increasing
        function \(h:\cf(\beta)\to\beta\). We define
        \[
            h(\alpha)=\begin{cases}
                h(0)           & \alpha=0,                                       \\
                h(\alpha)      & \textnormal{\(\alpha\) is a successor ordinal,} \\
                \sup h[\alpha] & \textnormal{\(\alpha\) is a limit ordinal.}
            \end{cases}
        \]
        This function is clearly continuous by construction. It must be strictly
        increasing on the successor ordinals because \(g\) is. For the limit
        ordinal case we have that for any \(\gamma<\alpha\) we see
        \(h(\gamma)<h(\gamma+1)\leq\sup f[\alpha]=h(\alpha)\) so \(h\) is
        strictly increasing. We also see that \(h\) is cofinal because
        \(h(\alpha)=g(\alpha)\) for all non-limit \(\alpha\) so the image is
        unbounded.
    \end{proof}
\end{proposition}

\begin{proposition}
    Let \(\alpha,\beta\) be limit ordinals and \(f:\alpha\to\beta\) a strictly
    increasing cofinal map, then \(\cf(\beta)=\cf(\alpha)\).

    \begin{proof}
        Pick the map continuous strictly increasing map
        \(g:\cf(\alpha)\to\alpha\) witnessing cofinality and consider \(f\circ
        g\). This is a strictly increasing map, the image of which must be
        unbounded because \(f\) and \(g\) are both unbounded and strictly
        increasing. This proves \(\cf(\beta)\leq\cf(\alpha)\).

        To show \(\cf(\alpha)\leq\cf(\beta)\) we construct a cofinal map
        \(h:\cf(\beta)\to\alpha\) using a strictly increasing cofinal
        \(g:\cf(\beta)\to\beta\) and \(f\). We do this by defining it as
        follows:
        \[
            h(\gamma)=\min\setwith{\delta}{f(\delta)>g(\gamma)}.
        \]
        This map is strictly increasing because both \(f\) and \(g\) are. The
        image of this map is also unbounded: let \(\gamma\in\alpha\) be any
        ordinal. Then there is a \(\delta\in\cf(\beta)\) such that
        \(g(\delta)>f(\gamma)\in\beta\) by cofinality of \(g\). Then
        \(f(h(\delta))>g(\delta)>f(\gamma)\) so because \(f\) is strictly
        increasing \(h(\delta)>\gamma\). This proves cofinality of \(h\).
    \end{proof}
\end{proposition}

\begin{corollary}
    For all ordinals \(\beta\) we have \(\cf(\cf(\beta))=\cf(\beta)\).

    \begin{proof}
        For successor ordinals, this is immediately clear because
        \(\cf(\beta)=1\) which has cofinality \(1\) as well.

        For limit ordinals \(\cf(\beta)\) is a limit ordinal as well, else it
        would be a successor and thus \(\beta\) would be as well. This means
        that we can apply the previous proposition to get
        \(\cf(\cf(\beta))=\cf(\beta)\).
    \end{proof}
\end{corollary}

\begin{definition}
    We call an ordinal \(\beta\) regular if \(\cf(\beta)=\beta\) and
    singular if not.
\end{definition}

\begin{proposition}
    The following facts are true about regular and singular ordinals:
    \begin{enumerate}
        \item Regular ordinals are cardinals,
        \item \(\cf(\beta)\) is regular for all ordinals \(\beta\in\ordinals\),
        \item \(\cf(\beta)\) is a cardinal for all ordinals \(\beta\),
        \item Given the axiom of choice, if \(\kappa\) is a cardinal
              \(\kappa^{+}\) is regular.
        \item For all limit ordinals \(\aleph_{\lambda}\) its cofinality
              is \(\cf(\lambda)\).
    \end{enumerate}

    \begin{proof}
        Let \(\alpha\) be a regular ordinal of cardinality \(\kappa<\alpha\).
        Let \(f:\kappa\to\alpha\) be a bijection, then the image of \(f\) is
        unbounded so \(\cf(\alpha)\leq\kappa<\alpha\) unless \(\kappa=\alpha\).
        This proves \(\alpha\) is a cardinal.

        By the previous corollary \(\cf(\cf(\beta))=\cf(\beta)\) for all
        ordinals so we see \(\cf(\beta)\) is regular.

        This is simply a corollary from the previous two remarks.

        Suppose there is a cofinal map \(f:\alpha\to\kappa^{+}\). Then
        \(\kappa^{+}=\bigcup\setwith{f(\gamma)}{\gamma<\alpha}\). We have
        \(\abs{f(\gamma)}\leq\kappa\) so the union has cardinality
        \(\kappa\cdot\abs{\alpha}=\max\set{\kappa,\abs{\alpha}}\). Therefore
        \(\abs{\alpha}=\kappa^{+}\).

        We have a cofinal strictly increasing map
        \(f:\lambda\to\aleph_{\lambda}\) given by \(f(\alpha)=\aleph_{\alpha}\).
        This proves \(\cf(\aleph_{\lambda})=\cf(\lambda)\).
    \end{proof}
\end{proposition}

\begin{definition}
    A cardinal is weakly inaccessible if it is a regular limit cardinal.

    A cardinal is strongly inaccessible if it is not the sum of less than
    \(\kappa\) ordinals of size less than \(\kappa\) and \(2^{\mu}<\kappa\) for
    all cardinals \(\mu<\kappa\).

    The existence of neither can be proven in ZFC. Assuming the generalized
    continuum hypothesis the two are equivalent.
\end{definition}

\begin{theorem}
    For any infinite cardinal \(\kappa\) we have
    \(\kappa<\kappa^{\cf(\kappa)}\).

    \begin{proof}
        Let \(f:\cf(\kappa)\to\kappa\) be a cofinal function and
        \(g:\kappa\to{}^{\cf(\kappa)}\kappa\) be arbitrary. We apply a
        diagonalisation argument to prove this inequality.

        We define \(h:\cf(\kappa)\to\kappa\) by
        \[
            h(\alpha)=\min\kappa\setminus\setwith{g(\beta)(\alpha)}{\beta<f(\alpha)}.
        \]
        The set \(\kappa\setminus\setwith{g(\beta)(\alpha)}{\beta<f(\alpha)}\)
        must be non-empty. We see this because \(f(\alpha)<\kappa\) and
        therefore also
        \(\abs{\setwith{g(\beta)(\alpha)}{\beta<f(\alpha)}}<\kappa\).

        This function \(h\) is not in the image of \(g\). To see this take any
        ordinal \(\beta\). We show there is some \(\alpha\) such that
        \(h(\alpha)\neq g(\beta)(\alpha)\). We take \(\alpha=f(\gamma)\) for
        some \(\gamma\) with \(f(\gamma)>\beta\). Then \(h(\alpha)\neq
        g(\beta)(\alpha)\).
    \end{proof}
\end{theorem}

\begin{corollary}
    For all cardinals \(\kappa\) it must be that \(\kappa<\cf(2^{\kappa})\).

    \begin{proof}
        If \(\lambda\geq\cf(2^{\lambda})\) then we would have
        \begin{align*}
            2^{\lambda} & =(2^{\lambda})^{\lambda}             \\
                        & \geq(2^{\lambda})^{\cf(2^{\lambda})} \\
                        & >2^{\lambda}
        \end{align*}
        which is a contradiction.
    \end{proof}
\end{corollary}

\subsection{Infinite sums and products}

\begin{definition}[Infinite cardinal sums and products]
    We define the sum of an infinite sequence of cardinals
    \(\sequence{\kappa_{\alpha}}{\alpha<\beta}\) as
    \[
        \sum_{\alpha<\beta}\kappa_{\alpha}=\abs*{\bigcup_{\alpha<\beta}\set{\alpha}\times\kappa_{\alpha}}
    \]
    which is the cardinality of the disjoint sum.

    We define the product of an infinite sequence of cardinals as
    \[
        \prod_{\alpha<\beta}\kappa_{\alpha}=\abs*{\prod_{\alpha<\beta}\kappa_{\alpha}}.
    \]
\end{definition}

\begin{proposition}
    For any infinite sequence of cardinals
    \(\sequence{\kappa_{\alpha}}{\alpha<\lambda}\) with \(\kappa_{\alpha}>0\)
    for all \(\alpha<\beta\) the infinite sum is given by
    \[
        \sum_{\alpha<\lambda}\kappa_{\alpha}=\lambda\cdot\sup_{\alpha}\kappa_{\alpha}.
    \]

    \begin{proof}
        Define \(\kappa=\sup_{\alpha}\kappa_{\alpha}\). We see
        \[
            \bigcup_{\alpha<\beta}\set{\alpha}\times\kappa_{\alpha}\subseteq\lambda\times\kappa
        \]
        the latter has cardinality \(\lambda\cdot\kappa\) so
        \(\sum_{\alpha<\beta}\kappa_{\alpha}\leq\lambda\cdot\kappa\).

        On top of that we have
        \[
            \lambda\times\set{0}\subseteq\lambda\cdot\sup_{\alpha<\beta}\kappa_{\alpha}
        \]
        and
        \[
            \kappa\leq\sum_{\alpha<\lambda}\kappa_{\alpha}
        \]
        so therefore
        \[
            \max\set{\lambda,\kappa}=\lambda\cdot\kappa=\sum_{\alpha<\lambda}\kappa_{\alpha}.
        \]
    \end{proof}
\end{proposition}

\begin{corollary}
    A cardinal \(\kappa\) is singular iff there is some increasing sequence of
    cardinals \(\sequence{\kappa_{\alpha}}{\alpha<\lambda}\) with
    \(\lambda<\kappa\) and \(\kappa_{\alpha}<\kappa\) for all
    \(\alpha<\lambda\).
\end{corollary}

\begin{proposition}
    If \(\sequence{\kappa_{\alpha}}{\alpha<\lambda}\) is a sequence with
    \(\kappa_{\alpha}\geq 2\) for all \(\alpha<\lambda\) then
    \[
        \sum_{\alpha<\lambda}\kappa_{\alpha}<\prod_{\alpha<\lambda}\kappa_{\alpha}.
    \]

    \begin{proof}
        No proof by boredom.
    \end{proof}
\end{proposition}

\begin{proposition}
    If \(\lambda\) is an infinite cardinal and
    \(\sequence{\kappa_{\alpha}}{\alpha<\lambda}\) is a sequence of
    non-decreasing non-zero cardinals then
    \[
        \prod_{\alpha<\lambda}\kappa_{\alpha}=\left(\,\sup_{\alpha<\lambda}\kappa_{\alpha}\right)^{\lambda}.
    \]

    \begin{proof}
        Taking \(\kappa=\sup_{\alpha<\lambda}\kappa_{\alpha}\) we have
        \[
            \prod_{\alpha<\lambda}\kappa_{\alpha}\subseteq\Hom(\lambda,\kappa).
        \]
        The other inclusion is more work: because
        \(\lambda=\lambda\cdot\lambda\) we can write \(\lambda\) as the disjoint
        union of \(\lambda\) many sets \(A_{\alpha}\) of cardinality
        \(\lambda\). Each set \(A_{\alpha}\) must be cofinal in \(\lambda\) so
        \(\kappa=\sup_{\gamma\in A_{\alpha}\kappa_{\alpha}}\). On top of that we
        also have
        \[
            \prod_{\gamma\in A_{\alpha}}\kappa_{\gamma}\geq\kappa
        \]
        by the same cofinality argument. Combining these two gives
        \begin{align*}
            \prod_{\alpha<\lambda}\kappa_{i} & =\prod_{\alpha<\lambda}\prod_{\gamma\in A_{\alpha}}\kappa_{\gamma} \\
                                             & \geq\prod_{\alpha<\lambda}\kappa                                   \\
                                             & =\kappa^{\lambda}.
        \end{align*}
        This proves both inequalities, which by Cantor-Schröder-Bernstein gives
        equipotence.
    \end{proof}
\end{proposition}

\begin{theorem}[König's inequality]
    Let \(\sequence{\kappa_{\alpha}}{\alpha<\lambda}\) and
    \(\sequence{\mu_{\alpha}}{\alpha<\lambda}\) be two sequences such that
    \(\kappa_{\alpha}<\lambda_{\alpha}\) for all \(\alpha<\lambda\). Then
    \[
        \sum_{\alpha<\lambda}\kappa_{\lambda}<\prod_{\alpha<\lambda}\mu_{\alpha}.
    \]

    \begin{proof}
        Let
        \(f:\bigcup_{\alpha<\lambda}\set{\alpha}\times\kappa_{\alpha}\to\prod_{\alpha\leq\lambda}\mu_{\alpha}\)
        be any map and consider the projection maps
        \(\pi_{\alpha}:\prod_{\alpha\leq\lambda}\mu_{\alpha}\to\mu_{\alpha}\)
        and injections
        \(\imath_{\alpha}:\kappa_{\alpha}\to\bigcup_{\alpha<\lambda}\set{\alpha}\times\kappa_{\alpha}\).

        The map \(\pi_{\alpha}\circ f\circ\imath_{\alpha}\) cannot be surjective
        because \(\kappa_{\alpha}<\lambda_{\alpha}\). This means
        \(V_{\alpha}=\lambda_{\alpha}\setminus(\pi_{\alpha}\circ
        f\circ\imath_{\alpha})[\kappa_{\alpha}]\neq\varnothing\).

        The function \(g(\alpha)=\min V_{\alpha}\) is then not in the image of
        \(f\) so \(f\) is not surjective.
    \end{proof}
\end{theorem}

\begin{corollary}
    This theorem gives many previously proven statements:
    \begin{align*}
        \kappa & <2^{\kappa},           \\
        \kappa & <\cf(2^{\kappa}),      \\
        \kappa & <\kappa^{\cf(\kappa)}.
    \end{align*}

    \begin{proof}
        For the first inequality take \(\kappa_{\alpha}=1\) and
        \(\mu_{\alpha}=2\) as sequences of length \(\kappa\).

        For the second take any sequence
        \(\sequence{\kappa_{\alpha}}{\alpha<\kappa}\) with
        \(\kappa_{i}<2^{\kappa}\) and \(\sequence{2^{\kappa}}{\alpha<\kappa}\).
        Then
        \[
            \kappa\cdot\sup_{\alpha<\kappa}\kappa_{\alpha}<(2^{\kappa})^{\kappa}=2^{\kappa}.
        \]
        This means there is no cofinal sequence in \(2^{\kappa}\) of length
        \(\kappa\).

        For the third let \(\lambda=\cf(\kappa)\) and
        \(\sequence{\kappa_{\alpha}}{\alpha<\lambda}\) be a sequence witnessing
        the cofinality. Then
        \[
            \kappa=\sum_{\alpha<\lambda}\kappa_{\alpha}<\prod_{\alpha<\lambda}\kappa=\kappa^{\cf(\kappa)}.
        \]
    \end{proof}
\end{corollary}

\subsection{The continuum function}
We now examine the exponentiation function
\[
    \kappa\mapsto 2^{\kappa}.
\]

\begin{proposition}
    Assuming the generalized continuum hypothesis, the following is true about
    the continuum function:
    \begin{enumerate}
        \item If \(\kappa<\lambda\)  then \(\kappa^{\lambda}=\lambda^{+}\),
        \item If \(\cf(\kappa)\leq\lambda<\kappa\) then
              \(\kappa^{\lambda}=\kappa^{+}\),
        \item If \(\lambda<\cf(\kappa)\) then \(\kappa^{\lambda}<\kappa\).
    \end{enumerate}

    \begin{proof}
        We have \(\kappa^{\lambda}\leq2^{\lambda}\) by \Cref{thm:exp-smaller}.
        The other inequality is trivial.

        The rest I don't want to prove.
    \end{proof}
\end{proposition}

\begin{definition}[Beth function]
    The Beth function is a class function defined recursively as:
    \begin{align*}
        \beth_{0}        & \aleph_{0},                           \\
        \beth_{\alpha+1} & =2^{\beth_{\alpha}}.                  \\
        \beth_{\lambda}  & =\sup_{\alpha<\lambda}\beth_{\alpha},
    \end{align*}
    where \(\lambda\) is a limit ordinal.
\end{definition}

\begin{definition}
    We define \(\kappa^{<\lambda}\) as the set of sequences with elements in
    \(\kappa\) of length at most \(\lambda\). Its cardinality is
    \[
        \sup_{\mu<\lambda}\kappa^{\mu}.
    \]

    We also define \([\kappa]^{\lambda}\) for \(\lambda\leq\kappa\) to be the
    set \(\setwith{x\subseteq\kappa}{\abs{x}=\lambda}\). This has cardinality
    \(\kappa^{\lambda}\).

    We also define the gimel function \(\gimel(\kappa)=\kappa^{\cf(\kappa)}\).
\end{definition}

For singular cardinals, there are some things we can say about exponentiation:

\begin{proposition}
    If \(\kappa\) is a singular cardinal and suppose \(\mu\mapsto 2^{\mu}\) is
    eventually constant on some interval \((\nu,\kappa]\). Then
    \(2^{\kappa}=2^{<\kappa}\). If the map is not eventually constant below
    \(\kappa\) then \(2^{\kappa}=\gimel(\kappa)\).

    If \(\kappa\) is regular \(2^{\kappa}=\gimel(\kappa)\).

    \begin{proof}
        Let \(\lambda=2^{\mu}\) for all \(\mu\in(\nu,\kappa]\), pick some
        \(\mu\geq\cf(\kappa)\) in that interval. Then
        \begin{align*}
            2^{\kappa} & =(2^{<\kappa})^{\cf(\kappa)} \\
                       & =(2^{\mu})^{\cf(\kappa)}     \\
                       & =2^{\mu}                     \\
                       & =\lambda.
        \end{align*}
        Else we have \(2^{<\kappa}=\sup_{\mu<\kappa}2^{\mu}\) so
        \(\cf(2^{<\kappa})=\cf(\kappa)\). This means
        \begin{align*}
            2^{\kappa} & =(2^{<\kappa})^{\cf(\kappa)} \\
                       & =\lambda^{\cf(\lambda)}      \\
                       & =\gimel(\lambda).
        \end{align*}

        If \(\kappa\) is a regular cardinal \(\kappa=\cf(\kappa)\) so
        \begin{align*}
            2^{\kappa} & =\kappa^{\kappa}      \\
                       & =\kappa^{\cf(\kappa)} \\
                       & =\gimel(\kappa).
        \end{align*}
    \end{proof}
\end{proposition}

\begin{theorem}[Hausdorff's formula]
    For any two ordinals \(\alpha,\beta\) we have
    \[
        \aleph_{\alpha+1}^{\aleph_{\beta}}=\aleph_{\alpha}^{\aleph_{\beta}}\cdot\aleph_{\alpha+1}.
    \]
\end{theorem}

\begin{proposition}
    For any limit cardinal \(\kappa\) and \(\lambda\geq\cf(\kappa)\) we have
    \[
        2^{\kappa}=(\lim_{\alpha\to\kappa}\alpha^{\lambda})^{\cf(\kappa)}.
    \]

    \begin{proof}
        Let \(\sequence{\kappa_{\alpha}}{\alpha<\cf(\kappa)}\) be a cofinal
        continuous strictly increasing sequence of cardinals with the property
        that \(\sum_{\alpha<\cf(\kappa)}\kappa_{\alpha}=\kappa\). Then
        \begin{align*}
            \kappa^{\lambda} & \leq\left(\prod_{\alpha<\cf(\kappa)}\kappa_{i}\right)^{\lambda} \\
                             & =\prod_{\alpha<\cf(\kappa)}\kappa_{i}^{\lambda}                 \\
                             & \leq\lim_{\alpha^{\lambda}}^{\cf(\kappa)}                       \\
                             & =(\kappa^{\lambda})^{\cf(\kappa)}                               \\
                             & =\kappa^{\lambda}.
        \end{align*}
    \end{proof}
\end{proposition}

\subsection{Infinatory combinatorics}
\begin{definition}[Filters and ideals]
    Let \(A\) be a set. Then \(\filter\subseteq\powset(A)\) is a filter
    if
    \begin{enumerate}
        \item \(A\in\filter\),
        \item \(\filter\) is closed under intersections,
        \item \(\filter\) is upward closed.
    \end{enumerate}
    A set \(\ideal\subseteq\powset(A)\) is an ideal if
    \begin{enumerate}
        \item \(\varnothing\in\ideal\),
        \item \(\ideal\) is closed under unions,
        \item \(\ideal\) is downward closed.
    \end{enumerate}
    These two concepts are dual: taking complements of each element of a filter
    gives an ideal and vice versa.

    A filter is \(\kappa\)-complete if for \(G\subseteq\filter\) of cardinality
    less than \(\kappa\) we have \(\bigcap G\in\filter\).
\end{definition}

\begin{definition}[Closed unbounded sets]
    Let \(\kappa\) be a limit cardinal.

    We say \(C\subseteq\kappa\) is closed if if \(\sup\delta\cap C=\delta\) then
    \(\delta\in C\) for all limit ordinals \(\delta<\kappa\). Alternatively a
    set is closed if it closed in the order topology applied to \(\kappa\).

    A set is called a closed unbounded set (cub/club) if \(C\subseteq\kappa\) is
    closed and unbounded. When talking about club sets, we will assume
    \(\cf(\kappa)>\aleph_{0}\). This is because we're usually interested
    in infinite sequences of length smaller than \(\cf(\kappa)\).
\end{definition}

\begin{proposition}
    The club sets of a cardinal \(\mu\) generate a \(\cf(\mu)\)-complete filter
    by collecting all sets containing a club set. This filter is written as
    \(\club(\mu)\).

    \begin{proof}
        This set is clearly upward closed and \(\mu\subset\mu\) is a club set.
        If we show that the set is closed under intersections of less than
        \(\cf(\mu)\) many club sets then it is a filter.

        Let \(\sequence{C_{\alpha}}{\alpha<\lambda<\cf(\mu)}\) be a sequence of club
        sets shorter than \(\cf(\mu)\).

        The intersection \(C=\bigcap_{\alpha<\lambda}C_{\alpha}\) is closed
        because it is the intersection of closed sets.

        To see it is unbounded take any \(\alpha<\mu\). We will show there is
        some element in \(C\) larger than \(\alpha\). To do this we construct
        \(\lambda\) many (strictly increasing) sequences of length \(\omega\).

        We define \(\gamma_{0,0}=\min\setwith{\beta\in C_{0}}{\beta>\alpha}\).
        Afterwards we recursively define
        \[
            \gamma_{0,n+1}=\min\setwith{\beta\in C_{0}}{\forall\eta<\lambda:\beta>\gamma_{\eta,n}}
        \]
        and
        \[
            \gamma_{\eta,n}=\min\setwith{\beta\in C_{\eta}}{\forall\zeta<\eta:\gamma_{\zeta,n}<\beta}.
        \]
        This will never run out of elements because
        \(\aleph_{0}\cdot\lambda<\cf(\mu)\). Then
        \(\beta=\sup_{\eta<\lambda,n<\omega}\gamma_{\eta,n}\) is an element of
        each \(C_{\eta}\) by closedness. Therefore the intersection is closed
        unbounded.
    \end{proof}
\end{proposition}

\begin{definition}
    A set \(S\subseteq\kappa\) is stationary if it intersects all cub sets of
    \(\kappa\) or equivalently if \(S\notin\club^{*}(\kappa)\). This ideal is
    called the non-stationary ideal because it contains exactly the
    non-stationary sets.
\end{definition}

\begin{proposition}
    Let \(\lambda\) be a regular cardinal and \(\mu\) be a cardinal with
    \(\cf(\mu)>\lambda\). Then
    \[
        S^{\mu}_{\lambda}=\setwith{\gamma<\mu}{\cf(\gamma)=\lambda}
    \]
    is stationary. This is also sometimes written as \(E^{\mu}_{\lambda}\).

    \begin{proof}
        Let \(\delta\) be the order type of a club set \(C\). Then
        \(\delta>\cf(\mu)>\lambda\). Let \(f:\delta\to C\) be the function
        witnessing this isomorphism. Then \(f(\lambda)\in C\cap
        S^{\mu}_{\lambda}\) because \(f\) is strictly increasing and continuous.
    \end{proof}
\end{proposition}

\begin{example}
    The set \(\omega_{2}\) has at least two disjoint stationary sets:
    \(S^{\omega_{2}}_{\omega}\) and \(S^{\omega_{2}}_{\omega_{1}}\).
\end{example}

\begin{proposition}[Ulam matrix]
    Let \(\kappa=\lambda^{+}\) be a successor cardinal. It has a family of
    \(\kappa\) many pairwise disjoint stationary sets.

    \begin{proof}
        Let \(\sequence{f_{\alpha}:\alpha\to\lambda}{\alpha<\kappa}\) be a
        sequence of injective functions (which must exist by the axiom of
        choice).

        For \(\alpha<\kappa\) and \(\xi<\lambda\) define
        \[
            X(\alpha,\xi)=\setwith{\beta>\alpha}{f_{\beta}(\alpha)=\xi}.
        \]
        %TODO
    \end{proof}
\end{proposition}

\begin{proposition}
    Let \(\mathcal{F}\) be a set of finitary functions on \(\kappa\) of
    cardinality less than \(\kappa\). Then
    \[
        C=\setwith{\gamma<\kappa}{\textnormal{\(\gamma\) is closed under \(\mathcal{F}\)}}
    \]
    is a club set.

    \begin{proof}
        To show \(C\) is closed suppose \(C\cap\delta\) is cofinal in a limit
        ordinal \(\delta\). Then for some \(f\in\mathcal{F}\) taking \(n_{f}\)
        arguments we have
        \begin{align*}
            f[\delta^{n_{f}}] & =\bigcup\setwith{f[\gamma^{n_{f}}]}{\gamma\in C\cap\delta} \\
                              & \subseteq\setwith{\gamma}{\gamma\in C\cap\delta}           \\
                              & =\delta.
        \end{align*}

        To show \(C\) is unbounded take any \(\xi\in\kappa\) and consider the
        closure of \(\xi\) under \(\mathcal{F}\). This has cardinality
        \(\abs{\xi}\cdot\abs{\mathcal{F}}<\kappa\) and is therefore bounded. Now
        for all \(\xi\) define \(g(\xi)\) such that \(G(\xi)\subseteq g(\xi)\).

        Then for all \(\alpha\in\kappa\) define \(\alpha_{0}=0\) and
        \(\alpha_{n+1}=g(\alpha_{n})\). Then \(\sup_{n\in\omega}\alpha_{n}\in
        C\) and this supremum is larger than \(\alpha\).
    \end{proof}
\end{proposition}

\begin{definition}
    Let \(\sequence{C_{\alpha}}{\alpha<\kappa}\) be a sequence of subsets of
    \(\kappa\). Then
    \[
        \bigtriangleup_{\alpha<\kappa}C_{\alpha}=\setwith{\delta\in\kappa}{\forall\alpha<\delta:\delta\in C_{\alpha}}
    \]
    is the diagonal intersection.
\end{definition}

\begin{proposition}
    If \(\kappa>\aleph_{0}\) is regular and
    \(\sequence{C_{\alpha}}{\alpha<\kappa}\) is a sequence of club sets then
    then
    \[
        C=\bigtriangleup_{\alpha<\kappa}C_{\alpha}
    \]
    is a club set.

    \begin{proof}
        The diagonal intersection doesn't change if we redefine
        \(C_{\alpha}=\bigcap_{\xi\leq\alpha}C_{\alpha}\) and therefore we can
        assume \(C_{\alpha}\subseteq C_{\beta}\) for \(\alpha\leq\beta\).

        To see \(C\) is closed take \(\delta\) such that \(C\cap\delta\) is
        cofinal in \(\delta\). Then \(C\cap\delta\subseteq C_{\alpha}\) for all
        \(\alpha<\delta\). This proves \(\delta=\sup C_{\alpha}\cap\delta\) for
        all \(\alpha<\delta\) and thus that \(\delta\in C_{\alpha}\) for all
        \(\alpha<\delta\) which means \(\delta\in C\).

        To see \(C\) is unbounded take \(\alpha\in\kappa\). We will construct a
        sequence \(\sequence{\alpha_{n}}{n<\omega}\) as follows: Take
        \(\alpha_{0}\in C_{0}\) to be larger than \(\alpha\) and choose
        \(\alpha_{n+1}\) recursively such that \(\alpha_{n+1}\in
        C_{\alpha_{n}}\) and \(\alpha_{n+1}>\alpha_{n}\). The limit
        \(\beta=\lim_{n<\omega}\alpha_{n}\) is an element of \(C\), in particular we
        will show that if \(\xi<\beta\) then \(\beta\in C_{\xi}\).

        Because \(\xi<\beta\) there is an \(n\) such that \(\xi<\alpha_{n}\).
        Each \(\alpha_{k}\) for \(k>n\) belongs to \(C_{\alpha_{n}}\) and
        therefore \(\beta\in C_{\beta_{n}}\). This means in particular that
        \(\beta\in C_{\xi}\) because \(\xi<\beta_{n}\) and \(C_{\xi}\subseteq
        C_{\beta_{n}}\).

        Ths proves that \(C\) is a club set.
    \end{proof}
\end{proposition}

\begin{definition}
    A function \(f:S\to\kappa\) for some \(S\subseteq\kappa\) is regressive
    if \(f(\alpha)<\alpha\).
\end{definition}

\begin{theorem}[Fodor's pressing down lemma]
    Let \(S\subseteq\kappa\) be a stationary set and \(f:S\to\kappa\) a
    regressive function. There is then some stationary \(T\subseteq S\) such
    that \(f(\alpha)=\gamma\) for all \(\alpha\in T\).

    \begin{proof}
        Assume that for each \(\gamma<\kappa\) the fibre \(f^{-1}[\gamma]\) is
        non-stationary. Then there is a club set \(C_{\gamma}\) such that
        \(\gamma\notin f[C_{\gamma}\cap S]\). Define
        \(C=\bigtriangleup_{\gamma<\kappa}C_{\gamma}\). Then for all \(\alpha\in
        S\cap C\) we see that \(f(\alpha)\neq\gamma\) for all \(\gamma<\alpha\).
        This means that \(S\cap C\) is empty: a contradiction because \(S\) is
        stationary. This means that at least one fibre of \(f\) must be
        stationary.
    \end{proof}
\end{theorem}

\begin{definition}
    Let \(A\) be a family of sets. A \(\Delta\)-system is a subset \(B\subseteq
    A\) with the property that for two different \(x,y\in B\) the intersection
    \(x\cap y\) is independent of \(x\) and \(y\).
\end{definition}

\begin{theorem}[Delta system lemma]
    An uncountable family of finite sets contains an uncountable
    \(\Delta\)-system.

    \begin{proof}
        We can assume that \(\bigcup A\subseteq\omega_{1}\) by just relabelling
        all elements of the sets in \(A\). Let
        \(\sequence{A_{\alpha}}{\alpha<\omega_{1}}\) be an enumeration of
        \(\aleph_{1}\) many elements of \(A\). We show this contains an
        uncountable \(\Delta\)-system.

        Define \(f:\omega_{1}\to\omega_{1}\) by \(f(\alpha)=\max\set{0}\cup
        A_{\alpha}\cap\alpha\). This maximum exists because \(A_{\alpha}\) is
        finite and \(f\) is regressive. Define
        \(C=\setwith{\delta<\omega_{1}}{\forall\alpha<\delta:\max
            A_{\alpha}<\delta}\). This is a club set, and thus in particular
        stationary. There must then be a stationary set \(S\subseteq C\) on
        which \(f\) is constant with value \(\beta\).

        For any \(\alpha\in S\) we have
        \(A_{\alpha}\subseteq[0,\beta]\cup[\alpha,\omega_{1}]\) and for
        \(\alpha_{1}<\alpha_{2}\) both elements of \(S\) we see
        \[
            A_{\alpha_{1}}\cap A_{\alpha_{2}}\subseteq[0,\beta]\cup[\alpha_{2},\omega_{1}].
        \]
        We assumed however that \(\max A_{\alpha_{1}}<\alpha_{1}\) so the
        (finite) intersection is contained in the countable set \([0,\beta]\).

        The set \([\beta]^{<\omega}\) is countable but we have uncountably many
        sets. This means there is at least one finite \(R\subseteq[0,\beta]\)
        such that \(B=\setwith{A_{\alpha}}{A_{\alpha}\cap[0,\beta]=R}\) is
        uncountable. This set \(B\) is the \(\Delta\)-system.
    \end{proof}
\end{theorem}

\begin{definition}
    Let \(f:X\to\powset(X)\) be a function with the property that \(y\notin
    f(y)\) for all \(y\in X\). A set \(M\subseteq X\) is free if for all \(y\neq
    z\) in \(M\) we have \(y\notin f(z)\) and \(z\notin f(y)\). Stated otherwise
    this means that \(f[y]\cap M=\varnothing\) for all \(y\in M\).
\end{definition}

\begin{theorem}[Hajnals free set lemma]
    If \(\kappa,\lambda\) are infinite cardinals with \(\lambda<\kappa\) and
    \(f:\kappa\to\powset(\kappa)\) is a function with
    \(\abs{f(\alpha)}=\lambda\) for all \(\alpha\in\kappa\). Then there exists a
    free set \(M\subseteq\kappa\) of cardinality \(\kappa\).
\end{theorem}

\subsection{Ramsey theory}

\begin{definition}
    Let \(\kappa,\lambda,\mu,\nu\) be cardinals. We write
    \[
        \ramsey{\kappa}{\lambda}{\mu}{\nu}
    \]
    if for any function \(F:[\kappa]^{\nu}\to\mu\) there is a set
    \(H\subseteq\kappa\) of cardinality \(\lambda\) such that \(F\) is constant
    on \([H]^{\nu}\). Usually \(\nu\) is a finite cardinal.
\end{definition}

\begin{proposition}
    For any two natural numbers \(n,m\in\omega\) we have
    \[
        \ramsey{\aleph_{0}}{\aleph_{0}}{n}{k}.
    \]

    \begin{proof}
        This will be a proof by induction on \(n\) and \(k\). The induction on
        \(k\) is trivial so we won't show it here. We only look at \(k=2\).

        For \(n=1\) let \(F:\aleph_{0}\to 2\) be any function. Then because
        \([\aleph_{0}]^{1}\) is countable either the fibre \(F^{-1}[0]\) or
        \(F^{-1}[1]\) is infinite.

        Assume the theorem holds for \(n\) and consider a function
        \(F:[\aleph_{0}]^{n+1}\to 2\). We define then for each \(a\in\omega\)
        \(F_{a}:[\aleph_{0}\setminus\set{a}]^{n}\to 2\) given by
        \(F_{a}(X)=F(X\cup\set{a})\). By assumption for every infinite
        \(S\subseteq\aleph_{0}\) there is an infinite \(H^{S}_{a}\subseteq S\)
        such that \(F\) is homogeneous on \(H^{S}_{a}\).

        Now we construct an infinite sequence \(\sequence{a_{n}}{n\in\omega}\)
        and helper sequence \(\sequence{S_{n}\subseteq\omega}{n\in\omega}\).

        Then we define \(S_{0}=\omega\) and \(a_{0}=0\) and continue recursively
        defining \(S_{n+1}=H^{S_{n}}_{a_{n}}\) and \(a_{n+1}=\min S_{n+1}\).

        For each \(a_{i}\) then, \(F_{a_{i}}\) is constant on
        \([\setwith{a_{m}}{m>i}]^{n}\). Define \(G(a_{i})\) to be this specific
        value. Then there is an infinite
        \(H\subseteq\setwith{a_{i}}{i\in\omega}\) such that \(G\) is constant on
        \(H\) (by the base case of the induction). The function \(F\) is then
        given on \(H^{n}\) by \(F(x)=G(\min x)\) which is constant on \(H\)
        by assumption.
    \end{proof}
\end{proposition}

This theorem does not generalise well:
\begin{enumerate}
    \item \(\nramsey{2^{\kappa}}{\kappa^{+}}{2}{2}\),
    \item \(\nramsey{2^{\aleph_{0}}}{3}{2}{\kappa}\),
    \item \(\nramsey{\kappa}{\aleph_{0}}{0}{\aleph_{0}}\).
\end{enumerate}

\begin{proposition}
    Any uncountable cardinal \(\kappa\) with \(\ramsey{\kappa}{\kappa}{2}{2}\)
    is a strong limit.

    \begin{proof}
        This is outside of the scope of the course and this summary.
    \end{proof}
\end{proposition}

\begin{theorem}[Erdős-Rado theorem]
    For all infinite cardinals we have
    \[
        \ramsey{(2^{\kappa})^{+}}{\kappa^{+}}{\kappa}{2}
    \]

    \begin{proof}
        %TODO: maybe
    \end{proof}
\end{theorem}

\subsection{Trees}

\begin{definition}
    A poset \((T,<)\) is a tree if all for all \(t\in T\) the set
    \[
        \setwith{s}{s<t}
    \]
    is well-ordered by \(<\). We say \(x\in T\) has level \(\alpha\) if
    \(\setwith{s}{s<t}\) has order type \(\alpha\). All elements of \(T\) of
    level \(\alpha\) are denoted by \(T_{\alpha}\).
\end{definition}

\begin{definition}
    Let \(A\) be a set and \(\alpha\) an ordinal. Then
    \[
        A^{<\alpha}=\bigcup_{\beta<\alpha}A^{\beta}
    \]
    is a tree with regards to the ordering \(\subseteq\).
\end{definition}

\begin{theorem}[Königs infinity lemma]
    Let \(T\) be an infinite tree with finite levels. Then there is a sequence
    \(\sequence{x_{i}}{i<\omega}\) such that \(x_{i}\in T_{i}\) and
    \(x_{i}<x_{j}\) for \(i<j\).

    \begin{proof}
        The set \(T\) is infinite so there is an \(x_{0}\) such that
        \(\setwith{x}{x>x_{0}}\) is infinite as well. Define \(x_{n+1}\)
        recursively given \(x_{n}\) as follows: the set \(\setwith{x\in
            T_{n+1}}{x\in T_{n+1}\wedge x_{n}<x}\) is finite but \(\setwith{x\in
            T}{x_{n}<x}\) is infinite. Therefore for one of the elements of the
        former set must have infinitely many elements above it. Pick this one
        for \(x_{n+1}\).

        This defines the infinite sequence \(\sequence{x_{i}}{i<\omega}\).
    \end{proof}
\end{theorem}

\begin{definition}
    A maximal chain of a tree \(T\) is called a branch.
\end{definition}

\begin{proposition}
    Given Königs lemma, Ramsey's theorem is true.

    \begin{proof}
        We will show Königs lemma implies
        \(\ramsey{\aleph_{0}}{\aleph_{0}}{2}{2}\). We define a subtree of
        \(2^{<\omega}\) defined by some function \(F:[\omega]^{2}\to2\) of which
        the infinite branch will define a homogeneous set.

        Let \(F:[\omega]^{2}\to2\) be a colouring. We define a sequence of
        elements \(\sequence{t_{n}}{n<\omega}\) of the tree \(2^{<\omega}\)
        which will form a subtree.

        We define \(t_{0}=\varnothing\in 2^{<\omega}\). Now we recursively build
        \(t_{n}\) given \(\sequence{t_{i}}{i<n}\). We build \(t\)
        recursively as well. We define \(t(0)=F(\set{0,n})\).

        If \(t\restrict k\) is not yet an element of the sequence then
        \(t_{n}=t\), if it is we define \(t(k)=F(\set{k,n})\).

        This recursively defines an infinite subtree of \(2^{<\omega}\) where
        \(t_{n}\) is at level \(\abs{t_{n}}\). This is because each \(t_{n}\) is
        unique. By Königs lemma there is now an infinite branch \(B\). If
        \(t_{n}<t_{m}\) for \(t_{n},t_{m}\in B\) then \(n<m\) and
        \(t_{m}(\abs{t_{n}})=F(\set{n,m})\). Now define
        \[
            H_{0}=\setwith{n\in\omega}{t_{n}\in B\wedge t_{n}\cup\set{(n,0)}\in B}.
        \]
        Then for all \(n,m\in H_{0}\) we must have \(F(\set{n,m})=0\). Defining
        \(H_{1}\) similarly replacing \(0\) with \(1\) gives another homogeneous
        set. We have \(B=H_{0}\cup H_{1}\) so at least one is infinite and
        homogeneous.
    \end{proof}
\end{proposition}

\begin{proposition}
    Given Ramsey's theorem, Königs lemma is true.

    \begin{proof}
        Let \(T\) be an infinite tree and immediately redefine it
        \(T=\bigcup_{n<\omega}T_{n}\). We want to show this has an infinite
        chain. We take two different orders on \(T\): \(<_{T}\) which is the
        partial order of the tree and \(<\) a linear order on \(T\) defined by
        some bijection \(\omega\to T\).

        We extend \(<_{T}\) to a linear order \(\prec\) on \(T\). We do this as
        follows: if \(s\) and \(t\) are \(<_{T}\)-comparable then \(\prec\)
        agrees on \(s\) and \(t\). If \(s\) and \(t\) are incomparable let take
        \(s'=\min\setwith{x<s}{x\leq_{T}s\wedge x\not<_{T}t}\) and
        \(t'=\min\setwith{x<s}{x\leq_{T}t\wedge x\not<_{T}s}\). Then \(s\prec
        t\) iff \(s'<t'\).

        Now define \(F:[T]^{2}\to\set{0,1}\) by
        \[
            F(\set{s,t})=\begin{cases}
                0 & s<t\Leftrightarrow s\prec t, \\
                1 & s<t\Leftrightarrow t\prec s.
            \end{cases}
        \]
        Let \(H\subseteq T\) be such that \(F\) is homogeneous on \([H]^{2}\).
        Define
        \[
            B=\setwith{t\in T}{\abs*{\setwith{h\in H}{t<_{T}h}\geq\aleph_{0}}}.
        \]
        Because each level is finite we must have that \(B\cap
        T_{n}\neq\varnothing\) for all levels. If \(t\in B\) and \(s<_{T}t\)
        then \(s\in B\) as well. If \(B\cap T_{n}\) has cardinality \(1\) then
        \(B\) is a branch.

        Suppose the cardinality is not \(1\). Take \(s,t\in B\cap T_{n}\) with
        \(s\prec t\). Then there are \(u,v\in H\) such that \(s<_{T}u\) and
        \(t<_{T}v\) and a \(w\in H\) such that \(s<_{T}w\) and
        \[
            u\prec v\prec w.
        \]
        This means that \(F(\set{u,v})=1\) and \(F(\set{v,w})=0\) which
        contradicts the assumption that \(H\) is homogeneous. This proves
        \(B\cap T_{n}\) has one element.
    \end{proof}
\end{proposition}

\begin{definition}
    Let \(\kappa\) be a cardinal number. An Aronszjajn tree is a tree of
    cardinality \(\kappa\) with levels \(\abs{T_{\alpha}}<\kappa\) and
    no branches of height \(\kappa\).
\end{definition}

\begin{proposition}
    There is an Aronszjajn tree of cardinality \(\aleph_{1}\).

    \begin{proof}
        Consider the set
        \[
            \setwith{s\in\mathbb{Q}^{<\omega_{1}}}{\textnormal{\(s\) is increasing}}.
        \]
        This is uncountable but has no branch of order type \(\omega_{1}\). It
        does have uncountable levels though, so we consider a subtree. We
        construct it maintaining the following invariant: for \(\beta<\alpha\),
        \(x\in U_{\beta}\) and \(q>\sup x\) there is a \(y\in U_{\alpha}\) such
        that \(x\subseteq y\) and \(q\geq\sup y\).

        Define \(T_{0}=\set{\varnothing}\) and given \(T_{\alpha}\) define
        \[
            T_{\alpha+1}=\setwith{s\cup\set{(\alpha+1,x)}}{s\in T_{\alpha},x>s(\alpha)}.
        \]

        For limit ordinals, this takes a bit more work. First of all we see that
        for each \(x\in\bigcup_{\beta<\alpha}T_{\beta}\) and \(q>\sup x\) there
        is some increasing \(\alpha\)-sequence \(y\) such that \(\sup y\leq q\)
        and \(x\subseteq y\).
    \end{proof}
\end{proposition}