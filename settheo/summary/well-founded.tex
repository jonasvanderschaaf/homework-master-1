\section{Well-founded universe}
\begin{definition}
    By ordinal recursion, we define the following sets:
    \begin{align*}
        V_{0}        & =\varnothing                        \\
        V_{\alpha+1} & =\powset(V_{\alpha})                \\
        V_{\lambda}  & =\bigcup_{\alpha<\lambda}V_{\alpha} \\
    \end{align*}
\end{definition}

\begin{theorem}
    The sets \(\sequence{V_{\alpha}}{\alpha\in\ordinals}\) satisfies the
    following properties:
    \begin{enumerate}
        \item \(V_{\alpha}\) is well-founded for all ordinals
              \(\alpha\in\ordinals\),
        \item \(V_{\beta}\subseteq V_{\alpha}\) for all
              \(\beta<\alpha\in\ordinals\).
    \end{enumerate}

    \begin{proof}
        For \(\alpha=0\) both statements are vacuously true.

        If both statements are true for \(\alpha\), consider \(\alpha+1\). The
        set \(V_{\alpha+1}\) is transitive because it is the powerset of a
        transitive set: take any \(x\in V_{\alpha+1}\), then \(x\subseteq
        V_{\alpha}\). This means that for all \(y\in x\) we have \(y\in
        V_{\alpha}\) and by transitivity of \(V_{\alpha}\) also \(y\subseteq
        V_{\alpha}\). This means that \(y\in V_{\alpha+1}\) so \(V_{\alpha+1}\)
        is transitive.

        Now take \(\beta<\alpha+1\), then \(\beta=\alpha\) or \(\beta<\alpha\).
        In the former case \(V_{\beta}\subseteq V_{\alpha}\subseteq
        V_{\alpha+1}\) by transitivity of \(V_{\alpha+1}\). If \(\beta=\alpha\)
        then \(V_{\alpha}\in V_{\alpha+1}=\powset(V_{\alpha})\) by definition.

        For a limit ordinal the induction step is rather trivial. The union of
        transitive sets is obviously transitive and for all \(\alpha<\lambda\)
        we get \(V_{\alpha}\subseteq V_{\lambda}\) by definition.
    \end{proof}
\end{theorem}

\begin{definition}
    We define \(\wf\) to be the class
    \(\bigcup_{\alpha\in\ordinals}V_{\alpha}\). This is called the Von-Neumann
    universe.

    For an \(x\in\wf\) we define the rank of \(x\) as
    \[
        \rank(x)=\min\setwith{\alpha\in\ordinals}{x\in V_{\alpha+1}}
    \]
\end{definition}

\begin{proposition}
    \label{prop:wf-sets-props}
    The following statements follow from the above definition almost
    immediately:
    \begin{enumerate}
        \item \(V_{\alpha}=\setwith{x\in\wf}{\rank(x)<\alpha}\),
        \item For \(x,y\in\wf\) with \(y\in x\) the inequality
              \(\rank(y)<\rank(x)\) holds,
        \item For \(x\in\wf\) we get \(\rank(x)=\sup\setwith{\rank(y)}{y\in
                  x}\),
        \item For all ordinals \(\alpha\in\ordinals\) we see \(\alpha\in\wf\)
              and \(\rank(\alpha)=\alpha\) or equivalently \(\wf\cap
              V_{\alpha}=\alpha\).
    \end{enumerate}
    On top of that the class \(\wf\) is a well-founded transitive model of set
    theory.

    \begin{proof}
        We will only prove the final statement using the properties stated
        above. Pick any set (or class) \(X\subseteq\wf\) and consider the set
        (or class) \(\setwith{\rank(y)}{y\in X}\). This has a minimal element
        \(\alpha\) because the ordinals are well ordered. Let \(y\) be a witness
        of this minimal ordinal. Then \(y\) is minimal by property 2 above.
    \end{proof}
\end{proposition}

\begin{definition}
    We now define another axiom of ZFC, the axiom of regularity:
    \[
        \forall x:(\exists y:y\in x)\to(\exists y\in x: y\cap x=\varnothing).
    \]
    This axiom prevents the existence of a sequence
    \(\sequence{x_{n}}{n\in\omega}\) with \(x_{i+1}\in x_{i}\) for all
    \(i\in\omega\). This is because the set \(x=\setwith{x_{n}}{n\in\omega}\)
    would then not contain any element \(y\) with \(y\cap x=\varnothing\). This
    demonstrates that the \(\in\) relation is well-founded in models of set
    theory where this axiom holds.

    More generally, this axiom says that for any non-empty set \(A\), the
    relation \(\in\) is well-founded. Taking any subset \(B\subseteq A\) there
    is a \(z\in B\) such that there is no \(y\in B\) with \(y\in z\).
\end{definition}

\begin{definition}
    For a set \(A\) we define \(\bigcup^{n}A\) recursively as:
    \begin{align*}
        \invisparen{\bigcup}^{0}A   & =A,                                             \\
        \invisparen{\bigcup}^{n+1}A & =\bigcup\left(\invisparen{\bigcup}^{n}A\right).
    \end{align*}
    Then the transitive closure of \(A\) is then defined as
    \[
        \trcl(A)=\bigcup_{n\in\omega}\invisparen{\bigcup}^{n}A.
    \]
    This is the smallest transitive set containing \(A\).
\end{definition}

\begin{proposition}
    The following statements are equivalent:
    \begin{enumerate}
        \item \(A\in\wf\),
        \item \(\trcl(A)\in\wf\),
        \item \(\in\) is a well-founded relation on \(A\).
    \end{enumerate}

    \begin{proof}
        \(1\Rightarrow3\) This is by \Cref{prop:wf-sets-props}: \(\in\) is a
        well-founded relation on all sets in \(\wf\).

        \(3\Rightarrow1\) Let \(A\) be a transitive set, suppose
        \(A\setminus\wf\) is non-empty, and let \(x\in A\setminus\wf\) be a
        \(\in\)-minimal element of this set. Then for all \(y\in x\) it must be
        that \(y\in\wf\), this gives the desired contradiction: \(A\) is
        transitive and so \(y\in A\) for all these \(y\in x\). The set
        \(\setwith{\rank(y)}{y\in x}\) must have a supremum which is the rank of
        \(x\notin\wf\). This is a contradiction so no such \(A\) exists.

        \(1\Rightarrow2\) The class \(\wf\) is transitive and contains \(A\).
        Because \(\trcl(A)\) is the smallest transitive set containing \(A\) we
        have \(\trcl(A)\subseteq\wf\). This means \(\trcl(A)\subseteq
        V_{\alpha}\) for some \(\alpha\in\ordinals\) so \(\trcl(A)\in
        V_{\alpha+1}\subseteq\wf\).

        \(2\Rightarrow1\) The set \(A\) is a subset of \(\trcl(A)\in\wf\) so it
        has rank strictly smaller than \(\trcl(A)\) and so particular
        \(A\in\wf\).
    \end{proof}
\end{proposition}

\begin{corollary}
    If the axiom of regularity holds, \(\wf\) contains all sets \(A\).

    \begin{proof}
        If the axiom of regularity holds, \(\in\) is well-founded on all sets \(A\)
        so we get \(A\in\wf\) by the previous proposition.
    \end{proof}
\end{corollary}

The axiom of regularity has many applications within set theory. For example:
\begin{enumerate}
    \item In ZF with regularity the axiom of choice is equivalent to the
          statement that \(\powset(\alpha)\) can be well-ordered for all
          ordinals \(\alpha\),
    \item Scott's trick: For an equivalence relation on the whole universe (such
          as cardinality) it is generally impossible to pick a representative.
          Using foundation it is however possible to pick a set of
          representatives: If \([a]\) is an equivalence class\footnote{Note that
              this is possibly the first case I've seen where an equivalence class
              could actually be a class.} then there is some least
          \(\alpha\in\ordinals\) such that \(V_{\alpha}\cap[a]\) is non-empty.
          This intersection is then a set of representatives of \([a]\).
\end{enumerate}