\documentclass{article}

\usepackage[utf8]{inputenc}
\usepackage[english]{babel}
\usepackage[shortlabels]{enumitem}
\usepackage{amsthm, amssymb, mathtools, amsmath, bbm, mathrsfs,stmaryrd}
\usepackage[margin=1in]{geometry}
\usepackage[parfill]{parskip}
\usepackage[hidelinks]{hyperref}

\usepackage{caption}
\usepackage{subcaption}
\usepackage{tikz}

\newtheorem*{lemma}{Lemma}

\theoremstyle{definition}
\newtheorem{question}{Exercise}
\newtheorem*{corollary}{Corollary}

\newcommand{\set}[1]{\left\{#1\right\}}
\newcommand{\setwith}[2]{\set{#1\colon#2}}

\newcommand{\powset}{\mathcal{P}}
\DeclareMathOperator{\dom}{dom}
\DeclareMathOperator{\range}{ran}
\DeclareMathOperator{\field}{field}

\newcommand{\restrict}{\upharpoonright}

\DeclarePairedDelimiter{\abs}{|}{|}

\title{Homework Set Theory}
\date{October 11, 2022}
\author{Jonas van der Schaaf}

\begin{document}
\maketitle

\begin{question}
    \begin{enumerate}[a.]
        \item Prove the axiom of choice implies \(T\).

              \begin{proof}
                  Let \(x,y\) be two sets. By the axiom of choice these can be
                  well-ordered by relations the relations \(<,\prec\)
                  respectively. Then as we have seen before \(x\) is an initial
                  segment of \(y\), \(y\) is an initial segment of \(x\), or
                  \(x\cong y\).

                  If \(x\) is an initial segment of \(y\), then we have the
                  injective embedding \(x\hookrightarrow y\) mapping \(x\) to
                  its isomorphic initial segment in \(y\). If \(x\cong y\) then
                  in particular the isomorphism is injective which gives the
                  injective map. The last case is identical to the first with
                  the roles of \(x,y\) switched around.
              \end{proof}

        \item For every set \(x\) there is an ordinal \(\alpha\) such that there
              is no injective map \(\alpha\hookrightarrow x\).

              \begin{proof}
                  Consider the set
                  \[
                      A=\setwith{(y,R)\in\powset(x)\times\powset(x\times x)}{\text{\(R\) is a well-order on \(y\)}}.
                  \]
                  Then each pair \((y,R)\) is isomorphic to a unique ordinal
                  \(\beta\). Using the axiom of replacement, this means there is
                  a unique set of ordinals \(A'\) which is the image of \(A\)
                  under the class function mapping to the unique ordinal.

                  This set is transitive for the following reason: take any
                  \(\beta\in A'\) and \(\gamma\in\beta\). Then
                  \(\gamma\subseteq\beta\) so \(\gamma\) is well-orderable by
                  the induced relation. Because \(\beta\) is equipotent to a
                  subset of \(x\) we get that \(\gamma\subseteq\beta\) is as
                  well to some \(y\subseteq x\). This \(y\) can be endowed with
                  the well-order induced by the bijection to \(\gamma\).
                  Therefore \(\gamma\in A'\). This means \(A'\) is a transitive
                  set of ordinals which can be well-ordered by \(\in\).
                  Therefore \(\alpha:=A'\) is an ordinal.


                  Now suppose there is an injection \(f:\alpha\hookrightarrow
                  x\). Then \(f[\alpha]\subseteq x\) can be well-ordered with
                  the well-order induced by \(f\). In particular
                  \(f[\alpha]\cong \alpha\) as a well-order. This means that as
                  an ordinal \(\alpha\in\alpha\) which contradicts the fact that
                  \(\in\) well-orders the ordinals.
              \end{proof}

        \item Prove \(T\) implies the axiom of choice.

              \begin{proof}
                  Consider any set \(x\) and take \(\alpha\) as in part b. Then
                  there is no injective \(\alpha\hookrightarrow x\), therefore
                  there is an injective \(x\hookrightarrow\alpha\). This induces
                  a well-order on \(x\). Therefore any set can be well-ordered.
                  This is equivalent to the axiom of choice.
              \end{proof}
    \end{enumerate}
\end{question}

\begin{question}
    Let \(\gamma:\powset(M)\setminus\set{\varnothing}\to M\) be a choice
    function. A subfamily \(\mathcal{A}\) of \(\powset(M)\) is a
    \(\gamma\)-chain if
    \begin{enumerate}[(1)]
        \item \(M\in\mathcal{A}\),
        \item if \(A\in\mathcal{A}\) then \(A'=A\setminus\set{\gamma(A)}\)
              belongs to \(\mathcal{A}\) too,
        \item if \(\mathcal{A}'\subseteq\mathcal{A}\) then
              \(\bigcap\mathcal{A}'\in\mathcal{A}\).
    \end{enumerate}

    \begin{enumerate}[a.]
        \item Verify that \(\powset(M)\) is a \(\gamma\)-chain.

              \begin{proof}
                  We have \(M\subset M\) so \(M\in\powset(M)\). For any
                  \(A\subseteq M\) we have
                  \(A'=A\setminus\set{\gamma(A)}\subseteq M\) so
                  \(A'\in\powset(M)\). If \(\mathcal{A}'\subseteq\powset(A)\)
                  then each \(A\in\mathcal{A}'\) is an element of
                  \(\powset(M)\). Therefore
                  \(\bigcap\mathcal{A}'\in\powset(M)\).
              \end{proof}

        \item Show if \(\mathfrak{U}\) is a family of \(\gamma\)-chains, then
              \(\bigcap\mathfrak{U}\) is a \(\gamma\)-chain.

              \begin{proof}
                  Because each \(\mathcal{A}\in\mathfrak{U}\) is a
                  \(\gamma\)-chain, \(M\in\mathcal{A}\) for each
                  \(\mathcal{A}\in\mathfrak{U}\). Therefore
                  \(M\in\bigcap\mathfrak{U}\).

                  If \(A\in\mathcal{A}\) for each \(\mathcal{A}\in\mathfrak{U}\)
                  then also \(A'=A\setminus\set{\gamma(A)}\in\mathcal{A}\) for
                  each chain. Therefore \(A'\in\bigcap\mathfrak{U}\).

                  If \(\mathcal{A}'\subseteq\bigcap\mathfrak{U}\), then
                  \(\mathcal{A}'\subseteq\mathcal{A}\) for each
                  \(\mathcal{A}\in\mathfrak{U}\). Therefore
                  \(\bigcap\mathcal{A}'\in\mathcal{A}\) for each
                  \(\mathcal{A}\in\mathfrak{U}\). Therefore
                  \(\bigcap\mathcal{A}\in\bigcap\mathfrak{U}\).
              \end{proof}
    \end{enumerate}
    Now let \(\mathcal{W}\) be the intersection of the collection of all
    \(\gamma\)-chains. Call \(A\in\mathcal{W}\) comparable if for all
    \(U\in\mathcal{W}\) \(A\subseteq U\) or \(U\subseteq A\).
    \begin{enumerate}[a.,resume]
        \item Prove every member of \(\mathcal{W}\) is comparable.

              \begin{proof}
                  Consider the set \(\mathcal{W}'=\setwith{A\in
                      \mathcal{W}}{\text{\(A\) is comparable}}\). Then this is a
                  \(\gamma\)-chain. It is clear that \(M\) is comparable
                  because every possible element of a \(\gamma\)-chain is a
                  subset of \(M\).

                  Take \(A\in\mathcal{W}\), define
                  \(A'=A\setminus\set{\gamma(A)}\) and consider
                  \(\mathcal{U}=\setwith{U\in\mathcal{W}}{U\subseteq A'\vee
                      A'\subseteq U}\). I will prove this is a chain.

                  \begin{lemma}
                      The set \(\mathcal{U}\) is a chain.

                      \begin{proof}
                          It is clear that \(M\in\mathcal{U}\).

                          If \(U\in\mathcal{U}\) then either \(U\subseteq A'\)
                          or \(A'\subseteq U\subseteq A\) or \(A'\subseteq
                          A\subseteq U'\). If \(U\subseteq A'\) also
                          \(U'\subseteq A'\). If \(A'\subseteq U\subseteq A\)
                          either \(A'=U\) in which case \(U'\subseteq A'\) or
                          \(U=A\) in which case \(U'=A'\). If \(A'\subseteq
                          A\subseteq U\) then we have \(A\subseteq U'\) in which
                          case \(A'\subseteq U'\) or \(U'\subseteq A\subseteq
                          U\) which means that \(U'=A\) or \(U=A\). In these
                          cases we either get \(A'\subseteq U'\) or \(U'=A'\)
                          respectively so \(U'\in\mathcal{U}\).

                          Now take any \(\mathcal{U}'\subseteq \mathcal{U}\).
                          Then \(\bigcap\mathcal{U}'\in\mathcal{W}\) because it
                          is a chain. If for each \(U\in\mathcal{U'}\) we have
                          \(A'\subseteq U\) then
                          \(A'\subseteq\bigcap\mathcal{U'}\) so
                          \(\bigcap\mathcal{U'}\in\mathcal{U}\). If there is any
                          \(U\in\mathcal{U'}\) with \(U\subsetneq A'\) then
                          \(\bigcap\mathcal{U}'\subseteq A'\) so
                          \(\bigcap\mathcal{U}'\in\mathcal{U}\).

                          This proves all properties of a \(\gamma\) chain hold
                          for \(\mathcal{U}\).
                      \end{proof}
                  \end{lemma}

                  Therefore \(\mathcal{U}\subseteq\mathcal{W}\) is a chain so
                  \(\mathcal{U}=\mathcal{W}\) and \(A'\) is comparable.

                  Take any \(\mathcal{A}'\subseteq\mathcal{W}'\). Then
                  \(\bigcap\mathcal{A}'\in\mathcal{W}\). Take any
                  \(U\in\mathcal{W}\). If for all \(A\in\mathcal{A}'\) we have
                  \(U\subseteq A\) we get \(U\subseteq\bigcap\mathcal{A}'\). If
                  there is an \(A\in\mathcal{A}'\) such that \(A\subsetneq U\)
                  then \(\bigcap\mathcal{A}'\subseteq U\). This means that
                  \(\bigcap\mathcal{A}'\) is comparable so an element of
                  \(\mathcal{W}'\).

                  This means that \(\mathcal{W}'\subseteq\mathcal{W}\) is a
                  chain so \(\mathcal{W}=\mathcal{W}'\) i.e. every element of
                  \(\mathcal{W}\) is comparable.
              \end{proof}

        \item Let \(Y\subseteq M\) be non-empty. Show there is a unique member
              \(W\in\mathcal{W}\) such that \(Y\subseteq W\) and \(\gamma(W)\in
              Y\).

              \begin{proof}
                  Consider \(W=\bigcap\setwith{W'\in\mathcal{W}}{Y\subseteq
                      W'}\). Then \(W\in\mathcal{W}\) by property 3 of
                  \(\gamma\)-chains. If \(\gamma(W)\notin Y\) then
                  \(Y\subseteq W'=W\setminus\set{\gamma(W)}\subsetneq W\).
                  This means that \(W\subseteq W'\subsetneq W\) which is
                  impossible.

                  To prove uniqueness, take two \(W,X\in\mathcal{W}\) with the
                  desired properties. Then without loss of generality
                  \(W\subseteq X\) by comparability. We get \(W\nsubseteq X'\)
                  because \(X\not\ni\gamma(X)\in Y\subseteq W\). By
                  comparibility this means \(X'\subsetneq W\) because
                  \(\gamma(X)\notin X\). Therefore \(X'\subsetneq W\subseteq X\)
                  so \(W=X\) which proves uniqueness.
              \end{proof}
    \end{enumerate}
    In particular for each \(x\in M\) we have a unique \(W_{x}\in\mathcal{W}\)
    such that \(x\in W_{x}\) and \(\gamma(W_{x})=x\). Define \(x\prec y\) iff
    \(y\in W_{x}'\).
    \begin{enumerate}[a.,resume]
        \item Prove that \(\prec\) is a well-order on \(M\).

              \begin{proof}
                  Let \(S\subset M\). By part d there is a unique
                  \(W\in\mathcal{W}\) such that \(S\subseteq W\) and
                  \(\gamma(W)\in S\). This \(x:=\gamma(S)\) will be the minimal
                  element.

                  This unique \(W\) must also be \(W_{x}\) because it satisfies
                  the same properties. Therefore for any \(y\in
                  S\setminus\set{x}\) we have that \(y\in
                  W_{x}\setminus\set{x}\). This means that \(x\prec y\)
                  for all \(y\in S\) different from \(x\).
              \end{proof}
    \end{enumerate}
\end{question}
\end{document}