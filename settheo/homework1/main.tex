\documentclass{article}

\usepackage[utf8]{inputenc}
\usepackage[english]{babel}
\usepackage[shortlabels]{enumitem}
\usepackage{amsthm, amssymb, mathtools, amsmath, bbm, mathrsfs,stmaryrd}
\usepackage[margin=1in]{geometry}
\usepackage[parfill]{parskip}
\usepackage[hidelinks]{hyperref}

\usepackage{caption}
\usepackage{subcaption}
\usepackage{tikz}

\newtheorem*{lemma}{Lemma}

\theoremstyle{definition}
\newtheorem{question}{Exercise}
\newtheorem*{corollary}{Corollary}

\newcommand{\set}[1]{\left\{#1\right\}}
\newcommand{\setwith}[2]{\set{#1\colon#2}}

\newcommand{\powset}{\mathcal{P}}
\DeclareMathOperator{\dom}{dom}
\DeclareMathOperator{\range}{ran}
\DeclareMathOperator{\field}{field}

\newcommand{\restrict}{\upharpoonright}

\DeclarePairedDelimiter{\abs}{|}{|}

\title{Homework Set Theory}
\date{\today}
\author{Jonas van der Schaaf}

\begin{document}
\maketitle

\begin{question}
    Prove: if \(x\in Z_{0}\), then \(x=\varnothing\) or \(x=\set{y}\) for some
    \(y\in Z_{0}\).

    \begin{proof}
        Take \(x\in Z_{0}\) and suppose it is not \(\varnothing\) or \(\set{y}\)
        for some \(y\in Z_{0}\). Then \(Z_{0}\setminus\set{x}\) is still closed
        under the map \(x\mapsto\set{x}\) but is strictly smaller. This
        contradicts the assumption that \(Z_{0}\) is the smallest set with these
        desired properties.
    \end{proof}
\end{question}

\begin{question}
    Prove that \(A_{x}\) is indeed defined for all \(x\in Z_{0}\) and that
    \(A_{x}\subseteq Z_{0}\) for all \(x\in Z_{0}\).

    \begin{proof}
        Take \(Z'=\setwith{x\in Z_{0}}{\textnormal{\(A_{x}\) exists}}\).

        Then \(\varnothing\in Z'\) because \(\varnothing\) exists. Also if
        \(A_{x}\) exists, then \(\set{x}\) exists by axiom of pair, so by axiom
        of pair and union \(A_{\set{x}}=A_{x}\cup\set{x}\) exists as well.
        Because \(Z_{0}\) is the smallest set with the desired property we have
        \(Z_{0}\subseteq Z'\) so \(Z'=Z_{0}\).

        To show \(A_{\set{x}}\subseteq Z_{0}\), consider the set
        \(Z'=\setwith{x\in Z_{0}}{A_{x}\subseteq Z_{0}}\). We know that
        \(A_{\varnothing}=\varnothing\subseteq Z_{0}\). On top of that if
        \(A_{x}\subseteq Z_{0}\) for \(z\in Z_{0}\), then
        \(A_{\set{x}}=A_{x}\cup\set{x}\subseteq Z_{0}\). By the same reasoning
        as before we get equality of the sets \(Z'\) and \(Z_{0}\):
        \(Z'=Z_{0}\).
    \end{proof}
\end{question}

\begin{question}
    Prove that for all \(x\in Z_{0}\) we have: if \(y\in A_{x}\), then
    \(y=\varnothing\) or there is a \(z\in A_{x}\) such that \(y=\set{z}\).

    \begin{proof}
        Define
        \[
            Z'=\setwith{x\in Z_{0}}{\forall y\in A_{x}:y=\varnothing\vee\exists z\in A_{x}:y=\set{z}}.
        \]

        Then \(\varnothing\in Z'\) because the required property is vacuously
        true.

        Suppose \(x\in Z'\). Then consider \(A_{\set{x}}=A_{x}\cup\set{x}\). The
        statement is true by assumption for all \(y\in A_{x}\). If \(y=x\) then
        there is some \(z\in Z_{0}\) such that \(x=\set{z}\). Then
        \(A_{\set{z}}=A_{x}\) and \(z\in A_{x}\). This is the necessary element
        \(z\) of \(A_{\set{x}}\). This means that \(Z'\) is closed under the map
        \(z\mapsto\set{z}\) so it is inductive and \(Z'=Z_{0}\).
    \end{proof}
\end{question}

\begin{question}
    Prove: for all \(x\in Z_{0}\) we have: if \(y\in A_{x}\) then
    \(A_{y}\subseteq A_{x}\).

    \begin{proof}
        Consider the set
        \[
            Z'=\setwith{x\in Z_{0}}{\forall y\in A_{x}:A_{y}\subseteq A_{x}}.
        \]

        We have \(\varnothing\in Z'\) because \(A_{\varnothing}=\varnothing\) so
        the statement is vacuously true.

        Now suppose the statement is true for some \(x\) and consider \(y\in
        A_{\set{x}}=A_{x}\cup\set{x}\). Then \(y\in A_{x}\) in which case
        \(A_{y}\subseteq A_{x}\subseteq A_{\set{x}}\) by assumption. If \(y=x\)
        then \(A_{x}\subseteq A_{x}\cup\set{x}=A_{\set{x}}\) so \(\set{x}\in
        Z'\). Yet again by the same reasoning therefore \(Z'=Z_{0}\).
    \end{proof}
\end{question}

\begin{lemma}
    For any \(x,y\in Z_{0}\) if \(A_{x}=A_{y}\) then \(x=y\).

    \begin{proof}
        Consider the set \(Z'=\setwith{x\in Z_{0}}{\forall y\in
        Z_{0}:A_{x}=A_{y}\Leftarrow x=y}\).

        Then \(\varnothing\in Z'\) because only \(A_{\varnothing}\) is empty.

        If \(x\in Z'\) is nonempty then consider \(A_{\set{x}}\). If
        \(A_{y}=A_{\set{x}}\) then \(y\neq\varnothing\) by the base case of the
        induction. Therefore there is a \(z\in Z_{0}\) such that \(y=\set{z}\).

        This means that \(A_{\set{z}}=A_{\set{x}}=A_{x}\cup\set{x}\). This means
        that \(z\in A_{x}\) or \(z=x\). Either way we get \(A_{z}\subseteq
        A_{x}\). By the same reasoning we also obtain \(A_{x}\subseteq A_{z}\).
        This means that \(x=z\) so \(\set{x}=\set{z}=y\). Therefore \(\set{x}\in
        Z'\).
    \end{proof}
\end{lemma}

\begin{lemma}
    If for any two \(x,y\in Z_{0}\) with \(A_{y}\subseteq A_{x}\) and \(x\neq
    y\) then \(y\in A_{x}\).

    \begin{proof}
        Define \(Z'=\setwith{y\in Z_{0}}{\forall x\in Z_{0}:A_{x}\subseteq
        A_{y}\wedge x\neq y\to x\in y}\).

        Then trivially \(\varnothing\in Z'\). Now assume \(x\in Z'\) and
        consider \(\set{x}\). Take any \(y\in Z_{0}\) such that \(A_{y}\subseteq
        A_{\set{x}}\) and \(y\neq\set{x}\). We consider two separate cases
        \(x\in A_{y}\) or \(x\notin A_{y}\).

        If \(x\in A_{y}\) then \(A_{x}\subseteq A_{y}\) and \(\set{x}\subseteq
        A_{y}\). Therefore \(A_{\set{x}}\subseteq A_{y}\) so
        \(A_{\set{x}}=A_{y}\). By the previous lemma this gives that
        \(\set{x}=y\) which contradicts the assumption.

        If \(x\notin A_{y}\) then because we assumed \(A_{y}\subseteq
        A_{x}\cup\set{x}\) we know that \(A_{y}\subseteq A_{x}\). By induction
        this means that \(y\in A_{x}\) so also \(y\in A_{\set{x}}\).
    \end{proof}
\end{lemma}

\begin{question}
    Prove: for all \(x,y\in Z_{0}\) we have \(A_{x}\subseteq A_{y}\) or
    \(A_{y}\subseteq A_{x}\).

    \begin{proof}
        Define
        \[
            Z'=\setwith{x\in Z'}{\forall y\in Z_{0}:A_{x}\subseteq A_{y}\vee A_{y}\subseteq A_{x}}.
        \]

        Then trivially \(\varnothing\in Z'\) because
        \(A_{\varnothing}=\varnothing\) is trivially a subset of all sets.

        Now suppose the statement is true for \(x\in Z_{0}\) and consider
        \(\set{x}\in Z_{0}\). Suppose there is a \(y\in Z_{0}\) such that
        neither \(A_{\set{x}}\) nor \(A_{y}\) are a subset of the other. We can
        assume that \(x\neq y\) because else \(A_{y}=A_{x}\subseteq
        A_{\set{x}}\) which contradicts the assumption. It also can't be the
        case that \(A_{y}\subseteq A_{x}\) because then \(A_{y}\subseteq
        A_{\set{x}}\) by transitivity of \(\subseteq\). Therefore
        \(A_{x}\subseteq A_{y}\) but \(A_{\set{x}}\nsubseteq A_{y}\). This means
        that \(x\notin A_{y}\). However because \(A_{x}\subseteq A_{y}\) by
        contraposition of the previous lemma we have that \(x=y\). Therefore
        \(A_{\set{x}}\supseteq A_{y}\) which is a contradiction with the
        assumption that this wasn't true. Therefore \(\set{x}\in Z'\) so \(Z'\)
        is inductive so \(Z'=Z_{0}\).
    \end{proof}
\end{question}

\begin{question}
    Prove: if \(x\in Z_{0}\) and \(A\subseteq A_{x}\) is nonempty then there is
    a \(y\in A\) such that \(A_{y}\subseteq A_{z}\) for all \(z\in A\).

    \begin{proof}
        Consider the set \(Z'=\setwith{x\in Z_{0}}{\forall A\subseteq
            A_{x}\exists y\in A\forall z\in A: A_{y}\subseteq A_{z}}\). Then
        vacuously \(\varnothing\in Z'\). Also if \(x\in Z'\) consider
        \(A\subseteq A_{\set{x}}\). If \(A\subseteq A_{x}\) then we are
        finished by assumption.

        If not then \(x\in A\). If \(A=\set{x}\) then \(x\) is clearly the
        minimal element such that all \(z\in A\) have \(A_{x}\subseteq A_{z}\).
        If not then there is some \(z\in A\cap A_{x}\subseteq A_{\set{x}}\) such
        that for all \(y\in A\cap A_{x}\) we have \(A_{z}\subseteq A_{y}\).
        Because \(z\in A_{x}\subseteq A_{\set{x}}\) we also have
        \(A_{z}\subseteq A_{\set{x}}\) so \(z\) has the desired property for all
        \(y\in A\).
    \end{proof}
\end{question}

\begin{question}
    Prove if \(A\subseteq Z_{0}\) is nonempty then there is a \(y\in A\) such
    that for all \(z\in A\) with \(A_{y}\subseteq A_{z}\).

    \begin{proof}
        Take any \(y\in A\). If \(A_{y}\cap A=\varnothing\) then by question 5,
        we have \(A_{z}\subseteq A_{y}\) or \(A_{y}\subseteq A_{x}\) for any
        \(z\in A\). It is clear that \(A_{y}\subseteq A_{y}\), but if \(x\neq
        y\) and \(A_{y}\nsubseteq A_{z}\) then \(A_{z}\subseteq A_{y}\) so by
        the previously proven lemma \(z\in A_{y}\). This contradicts the
        assumption that \(A_{y}\cap A=\varnothing\) so no such \(z\) can exist.

        If \(A_{y}\cap A\neq\varnothing\) then by question \(6\) there is an
        \(x\in A_{y}\) such that \(A_{x}\subseteq A_{z}\) for all \(z\in
        A_{y}\). Take any \(z\neq y\in A\) then if \(z\in A_{y}\) we know
        \(A_{x}\subseteq A_{z}\). If \(z\notin A_{y}\) then we know that
        \(A_{y}\subseteq A_{z}\) so \(A_{x}\subseteq A_{y}\subseteq A_{z}\).

        This proves that \(x\) is the minimal element.
    \end{proof}
\end{question}

\begin{question}
    Prove the existence of \(x\times y\) by applying the Power Set Axiom and a
    suitable instance of the Separation axiom.

    \begin{proof}
        Consider the set \(Z=\powset(\bigcup\set{x,y})\). Then for all \(a\in
        x\) the set \(\set{a}\) is an element of \(Z\). Also for all \(a\in x\)
        and \(b\in y\) and \(\set{a,b}\in Z\). Therefore for any \(a\in x\) and
        \(b\in y\) it must be true that
        \(\set{\set{a},\set{a,b}}\in\powset(Z)\). Therefore we can define
        \[
            x\times y=\setwith{c\in \powset(Z)}{\exists a\in x:\exists b\in y:c=\set{\set{a},\set{a,b}}}.
        \]
        Every element of \(x\times y\) is clearly a tuple, and by the
        aforementioned reasoning all tuples are in the set.
    \end{proof}
\end{question}
\end{document}