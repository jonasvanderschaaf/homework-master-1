\documentclass{article}

\usepackage[utf8]{inputenc}
\usepackage[english]{babel}
\usepackage[shortlabels]{enumitem}
\usepackage{amsthm, amssymb, mathtools, amsmath, bbm, mathrsfs,stmaryrd}
\usepackage[margin=1in]{geometry}
\usepackage[parfill]{parskip}
\usepackage[hidelinks]{hyperref}
\usepackage{cleveref}

\usepackage{caption}
\usepackage{subcaption}
\usepackage{tikz}

\newtheorem*{lemma}{Lemma}

\theoremstyle{definition}
\newtheorem{question}{Exercise}
\newtheorem*{corollary}{Corollary}

\newcommand{\set}[1]{\left\{#1\right\}}
\newcommand{\setwith}[2]{\set{#1\colon#2}}
\newcommand{\sequence}[2]{\left<#1:#2\right>}

\newcommand{\powset}{\mathcal{P}}
\DeclareMathOperator{\dom}{dom}
\DeclareMathOperator{\range}{ran}
\DeclareMathOperator{\field}{field}
\DeclareMathOperator{\cof}{cf}
\DeclareMathOperator{\trcl}{trcl}
\newcommand{\ordinals}{{\bf On}}

\newcommand{\restrict}{\upharpoonright}

\DeclarePairedDelimiter{\abs}{|}{|}

\newenvironment{solution}{\begin{proof}[Solution]}{\end{proof}}

\title{Homework Set Theory}
\date{Novenber 21, 2022}
\author{Jonas van der Schaaf}

\begin{document}
\maketitle

\begin{question}
    For an infinite cardinal \(\kappa\) let
    \(H(\kappa)=\setwith{x}{\abs{\trcl(x)}<\kappa}\). Prove the following about
    \(H(\kappa)\).

    \begin{enumerate}[a.]
        \item \(H(\kappa)\) is transitive.

              \begin{proof}
                  Let \(x\in H(\kappa)\), i.e. \(\trcl(x)<\kappa\). Then for
                  \(y\in x\) we have \(\trcl{y}\subseteq\trcl{x}\) so
                  \(\abs{\trcl{y}}<\kappa\). This proves that \(H(\kappa)\) is
                  transitive.
              \end{proof}

        \item \(H(\kappa)\cap \ordinals=\kappa\).

              \begin{proof}
                  For any ordinal \(\alpha\) we have \(\trcl(\alpha)=\alpha\)
                  because ordinals are transitive. This proves \(\alpha\in
                  H(\kappa)\) iff
                  \begin{align*}
                      \abs{\trcl{\alpha}} & =\abs{\alpha} \\
                                          & =\alpha       \\
                                          & <\kappa.
                  \end{align*}
              \end{proof}

        \item If \(x\in H(\kappa)\) and \(y\subseteq x\) then \(y\in
              H(\kappa)\).

              \begin{proof}
                  If \(x\in H(\kappa)\) and \(y\subseteq x\) then
                  \(\trcl(y)\subseteq\trcl{x}\) so
                  \(\abs{\trcl(y)}\leq\abs{\trcl{x}}<\kappa\) so \(y\in
                  H(\kappa)\).
              \end{proof}

        \item Show \(H(\kappa)\) is closed under the Gödel operations.

              \begin{proof}
                  For \(x,y\in H(\kappa)\) the set
                  \(\trcl(\setwith{x,y})=\trcl{x}\cup\trcl{y}\cup\set{x,y}\).
                  We have \(\abs{trcl(x)},\abs{\trcl(y)}<\kappa\) so
                  \(\abs{\trcl{x,y}}\leq\abs{trcl(x)}+\abs{\trcl(y)}+2<\kappa\).
                  This proves \(H\) is closed under \(G_{1}\).

                  For \(x,y\in H(\kappa)\) the transitive closure of \(x\times
                  y\) is given by
                  \[
                      \trcl{x}\cup\trcl{y}\cup\set{x\times y}\cup x\times y\cup\setwith{\trcl((a,b))}{a\in x, b\in y}.
                  \]
                  Each of these sets has cardinality less than kappa (in
                  particular the tuples have this because it's the transitive
                  closure of their coordinates with two extra sets). Therefore
                  the total transitive closure has cardinality less than
                  \(\kappa\) and therefore \(H(\kappa)\) is closed under
                  \(G_{2}\).

                  The set \(G_{3}(x,y)\) is a subset of \(G_{2}(x,y)\) and
                  therefore its transitive closure is a subset of the
                  aforementioned one. This means its cardinality is also less
                  than \(\kappa\) so \(H(\kappa)\) is closed under \(G_{3}\).

                  If \(x,y\in H(\kappa)\) then \(\trcl{x-y},\trcl{x\cap
                      y}\subseteq\trcl{x}\) which has cardinality less than
                  \(\kappa\) so \(\trcl{x-y}\) and \(\trcl{x\cap y}\) do as
                  well. This means \(H(\kappa)\) is closed under \(G_{4}\) and
                  \(G_{5}\).

                  If \(x\in H(\kappa)\) then \(\trcl{\bigcup
                      x}\subseteq\trcl{x}\). This means the former has cardinality
                  of at most the latter one, which is less than \(\kappa\).
                  Therefore \(H(\kappa)\) is closed under \(G_{6}\).

                  For any set \(x\) and a tuple \((a,b)\in x\) \(a\in\trcl{x}\)
                  so \(\trcl{\dom(x)}\subseteq\trcl{x}\) the latter of which has
                  cardinality less than \(\kappa\), so \(\trcl{\dom(x)}\) does
                  as well. This proves \(H(\kappa)\) is closed under \(G_{7}\).

                  Careful examination shows that for any tuple \((a,b)\) we have
                  \(\trcl((a,b))\setminus\set{(a,b)}=\trcl((b,a))\setminus\set{(b,a)}\).
                  This proves that the only difference in the transitive closure
                  of \(x\) and \(G_{8}(x)\) is the set \(\setwith{z\in
                      x}{\textnormal{\(z\) is a tuple}}\) which is replaced with
                  \(\setwith{z\in x}{\textnormal{\(z\) is a tuple with flipped
                          coordinates in \(x\)}}\). Because flipping coordinates is
                  invertible this means that the transitive closures of \(x\)
                  and \(G_{8}(x)\) have the same cardinality so \(H(\kappa)\) is
                  closed under \(G_{8}\).

                  For a tuple
                  \((x,y,z)=(x,(y,z))=\set{x,\set{x,\set{y,\set{y,z}}}}\)
                  swapping \(y\) and \(z\) results in the set written as
                  \(\set{x,\set{x,\set{z,\set{y,z}}}}\). The transitive closure
                  of this swapped tuple \((x,(z,y))\) is given by
                  \[
                      \trcl(x)\cup\trcl(y)\cup\trcl(z)\cup
                      \set{\set{x,\set{x,\set{y,\set{y,z}}}},\set{y,\set{y,z}},\set{y,z}}
                  \]
                  Because the last set of that union has cardinality \(3\) we
                  can easily see that
                  \begin{align*}
                      \abs{\trcl(G_{9}(x))} & \leq\abs{\trcl(x)}+\abs{3\cdot x} \\
                                            & <\kappa+\kappa                    \\
                                            & =\kappa.
                  \end{align*}

                  Similarly for a tuple \((x,y,z)\) applying \(G_{10}\) gives
                  \((y,z,x)=\set{y,\set{y,\set{z,\set{z,x}}}}\). The transitive
                  closure of this is then
                  \[
                      \trcl(x)\cup\trcl(y)\cup\trcl(z)\cup\set{\set{y,\set{y,\set{z,\set{z,x}}}},\set{y,\set{z,\set{z,x}}},\set{z,x}}.
                  \]
                  Therefore
                  \begin{align*}
                      \abs{\trcl{G_{10}(x)}} & \leq\abs{\trcl(x)}+\abs{3\cdot x} \\
                                             & <\kappa+\kappa                    \\
                                             & =\kappa.
                  \end{align*}

                  This proves \(H(\kappa)\) is closed under \(G_{9}\) and
                  \(G_{10}\) as well so \(H(\kappa)\) is closed under all Gödel
                  operations.
              \end{proof}

        \item (Using choice) If \(\kappa\) is regular then \(x\in H(\kappa)\) if
              and only if \(x\subseteq H(\kappa)\) and \(\abs{x}\leq\kappa\).

              \begin{proof}
                  For \(x\in H(\kappa)\) we have \(x\subseteq\trcl(x)\) so
                  \(\abs{x}\leq\abs{\trcl(x)}<\kappa\). Because \(H(\kappa)\) is
                  transitive we also immediately get \(x\subseteq H(x)\).

                  Conversely let \(x\subseteq H(\kappa)\) be a set with
                  \(\abs{x}<\kappa\). Then
                  \(\trcl{x}=\bigcup\setwith{\trcl{y}}{y\in x}\cup{x}\). However
                  because \(\abs{x}<\kappa\) and \(\trcl{y}<\kappa\) for all
                  \(y\in x\) the set \(\setwith{\abs{\trcl{y}}}{y\in x}\) must
                  have some supremum \(\lambda<\kappa\).

                  Then we have
                  \begin{align*}
                      \abs{\trcl{x}} & =\abs{\bigcup_{y\in x} \trcl{y}\cup\set{x}} \\
                                     & =\abs{\bigcup_{y\in x} \trcl{y}}+1          \\
                                     & \leq\sum_{y\in x}\abs{\trcl{y}}             \\
                                     & =\abs{x}\lambda                             \\
                                     & <\kappa
                  \end{align*}
                  so \(x\in\trcl{\kappa}\).
              \end{proof}

        \item (Using choice) If \(\kappa\) is regular and uncountable then
              \(H(\kappa)\) is a model of ZFC-P.

              \begin{proof}
                  The axiom of extensionality is immediately satisfied because
                  it is expressed in terms of only universal quantifiers and
                  therefore preserved under substructure.

                  The axiom of union and pair are satisfied because
                  \(H(\kappa)\) is closed under the Gödel operations.

                  The axiom of regularity is satisfied because
                  \(H(\kappa)\subseteq V\) is a subset of the well-founded
                  universe because the class of sets is well-founded.

                  The axiom of infinity is satisfied because
                  \(\abs{\trcl{\omega}}=\abs{\omega}=\aleph_{0}<\kappa\).

                  The axiom schema of separation is satisfied because for any
                  \(y\subseteq x\subseteq H(\kappa)\) with \(\abs{x}<\kappa\) we
                  have that \(\abs{y}\leq\abs{x}<\kappa\) and \(y\subseteq
                  H(\kappa)\) so \(y\in H(\kappa)\).

                  The axiom of choice is satisfied because for any set \(x\) and
                  any choice function \(f\) has cardinality \(\abs{x}<\kappa\)
                  and \(f\subseteq H(\kappa)\). The equality of \(\abs{f}\) and
                  \(\abs{x}\) is simply by functionality of the relation \(f\).
                  We have \(f\subseteq x\times\bigcup x\subseteq H(\kappa)\)
                  because \(H(\kappa)\) is closed under Gödel operations and
                  transitive. This proves \(f\in H(\kappa)\) for all choice
                  functions.

                  To show the axiom schema of replacement is satisfied let \(x\)
                  be any set and \(\varphi\) a functional formula on \(x\). If
                  for each \(z\in x\) we have (excusing terrible notation)
                  \(\varphi(z)\in H(\kappa)\), then the set \(y\) obtained from
                  the axiom of replacement in our original model is a subset of
                  \(H(\kappa)\). The cardinalities \(\abs{\trcl(\varphi(z))}\)
                  are smaller than \(\kappa\) for all \(z\in x\). By the same
                  sequence of (in)equalities as part e this means that
                  \(\abs{\trcl{y}}<\kappa\). This proves \(y\in H(\kappa)\) so
                  it satisfies the axiom of replacement.

                  This demonstrates all axioms except the powerset axiom and
                  thus that \(H(\kappa)\) is a model for ZFC-P.
              \end{proof}

        \item Conclude that ZFC-P is consistent with the fact that every set is
              countable.

              \begin{proof}
                  Consider \(H(\aleph_{1})\). This is a model of ZFC-P. We will
                  show that for each infinite set there is a bijection to the
                  natural numbers. This will prove all sets are countable.

                  In the outer model we have
                  \(\abs{x}\leq\abs{\trcl(c)}<\aleph_{1}\) for each infinite set
                  \(x\). This shows that in the outer model there is a bijection
                  \(f:\aleph_{1}\to x\). By reasoning similar to the one given
                  in part f for the truth of the axiom of choice in
                  \(H(\kappa)\) we must have that \(\abs{\trcl{f}}<\kappa\) and
                  \(f\subseteq H(\kappa)\) so \(f\in H(\kappa)\). This proves
                  \(\abs{x}=\aleph_{0}\) for each infinite set \(x\).
              \end{proof}
    \end{enumerate}
\end{question}

\begin{question}
    This exercise proves that ``\(x\) is finite'' is a \(\Delta_{1}\) property.

    \begin{enumerate}[a.]
        \item Verify that the definition of finiteness can be expressed as a
              \(\Sigma_{1}\) formula.

              \begin{proof}
                  First we define a couple helper formulas which are all
                  \(\Delta_{0}\):
                  \begin{align*}
                      \psi(z,a,b) & \equiv z=(a,b)                                                                                                                               \\
                      \chi(z)     & \equiv\textnormal{\(z\) is an ordinal}                                                                                                       \\
                      \xi(z)      & \equiv\chi(z)\wedge(\exists a\in z:\top)\wedge(\forall a\in z:a\cup\set{a}\in z\wedge (a\neq\varnothing\to(\exists b\in z:a=b\cup\set{b}))).
                  \end{align*}
                  The formula \(\xi\) is equivalent to \(z\) being \(\omega\):
                  it is a non-empty ordinal containing the successor of each of
                  its elements, so a limit ordinal, and all non-zero elements
                  are a successor so it must be the smallest limit ordinal.

                  Now we define the \(\Sigma_{1}\) formula ``\(x\) is finite''
                  as follows:
                  \begin{align*}
                      \exists f,w:\xi(w)\wedge f\subseteq w\times x
                       & \wedge(\exists n\in w:n=\dom(f) \wedge (\forall m\in n:\exists a\in x,z\in f:\psi(z,m,a)) \\
                       & \wedge(\forall a\in x:\exists m\in n:\exists z\in f:\psi(z,m,a))).
                  \end{align*}
                  This formula is equivalent to the existence of a surjective
                  \(f:n\to x\) for some \(n\in \omega\) which is equivalent to
                  finiteness\footnote{But worded slightly weirdly, if you wanted
                      something nice to read you should have given different
                      homework.}.

                  Therefore finiteness is a \(\Sigma_{1}\) property.
              \end{proof}

        \item Verify that the definition of finiteness can be expressed as a
              \(\Pi_{1}\) formula.

              \begin{proof}
                  I will prove that \(T\)-finiteness, which is equivalent to
                  finiteness, is \(\Pi_{1}\)-expressible. First I will define a
                  couple of helper \(\Delta_{0}\) formulas:
                  \begin{align*}
                      \varphi(x,y) & =\forall z\in y:\forall t\in z:t\in x                             \\
                      \psi(y)      & =\exists z\in y:\forall t\in y:t=z\vee \neg\forall a\in z:a\in t.
                  \end{align*}
                  The formula \(\varphi(x,y)\) means that \(x\) is a set of
                  subsets of \(x\). The formula \(\psi\) means that there is an
                  element \(z\in y\) such that it is equal to any other \(t\in
                  y\) it is a subset of: \(y\) has a maximal element.

                  By homework 3 we conclude that \(\forall
                  y:\varphi(x,y)\to\psi(y)\) is equivalent to finiteness of
                  \(x\).\

                  Therefore that finiteness is \(\Pi_{1}\) and also
                  \(\Delta_{1}\) in combination with part a.
              \end{proof}
    \end{enumerate}
\end{question}

\begin{question}
    We work in \(H(\aleph_{2})\). One can extend the methods used in class to
    prove the following: if \(A\in H(\aleph_{2})\) is countable then there is a
    countable set \(M\in H(\aleph_{2})\) that contains \(A\) and that satisfies
    the equivalence
    \[
        \phi^{M}(m_{1},\ldots,m_{k})\leftrightarrow\phi^{H(\aleph_{2})}(m_{1},\ldots,m_{k})
    \]
    for all formulas \(\phi\) and all \(m_{1},\ldots,m_{k}\in M\).

    Now let \(f:\omega_{1}\to\omega_{1}\) be a regressive function and let \(M\)
    be as above for the countable set \(f\).
    \begin{enumerate}[a.]
        \item Verify that \(\omega_{1}\in M\).

              \begin{proof}
                  Let \(\varphi\) be the formula given by \(\varphi(x)=\exists
                  y:y=\dom(x)\). Then \(\varphi^{H(\aleph_{2})}(f)\) is clearly
                  satisfied and \(\dom(f)=\omega_{1}\). This means
                  \(\varphi^{M}(f)=\exists y\in M:y=\dom(f)\), let the element
                  satisfying this be called \(x\). Then
                  \((x=\dom(f))^{H(\kappa_{2})}\) as well, so by absoluteness of
                  \(\dom\) under transitivity \(x=\dom(f)=\omega_{1}\) in
                  \(H(\aleph_{2})\) so \(x=\omega_{1}\in M\).
              \end{proof}

        \item Prove that \(\omega\in M\) and \(\omega\subseteq M\).

              \begin{proof}
                  We borrow the formulas \(\chi,\xi\) from exercise 2.a we see that
                  the formula
                  \[
                      \varphi(x)=\exists z:(\exists t\in z:\top)\wedge z\in\dom(x)\wedge\chi(z)\wedge\xi(z)
                  \]
                  clearly has the property that
                  \[
                      (\varphi(f))^{H(\aleph_{2})}.
                  \]
                  This means that
                  \[
                      (\varphi(f))^{M}.
                  \]
                  as well, which means there is an ordinal in \(M\) which
                  contains at least one element only successor ordinals and zero
                  and is closed under taking successors. This must be
                  \(\omega\). Therefore \(\omega\in M\).

                  For the second statement we first prove \(\varnothing\in M\)
                  and proceed via induction. We have the formula \(\exists
                  x:\forall y:y\notin x\). This denotes the existence of
                  something acting like an empty set. It is clearly true in
                  \(H(\aleph_{2})\), so there is an \(x\in M\) satisfying
                  \(\forall y\in M:y\notin x\). This does not mean \(x\) is the
                  empty set. We get this by stating that \((\forall y:y\notin
                  x)^{M}\) so also \((\forall y:y\notin x)^{H(\aleph_{2})}\)
                  so \(x=\varnothing\).

                  Now let \(n\in M\) be a natural number. Then \(\exists
                  x:x=z\cup\set{z}\) is satisfied in \(H(\aleph_{2})\) for
                  \(z=n\). Therefore there is also a \(x\in M\) which has
                  \(\exists x\in M:x=z\cup\set{z}\). We also have
                  \(x=n\cup\set{n}=n+1\) in the model \(H(\aleph_{2})\)
                  and therefore \(x=n+1\).

                  This proves by induction that \(\omega\subseteq M\).
              \end{proof}

        \item I'm sorry I don't really have time for the rest even though you
              essentially spoiled the answer in the lecture.
    \end{enumerate}
\end{question}
\end{document}