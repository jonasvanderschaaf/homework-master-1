\documentclass{article}

\usepackage[utf8]{inputenc}
\usepackage[english]{babel}
\usepackage[shortlabels]{enumitem}
\usepackage{amsthm, amssymb, mathtools, amsmath, bbm, mathrsfs,stmaryrd}
\usepackage[margin=1in]{geometry}
\usepackage[parfill]{parskip}
\usepackage[hidelinks]{hyperref}
\usepackage{cleveref}

\usepackage{caption}
\usepackage{subcaption}
\usepackage{tikz}

\newtheorem*{lemma}{Lemma}

\theoremstyle{definition}
\newtheorem{question}{Exercise}
\newtheorem*{corollary}{Corollary}

\newcommand{\set}[1]{\left\{#1\right\}}
\newcommand{\setwith}[2]{\set{#1\colon#2}}

\newcommand{\powset}{\mathcal{P}}
\DeclareMathOperator{\dom}{dom}
\DeclareMathOperator{\range}{ran}
\DeclareMathOperator{\field}{field}
\DeclareMathOperator{\cof}{cf}

\newcommand{\restrict}{\upharpoonright}

\DeclarePairedDelimiter{\abs}{|}{|}

\title{Homework Set Theory}
\date{October 11, 2022}
\author{Jonas van der Schaaf}

\begin{document}
\maketitle

\begin{question}
    In this problem we show that, assuming the Axiom of Foundation, the Axiom of
    Choice is equivalent to the statement that for every ordinal \(\alpha\) the
    power set \(\powset(\alpha)\) can be well-ordered. From now on we assume the
    latter statement.
    \begin{enumerate}[a.]
        \item Prove that it suffices to show that for every ordinal \(\alpha\)
              the set \(V_{\alpha}\) is well-orderable.

              \begin{proof}
                  Let \(X\) be a set of Mirimanoff rank \(\alpha\), that is
                  \(X\in V_{\alpha+1}=\powset(V_{\alpha})\) but \(X\notin
                  V_{\alpha}\). Then \(X\subseteq V_{\alpha}\). Because
                  \(V_{\alpha}\) can be well-ordered by some relation \(R\),
                  restricting \(R\) to \(X\) is a well-order on \(X\): every
                  subset has a minimal element and the restriction of a linear
                  order is still linear.
              \end{proof}

        \item Verify \(V_{0}\) can be well-ordered.

              \begin{proof}
                  For \(V_{0}=\varnothing\), the empty relation is a well-order.
                  It is clearly a total order because being total is vacuous in
                  this case. There is no non-empty subset of \(\varnothing\)
                  either, so being a well-order is a totally vacuous statement
                  on \(\varnothing\).
              \end{proof}

        \item Show: if \(V_{\alpha}\) can be well-ordered then \(V_{\alpha+1}\)
              can be well-ordered.

              \begin{proof}
                  Suppose \(V_{\alpha}\) is well-ordered by \(R\) such that
                  \((V_{\alpha},R)\cong(\beta,\in)\) for some unique ordinal
                  \(\beta\). Then \(\powset(\beta)\) is equipotent to
                  \(\powset(V_{\alpha})\) through some bijection
                  \(f:V_{\alpha+1}\to\powset(\beta)\). By assumption the former
                  can be well-ordered. The latter can then be well-ordered by
                  lifting the well-order through the aforementioned bijection
                  \(f\). This means \(V_{\alpha+1}\) can be well-ordered as
                  well.
              \end{proof}
    \end{enumerate}
    Now let \(\alpha\) be a limit and assume that for all \(\beta<\alpha\) the
    set \(V_{\beta}\) can be well-ordered. Let \(\kappa=\aleph(V_{\alpha})\).
    \begin{enumerate}[a., resume]
        \item Show that \(\powset(\kappa\times\kappa)\) can be well-ordered.

              \begin{proof}
                  The set \(\kappa\times\kappa\) can be well-ordered according
                  to Hessenberg's theorem and is therefore isomorphic to some
                  ordinal \(\gamma\). By assumption then \(\powset(\gamma)\) can
                  be well-ordered, and therefore so can
                  \(\powset(\kappa\times\kappa)\).
              \end{proof}

        \item Show: if \(\beta<\alpha\) then there is a subset \(R\) of
              \(\kappa\times\kappa\) such that \((\field R, R)\) is isomorphic
              to \((V_{\beta},\in)\).

              \begin{proof}
                  Because \(V_{\beta}\) can be well-ordered and so can
                  \(\kappa\times\kappa\) there is either an injective function
                  \(V_{\beta}\to\kappa\times\kappa\) or \(\kappa\times\kappa\to
                  V_{\beta}\). The latter cannot occur because the inclusion map
                  \(\imath:V_{\beta}\to V_{\alpha}\) is injective, which would
                  lead to an injective \(\kappa\times\kappa\to V_{\alpha}\).
                  This cannot happen because there is not even an injective map
                  \(\kappa\to V_{\alpha}\).

                  This means that there is an injective map
                  \(f:V_{\beta}\hookrightarrow\kappa\times\kappa\). Defining
                  \[
                      R=\setwith{(f(x),f(y))\in\kappa\times\kappa}{x,y\in V_{\beta},x<y}
                  \]
                  gives a relation \(R\) such that \((V_{\beta},<)\) and
                  \(\field R,R\) are clearly isomorphic because \(x<y\) exactly
                  iff \(f(x)<f(y)\) and \(f\) is bijective onto its range
                  because it is injective.
              \end{proof}

        \item Prove: If \(R\) is as in the previous parts and if
              \(f:V_{\beta}\to\field R\) such that \((\field R,R)\) is
              isomorphic to \((V_{\beta},\in)\), then the inverse is given,
              recursively, by \(g(x)=\setwith{g(y)}{yRx}\).

              \begin{proof}
                  Take \(x\in\kappa\times\kappa\). Then \(z\in f^{-1}(x)\) is
                  equivalent to \(z<f^{-1}(x)\), which in turn is equivalent to
                  \(f(z)<x\) by properties of isomorphisms. This means that the
                  elements of \(f^{-1}(x)\) are exactly those
                  \(f(y)\in\kappa\times\kappa\) such that \(f(y)<x\).
                  Therefore \(f^{-1}(x)=\setwith{f(y)}{yRx}\).
              \end{proof}

        \item I don't want to write out this question, but I'm generous so I
              will give a proof of the statement.

              \begin{proof}
                  We know for each \(\beta<\alpha\) there is at least one
                  well-order \(R\in\powset(\kappa\times\kappa)\) such that there
                  is an isomorphism \((\field R,R)\cong (V_{\beta},\in)\). Each
                  set \(\setwith{R\in\powset(\kappa\times\kappa)}{(\field
                      R,R)\cong(V_{\beta},\beta)}\) is non-empty and has a minimal
                  element \(\prec_{\beta}\) because
                  \(\powset(\kappa\times\kappa)\) can be well-ordered. These can
                  be chosen because we can fix one well-order on the
                  encompassing set \(\powset(\kappa\times\kappa)\). Collecting
                  these using replacement will give the sequence of well-orders
                  \(\left<R_{\beta},\prec_{\beta}\right>\).
              \end{proof}

        \item Define a well-order of \(V_{\alpha}\).

              \begin{proof}
                  We will define a relationship as follows: For \(x,y\in
                  V_{\alpha}=\bigcup_{\beta<\alpha}V_{\beta}\) if the rank of
                  \(x\) and \(y\) are different and \(\beta\) and \(\gamma\)
                  respectively we define \(x<y\) if \(\beta<\gamma\) and \(x>y\)
                  if \(\beta>\gamma\). If \(\beta=\gamma\) then we take the
                  well-order \(\prec_{\beta}\) and say \(x<y\) iff
                  \(x\prec_{\beta}y\).

                  This defines a well-order. To see this, first notice that this
                  is a linear order. It is irreflexive because each
                  \(\prec_{\beta}\) is. For \(x,y,z\) with \(x<y<z\) it is
                  transitive because if the ranks of two of \(x,y,z\in
                  V_{\alpha}\) are different, then the ranks of \(x,z\) are
                  different where \(z\) must have the higher rank so \(x<z\). If
                  all ranks are the same then \(x\prec_{\beta}y\prec_{\beta}z\)
                  so \(x\prec_{\beta}z\) and thus \(x<y\) by transitivity of
                  \(\prec_{\beta}\). The relation is total because for \(x,y\in
                  V_{\alpha}\) of different rank they are clearly ordered (the
                  one with the lower rank is smaller). If they have the same
                  rank \(\beta\) then one must be smaller because
                  \(\prec_{\beta}\) is total.

                  To see that every set has a minimal element take any
                  \(X\subseteq V_{\alpha}\). Using replacement we can construct
                  the set of all ranks of \(x\in X\): each element of \(X\) has
                  a unique rank and thus these ordinals can be aggregated into a
                  set using replacement. Call this set \(\Omega\). Because the
                  ordinals are well-ordered take the least \(\beta\in\Omega\).
                  Taking the least element of \(V_{\beta}\cap X\) using the well
                  order \(\prec_{\beta}\) will give a least element \(x\in
                  V_{\beta}\cap X\). This is the least element of \(X\).

                  To see this notice that for every \(y\in X\) with rank
                  \(\gamma>\beta\) we automatically get \(x<y\). If \(y\) has
                  rank \(\beta\) as well then \(x\prec_{\beta}y\) by assumption
                  that \(x\) is minimal in \(V_{\beta}\cap\Omega\).

                  This means that \(V_{\alpha}\) is well-ordered by \(<\).
              \end{proof}
    \end{enumerate}
\end{question}

The next two questions were made in collaboration with Paul Talma.

\begin{question}
    Prove the following statements:
    \begin{enumerate}[a.]
        \item \(\aleph_{\omega}^{\aleph_{1}}=\aleph_{\omega}^{\aleph_{0}}\cdot
              2^{\aleph_{1}}\).

              \begin{proof}
                  Using monotonicity of exponentiation in both variables and the
                  fact that \(\aleph_{1}>\aleph_{0}\) we get
                  \(\aleph_{\omega}^{\aleph_{1}}\geq\aleph_{\omega}^{\aleph_{0}}\).
                  Similarly \(\aleph_{\omega}^{\aleph_{1}}\geq 2^{\aleph_{1}}\)
                  is true. Therefore
                  \begin{equation}
                      \label{for:ineq1}
                      \aleph_{\omega}^{\aleph_{0}}\cdot 2^{\aleph_{1}}
                      =\max\set{\aleph_{\omega}^{\aleph_{0}},2^{\aleph_{1}}}\leq\aleph_{\omega}^{\aleph_{1}}.
                  \end{equation}

                  There are two possible cases, either there is some finite
                  \(\alpha<\omega\) such that
                  \(\aleph_{\alpha}^{\aleph_{1}}\geq\aleph_{\omega}\) or for all
                  \(\alpha<\omega\) we have
                  \(\aleph_{\alpha}^{\aleph_{1}}<\aleph_{\omega}\).

                  First assume the former case: take \(\alpha\) such that
                  \(\aleph_{\alpha}^{\aleph_{1}}\geq\aleph_{\omega}\). If
                  \(\alpha=0\) then
                  \(\aleph_{0}^{\aleph_{1}}=2^{\aleph_{1}}\geq\aleph_{\omega}^{\aleph_{1}}\).
                  Combining this inequality with \Cref{for:ineq1} gives equality
                  by a permutation of the Dedekind-Schröder-Bernstein-Cantor
                  theorem:
                  \begin{align*}
                      \aleph_{\omega}^{\aleph_{0}}\cdot2^{\aleph_{1}} & \leq\aleph_{\omega}^{\aleph_{1}}                    \\
                                                                      & \leq 2^{\aleph_{1}}                                 \\
                                                                      & \leq\aleph_{\omega}^{\aleph_{0}}\cdot2^{\aleph_{1}}
                  \end{align*}
                  so \(\aleph_{\omega}^{\aleph_{0}}\cdot2^{\aleph_{1}}=\aleph_{\omega}^{\aleph_{1}}\).

                  If \(\aleph<\omega\) is a successor ordinal then we can apply
                  the Haussdorff formula finitely many times to obtain the
                  following inequality:
                  \begin{align*}
                      \aleph_{\omega}^{\aleph_{1}} & \leq\aleph_{\alpha}^{\aleph_{1}}                                        \\
                                                   & =\aleph_{0}^{\aleph_{1}}\cdot\prod_{1\leq\beta\leq\alpha}\aleph_{\beta} \\
                                                   & =\aleph_{0}^{\aleph_{1}}\cdot\aleph_{\alpha}.
                  \end{align*}
                  However because \(\aleph_{\alpha}<\aleph_{\omega}\) we must
                  get
                  \(\aleph_{\omega}^{\aleph_{1}}\leq\aleph_{0}^{\aleph_{1}}\).
                  Applying the same permutation of
                  Bernstein-Schröder-Dedekind-Cantor in the same manner as
                  before gives the same desired equality:
                  \(\aleph_{\omega}^{\aleph_{1}}=\aleph_{\omega}^{\aleph_{0}}\cdot
                  2^{\aleph_{1}}\).

                  In the latter case we have
                  \(\aleph_{\alpha}^{\aleph_{1}}<\aleph_{\omega}\) for all
                  \(\alpha<\omega\). Then for the cofinality we have the
                  following inequality:
                  \(\cof(\aleph_{\omega})=\aleph_{0}\leq\aleph_{1}\). This means
                  that
                  \(\aleph_{\omega}^{\aleph_{1}}=\aleph_{\omega}^{\aleph_{0}}\).
                  Combining this with the inequality of \Cref{for:ineq1} we
                  obtain the desired equality in this case:
                  \(\aleph_{\omega}^{\aleph_{1}}=\aleph_{\omega}^{\aleph_{0}}\cdot
                  2^{\aleph_{1}}\).
              \end{proof}

        \item If \(2^{\aleph_{1}}=\aleph_{2}\) and
              \(\aleph_{\omega}^{\aleph_{0}}>\aleph_{\omega_{1}}\) then
              \(\aleph_{\omega_{1}}^{\aleph_{1}}=\aleph_{\omega}^{\aleph_{0}}\).

              \begin{proof}
                  By part a and because \(\aleph_{2}<\aleph_{\omega}\) we have
                  \begin{align*}
                      \aleph_{\omega_{1}}^{\aleph_{1}} & =\aleph_{\omega}^{\aleph_{0}}\cdot 2^{\aleph_{1}} \\
                                                       & =\aleph_{\omega}^{\aleph_{0}}\aleph_{2}           \\
                                                       & =\aleph_{\omega}^{\aleph_{0}}.
                  \end{align*}
              \end{proof}

        \item If \(2^{\aleph_{0}}\geq\aleph_{\omega_{1}}\) then
              \(\gimel(\aleph_{\omega})=2^{\aleph_{0}}\) and
              \(\gimel(\aleph_{\omega_{1}})=2^{\aleph_{1}}\).

              \begin{proof}
                  We have \(\cof(\aleph_{\omega})=\aleph_{0}\) and so
                  \(\gimel(\aleph_{\omega})=\aleph_{\omega}^{\aleph_{0}}\).
                  Because \(2^{\aleph_{0}}\geq\aleph_{\omega_{1}}\) by
                  monotonicity of exponentiation we get
                  \(2^{\aleph_{0}}=\left(2^{\aleph_{0}}\right)^{\aleph_{0}}\geq\aleph_{\omega_{1}}^{\aleph_{0}}\geq\aleph_{\omega}^{\aleph_{0}}\).
                  Monotonicity of exponentiation will give the other inequality:
                  because \(2<\aleph_{\omega_{1}}\) we have
                  \(2^{\aleph_{0}}\leq\aleph_{\omega_{1}}^{\aleph_{0}}\) as
                  well. This means
                  \(\gimel(\aleph_{\omega})=\aleph_{\omega}^{\aleph_{0}}=2^{\aleph_{0}}\).

                  For the other inequality, first notice that
                  \(\cof(\aleph_{\omega_{1}})=\omega_{1}=\aleph_{1}\). By almost
                  the same reasoning as before we get because
                  \(2^{\aleph_{0}}\geq\aleph_{\omega_{1}}\) we get
                  \begin{align*}
                      2^{\aleph_{1}} & =\left(2^{\aleph_{0}}\right)^{\aleph_{1}} \\
                                     & \geq\aleph_{\omega_{1}}^{\aleph_{1}}.
                  \end{align*}
                  Monotonicity of exponentiation automatically gives
                  \(2^{\aleph_{1}}\leq\aleph_{\omega_{1}}^{\aleph_{1}}\).
                  Therefore
                  \begin{align*}
                      2^{\aleph_{1}} & =\aleph_{\omega_{1}}^{\aleph_{1}}                \\
                                     & =\aleph_{\omega_{1}}^{\cof(\aleph_{\omega_{1}})} \\
                                     & =\gimel(\aleph_{\omega_{1}}).
                  \end{align*}
              \end{proof}
    \end{enumerate}
\end{question}

\begin{question}
    Prove: If \(\beta\) is such that
    \(2^{\aleph_{\alpha}}=\aleph_{\alpha+\beta}\) then \(\beta<\omega\).
    Complete the following steps. Assume \(\beta\geq\omega\).
    \begin{enumerate}[a.]
        \item Let \(\alpha\) be minimal such that \(\alpha+\beta>\beta\).
              Show that \(\beta\) is a limit.

              \begin{proof}
                  If \(\alpha=\gamma+1\) for some ordinal \(\gamma\) then
                  \begin{align*}
                      \gamma+\beta & =\gamma+(1+\beta) \\
                                   & =\gamma+1+\beta   \\
                                   & =\alpha+\beta     \\
                                   & >\beta.
                  \end{align*}
                  This means that \(\alpha\) cannot be minimal such that
                  \(\alpha+\beta>\beta\) if \(\alpha\) is a successor. Therefore
                  it is a limit.
              \end{proof}

        \item Let \(\kappa=\aleph_{\alpha+\alpha}\). Show that \(\kappa\) is
              singular.

              \begin{proof}
                  We will show
                  \(\cof(\aleph_{\alpha+\alpha})=\alpha<\aleph_{\alpha+\alpha}\).

                  There is a cofinal map \(f:\alpha\to\kappa\) given by
                  \(f(\gamma)=\aleph_{\alpha+\gamma}\). We have
                  \(\sup_{\gamma<\alpha}f(\gamma)\leq\kappa\) because
                  \(\gamma+\alpha\leq\alpha+\alpha\) for all \(\gamma<\alpha\).
                  We obtain the other inequality by noticing that for each
                  \(\delta<\alpha\) there must be some \(\gamma<\alpha\) such
                  that \(\alpha+\gamma=\delta\). Then \(\alpha+\gamma+1>\delta\)
                  so the supremum of
                  \(\setwith{f(\gamma)\in\aleph_{\alpha}}{\gamma<\alpha}\) is
                  larger than each cardinal smaller than
                  \(\aleph_{\alpha+\alpha}\).

                  We get that \(\alpha\leq\aleph_{\alpha}\) because the
                  \(\aleph_{\alpha}\) functions are strictly increasing and
                  continuous by definition. We will show
                  \(\alpha+\alpha>\alpha\) to get
                  \(\aleph_{\alpha}<\aleph_{\alpha+\alpha}\) to get the required
                  inequality. Assume \(\alpha+\alpha=\alpha\), then
                  \[
                      \alpha+\beta=\alpha+\alpha+\beta>\alpha+\beta
                  \]
                  which is an obvious contradiction. Therefore
                  \[
                      \cof(\kappa)\leq\alpha\leq\aleph_{\alpha}<\aleph_{\alpha+\alpha}
                  \]
                  so \(\kappa\) is singular.
              \end{proof}

        \item Prove \(2^{\alpha_{\alpha+\xi}}=\aleph_{\alpha+\beta}\) for all
              \(\xi<\alpha\).

              \begin{proof}
                  Because \(\xi<\alpha\) \(\beta\leq\xi+\beta\leq\beta\).
                  Therefore \(\xi+\beta=\beta\). This means that we obtain the
                  following two equalities:
                  \(2^{\aleph_{\alpha+\xi}}=\aleph_{\alpha+\xi+\beta}=\aleph_{\alpha+\beta}\).
              \end{proof}

        \item Calculate \(2^{\kappa}\) and derive a contradiction.

              \begin{proof}
                  The assumption gives us that
                  \(2^{\kappa}=2^{\aleph_{\alpha+\alpha}}=\aleph_{\alpha+\alpha+\beta}>\aleph_{\alpha+\beta}\).

                  Because \(\kappa\) is singular and \(2^{\aleph_{\gamma}}\) is
                  eventually constant and equal to \(\aleph_{\alpha+\beta}\)
                  below \(\kappa\) we also have that
                  \(2^{\kappa}=\aleph_{\alpha+\beta}\). This gives the obvious
                  contradiction \(\aleph_{\alpha+\beta}<\aleph_{\alpha+\beta}\).

                  This means that \(\beta<\omega\).
              \end{proof}
    \end{enumerate}
\end{question}
\end{document}