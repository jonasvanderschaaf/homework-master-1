\documentclass{article}

\usepackage[utf8]{inputenc}
\usepackage{enumerate}
\usepackage{scalerel}
\usepackage{amsthm, amssymb, mathtools, amsmath, bbm, mathrsfs, stmaryrd}
\usepackage[margin=1in]{geometry}
\usepackage[parfill]{parskip}
\usepackage[hidelinks]{hyperref}
\usepackage{quiver}
\usepackage{float}
\usepackage{nicefrac}

\newcommand{\N}{\mathbb{N}}
\newcommand{\Z}{\mathbb{Z}}
\newcommand{\NZ}{\mathbb{N}_{0}}
\newcommand{\Q}{\mathbb{Q}}
\newcommand{\R}{\mathbb{R}}
\newcommand{\C}{\mathbb{C}}

\newcommand{\F}{\mathbb{F}}
\newcommand{\incl}{\imath}

\newcommand{\tuple}[2]{\left\langle#1\colon #2\right\rangle}

\DeclareMathOperator{\order}{orde}
\DeclareMathOperator{\Id}{Id}
\DeclareMathOperator{\im}{im}
\DeclareMathOperator{\coker}{coker}
\DeclareMathOperator{\ggd}{ggd}
\DeclareMathOperator{\kgv}{kgv}
\DeclareMathOperator{\degree}{gr}

\DeclareMathOperator{\gl}{GL}

\DeclareMathOperator*{\bigplus}{\scalerel*{+}{\sum}}

\DeclareMathOperator{\Aut}{Aut}
\DeclareMathOperator{\End}{End}
\DeclareMathOperator{\Hom}{Hom}
\newcommand{\isom}{\overset{\sim}{\longrightarrow}}

\newcommand{\Grp}{{\bf Grp}}
\newcommand{\Ab}{{\bf Ab}}
\newcommand{\cring}{{\bf CRing}}
\DeclareMathOperator{\modules}{{\bf Mod}}
\newcommand{\catset}{{\bf Set}}
\newcommand{\cat}{\mathcal{C}}
\newcommand{\chains}{{\bf Ch}}
\newcommand{\homot}{{\bf Ho}}
\DeclareMathOperator{\objects}{Ob}
\newcommand{\gen}[1]{\left<#1\right>}
\DeclareMathOperator{\cone}{Cone}
\newcommand{\set}[1]{\left\{#1\right\}}
\newcommand{\setwith}[2]{\left\{#1:#2\right\}}
\DeclareMathOperator{\Ext}{Ext}
\DeclareMathOperator{\nil}{Nil}
\DeclareMathOperator{\idem}{Idem}
\DeclareMathOperator{\JRad}{JRad}
\DeclareMathOperator{\Nil}{Nil}
\DeclareMathOperator{\Idem}{Idem}
\DeclareMathOperator{\Rad}{Rad}
\DeclareMathOperator{\Ann}{Ann}
\DeclareMathOperator{\Mat}{Mat}
\DeclareMathOperator{\ch}{ch}

\newcommand{\maxid}{\mathfrak{m}}
\newcommand{\primeid}{\mathfrak{p}}
\newcommand{\ideal}{\triangleleft}

\newenvironment{solution}{\begin{proof}[Solution]\renewcommand\qedsymbol{}}{\end{proof}}

\newtheorem{theorem}{Theorem}[section]
\newtheorem{lemma}[theorem]{Lemma}
\newtheorem{proposition}[theorem]{Proposition}

\newtheorem{question}{Question}

\theoremstyle{definition}
\newtheorem{definition}[theorem]{Definition}
\newtheorem{remark}[theorem]{Remark}
\newtheorem{example}[theorem]{Example}
\newtheorem{corollary}[theorem]{Corollary}

\title{Homework Commutative Algebra}
\author{Jonas van der Schaaf}
\date{October 24, 2022}

\begin{document}
\maketitle

\begin{question}
    Let \(A\) be a ring and \(S\subseteq A\) a multiplicative subset. Prove that
    \(S^{-1}A[x]\cong S^{-1}(A[x])\) as \(A\)-algebras using the universal
    property.

    \begin{proof}
        Let \(B\) be an \(A\)-algebra such that \(f:A[x]\to B\) is a morphism
        such that \(f(s)\in B^{*}\) for all \(s\in S\). We will prove \(f\)
        factors uniquely through \(S^{-1}A[x]\) by the trivial inclusion map
        \[
            \incl:A[x]\hookrightarrow S^{-1}A[x]:\sum_{i=0}^{n}a_{i}x^{i}\mapsto\sum_{i=0}^{n}\frac{a_{i}}{1}x^{i}.
        \]

        We define a map \(\tilde{f}:S^{-1}A[x]\to B\) as follows: for
        \(b=\sum_{i=0}^{n}\nicefrac{a_{i}}{s_{i}}x^{i}\) define
        \[
            \tilde{f}\left(\sum_{i=0}^{n}\frac{a_{i}}{s_{i}}x^{i}\right)=\sum_{i=0}^{n}\frac{f(a_{i})}{f(s_{i})}f(x)^{i}.
        \]
        Note that this is well defined because \(f(b_{i})\) is a unit in \(B\)
        for all \(s\in S\). Then \(\tilde{f}\) is clearly an additive
        multiplicative function and
        \begin{align*}
            \tilde{f}\left(c\sum_{i=0}^{n}\frac{a_{i}}{s_{i}}x^{i}\right) & =\tilde{f}\left(\sum_{i=0}^{n}\frac{ca_{i}}{s_{i}}x^{i}\right) \\
                                                                          & =\sum_{i=0}^{n}\frac{f(ca_{i})}{f(s_{i})}f(x)^{i}              \\
                                                                          & =c\sum_{i=0}^{n}\frac{f(a_{i})}{f(s_{i})}f(x)^{i}
        \end{align*}
        by linearity of \(f\) in \(A\).
        Therefore \(\tilde{f}\) is a well-defined \(A\)-algebra homomorphism.

        Because
        \[
            \tilde{f}\imath\left(\sum_{i=0}^{n}a_{i}x^{i}\right)=\sum_{i=0}^{n}f(a_{i})f(x)^{i}
        \]
        we get that \(\tilde{f}\imath=f\).

        To prove uniqueness of \(\tilde{f}\) notice that for each \(a\in A\) we
        must have
        \begin{align*}
            f(a) & =\tilde{f}\imath(a)                 \\
                 & =\tilde{f}\left(\frac{a}{1}\right).
        \end{align*}
        Because ring morphisms preserve units
        \(\tilde{f}(s^{-1})=\tilde{f}(s)^{-1}=f(s)^{-1}\) for all \(s\in S\).
        Therefore \(\tilde{f}\) applied to the coefficients of any polynomial is
        fixed. Also \(\tilde{f}(x)=\tilde{f}\imath(x)=f(x)\) is fixed uniquely
        by \(f\). Therefore \(\tilde{f}\) is determined uniquely by the original
        map \(f\).
    \end{proof}
\end{question}

\begin{question}
    Compute the cardinality of the module
    \(\Z/525\Z\otimes_{\Z}\Z\left[\nicefrac{1}{33}\right]\).

    \begin{proof}
        Because \(33\) is coprime with \(525\) it's a unit in \(\Z/525\Z\).
        Therefore the localization of \(\nicefrac{1}{33}\) in \(\Z/525\Z\) is
        just \(\Z/525\Z\) again. Rewriting the module above using properties of
        the tensor product we obtain:
        \begin{align*}
            \Z/525\Z\otimes_{\Z}\Z\left[\nicefrac{1}{33}\right] & \cong\Z\left[\frac{1}{33}\right]/\left(525\Z\left[\frac{1}{33}\right]\right)                                        \\
                                                                & \cong\Z/525\Z\left[\frac{1}{33}\right]                                                                              \\
                                                                & \cong\left(\Z/3\Z\oplus\Z/5^{2}\Z\oplus\Z/7\Z\right)\left[\frac{1}{33}\right]                                       \\
                                                                & \cong\Z/3\Z\left[\frac{1}{33}\right]\oplus\Z/5^{2}\Z\left[\frac{1}{33}\right]\oplus\Z/7\Z\left[\frac{1}{33}\right].
        \end{align*}
        Because \(33\equiv0\mod 3\) the first localization is \(0\). Because
        \(33\) is coprime with \(5^{2}\) and \(7\) it must be invertible in the
        rings \(\Z/5^{2}\Z\) and \(\Z/7\Z\) by elements \(y\in\Z/5^{2}\Z\) and
        \(z\in\Z/7\Z\). Then in both rings the elements \(x/33^{i}\) already
        exist and are given by \(xy^{i}\) and \(xz^{i}\) respectively so the
        module is isomorphic to \(\Z/5^{2}\Z\oplus\Z/7\Z\) which has
        \(175\) elements.
    \end{proof}
\end{question}

\begin{question}
    For each of the following rings determine whether it is a field:
    \begin{enumerate}[(a)]
        \item \(\Q(\sqrt{2})\otimes_{\Q}\R\)

              \begin{solution}
                  Consider the map \(f:\Q(\sqrt{2})\times\R\twoheadrightarrow\R\)
                  given by \(f(x,y)=xy\). This is clearly a surjective bilinear
                  map to a field. The lifted map must therefore also be
                  surjective so if the tensor product is a field the map must be
                  injective (the kernel cannot be the entire ring). We have
                  \(\sqrt{2}\otimes\sqrt{2}\neq 2\otimes 1\) because there is no
                  \(x\in\Q\) such that \((\sqrt{2}x,\sqrt2)=(2,1)\). However
                  \(\tilde{f}(\sqrt{2}\otimes\sqrt{2}-2\otimes
                  1)=\sqrt{2}^{2}-2=0\). Therefore \(\tilde{f}\) is not
                  injective so \(\Q(\sqrt{2})\otimes_{\Q}\R\) is not a field.
              \end{solution}

        \item \(\Q(\sqrt{-2})\otimes_{\Q}\Q(\sqrt{-1})\)

              \begin{solution}
                  I will prove \(\Q(\sqrt{-2},\sqrt{-1})\) is isomorphic to the
                  tensor product as \(\Q\)-algebra. We have a surjective
                  bilinear map of modules
                  \[
                      f:\Q(\sqrt{-2})\times\Q(\sqrt{-1})\twoheadrightarrow\Q(\sqrt{-1},\sqrt{-2}):(x,y)\mapsto xy.
                  \]
                  This map is an isomorphism because both have equal dimension
                  as \(\Q\)-vector spaces (tensor product is multiplicative in
                  dimension) and \(f\) is injective because the codomain is an
                  integral domain. Therefore also a surjective
                  \[
                      \tilde{f}:\Q(\sqrt{-2})\otimes_{\Q}\Q(\sqrt{-1})\twoheadrightarrow\Q(\sqrt{-1},\sqrt{-2}):x\otimes y\mapsto xy.
                  \]
                  which is an isomorphism of \(\Q\)-modules by the same
                  dimension argument.

                  The map \(\tilde{f}\) however is also multiplicative with
                  respect to the ring structures of both modules as algebras,
                  therefore \(\tilde{f}\) is a bijective algebra morphism, so
                  \(\Q(\sqrt{-2},\sqrt{-1})\) is the tensor product, so it is a
                  field.
              \end{solution}
    \end{enumerate}
\end{question}

\begin{question}
    Let \(k\) be a field and \(R=k[x,y]\) a polynomial ring, and
    \(I=(x,y)\subseteq R\). We begin with the following sequence:
    \[
        0\to R\overset{f}{\to}R\oplus R\overset{g}{\to}I\to 0
    \]
    where \(f(1)=(y,-x)\), \(g(1,0)=x\), and \(g(0,1)=y\).

    \begin{enumerate}[(a)]
        \item Show that the above sequence is exact.

              \begin{proof}
                  Because \(R\) is an integral domain any \(h\in R\) with
                  \((hx,hy)=(0,0)\) must have \(h=0\) so therefore we get
                  \(\ker(f)=(0)\).

                  The map \(g\) is surjective because for
                  \(h=\sum_{i,j}a_{ij}x^{i}y^{j}\) the element
                  \(\left(a_{11},\sum_{i\neq 1,j}a_{ij}x^{i}y^{j-1}\right)\) is
                  clearly mapped onto \(h\).

                  Also for \(h\in R\) we have
                  \begin{align*}
                      gf(h) & =g(yh,-xh) \\
                            & =xyh-yxh   \\
                            & =0
                  \end{align*}
                  so \(\im(f)\subseteq\ker(g)\).

                  For some \((a,b)\in\ker(g)\) we have \(xa+yb=0\). Because
                  \(R\) is a unique factorization domain \(xa\) and \(yb\) must
                  have a unique decomposition. In particular \(x|b\) and \(y|a\)
                  such that \(b/x=d\) and \(a/y=-d\). Therefore
                  \((a,b)=d(x,y)\).

                  Therefore \(f\) is injective, \(g\) is surjective and
                  \(\im(f)=\ker(g)\): the sequence is exact.
              \end{proof}

        \item Show that \(I\) is not flat as an \(R\)-module.

              \begin{proof}
                  Because the tensor product is right-exact the sequence
                  \[
                      R\otimes I\to (R\oplus R)\otimes I\to I\otimes I\to 0
                  \]
                  is exact. We obtain the following commutative diagram:
                  \begin{figure}[H]
                      \[
                          \begin{tikzcd}[ampersand replacement=\&]
                              \&\& {R\otimes I} \&\& {(R\oplus R)\otimes I} \&\& {I\otimes I} \&\& 0 \\
                              \\
                              0 \&\& R \&\& {R\oplus R} \&\& I \&\& 0
                              \arrow["f\otimes\Id", from=1-3, to=1-5]
                              \arrow["g\otimes\Id", from=1-5, to=1-7]
                              \arrow[from=1-7, to=1-9]
                              \arrow["{a\otimes b\mapsto ab}", from=1-3, to=3-3]
                              \arrow["{(a\oplus b)\otimes c\mapsto ac\oplus bc}", from=1-5, to=3-5]
                              \arrow["{\varphi:a\otimes b\mapsto ab}", from=1-7, to=3-7]
                              \arrow["f", from=3-3, to=3-5]
                              \arrow["g", from=3-5, to=3-7]
                              \arrow[from=3-1, to=3-3]
                              \arrow[from=3-7, to=3-9]
                          \end{tikzcd}
                      \]
                      \caption{A commutative diagram}
                  \end{figure}

                  The snake lemma then induces the following diagram with exact
                  columns where the long sequence of the top and bottom rows are
                  also exact:
                  \begin{figure}[H]
                      \[\begin{tikzcd}[ampersand replacement=\&]
                              \&\& 0 \&\& 0 \&\& {\ker(\varphi)} \\
                              \\
                              \&\& {R\otimes I} \&\& {(R\oplus R)\otimes I} \&\& {I\otimes I} \&\& 0 \\
                              \\
                              0 \&\& R \&\& {R\oplus R} \&\& I \&\& 0 \\
                              \\
                              \&\& {R/I} \&\& {R/I\oplus R/I} \&\& {I/I^{2}}
                              \arrow["f\otimes\Id", from=3-3, to=3-5]
                              \arrow["g\otimes\Id", from=3-5, to=3-7]
                              \arrow[from=3-7, to=3-9]
                              \arrow["{a\otimes b\mapsto ab}", from=3-3, to=5-3]
                              \arrow["{(a\oplus b)\otimes c\mapsto ac\oplus bc}", from=3-5, to=5-5]
                              \arrow["{\varphi:a\otimes b\mapsto ab}", from=3-7, to=5-7]
                              \arrow["f", from=5-3, to=5-5]
                              \arrow["g", from=5-5, to=5-7]
                              \arrow[from=5-5, to=7-5]
                              \arrow[from=5-7, to=7-7]
                              \arrow[from=5-3, to=7-3]
                              \arrow[from=1-3, to=3-3]
                              \arrow[from=1-5, to=3-5]
                              \arrow[from=1-3, to=1-5]
                              \arrow[from=1-5, to=1-7]
                              \arrow[from=1-7, to=3-7]
                              \arrow["0"{description}, from=7-3, to=7-5]
                              \arrow[from=7-5, to=7-7]
                              \arrow[from=1-7, to=7-3, out=0, in=180, looseness=2, overlay]
                              \arrow[from=5-1, to=5-3]
                              \arrow[from=5-7, to=5-9]
                          \end{tikzcd}\]
                      \caption{Yet another commutative diagram}
                  \end{figure}
                  In particular the map \(R/I\to R/I\oplus R/I\) in the bottom
                  left square is the zero map. This is because
                  \(\Im(f)=\subseteq I\oplus I\) so \(\overline{f(x)}=0\in
                  R/I\oplus R/I\) for all \(x\in R\). Because the projection
                  \(R\to R/I\) is surjective this means the map must be zero.

                  By exactness of the long sequence this means that the map
                  \(\ker(\varphi)\to R/I\) is surjective and therefore
                  \(\ker(\varphi)\neq 0\).

                  Now consider the trivially exact sequence
                  \[
                      0\to I\to R\to R/I\to 0.
                  \]
                  Tensored with \(I\) this will immediately give the sequence
                  \[
                      I\otimes I\to R\otimes I\to R/I\to 0
                  \]
                  is exact. Because \(R\otimes I\cong I\) by the isomorphism
                  \(r\otimes i\mapsto ri\) we get the same exact sequence
                  \[
                      I\otimes I\overset{\varphi}{\to} I\to I/I^{2}\to 0.
                  \]
                  However \(\varphi\) is not injective, so the original map
                  \(I\otimes I\to R\otimes I\) was not either. Therefore the
                  functor \(-\otimes_{R}I\) does not preserve injectiveness of
                  functions and is not exact. This means \(I\) is not flat.
              \end{proof}
    \end{enumerate}
\end{question}
\end{document}