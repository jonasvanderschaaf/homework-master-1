\documentclass{article}

\usepackage[utf8]{inputenc}
\usepackage[shortlabels]{enumitem}
\usepackage{scalerel}
\usepackage{amsthm, amssymb, mathtools, amsmath, bbm, mathrsfs, stmaryrd,nicefrac}
\usepackage[margin=1in]{geometry}
\usepackage[parfill]{parskip}
\usepackage[hidelinks]{hyperref}
\usepackage{quiver}
\usepackage{float}

\newcommand{\N}{\mathbb{N}}
\newcommand{\Z}{\mathbb{Z}}
\newcommand{\NZ}{\mathbb{N}_{0}}
\newcommand{\Q}{\mathbb{Q}}
\newcommand{\R}{\mathbb{R}}
\newcommand{\C}{\mathbb{C}}

\newcommand{\F}{\mathbb{F}}
\newcommand{\incl}{\imath}

\newcommand{\tuple}[2]{\left\langle#1\colon #2\right\rangle}

\DeclareMathOperator{\order}{orde}
\DeclareMathOperator{\Id}{Id}
\DeclareMathOperator{\im}{im}
\DeclareMathOperator{\coker}{coker}
\DeclareMathOperator{\ggd}{ggd}
\DeclareMathOperator{\kgv}{kgv}
\DeclareMathOperator{\degree}{gr}

\DeclareMathOperator{\gl}{GL}

\DeclareMathOperator*{\bigplus}{\scalerel*{+}{\sum}}

\DeclareMathOperator{\Aut}{Aut}
\DeclareMathOperator{\End}{End}
\DeclareMathOperator{\Hom}{Hom}
\newcommand{\isom}{\overset{\sim}{\longrightarrow}}

\newcommand{\Grp}{{\bf Grp}}
\newcommand{\Ab}{{\bf Ab}}
\newcommand{\cring}{{\bf CRing}}
\DeclareMathOperator{\modules}{{\bf Mod}}
\newcommand{\catset}{{\bf Set}}
\newcommand{\cat}{\mathcal{C}}
\newcommand{\chains}{{\bf Ch}}
\newcommand{\homot}{{\bf Ho}}
\DeclareMathOperator{\objects}{Ob}
\newcommand{\gen}[1]{\left<#1\right>}
\DeclareMathOperator{\cone}{Cone}
\newcommand{\set}[1]{\left\{#1\right\}}
\newcommand{\setwith}[2]{\left\{#1:#2\right\}}
\DeclareMathOperator{\Ext}{Ext}
\DeclareMathOperator{\nil}{Nil}
\DeclareMathOperator{\idem}{Idem}
\DeclareMathOperator{\JRad}{JRad}
\DeclareMathOperator{\Nil}{Nil}
\DeclareMathOperator{\Idem}{Idem}
\DeclareMathOperator{\Rad}{Rad}
\DeclareMathOperator{\Ann}{Ann}
\DeclareMathOperator{\Ass}{Ass}
\DeclareMathOperator{\Spec}{Spec}
\DeclareMathOperator{\lc}{lc}

\newcommand{\maxid}{\mathfrak{m}}
\newcommand{\primeid}{\mathfrak{p}}
\newcommand{\primeidd}{\mathfrak{q}}
\newcommand{\ideal}{\triangleleft}

\newenvironment{question}[1][]{\begin{paragraph}{Question #1}}{\end{paragraph}}
\newenvironment{solution}{\begin{proof}[Solution]\renewcommand\qedsymbol{}}{\end{proof}}

\newtheorem{theorem}{Theorem}[section]
\newtheorem{lemma}[theorem]{Lemma}
\newtheorem{proposition}[theorem]{Proposition}


\theoremstyle{definition}
\newtheorem{definition}[theorem]{Definition}
\newtheorem{remark}[theorem]{Remark}
\newtheorem{example}[theorem]{Example}
\newtheorem{corollary}[theorem]{Corollary}

\title{Notes Commutative Algebra}
\author{Jonas van der Schaaf}
\date{2022}

\begin{document}
\maketitle

\tableofcontents

\section{Rings}

\begin{definition}
    We only consider commutative rings with unit. For more information see an
    introductory course on rings.
\end{definition}

Important subsets of rings:

\begin{enumerate}
    \item Units: \(A^{*}:=\setwith{a\in A}{\exists b\in A:ab=1}\),
    \item Zero divisors: \(\setwith{a\in A}{\exists b\in A:b\neq0\wedge ab=0}\),
    \item Nilpotent elements: \(\nil(A):=\setwith{a\in A}{\exists m\in\Z_{>0},a^{m}=0}\)
    \item Idempotent elements: \(\idem(A):=\setwith{a\in A}{a^{2}=a}\).
\end{enumerate}

The units are an abelian group under multiplication. What about the others?

\begin{proposition}
    The following are equivalent formulations of a ring \(A\) being a field:
    \begin{enumerate}
        \item Every non-zero element is a unit,
        \item The only ideals of \(A\) are \((0)\) and \((1)\),
        \item Every ring morphism \(A\to B\neq 0\) is injective.
    \end{enumerate}
\end{proposition}

\begin{definition}
    An integral domain is a ring with no non-zero zero divisors.
\end{definition}

\begin{proposition}
    There is a bijection between product decompositions
    \[
        \set{A=A_{1}\times A_{2}}\leftrightarrow\set{\text{Pairs of idempotents \(e_{1},e_{2}\) with \(e_{1}e_{2}=0\) and \(e_{1}+e_{2}=0\)}}.
    \]

    \begin{proof}
        \begin{align*}
            (A_{1},A_{2})               & \mapsto e_{1} =(1,0), e_{2}=(0,1) \\
            A=A/(e_{2})\times A/(e_{1}) & \mapsfrom (e_{1},e_{2})
        \end{align*}
    \end{proof}
\end{proposition}

\begin{proposition}
    If \(f:A\to B\) ring hom and \(p\subseteq B\) prime, then
    \(f^{-1}(p)\subseteq A\) prime.

    The same doesn't hold for maximal ideals.
\end{proposition}

\begin{theorem}
    Every non-zero ring \(A\) has a maximal ideal.

    Consider the set \(\Sigma\) of proper ideals of \(A\). We have
    \((0)\in\Sigma\) so it's non empty. It is ordered by \(\subseteq\).

    Consider a chain \((I_{\alpha})_{\alpha}\), and take the ideal
    \(\bigcup_{\alpha}I_{\alpha}\). This is an upper bound. By Zorn's lemma
    the chain has a maximal element, which is a maximal ideal.
\end{theorem}

\begin{corollary}
    Every proper ideal \(I\) is contained in a maximum ideal.

    \begin{proof}
        Apply the theorem to \(A/I\).
    \end{proof}
\end{corollary}

\begin{corollary}
    Every non-unit \(x\) is contained in a maximal ideal.

    \begin{proof}
        Apply previous corollary to \(x\).
    \end{proof}
\end{corollary}

\begin{corollary}
    \(A\cong A^{*}\sqcup\bigcup I\) where \(I\) are the maximal ideals.
\end{corollary}

\begin{definition}
    A ring is local if it has exactly one maximal ideal.
\end{definition}

\begin{example}
    Localisation
    \[
        \Z_{(p)}=\setwith{q\in\Q}{\exists a,b:q=\frac{a}{b}, p\nmid b}
    \]

    This has unique maximal ideal \(\Z_{(p)}\).

    \begin{proof}
        We have \(f:\Z_{(p)}\to\F_{p}:a/b\mapsto a/b\) because \(p\neq b\). The
        kernel is \(p\Z_{(p)}\) and \(\F_{p}\) is a field so that is maximal.

        Suppose there is an \(x\in\Z_{(p)}\setminus p\Z_{(p)}\). Then
        \(1/x\in\Z_{(p)}\) so \(x\) is a unit so it cannot be contained in a
        maximum ideal.
    \end{proof}
\end{example}

\begin{proposition}
    A ring is a local ring iff \(A\setminus A^{*}\) is an ideal.

    \begin{proof}
        \(\Rightarrow\)

        If \(A\setminus A^*\) is an ideal, then it must be maximal, because if
        you add any element, it is a unit so the ideal is not proper.
    \end{proof}
\end{proposition}

\begin{proposition}
    \(\nil(A)\) is an ideal called nilradical and is given by \(\bigcap_{I\text{
            prime}}I\).
\end{proposition}

\begin{definition}
    The Jacobson radical is given by
    \[
        \JRad(A)=\bigcap_{I\text{ maximal}}I.
    \]
\end{definition}

\section{Radical ideals}
\begin{definition}
    For any ideal \(I\), define \(r(i)=\sqrt{I}=\setwith{x\in A}{\exists
        n\in\N:x^{n}\in I}\).
\end{definition}

\begin{example}
    For \(\primeid\triangleleft A\), \(\sqrt{\primeid}=\primeid\).
\end{example}

\begin{proposition}
    Let \(A\) be a ring. Then
    \[
        \Nil(A)=\bigcap_{\primeid\triangleleft A,\textnormal{\(\primeid\) prime}}.
    \]
\end{proposition}

\begin{proposition}
    For any \(I\ideal A\) we have
    \(\sqrt{I}=\bigcap_{I\subseteq\primeid,\textnormal{\(\primeid\)
            prime}}\primeid\).
\end{proposition}

\begin{definition}
    A ring is reduced if \(\sqrt{(0)}=(0)\). We also have
    \[
        \setwith{p\ideal A}{\textnormal{\(p\) prime}}\cong
        \setwith{p\ideal A/\sqrt{(0)}}{\textnormal{\(p\) prime}}
    \]
\end{definition}

\(A\) is an integral domain iff \(\sqrt{0}=(0)\).

\begin{example}
    \(T\) a top space, then \(C^{0}(T)=\set{\textnormal{cont. func.
            \(T\to\R\)}}\).

    Define \(\maxid_{t}=\setwith{f\in C^{0}(T)}{f(t)=0}\). Then
    \(C^{0}(T)\cong\R\) by \(\varphi:f\mapsto f(t)\) has kernel \(\maxid_{t}\).

    Then \(\bigcup_{t\in T}\maxid_{t}=(0)\). The Jacobson radical must be
    contained in \((0)\), so \((0)\subseteq\JRad\subseteq(0)\).
\end{example}

\begin{example}
    Let \(A=k[x_{1},\ldots,x_{n}]\) be an infinite field. Then
    \(\maxid_{a}=\bigplus_{1\leq i\leq n}(x_{i}-a_{i})\) which is the kernel
    of \(\varphi:A\twoheadrightarrow k:f\mapsto f(a)\).

    Because we work in an infinite field \(\bigcap_{a\in k^{n}}\maxid_{a}=(0)\).

    We have \(\JRad(A)\subseteq\bigcap_{a}\maxid_{a}=(0)\).
\end{example}

\begin{example}
    \[
        \Nil(\Z/4\Z)=(2)=\JRad(\Z/4\Z)
    \]
    \[
        \Nil(k[x]/(x^{2}))=(x)=\JRad(k[x]/x^{2}).
    \]
    In a local ring, the Jacobson radical is a maximal ideal.
\end{example}

\begin{example}
    \begin{align*}
        \Nil(\Z_{(p)})  & =(0)  \\
        \JRad(\Z_{(p)}) & =(p).
    \end{align*}
    Because \(\Z_{p}\) is a local ring and an integral domain.
\end{example}

\begin{proposition}[Weird but useful]
    Let \(A\) be a ring and \(x\in A\). Then \(x\in\JRad(A)\) iff \(\forall y\in
    A:1-xy\in A^{*}\).

    \begin{proof}
        If \(x\in\JRad(A)\) and \(1-xy\notin A^{*}\) for some \(y\). Then there is a
        \(\maxid\ideal A\) maximal with \(1-xy\in\maxid\) so \(q\in\maxid\) which is
        a contradiction.

        If \(x\notin\JRad(A)\), then there is  a \(\maxid\ideal A\) maximal ideal
        with \(x\notin\maxid\) so \((x)+\maxid=A\). Therefore there is some \(a\) such that
        \(xy+a=1\) for some \(a\in\maxid\), \(y\in A\).

        Therefore \(1-xy\in\maxid\) so \(1-xy\) is not a unit.
    \end{proof}
\end{proposition}

\begin{proposition}
    Let \(A\) be a ring and \(I,J\ideal A\). Then
    \(\sqrt{IJ}=\sqrt{I}\cap\sqrt{J}\).
\end{proposition}

\begin{definition}
    For any ring \(A\) and \(I,J\ideal A\), \(I,J\) are coprime iff \(I+J=A\).

    Equivalently \(x\) is a unit in \(A/(y)\) or \(y\) is a unit in \(A/(x)\).
\end{definition}

\begin{proposition}
    For any ideals \(I,J\ideal A\), then \(\sqrt{I}+\sqrt{J}=(1)\) iff
    \(I+J=1\).
\end{proposition}

\begin{proposition}
    Let \(A\) be a ring and \(I_{1},\ldots,I_{n}\ideal A\). Then
    \[
        \varphi:A\to\prod_{i}A/I_{i}:a\mapsto (a+I_{1},\ldots,a+I_{n})
    \]
    has:
    \begin{enumerate}
        \item \(\varphi\) is injective iff \(\bigcap_{i}I_{i}=(0)\),
        \item \(\varphi\) is surjective iff \(\forall i,j:I_{i}+I_{j}\neq(1)\).
    \end{enumerate}
\end{proposition}

\begin{proposition}
    Let \(A\) be a ring.
    \begin{enumerate}[i.]
        \item If \(\primeid_{1},\ldots,\primeid_{n}\ideal A\) prime ideals and
              \(I\subseteq\bigcup_{i}\primeid_{i}\), then
              \(I\subseteq\primeid_{i}\) for some \(i\).

        \item If \(I_{1},\ldots,I_{n}\ideal A\), \(\primeid\ideal A\) a prime
              ideal and \(\bigcap_{i}I_{i}\subseteq\primeid\), then
              \(I_{i}\subseteq\primeid\) for some \(i\).
    \end{enumerate}
\end{proposition}

\begin{definition}
    Let \(A\) be a ring and \(I\subseteq J\ideal A\). Then
    \[
        (I:J):=\setwith{x\in A}{xJ\subseteq I}.
    \]
    Think of \(x=\frac{a\in I}{b\in J}\). This is an ideal.
\end{definition}

\begin{example}
    We have \(J\subseteq I\ideal A\) iff \((I:J)=A\).

    \((I:A)=I\).

    \(((0):I)=\setwith{x\in A}{xJ=0}=\Ann(I)\).
\end{example}

\begin{definition}
    For a ring \(A\) define \(D=\setwith{x\in A}{\exists y:xy=0}\).
    Then
    \begin{align*}
        D & =\bigcup_{x\neq0}\Ann(x)        \\
          & =\bigcup_{x\neq0}\sqrt{\Ann(x)} \\
          & =\sqrt{D}
    \end{align*}
    where \(\sqrt{D}\) is just the set theoretic radical.
\end{definition}

\begin{definition}
    Let \(A,B\) be rings and \(f\in\Hom(A,B)\). Take \(I\ideal A\).
    Then \(f[I]\) is not necessarily an ideal, so we define \(I^{e}:=(f[I])\).

    If \(J\ideal B\) then \(J^{c}=f^{-1}[J]\) is an ideal in \(A\).
\end{definition}

\begin{proposition}
    If \(J\) is prime then \(J^{c}\) is prime. The same does not hold for
    extensions.
\end{proposition}

\begin{proposition}
    \begin{enumerate}[i.]
        \item \(I\subseteq I^{ec}\),
        \item \(J\supset J^{ce}\),
        \item \(I^{ece}=I^{e}\),
        \item \(J^{c}=J^{cec}\).
    \end{enumerate}

    \[
        \setwith{J^{c}}{J\ideal B}\cong \setwith{I^{c}:I\ideal A}.
    \]
\end{proposition}

\section{Modules and Nakayama's lemma}
\begin{definition}
    The definition of a module is trivial. So is the definitions of their
    morphisms.
\end{definition}

\begin{definition}
    For any two \(N,M\in\modules_{R}\) the homset \(\Hom(M,N)\) is an
    \(R\)-module by
    \begin{align*}
        r\times f & \mapsto (x\mapsto f(rx))     \\
        f+g       & \mapsto(x\mapsto f(x)+g(x)).
    \end{align*}

    For any \(u:M'\to M,v:N\to N'\) we have
    \[
        \alpha:\Hom(M,N)\to\Hom(M',N'):f\mapsto u\circ f\circ v.
    \]
\end{definition}

\begin{proposition}
    For all \(M\) we have
    \[
        \Hom_{R}(R,M)\cong M
    \]
    by
    \[
        \alpha:f\mapsto f(1).
    \]
\end{proposition}

\begin{definition}
    For \(M\in\modules_{R}\) we define a submodule as a subgroup
    \(N\subseteq M\) closed under multiplication by \(R\).

    The quotient \(M/N\) is also an \(R\)-module. The quotient map
    \(M\twoheadrightarrow M/N\) is \(R\)-linear.
\end{definition}

\begin{definition}
    If \(f\in\Hom(M,N)\) then \(\ker f\subseteq M\) is the largest subset with
    \(f(\ker(f))=0\) and is a module.

    The image \(f[M]\subseteq N\) is also a module.

    The cokernel \(\coker(f)=N/\im(f)\) is also an \(R\) module.

    For any \((M_{i})_{i\in I}\subseteq M\) we define
    \[
        \sum_{i\in I}M_{i}=\bigoplus_{i\in I}M_{i}.
    \]
\end{definition}

\begin{example}
    More ways to build modules. If \(R\) is a ring, \(I\ideal R\) is an ideal
    and \(M\in\modules_{R}\) then \(IM=\setwith{im}{i\in I,m\in M}\subseteq M\)
    is also a module and \(M/(IM)\) is a \(R/I\) module.
\end{example}

\begin{example}
    For any \(M\in\Hom_{R}\) and \(N,P\subseteq M\) submodules then
    \[
        (N:P)=\setwith{r\in R}{rP\subseteq N}
    \]
    is an ideal.

    In particular \((0:M):=\Ann_{R}(M)\) are the annihilators of \(M\).
\end{example}

\begin{definition}
    The module \(M\in\modules_{R}\) is faithful if \(\Ann(M)=0\). We can regard
    \(M\) as a module over \(R/\Ann(M)\) by the remark that \(M/IM\) is a module
    over \(R/I\).
\end{definition}

\begin{definition}
    For any collection \((M_{i})_{i\in I}\) we define
    \[
        \bigoplus_{i}M_{i}=\setwith{(m_{i})_{i\in I}}{\text{The tuple has finite support.}}
    \]
    And also
    \[
        \prod_{i}M_{i}=\set{(m_{i})_{i\in I}}.
    \]
\end{definition}

\begin{definition}
    A module \(M\in\modules_{R}\) is free if
    \[
        M\cong R^{(I)}:=\bigoplus_{i}R
    \]
    for some set \(I\).
\end{definition}

\begin{definition}
    A module \(M\in\modules_{R}\) is finitely generated if for some finite set
    \(I\) there is an \(f:R^{(I)}\twoheadrightarrow M\).
\end{definition}

\begin{lemma}[Nakayama's lemma]
    Let \(R\) be a ring and \(M\in\modules_{R}\) finitely generated. \(I\ideal
    R\) and \(\varphi:M\to M\) such that \(\varphi(M)\subseteq IM\). Then
    \(\exists n\in\N\) and \(a_{1},\ldots,a_{n}\in I\) such that
    \[
        \varphi^{n}+a_{1}\varphi^{n-1}+\cdots+a_{n-1}\varphi+a_{n}=0.
    \]

    \begin{proof}
        Let \(x_{1},\ldots,x_{n}\) generate \(M\). We have \(\varphi(x_{i})\in
        IM\) so \(\varphi(x_{i})=\sum_{j=1}^{n}a_{ij}x_{j}\). Define the matrix
        \(P_{ij}=(\delta_{ij}\varphi-a_{ij})_{ij}\). The entries of \(P\) are
        elements of \(\End(M)\): \(P_{ij}\in \End(M)\). Then
        \[
            P\cdot\begin{bmatrix}
                x_{1}  \\
                \vdots \\
                x_{n}
            \end{bmatrix}=\begin{bmatrix}
                0      \\
                \vdots \\
                0.
            \end{bmatrix}
        \]
        Therefore \(P^{adj}\) is the adjugate matrix.
        \(P^{adj}P=\det(P)\cdot\Id_{n\times n}\). \(\det(P)\) is an element of
        \(\End(M)\) with \(\det(P)x_{i}=0\) for all \(x_{i}\) and therefore
        \(\det(P)\in\Ann(M)\).

        The determinant \(\det(P)\) is monically polynomial in \(\varphi\) with
        coefficients which are elements of \(I\). This means that \(\det(P)\) is
        the zero map in \(\End(M)\).
    \end{proof}
\end{lemma}

\begin{corollary}
    Let \(M\in\modules_{R}\) and \(I\ideal R\) such that \(IM=M\). Then there is
    an \(x\in 1+I\) such that \(xM=0\).

    \begin{proof}
        Apply the lemma with \(\varphi=\Id\). Then
        \[
            x:=\varphi^{n}+a_{1}\varphi^{n-1}+\ldots=1+a_{1}+\ldots+a_{n}
        \]
        with \(a_{1}+\ldots+a_{n}\in I\) which is zero in \(\End(M)\).
    \end{proof}
\end{corollary}

\begin{corollary}
    Let \(M\in\modules_{R}\) and \(I\ideal R\) with \(I\subseteq\JRad(R)\)
    and suppose \(IM=M\). Then \(M=0\).

    \begin{proof}
        By the corollary there is an \(x\in 1+I\) such that \(xM=0\). But
        \(1+I\) is a unit so \(M\) must be a unit.
    \end{proof}
\end{corollary}

\begin{remark}
    For any local ring, the previous corollary is true for any ideal with
    \(IM=M\).
\end{remark}

\begin{corollary}
    Let \(M\in\modules_{R}\) be finitely generated and \(N\subseteq M\) and
    \(I\ideal R\) contained in \(\JRad(R)\) such that \(M=IM+N\). Then \(M=N\).

    \begin{proof}
        Apply the previous corollary to \(M/N\).
    \end{proof}
\end{corollary}

\begin{example}
    If \(R\) is local and \(\maxid\ideal R\) maximal. Then \(M/\maxid M\) is
    annihilated by \(\maxid\). Therefore \(M/\maxid M\) is a module over the
    field \(R/\maxid\) so a vector space. If \(M\) is finitely generated then
    \(M/\maxid M\) is a finitely generated vector space.
\end{example}

\begin{proposition}
    Let \(M\in\modules_{R}\) be finitely generated and \((x_{i})_{i}\) be such
    that their images in \(M/\maxid M\) generate as an \(R/\maxid\) module. Then
    \(x_{i}\) generate \(M\).

    \begin{proof}
        Let \(N=\gen{x_{1},\ldots,x_{n}}\subseteq M\) Then
        \[
            N\to M\to M/\maxid M
        \]
        is surjective so \(N+\maxid M=M\). Therefore \(M=N\) by a previous
        corollary.
    \end{proof}
\end{proposition}

\begin{definition}
    Let \(R\) be a ring and \((M_{i})_{i\in I}\) a sequence with maps
    \(f_{i}:M_{i-1}\to M_{i}\) is exact if \(\im(f_{i-1})=\ker(f_{i})\).
\end{definition}

\begin{proposition}
    Let \(R\) be a ring and \(M,M',M''\in\modules_{R}\).

    \begin{enumerate}
        \item Let \(M'\to M\to M''\to 0\) be a sequence. It is exact iff for all
              \(N\in\modules_{R}\) the sequence
              \[
                  0\to\Hom(M'',N)\to\Hom(M,N)\to\Hom(M',N)
              \]
              is exact.

        \item Let \(0\to M'\to M\to M''\) be a sequence. Then this is exact iff
              for all \(N\) the sequence
              \[
                  0\to\Hom(N,M')\to\Hom(N,M)\to\Hom(N,M'')
              \]
              is exact.
    \end{enumerate}

    \begin{proof}
        Proof by go fuck yourself.
    \end{proof}
\end{proposition}

\section{Tensor products and flatness}
\begin{definition}
    Let \(M,N,R\in\modules_{R}\). Then \(\varphi:M\times N\to P\) is bilinear
    if
    \begin{itemize}
        \item For all \(m\in M\) the map \(\varphi_{m}:N\to
              P:n\mapsto\varphi(m,n)\) is \(R\)-linear.
        \item For all \(n\in N\) the map \(\varphi_{n}:M\to
              P:m\mapsto\varphi(m,n)\) is \(R\)-linear.
    \end{itemize}
\end{definition}

\begin{definition}
    Let \(M,N\in\modules_{R}\). The tensor product of \(M,N\) is a pair
    \((M\otimes N,g:M\times N\to M\otimes N)\) such that every bilinear map from
    \(M\times N\) factors uniquely through \(M\otimes N\) by \(g\).
\end{definition}

\begin{proposition}
    The tensor product exists and is unique up to unique isomorphism.
\end{proposition}

\begin{proposition}
    Take \(M,N,P\in\modules_{R}\). Then
    \begin{align*}
        M\otimes N           & \cong N\otimes M                    \\
        (M\otimes N)         & \otimes P\cong M\otimes(N\otimes P) \\
        (M\oplus N)\otimes P & \cong M\otimes P\oplus N\otimes P   \\
        R\otimes M           & \cong M
    \end{align*}
    sending respectively
    \begin{align*}
        m\otimes n            & \mapsto n\otimes n             \\
        (m\otimes n)\otimes p & \mapsto m\otimes(n\otimes p)   \\
        (m,n)\otimes p\otimes & \mapsto(m\otimes p,n\otimes p) \\
        r\otimes m            & \mapsto rm
    \end{align*}
\end{proposition}

\begin{definition}[Restriction of scalars]
    Take \(f:R\to S\) for \(R,S\) rings. If \(N\in\modules_{S}\) then
    \(N\) is an \(R\)-module with \(r\cdot m=f(r)m\).
\end{definition}

\begin{proposition}
    If \(f:R\to S\) is a ring morphism,\(S\) is finitely generated as
    \(R\)-module, and that \(M\) is finitely generated as an \(S\)-module. Then
    \(N\) is finitely generated as an \(R\)-module.
\end{proposition}

\begin{definition}[Extension of scalars]
    If \(f:R\to S\) is a ring morphism and \(M\in\modules_{R}\) then
    \(S\otimes_{R}M\) is an \(S\)-module by \(s(s'\otimes m)=(ss')\otimes m\).
\end{definition}

\begin{proposition}
    If \(M\) is finitely generated \(R\)-module then \(S\otimes_{R} M\) is
    finitely generated as \(S\)-module.
\end{proposition}

\begin{proposition}
    Let \(M\in\modules_{R}\) and \(N,P\in\modules_{S}\) then
    \((M\otimes_{R}N)\otimes_{S}P\cong M\otimes_{R}(N\otimes_{S}P)\) as
    \(S\)-modules.
\end{proposition}

\begin{proposition}
    Let \(M,M',N,N'\in\modules_{R}\) with maps \(f:M\to M',g:N\to N'\) then
    there is a unique \(R\)-linear map \(f\otimes g:M\otimes N\to M'\otimes N'\)
    with \(m\otimes n\mapsto f(m)\otimes g(m)\). This is compatible with
    composition.
\end{proposition}

\begin{proposition}[Tensor-\(\Hom\) adjunction]
    Let \(R\) be a ring and \(M,N\in\modules_{R}\).

    Then there is a unique \(R\)-linear isomorphism
    \[
        \varphi:\Hom(M\otimes N, P)\to Hom(M,\Hom(N,P))
    \]
    with
    \[
        f\mapsto(m\mapsto(n\mapsto f(m\otimes n))).
    \]
    The inverse is given by
    \[
        f\mapsto (m\otimes n\mapsto f(m)(n)).
    \]
\end{proposition}

\subsection{Tensor products and exact sequences}
\begin{proposition}
    If
    \[
        M'\to M\to M''\to 0
    \]
    is an exact sequence of \(R\)-modules, then for \(N\in\modules_{R}\)
    \[
        M'\otimes N\to M\otimes N\to M''\otimes N\to 0
    \]

    \begin{proof}
        The proof is by tensor-\(\Hom\) adjunction. We know \(\Hom\) is right
        (or left I don't know) exact, so \(\otimes N\) is that as well.
    \end{proof}
\end{proposition}

\begin{example}
    Take \(R=\Z\). Then
    \[
        0\to \Z\overset{\cdot 2}{\to} \Z\to \Z/2\Z\to 0
    \]
    is exact but
    \[
        0\to \Z\otimes\Z/2\Z\overset{\cdot2}{\to}\Z\otimes \Z/2\Z\to \Z/2\Z\otimes\Z/2\Z\to 0
    \]
    is not because the second map is \(0\) so it is not injective.
\end{example}

\begin{definition}
    Let \(R\) be a ring and \(N\in\modules_{R}\). We say \(N\) is flat if any of the following exact conditions hold:
    \begin{enumerate}[i]
        \setcounter{enumi}{1}
        \item If
              \[
                  0\to M'\to M\to M''\to 0
              \]
              is exact then
              \[
                  0\to M'\otimes N\to M\otimes N\to M''\otimes N\to 0
              \]
              is exact.

        \item If \(f:M\to M'\) is injective then \(f\otimes N:M\otimes N\to
              M'\otimes N\) is injective.

        \item If \(f:M'\to M\) is injective and \(M,M'\) are finitely generated
              then \(f\otimes N:M'\otimes N\to M\otimes N\) is injective.
    \end{enumerate}
\end{definition}

\begin{example}
    The module \(\Z/n\Z\) is flat over \(\Z\) iff \(n=1\).

    The module \(\Z/n\Z\) is flat over \(\Z/n\Z\).
\end{example}

\begin{proposition}
    Free modules are flat.

    \begin{proof}
        Let \(N=\bigoplus_{I}R\). Then
        \[
            f\otimes N:(\bigoplus_{I}R)\otimes M'\to (\oplus_{I}R)M
        \]
        has equivalent map
        \[
            f\otimes N:\bigoplus_{I} R\otimes M'=\bigoplus_{I}M'\to \bigoplus_{I}M
        \]
        given by
        \[
            (m_{i})_{i\in I}\mapsto (f(m_{i}))_{i\in I}
        \]
        which is injective so \(f\otimes N\) is injective.
    \end{proof}
\end{proposition}

\begin{definition}
    An \(R\)-algebra is a a pair \((A,f)\) with \(A\) a ring and \(f:R\to A\) a
    ring hom.
\end{definition}

\begin{definition}
    Let \(R\) be a ring and \(A\) an \(R\)-algebra.
    \begin{itemize}
        \item We say \(A\) is finite if it is finitely generated as
              \(R\)-module.

        \item We say \(A\) is of finite type if it is finitely generated as an
              \(R\)-algebra. This is equivalent to existance of some
              \(g:R[x_{1},\ldots,x_{n}]\twoheadrightarrow A\).
    \end{itemize}
\end{definition}

\begin{definition}
    Let \(R\) be a ring and \(A,B\) two \(R\)-algebras. Then the tensor product
    over \(R\) is a triple \((A\otimes B, f:A\to A\otimes B, g:B\to A\otimes
    B)\) such that for every diagram there is a unique dashed arrow
    making it commute:
    \begin{figure}[H]
        \[
            \begin{tikzcd}[ampersand replacement=\&]
                R \&\& A \\
                \\
                B \&\& {A\otimes B} \\
                \&\&\& C
                \arrow[from=1-1, to=1-3]
                \arrow[from=1-1, to=3-1]
                \arrow["f", from=1-3, to=3-3]
                \arrow["g", from=3-1, to=3-3]
                \arrow[curve={height=-12pt}, from=1-3, to=4-4]
                \arrow[curve={height=12pt}, from=3-1, to=4-4]
                \arrow["{\exists !}"{description}, dashed, from=3-3, to=4-4]
            \end{tikzcd}
        \]
        \caption{A commutative diagram.}
        \label{fig:alg-tensor}
    \end{figure}
\end{definition}

\begin{remark}
    It turns out the algebra tensor product \(A\otimes B\) as modules is isomorphic
    to module tensor product.
\end{remark}

\section{Rings and Modules of Fractions}
\begin{definition}
    Let \(A\) be a ring. Then \(S\subseteq A\) is multiplicatively closed if
    \(S\) is a monoid with respect to multiplication.
\end{definition}

\begin{theorem}[Localisation]
    Let \(S\) be a multiplicatively closed set. Then there is a unique pair
    \((S^{-1}A,f)\) where \(S^{-1}A\) is a ring and \(f:A\to S^{-1}A\) such that
    the following diagram commutes for all maps \(g:A\to B\) such that \(g[S]\)
    is a group:
    \begin{figure}[H]
        \[
            \begin{tikzcd}[ampersand replacement=\&]
                A \&\& {S^{-1}A} \\
                \\
                \&\& B
                \arrow["f", from=1-1, to=1-3]
                \arrow["g"', from=1-1, to=3-3]
                \arrow["\exists!", dashed, from=1-3, to=3-3]
            \end{tikzcd}
        \]
        \caption{The universal property of localization.}
        \label{fig:localization}
    \end{figure}

    \begin{proof}
        Uniqueness: by universal property.

        Existence: Define \(S^{-1}A=\nicefrac{A\times A}{\sim}\) where
        \((a,s)\sim (a',s')\) iff \(\exists u\in S\) with \(u(as'-a's)=0\). This
        is an equivalence relation.

        To define ring operations we define \((a,s)+(b,s')=(as'+bs,ss')\) and
        \((a,s)\cdot (a',s')=(aa',ss')\). This is well-defined on the
        equivalence classes. This is unital ring by \(1=(1,1)\),
        \(0=(0,1)\), \(-(a,s)=(-a,s)\).

        Define \(f:A\to S^{-1}A\) by \(f(a)=(a,1)\). This is a ring morphism
        which maps \(a\in A\) to a unit in \(S^{-1}A\).

        To demonstrate the universal property let \(g:A\to R\) be a map such
        that \(f[S]\) is a group. Define \(\tilde{g}(a,s)=g(a)g(s)^{-1}\). This
        is a well-defined ring morphism. It is unique by \(g(1)=1\).
    \end{proof}
\end{theorem}

\begin{example}
    The following are examples of localisation:
    \begin{itemize}
        \item \(A\) integral domain \(S=A\setminus\set{0}\). Then
              \(S^{-1}A=Q(A)\).

        \item \(A=\Z\) and \(S=\setwith{2^{m}}{m\in\N}\) then
              \(S^{-1}A=\Z[\nicefrac{1}{2}]\).

        \item \(A=\Z\), \(S=\Z\setminus(2)\), then \(S^{-1}A=\Z_{(2)}\).

        \item Let \(A\) be a ring, then \(0\in S\) iff \(S^{-1}A=0\).

        \item Also \(a/1=0\) in \(S^{-1}A\) iff \(\exists u\in S\) with
              \(ua=0\).

        \item If \(\primeid\ideal A\) and \(S=A\setminus\primeid\) then
              \(S^{-1}A=:A_{\primeid}\) is local:
              \(A_{\primeid}=A_{\primeid}^{*}\sqcup \primeid A_{\primeid}\).
    \end{itemize}
\end{example}

\begin{definition}
    Let \(A\) be a ring \(S\subseteq A\) mult. closed and \(M\) an \(A\)-module.

    Then \(S^{-1}M=\nicefrac{M\times S}{\sim}\) where \((m,s)\sim(m',s')\) iff
    \(\exists u\) such that \(u(sm'-s'm)=0\). The set \(S^{-1}M\) is an
    \(S^{-1}A\)-module.

    Any \(\varphi:M\to N\) induces a map \(S^{-1}\varphi:S^{-1}M\to
    S^{-}N:(m,s)\mapsto (\varphi(m),s)\).

    The map \(S^{-1}\) is a functor \(\modules_{A}\to \modules_{S^{-1}A}\).
\end{definition}

\begin{proposition}
    The functor \(S^{-1}\) is exact.
\end{proposition}

\begin{proposition}
    Let \(A\) be a ring \(S\) multiplicatively closed.
    \begin{itemize}
        \item for \(M,N\ideal P\) we have \(S^{-1}(M+ N)\cong S^{-1}M+ S^{-1}N\)
        \item for \(M,N\ideal P\) we have \(S^{-1}(M+ N)\cong S^{-1}M\cap
              S^{-1}N\)
        \item for \(N\ideal M\) we have \(S^{-1}(M/N)\cong S^{-1}M/S^{-1}N\)
        \item \(S^{-1}(M\oplus N)\cong S^{-1}M\oplus S^{-1}N\)
        \item \(S^{-1}(M\otimes_{A} N)\cong S^{-1}M\otimes_{S^{-1}A}S^{-1}N\)
        \item \(S^{-1}A\) as a ring is an \(A\)-module so
              \(S^{-1}A\otimes_{A}M\cong S^{-1}M\)
        \item \(S^{-1}S^{-1}M\cong S^{-1}M\)
        \item \(S^{-1}A\) is flat as an \(A\)-module because localization
              functor is exact and \(S^{-1}M\cong S^{-1}A\otimes_{A}M\)
    \end{itemize}
\end{proposition}

\begin{proposition}[Local properties]
    Let \(A\) be a ring, \(S\subseteq A\) mult. closed and \(M\) an \(A\)-module.
    \begin{itemize}
        \item
              \begin{align*}
                  M=0 & \Leftrightarrow \forall\primeid\ideal A:M_{\primeid}=(A\setminus\primeid)^{-1}M=0 \\
                      & \Leftrightarrow \forall\maxid\ideal A:M_{\maxid}=(A\setminus\maxid)^{-1}M=0,
              \end{align*}

        \item \(\varphi:M\to N\) is injective/surjective iff
              \(\varphi_{\primeid}:M_{\primeid}\to N_{\primeid}\) is injective
              \(\forall\primeid\ideal A\). Same for all \(\maxid\ideal A\),

        \item Same for flatness of \(M\).
    \end{itemize}
\end{proposition}

\begin{proposition}
    Let \(f:A\to S^{-1}A\) be the universal map. If \(I\ideal A\) and \(J\ideal
    S^{-1}A\) then
    \begin{itemize}
        \item \(J^{ce}=J\)
        \item \(I^{ec}=\bigcup_{s\in S}(I:(s))\)
        \item \(I^{e}=(1)\) iff \(I\cap S\neq\varnothing\)
        \item No element of \(s\) is a zero-divisor in \(\nicefrac{A}{J^{c}}\)
        \item
              \[
                  \setwith{\primeid\ideal A}{\primeid\cap S=\varnothing}\overset{\sim}{\leftrightarrow}\set{\primeid\ideal S^{-1}A}.
              \]

        \item \(S^{-1}\) commutes with sums, products, radicals, and
              intersection of ideals,
        \item \(\Nil(S^{-1}A)=S^{-1}\Nil(A)\)
        \item \(Q(A/\primeid)=(A\setminus\primeid)^{-1}A/\primeid\cong\nicefrac{A_{p}}{pA_{\primeid}}\)
        \item if \(M\in\modules_{A}\) then \(S^{-1}\Ann(M)=\Ann(S^{-1}M)\)
        \item \(S^{-1}(N:P)=(S^{-1}N:S^{-1}P)\)
    \end{itemize}
\end{proposition}

\begin{proposition}
    Take \(f:A\to B\) and \(\primeid A\) a prime ideal. Then \(\primeid\) is a
    contraction of a prime ideal of \(B\) iff \(\primeid^{ec}=\primeid\).

    \begin{proof}
        If \(p=q^{c}\) with \(q\) prime, then
        \(\primeid^{ec}=q^{cec}=q^{c}=\primeid\).
    \end{proof}
\end{proposition}

\section{Primary decomposition}

\begin{example}
    In algebraic geometry if \(I,J\ideal k[x_{1},\ldots,x_{n}]\). Then \(Z(I\cap
    J)=Z(I)\cup Z(J)\). Primary decomposition will correspond to decomposition
    in irreducibles.
\end{example}

\begin{example}
    In number theory, any integer is a unique product of primes. The unique
    decomposition of ideals in rings of integers is a specific instance of
    primary decomposition.
\end{example}

\begin{definition}
    Given an ideal \(I\ideal A\). Then \(I\) is a primary ideal if
    \(I\neq A\) and for all \(a,b\in A\) if \(ab\in A\) then \(a\in I\)
    or \(b\in\sqrt{I}\).
\end{definition}

\begin{remark}
    The following is true for primary ideals:
    \begin{itemize}
        \item For a prime ideal \(I\ideal A\), the radical \(\sqrt{I}\) is a prime ideal.

              \begin{proof}
                  Take \(xy\in\sqrt{I}\). Then \(x^{n}y^{n}\in I\). Therefore \(x^{n}\in
                  I\) or \(y^{n}\in\sqrt{I}\). In both cases \(x\in\sqrt{I}\) or
                  \(y\in\sqrt{I}\).
              \end{proof}

        \item If \(I\ideal A\) is primary then all zero divisors of \(A/I\) are
              nilpotents.
    \end{itemize}
\end{remark}

\begin{proposition}
    If \(I\ideal A\) is a primary ideal. Then \(\sqrt{I}\) is the smallest prime
    ideal that contains \(I\).

    \begin{proof}
        By the previous remark \(\sqrt{I}\) is prime.

        Then consider \(\sqrt{I}=\bigcap_{\primeid\supseteq I}\primeid\).
    \end{proof}
\end{proposition}

\begin{definition}
    If \(\primeid\) we say \(I\) is \(\primeid\)-primary if
    \(\sqrt{I}=\primeid\) and \(I\) is primary.
\end{definition}

\begin{example}
    Examples of primary ideals:
    \begin{itemize}
        \item Prime ideals are primary
    \end{itemize}
\end{example}

\begin{example}
    Powers of primary ideal are not necessarily primary:
    \(A=\Q[x,y,z]/(xy-z^{2})\) with \(I=(x,z)^{2}\). This is not primary but
    \((x,z)\) is primary.

    Primary ideals are not necessarily prime: \(A=k[x,y]\) and \((x,y^{2})\) is
    \((x,y)\)-primary.
\end{example}

\begin{proposition}
    If \(I\ideal A\) has the property that \(\sqrt{I}\) is maximal, then \(I\)
    is primary.

    \begin{proof}
        We have that \(\sqrt{I}=\bigcap_{\primeid\supseteq I}\primeid\) is a
        maximal ideal, so \(\primeid=\sqrt{I}\) is the only prime ideal
        containing \(I\). Therefore \(A/I\) has exactly one prime ideal:
        \(\sqrt{I}/I\). Therefore \(A/I\) is local so \(\Nil(A/I)=\sqrt{I}/I\).

        Then \(A/I=(A/I)^{*}\sqcup\Nil(A/I)\). Zero divisors are not units,
        therefore they must be nilpotent so \(I\) is primary.
    \end{proof}
\end{proposition}

\begin{example}
    For \(A=k[x_{1},\ldots,x_{n}]\) the ideals
    \((x_{1}^{m_{1}},\ldots,x_{n}^{m_{n}})\ideal A\) are primary.
\end{example}

\begin{example}
    For \(A=k[x,y]\) the ideal \(I=(x^{2},xy)\) is not primary because
    \(\sqrt{I}=(x)\) and \(xy\in I\) but \(x\notin I\) and
    \(y\notin\sqrt{I}\).
\end{example}

\begin{example}
    Take \(I\ideal A\) an ideal. Call \(I\) irreducible if it cannot be written
    as \(I=I_{1}\cap I_{2}\) with \(I_{1},I_{2}\ideal A\) proper ideals of \(A\)
    then \(I\) is irreducible.

    Irreducible ideals in Noetherian ring (such as \(k[x_{1},\ldots,x_{n}]\))
    are primary.
\end{example}

\begin{lemma}
    If \(I_{1},I_{2}\) are \(\primeid\)-primary ideals in \(A\) for some prime
    \(\primeid\ideal A\). Then \(I_{1}\cap I_{2}\) is also \(\primeid\)-primary.

    \begin{proof}
        If \(ab\in I_{1}\cap I_{2}\) then if not \(a\in I_{1}\cap I_{2}\) we
        have \(b\in\sqrt{I_{i}}=\primeid\) for \(i\in\set{1,2}\). Therefore
        \(b\in\sqrt{I_{3-i}}\). Because
        \begin{align*}
            \sqrt{I_{1}\cap I_{2}} & =\sqrt{I_{1}}\cap\sqrt{I_{2}} \\
                                   & =\primeid
        \end{align*}
        we have \(b\in\sqrt{I_{1}\cap I_{2}}\).
    \end{proof}
\end{lemma}

\begin{lemma}
    Let \(I\ideal\primeid\) be primary and \(x\in A\).
    \begin{itemize}
        \item If \(x\in I\) then \((I:x)=(1)\).

              \begin{proof}
                  For all \(a\in A\) we have \(ax\in I\) so \((I:x)=(1)=A\).
              \end{proof}

        \item If \(x\notin I\) then \((I:x)\) is \(\primeid\)-primary.

              \begin{proof}
                  If \(ab\in(I:(x))\) and \(a\notin(I:x)\), then \(abx\in I\)
                  and \(a\notin I\).

                  We have \(\sqrt{I}\subseteq\sqrt{(I:x)}\). If
                  \(a\in\sqrt{(I:x)}\) then \(a^{n}\in(I:x)\) for some \(n\).
                  Because \(x\notin I\) this gives that \(a^{n}\in I\) so
                  \(a\in\sqrt{I}\).
              \end{proof}

        \item If \(x\notin\primeid\) then \((I:x)=I\).

              \begin{proof}
                  Use definition and observation that \(x\notin\primeid\) then
                  \(\overline{x}\in A/I\) is a unit. Then \(b^{n}\in I\) for
                  some \(n>0\).
              \end{proof}
    \end{itemize}
\end{lemma}

\begin{definition}
    An ideal \(I\ideal A\) is decomposable if there are
    \(q_{1},\ldots,q_{n}\) primary ideals such that \(I=\bigcap_{i}q_{i}\).
    This is called the primary decomposition.

    A primary decomposition is minimal if \(\sqrt{q_{i}}\neq\sqrt{q_{j}}\) for
    \(i\neq j\) and \(\bigcap_{j\neq i}q_{j}\nsubseteq q_{i}\) for all \(i\).
\end{definition}

\begin{example}
    For \(I=(x^{2},xy)\ideal k[x,y]\) we have that \(I=(x)\cap (x^{2},y)\) is a
    minimal primary decomposition. So is \((x)\cap(x,y^{2})^{2}\). So is
    \((x)\cap(x^{2},xy,y^{1},y^{2},y^{3},\ldots)\).

    The minimal primary decomposition is not unique.
\end{example}

\begin{example}
    Not all ideals are decomposable: consider \(C^{0}(T)\) for \(T\) compact and
    Hausdorff. Then \((0)\) is not decomposable.
\end{example}

\begin{theorem}
    If \(I\ideal A\) is decomposable with minimal decomposition
    \(I=\bigcap_{i}q_{i}\) then
    \[
        \Ass(I)=\set{\sqrt{q_{1}},\ldots,\sqrt{q_{m}}}=\setwith{\sqrt{(I:x)}}{x\in A}\cap\Spec(A)
    \]
    is independent of the decomposition

    \begin{proof}
        We have
        \begin{align*}
            (I:x)        & =\bigcap_{i}(q_{i}:x)                   \\
            \sqrt{(I:x)} & =\bigcap_{i}\sqrt{(q_{i}:x)}            \\
                         & =\bigcap_{i,x\notin q_{i}}\sqrt{q_{i}}.
        \end{align*}
        by def of minimal decomposition so choose an \(x\) such that
        \(x\in\bigcap_{j\neq i}q_{j}\). If \(\sqrt{(I:x)}\) is prime there is an
        \(i\) such that \(q_{i}=\sqrt{(I:x)}\).
    \end{proof}.
\end{theorem}

\begin{definition}
    The prime ideals in \(\Ass(I)\) are called the associative primes.

    If \(\primeid\in\Ass(I)\) and it is minimal for inclusion in \(\Ass(I)\)
    then it is called an isolated prime or minimal prime.

    If not then \(\primeid\) is called an embedded prime.
\end{definition}

\begin{proposition}
    If \(I\ideal A\) is decomposable then
    \[
        \bigcup_{\primeid\in\Ass(I)}\primeid=\setwith{x\in A}{(I:x)\neq I}.
    \]

    \begin{proof}
        Use definition, previous lemma (\((q_{i}:x)\)), and minimal decomposition.
    \end{proof}
\end{proposition}

\begin{proposition}
    If \(S\subseteq A\) multiplicatively closed set, \(q\ideal A\) primary.
    \begin{itemize}
        \item If \(S\cap\sqrt{q}\neq\varnothing\) then \(S^{-1}q=(1)\).
        \item If \(S\cap\sqrt{q}=\varnothing\) then \(S^{-1}q\) is
              \(S^{-1}\sqrt{q}\)-primary and \((S^{-1}q)^{c}=q\) under \(A\to
              S^{-1}A\).
    \end{itemize}
\end{proposition}

\begin{proposition}
    If \(S\subseteq A\) multiplicatively closed and \(A\to S^{-1}A\). If
    \(I\ideal A\) decomposable by \(\bigcap_{i}q_{i}\) then
    \begin{align*}
        S^{-1}I       & =\bigcap_{i,S\cap\sqrt{q_{i}}=\varnothing}S^{-1}q_{i} \\
        (S^{-1}I)^{c} & =\bigcap_{i,S\cap\sqrt{I}=\varnothing}q_{i}
    \end{align*}
    are primary decompositions.
\end{proposition}

\begin{definition}
    A set \(\Sigma\subseteq\Ass(I)\) is isolated if
    \(\primeid\in\Sigma,\primeid'\in\Ass(I)\) such that
    \(\primeid'\ideal\primeid\) then \(\primeid'\in\Sigma\).
\end{definition}

\begin{proposition}
    If \(I\ideal A\) has minimal decomposition \(\bigcap_{i}q_{i}\) and
    \(\Sigma\subseteq\Ass(I)\) an isolated set
    \(\set{\sqrt{q_{i_{1}}},\ldots\sqrt{q_{i_{s}}}}\). Then
    \(\bigcap_{j}q_{i_{j}}\) is independent of the decomposition.

    \begin{proof}

    \end{proof}
\end{proposition}

\section{Integral extensions}
\begin{definition}
    Let \(A\subseteq B\) be an inclusion of rings. Then \(b\in B\) is integral
    if there are \(a_{1},\ldots,a_{n}\) such that
    \(b^{n}+a_{1}b^{n-1}+\ldots+a_{n}=0\): i.e. \(b\) is the root
    of a monic polynomial \(f\in A[x]\).
\end{definition}

\begin{proposition}
    For an inclusion \(A\subseteq B\), the following statements are equivalent:
    \begin{enumerate}
        \item \(b\) is integral over \(A\),
        \item \(A[b]\) is a finitely generated \(A\)-module,
        \item \(A[b]\) is contained in a submodule \(C\subseteq B\) which is
              finitely generated as \(A\)-module,
        \item There is a faithful \(A[b]\)-module \(M\) that is finitely
              generated as \(A\)-module.
    \end{enumerate}

    \begin{proof}
        \((1)\Rightarrow(2)\)Let \(f\in A[x]\) be a monic with \(f(b)=0\). Then
        \(1,b,\ldots,b^{\deg f-1}\) generate \(A[b]\).

        \((2)\Rightarrow(3)\) Take \(C=A[b]\).

        \((3)\Rightarrow(4)\) We want to show \(\Ann(M)=0\). Take \(y\in A\)
        such that \(yC=0\). Then \(y\cdot1 = 0\) so \(y=0\).

        \((4)\Rightarrow(1)\) Let \(M\) be a faithful \(A[b]\) module and
        \(\varphi:M\to M\) the map \(\varphi(m)=bm\). By Nakayama chapter
        \(b^{n}+a_{1}b^{n-1}+\ldots+a_{n}\in\Ann(M)=0\). Therefore
        \(b\) is integral.
    \end{proof}
\end{proposition}

Useful to know:
\begin{itemize}
    \item If \(b_{1},\ldots,b_{r}\in B\) are integral over \(A\) then
          \(A[b_{1},\ldots,b_{r}]\) are finitely generated as \(A\)-module.
\end{itemize}

\begin{definition}
    The integral closure of \(A\) in \(B\): \(\setwith{b\in B}{\textnormal{\(b\)
            integral over \(A\)}}\) is a subring of \(B\).

    The ring \(B\) is integral over \(A\) if \(B\) is integral closure of \(A\).
    We say \(A\) is integrally closed over \(B\).
\end{definition}

\begin{itemize}
    \item An integral domain \(A\) is integrally closed in \(K(A)\).
    \item Being integrally closed is a local property.
\end{itemize}

\begin{definition}
    A morphism \(f:A\to B\) is integral if \(f(A)\subseteq B\) is integral.
\end{definition}

\begin{corollary}
    A ring \(B\) of finite type and an integral extension over \(A\) is finitely
    generated as an \(A\)-module.
\end{corollary}

\begin{proposition}
    If \(A\subseteq B\subseteq C\) are integral extensions then
    \(A\subseteq C\) is integral.

    The integral closure of \(A\) in \(B\) is integrally closed in \(B\).
\end{proposition}

\begin{proposition}
    If \(A\subseteq B\) is integral and \(J\ideal B\) then
    \(A/J\cap A\subseteq B/J\) is integral.

    If \(S\) is multiplicatively closed then
    \(S^{-1}A\subseteq S^{-1}B\) is integral.
\end{proposition}

\begin{proposition}
    If \(A\subseteq B\) and \(C\subseteq B\) integral closure of \(A\)
    then \(S^{-1}C\) is integral closure of \(S^{-1}A\) in \(S^{-1}B\).
\end{proposition}

\begin{proposition}
    \(A\subseteq B\) integral extension of integral domains, then \(A\) is a
    field iff \(B\) is a field.
\end{proposition}

\begin{corollary}
    \(A\subseteq B\) integral extension, \(\primeid\ideal B\) then \(\primeid\)
    is maximal iff \(\primeid\cap A\) is maximal.
\end{corollary}

\begin{theorem}[Going up]
    If \(A\subseteq B\) is an integral extension and
    \(\primeid_{1}\subseteq\primeid_{2}\ideal A\) prime ideals of \(A\) and
    \(\primeid_{1}'\ideal B\) prime ideal such that \(\primeid_{1}\cap
    A=\primeid_{1}\). Then there is a \(\primeid_{2}'\ideal B\) prime such that
    \(\primeid_{1}'\subseteq\primeid_{2}'\) and \(\primeid_{2}'\cap
    A=\primeid_{2}\).
\end{theorem}

\begin{proposition}
    If \(A\subseteq B\) is an integral extension and
    \(\primeid\subseteq\primeid'\ideal B\) such that \(\primeid\neq\primeid'\)
    then \(\primeid\cap A\neq q'\cap A\).

    \begin{proof}
        Assume \(\primeid\cap A=\primeid'\cap A\). We get the diagram:
        \[
            \begin{tikzcd}[ampersand replacement=\&]
                \&\& A \&\& B \\
                \\
                {\primeid A_{\primeid}}\&\& {A_{\primeid}} \&\& {B_{\primeid}}
                \arrow[hook, from=1-3, to=1-5]
                \arrow[from=1-3, to=3-3]
                \arrow[from=1-5, to=3-5]
                \arrow[hook, from=3-3, to=3-5]
                \arrow[hook, from=3-1, to=3-3]
            \end{tikzcd}
        \]
        then \(\primeid\cap A A_{\primeid}\) is a maximal ideal of \(A\).
        Therefore \(\primeid B_{\primeid}\subseteq\primeid' B_{\primeid}\) both
        restrict to \((\primeid\cap A) A_{\primeid}\). Therefore both are maximal
        so they are equal.
    \end{proof}
\end{proposition}

\begin{proposition}
    If \(A\subseteq B\) is an integral extension, \(\primeid\ideal A\) a prime
    ideal then there is a \(\primeid'\ideal B\) prime such that \(\primeid'\cap
    A=\primeid\).
\end{proposition}

\begin{theorem}[Going down]
    Let \(A\subseteq B\) be an integral extension of integral domains and \(A\)
    integrally closed.
\end{theorem}

\begin{definition}
    If \(A\subseteq B\) is an integral extension, then there is a finite
    morphism \(\Spec B\to\Spec A\) with dense image.

    If \(A\subseteq A^{\nu}\subseteq Q(A)\) with \(A^{\nu}\) integral closure
    we get \(\Spec A^{\nu}\to\Spec A\) called the normalization.
\end{definition}

\begin{proposition}[Noether normalization]
    If \(k\) is a field, \(A\neq 0\) finite \(k\)-algebra. Then there are
    \(a_{1},\ldots,a_{r}\) which are algebraically
    independent\footnote{\(k[a_{1},\ldots,a_{r}]\cong k[x_{1},\ldots,x_{r}]\) as
    \(k\)-algebras.} over \(k\). Then \(k[a_{1},\ldots,a_{r}]\subseteq A\) is
    integral.
\end{proposition}

\section{Valuation rings}
\begin{definition}
    Let \(k\) be a field. A subring \(A\subseteq K\) is a valuation ring of
    \(k\) if
    \begin{itemize}
        \item \(k=Q(A)\),
        \item \(\forall x\in k, x\in A\vee x^{-1}\in A\).
    \end{itemize}
\end{definition}

\begin{example}
    For any field \(k\), \(k\) is a valuation rng.

    For \(k=\Q\), \(\Z_{(p)}\) is a valuation ring for \(p\) prime.
\end{example}

\begin{remark}
    Valuation rings have the following properties:
    \begin{itemize}
        \item Being a valuation ring implies being local (show that \(A\setminus A^{*}\)
              is an ideal).
        \item If \(A\subseteq B\subseteq k\) are subrings and \(A\) is a
              valuation ring then \(B\) is as well.
    \end{itemize}

    Motivation: for a field \(k\) and \(\Gamma\) a totally ordered abelian group
    \(v:K^{*}\to\Gamma\) is a valuation if \(v\) is a group morphism with
    \(v(x+y)\geq\min(v(x),v(y))\). We say \(v(k^{*})\) is the value group of
    \(k\).

    Then \(A=\setwith{x\in k^{*}}{v(x)\geq 0}\cup\set{0}\) is a valuation ring.
\end{remark}

\begin{theorem}
    Every valuation ring arises as above.

    \begin{proof}
        Take \(\Gamma=k^{*}/A^{*}\) with \(\overline{x}\geq\overline{y}\) iff
        \(xy^{-1}\in A\). This is a totall order satisfying the required
        property.
    \end{proof}
\end{theorem}

\begin{example}
    The following are examples of valuation rings

    \begin{itemize}
        \item Fields are valuation rings by \(k^{*}\to 0\).
        \item \(v:\Q((t))^{*}\to\Z\) with \(\sum_{i=d}^{\infty}a_{i}t^{i}\mapsto
              \min\setwith{i}{a_{i}\neq 0}\). Then \(\Q[[t]]\) is valuation
              ring.
    \end{itemize}
\end{example}

\begin{definition}
    If \(A,B\subseteq k\) are local subrings, we say \(B\) dominates \(A\) if
    \begin{itemize}
        \item \(A\subseteq B\),
        \item \(\maxid_{B}\cap A=\maxid_{A}\).
    \end{itemize}
\end{definition}

\begin{remark}
    Every local ring \(A\subseteq k\) is dominated by a valuation ring \(B\) of
    \(k\).

    \begin{proof}
        Consider the set \(\setwith{B\subseteq k}{A\subseteq B\wedge 1\notin
            \maxid_{A}B}\).

        This set is non-empty because it contains \(A\). Zorn's lemma gives a
        maximal element \(B\) w.r.t. inclusion and maximal ideal
        \(\maxid_{B}=\maxid_{A}B\neq(1)\). Because \(B\) is maximal w.r.t.
        inclusion \(\maxid_{A}B\) is maximal: invert an element of
        \(\maxid_{B}-\maxid_{A}B\). This gives a larger ring with the same
        property.

        To prove \(\maxid_{B}\cap A=\maxid_{A}\) we get \(\maxid_{B}\cap
        A\subseteq\maxid_{A}\). Also \(\maxid_{B}\neq 1\) so \(\maxid_{B}\cap
        A\neq(1)\). Therefore \(\maxid_{A}=\maxid_{B}\cap A\).
    \end{proof}
\end{remark}

\begin{proposition}
    The above ring \(B\) is integrally closed by maximality.

    \begin{proof}
        If not then \(B\subseteq C\subseteq k\) has integral extension.

        Then there is a prime \(q\ideal C\) such that \(q\cap B=\maxid_{B}\).
        Therefore \(B\subseteq C_{q}\) which is therefore local and satisfies
        the same properties as the maximal from before. Therefore \(B=C_{q}\)
        which means \(B\) is integrally closed.

        To prove \(B\) is a valuation ring first consider \(x\in K\) with
        \(x\notin B\). Then \(1\in\maxid_{A}B[x]\).

        Therefore \(1=u_{0}+\ldots+u_{n}x^{n}\) with
        \(u_{i}\in\maxid_{A}B\neq(1)\).

        Since \(B\) is local \(u_{0}-1\notin\maxid_{B}\) so
        \(u_{0}-1\in B^{*}\).

        \(x^{-n}+\nicefrac{u_{1}}{u_{0}-1}x^{-(n-1)}+\ldots+\nicefrac{u_{n}}{u_{0}-1}=0\).
        \(B\) is integrally closed so \(x^{-1}\in B\).
    \end{proof}
\end{proposition}

\begin{proposition}
    Let \(k\) be a field and \(\Omega\) algebraically closed field.
    Define
    \[
        \Sigma=\setwith{(A,f)}{A\subseteq k,f\in\Hom(A,\Omega)}
    \]
    with partial order \((A,f)\subseteq (A',f')\) iff \(A\subseteq A'\) and
    \(f'|_{A}=f\). All maximal elements
    \((A,f)\) of \(\Sigma\) are valuation rings for \(k\).

    \begin{proof}
        We have \(\primeid=\ker f\) is a prime ideal of \(A\) as \(\Omega\) is a
        field. Universal property gives unique extension of \(f\) to
        \(f_{\primeid}:A_{\primeid}\to\Omega\). We have \(A\subseteq
        A_{\primeid}\) and \(f_{\primeid}|_{A}=f\). Therefore \(A=A_{\primeid}\):
        \(A\) is local.

        By the exercise there is a \(B\subseteq K\) valuation
        ring that dominates \(A\).

        If \(A\subseteq B\) is not algebraic, then there is an \(x\in B\) not
        algebraic over \(A\) so \(A[t]\cong A[x]\).

        We have that \(f\) extends to \(A[x]\to\Omega\). By maximality
        \(A=A[x]\). Therefore \(A\subseteq B\) is an algebraic extension.

        We have the diagram
        \[
            \begin{tikzcd}[ampersand replacement=\&]
                B \&\& {B/\maxid_{B}} \\
                \\
                A \&\& {A/\maxid_{A}} \&\& \Omega
                \arrow[from=3-1, hook, to=1-1]
                \arrow[from=1-1, to=1-3]
                \arrow[from=3-3, hook, to=1-3]
                \arrow[from=3-1, to=3-3]
                \arrow[from=3-3, to=3-5]
                \arrow["f"', curve={height=18pt}, from=3-1, to=3-5]
                \arrow["{\exists!}", dashed, from=1-3, to=3-5]
            \end{tikzcd}
        \]

        By maximality of \(A\) therefore conclude that \(A=B\). \(B\) is a
        valuation ring and therefore so is \(A\).
    \end{proof}
\end{proposition}

\begin{remark}
    All valuation rings are integrally closed.
\end{remark}

\begin{corollary}
    If \(A\subseteq k\) with integral closure \(\overline{A}\), then
    \(\overline{A}=\bigcap_{A\subseteq B,\textnormal{\(B\) valuation ring}}B\)
\end{corollary}

\section{Chain conditions}
\begin{definition}
    Fix a ring \(A\) and \(M\) an \(A\)-module. Define
    \(\Sigma\) a partially ordered set. It has the ascending chain condition if
    every ascending chain has some \(n\) such that \(x_{r}=x_{n}\) for all
    \(r\geq n\).
\end{definition}

\begin{definition}
    If \(\set{\textnormal{\(A\)-submodules of \(M\)}}\) has
    ascending chain condition \(M\) is Noetherian. If descending then Artinian.
\end{definition}

\begin{example}
    Finite modules are Noetherian and Artinian \(Z\)-modules.

    \(\Z\) is Noetherian but not Artinian.
\end{example}

\begin{definition}
    A ring is Noetherian/Artinian if it is Noetherian/Artinian as an
    \(A\)-module of itself (submodules of \(A\) are ideals of \(A\)).
\end{definition}

\begin{proposition}
    Noetherian and Artinian is compatible with:
    \begin{enumerate}
        \item Taking submodules,
        \item taking quotients,
        \item short exact sequences,
        \item direct sums.
    \end{enumerate}
\end{proposition}

\begin{definition}
    A chain \(M=M_{0}\supseteq M_{1}\supseteq\ldots\supseteq M_{n}\) is a chain
    of length \(n\). A chain is \emph{maximal} or a \emph{composition series} if
    there are no intermediate submodules.

    We define \(\ell(M)\) to be the length of a composition series of \(M\).
    A module such that \(\ell(M)\) is finite is called of finite length.
\end{definition}

\begin{proposition}
    Every composition series of \(M\) has the same length.
\end{proposition}

\begin{proposition}
    A module \(M\) is Artinian and Noetherian iff it has a composition series of
    finite length.
\end{proposition}

\begin{proposition}
    Every chain can be extended to a composition series if \(M\) is both
    Artinian and Noetherian.
\end{proposition}

\begin{proposition}
    An \(A\)-module \(M\) is Noetherian iff any submodule is finitely generated.

    \begin{proof}
        \(\Longleftarrow\) Take a chain \(M_{1}\subseteq M_{2}\subseteq\ldots\)
        of submodules. Take \(\bigcup_{i}M_{i}\subseteq M\). This is finitely
        generated by \(x_{1},\ldots,x_{n}\). Take
        \(m=\min\setwith{i\in\N}{\forall j:x_{j}\in M_{i}}\). Then
        \(M_{m}=\bigcup_{i}M_{i}\) so the chain is constant after \(M_{m}\).

        \(\Longrightarrow\) Fix any \(N\subseteq M\) a submodule. Let \(\Sigma\)
        be the set of finitely generated submodules of \(N\). In particular
        \(0\in\Sigma\) so it is non-empty. Therefore it has a maximal element
        \(N_{0}\subseteq N\). If \(N_{0}\neq N\) then take \(x\in N\setminus
        N_{0}\). Then \(N_{0}[x]\supsetneq N_{0}\) which contradicts with the
        fact that \(N_{0}\) was maximal. Therefore \(N_{0}=N\) and thus
        \(N\in\Sigma\) and \(N\) is finitely generated.
    \end{proof}
\end{proposition}

\begin{corollary}
    A ring is Noetherian iff every ideal is finitely generated.
\end{corollary}

\begin{proposition}
    Let \(A\twoheadrightarrow B\) be a Noetherian ring. If \(A\) is Noetherian
    then \(B\) is Noetherian.
\end{proposition}

\begin{theorem}[Hilbert's basis theorem]
    If \(A\) is Noetherian then \(A[x]\) is Noetherian.

    \begin{proof}
        Let \(I\ideal A[x]\) be an ideal. For \(f=\sum_{i=0}^{n}a_{i}x^{i}\) we
        define \(\lc(f)=a_{i}\) where \(i\) is the highest \(i\leq n\) such that
        \(a_{i}\neq 0\).

        The ideal \(L=\setwith{\lc(f)}{f\in I}\ideal A\) must be finitely
        generated by some \(\lc(f_{1},\ldots,\lc(f_{n}))\). Define
        \(N=\max\set{\deg(f_{1}),\ldots,\deg(f_{n})}\) and
        \(L_{d}=\set{\lc(f):f\in I,\deg(f)\leq d}\ideal A\). The latter must be
        finitely generated by some \(\lc(f_{d,1}),\ldots,\lc(f_{d,n_{d}})\).
        Note that \(L_{N}=L\) and define \(I'=(\setwith{f_{d,i}}{0\leq d\leq
            N,1\leq i\leq d})\subseteq I\).

        Supose \(I'\neq I\) take \(f\in I\setminus I'\) of smallest degree.

        If \(d=\deg(f)\leq N\) then \(\lc(f)\in L_{d}\) and thus
        \(\lc(f)=\sum_{i=1}^{n_{d}}a_{i}\lc(f_{d,i})\) for some \(a_{i}\in A\).
        Also \(f-\sum_{i=1}^{n_{d}}a_{i}d_{d,i}x^{d-\deg(f_{i})}\) has degree
        smaller than \(d\) and is in the complement \(I\setminus I'\). But we
        chose \(d=\deg(f)\) minimal so this is a contradiction.

        If \(d=\deg(f)>N\) then \(\lc(f)\in L\). Therefore
        \(\lc(f)=\sum_{i=1}^{n}a_{i}\lc(f_{i})\) and thus
        \(f-\sum_{i=1}^{n}a_{i}f_{i}x^{d-\deg(f_{i})}\) must have degree smaller
        than \(d=\deg(f)\) which is impossible because \(d\) was chosen to be
        minimal.
    \end{proof}
\end{theorem}

\begin{corollary}
    Polynomial rings over Noetherian rings in finite variables are Noetherian.
    Combining this with the fact that quotients of Noetherian rings are
    Noetherian if \(B\) is a finitely generated \(A\)-algebra, then \(B\) is
    Noetherian as a ring.

    The \(A\)-module \(A[x]\) is not Noetherian.
\end{corollary}

\begin{proposition}
    If \(A\subseteq B\subseteq C\), \(C\) is a finitely generated \(A\)-algebra,
    and \(B\subseteq C\) is an integral extension then \(B\) is a finitely
    generated \(A\)-algebra.
\end{proposition}

\begin{lemma}[Zariski lemma]
    Let \(k\subseteq E\) be a field extension and \(E\) a finitely generated
    \(k\)-algebra. Then \(E/k\) is finite.
\end{lemma}

\begin{theorem}[Weak nullstellensatz]
    Let \(k\) be a field and \(A\) a finitely generated \(k\)-algebra and
    \(\maxid\) a maximal ideal of \(A\). Then \(A/\maxid\) is finite over
    \(k\).

    \begin{proof}
        This is a specific application of the Zariski lemma.
    \end{proof}
\end{theorem}

\begin{theorem}
    If \(A\) is a Noetherian ring every ideal of \(A\) is decomposable.

    \begin{proof}
        If \(I\ideal A\) is proper and irreducible then \(I\) is primary.

        If there are \(xy\in A/I\) such that \(xy=0\) and \(y\neq 0\), then
        \(y\) is nilpotent. Consider the chain
        \(y\in\Ann(x)\subseteq\Ann(x^{2})\subseteq\ldots\). Then
        \(\Ann(x^{n})=\Ann(x^{n+1})\). If \(a\in(x^{n})\cap(y)\), then
        \(a=bx^{n}\) and \(b^{n}x^{n+1}=ax=0\) has
        \(b\in\Ann(x^{n+1})=\Ann(x^{n})\). Therefore \(a=0\) so the intersection
        \((x^{n})\cap (y)\) is \(0\) which is irreducible. Because \(y\neq 0\)
        we get \(x^{n=0}\).
    \end{proof}
\end{theorem}

\begin{proposition}
    If \(A\) is Noetherian and \(I\ideal A\) then there is an \(r\) such that
    \(\sqrt{I}^{3}\subseteq I\).

    \begin{proof}
        We have \(\sqrt{I}=(x_{1},\ldots,x_{s})\) because all ideals are
        finitely generated. We have \(x_{i}^{r_{i}}\in\sqrt{I}\). Take
        \(r=\sum_{i}r_{i}\). Then
        \(\sqrt{I}^{r}=(\prod_{i}x_{i}^{\alpha_{i}}:\sum_{i}\alpha_{i}=r)\).
        Then \(\alpha_{i}\geq r_{i}\) for some \(i\) so the product is contained
        in \(I\).
    \end{proof}
\end{proposition}

\begin{proposition}
    If \(A\) is Noetherian and \(I\ideal A\) a proper ideal. Then
    \(\Ass(I)=\setwith{(I:x)}{x\in A}\cap\Spec(A)\).
\end{proposition}

\section{Artins rings \& Dimension}
\begin{definition}
    The dimension of a ring \(R\) is the length of the longest chain of prime
    ideals
    \[
        \primeid_{0}\subsetneq \primeid_{1}\subsetneq\ldots\subseteq \primeid_{n}.
    \]
    The above sequence has length \(n\) (the amount of inclusions). This the
    dimension is an element of the set \(\N\cup\set{\infty,-\infty}\).
\end{definition}

\begin{proposition}
    If \(R\) is an Artinian ring then every prime ideal is maximal.

    \begin{proof}
        Let \(\primeid\ideal R\) be prime. Then \(R/\primeid\) is an Artinian
        domain. We want to show it is a field. Let \(x\in R/\primeid\) be
        non-zero, then the sequence
        \[
            (x)\supseteq(x^{2})\supseteq\ldots
        \]
        stabilizes because \(R/\primeid\) is artin. This means that
        \(x^{n}=yx^{n-1}\) for some \(n\in\N_{>0}\) and \(y\in R/\primeid\).
        Because \(R/\primeid\) is a domain we get \(x^{n-1}(xy-1)=0\) so
        \(y=x^{-1}\) and therefore \(x\) is invertible.
    \end{proof}
\end{proposition}

\begin{corollary}
    The dimension of an Artin ring is \(0\).

    \begin{proof}
        Any prime ideal is maximal so any chain of prime ideals has length
        \(0\).
    \end{proof}
\end{corollary}

\begin{corollary}
    For an Artin ring \(R\), \(\JRad(R)=\nil(R)\).

    \begin{proof}
        We have \(\textnormal{Max}(R)=\Spec{R}\). This has the corollary as
        immediate consequence.
    \end{proof}
\end{corollary}

\begin{proposition}
    Let \(R\) be an Artin ring, then \(\textnormal{Max}(R)\) is finite.

    \begin{proof}
        Define
        \[
            S=\set{\textnormal{Finite intersection of maximal ideals}}.
        \]
        then all chains in \(S\) stabilize. Therefore \(S\) has a minimal
        element \(\maxid_{1}\cap\ldots\cap\maxid_{n}\).

        Take \(x\in\textnormal{Max}(R)\). Then
        \[
            \maxid\cap\maxid_{1}\cap\ldots\cap\maxid_{n}=\maxid_{1}\cap\ldots\cap\maxid_{n}
        \]
        by minimality.

        By earlier proposition this means there is an \(n\) such that
        \(\maxid=\maxid_{n}\).
    \end{proof}
\end{proposition}

\begin{proposition}
    If \(R\) is Artinian then \(\Nil(R)\) is nilpotent.
\end{proposition}

\begin{proposition}
    A ring \(R\) is Artinian iff it's Noetherian and has dimension \(0\).

    \begin{proof}
        Let \(R\) be Artinian. The dimension of \(R\) is zero by a previous
        corollary.

        There arefinitely many maximal ideals \(\maxid_{1},\ldots,\maxid_{n}\).
        Then
        \(\prod_{i}\maxid_{n}\subseteq\bigcap_{i}\maxid_{i}=\Nil(R)\).

        Therefore \((\prod_{i}\maxid_{i})^{k}\subseteq(0)^{k}\) and so \(R\) is
        Noetherian.

        Suppose \(R\) is Noetherian with dimension \(0\). We want to write \(0\)
        as the product of maximal ideals. The ideal \((0)\) has a primary
        decomposition \((0)=q_{1}\cap\ldots\cap q_{n}\) because \(R\) is
        Noetherian. Therefore \(R\) has only finitely many minimal primes.

        Write \(\Nil(R)=\sqrt{0}=\bigcap_{i}\primeid_{i}\) where
        \(\primeid_{i}\) are the minimal associated primes. All these primes are
        maximal so \(\primeid_{i}\) is maximal. This means
        \[
            \prod_{i}\primeid_{i}\subseteq\bigcap_{i}\primeid_{i}=\Nil(R)
        \]
        which is nilpotent so \(0\) is the product of finitely many maximal
        ideals. Therefore \(R\) is Artinian.
    \end{proof}
\end{proposition}

\begin{proposition}
    Every Artinian ring is a finite product of Artinian logcal rings.
\end{proposition}

\begin{corollary}
    Let \(R\) be an Aritinian local ring with unique maximal ideal \(\maxid\).
    Then \(\Spec(R)=\maxid\) and \(\maxid=\Nil(R)\). We know the nilradical is
    nilpotent so \(\maxid\) is nilpotent.
\end{corollary}

\begin{proposition}
    Let \(R\) be Noetherian local with maximal ideal \(\maxid\). Then one of the
    following holds:
    \begin{enumerate}
        \item \(\maxid^{n}\neq\maxid^{n+1}\) for some \(n\in\N\),
        \item there is an \(n\) such that \(\maxid^{n}=0\) in which case \(R\)
              is Artin.
    \end{enumerate}

    \begin{proof}
        Suppose 1 fails and so \(\maxid^{n}=\maxid^{n+1}\) for some \(n\).
        By Nakayama's lemma \(\maxid^{n}=0\).

        To prove it is Artin take \(\primeid\in\Spec(R)\). Then
        \((0)=\maxid^{n}\subseteq\primeid\). Therefore
        \(\maxid=\sqrt(\maxid^{n})=\sqrt{\primeid}=\primeid\).

        This means that \(\dim(R)=0\).
    \end{proof}
\end{proposition}

\section{Discrete valuation rings}
We want to study Noetherian rings of dimension \(1\) but these are hard.
Therefore we look at domains and (usually) integrally closed rings.

\begin{definition}
    Let \(k\) be a field. A discrete valuation on \(k\) is a group homomorphism
    \[
        v:k^{*}\twoheadrightarrow\Z
    \]
    such that \(v(x+y)\geq\min(v(x),v(y))\). Extend to map \(v:k\to
    \Z\cup\set{\infty}\) with \(v(0)=\infty\). The valuation ring of \(v\) is
    \[
        O(v)=\setwith{x\in k}{v(x)\geq 0}
    \]
    which is a subring of \(k\). The ring \(O_{v}\) is a valuation ring.
\end{definition}

\begin{definition}
    A discrete valuation ring is a domain \(R\) such that there is a discrete
    valuation \(v\) on \(Q(R)\) such that \(O_{v}=R\).
\end{definition}

\begin{proposition}
    Let \(R\) be a discrete valuation ring. Then
    \begin{enumerate}
        \item \(R\) is local,
        \item \(R\) is Noetherian,
        \item \(\dim(R)=1\),
        \item \(R\) is integrally closed.
    \end{enumerate}

    \begin{proof}
        Let \(k=Q(R)\) and \(v\) a discrete valuation with \(O_{v}=R\). Let
        \(\maxid=v^{-1}[\Z_{>0}]\). Then \(\maxid\) is a maximal ideal and
        \(R\setminus\maxid=v^{-1}[\set{0}]\). Every element of
        \(v^{-1[\set{0}]}\) is a unit, hence \(\maxid\) is maximal and \(R\) is
        local.

        Let \(I\ideal R\). Then \(I=\maxid^{n}\). Let
        \(k=\min\setwith{v(x)}{x\in I}\). Then \(I=v^{-1}(\Z_{\geq
            0})=\maxid^{k}\). The left to right inclusion is clear.

        Take \(x\in v^{-1}(\Z_{\geq k})\), suppose \(v(x)=k'\). Then there is a
        \(y\in I\) such that \(v(y)=k'=v(x)\). Claim: \(\nicefrac{x}{y}\in
        R^{*}\):
        \begin{align*}
            v\left(\frac{x}{y}\right) & =v(x)-v(y) \\
                                      & =k'-k'     \\
                                      & =0.
        \end{align*}
        This proves the claim. We have the chain of all ideals except \((0)\):
        \[
            \maxid\supsetneq\maxid^{1}\supseteq\ldots
        \]
        so ascending chain condition holds. Therefore there are two prime ideals
        \(\Spec(R)=\set{(0),\maxid}\) so \(\dim(R)=1\).
    \end{proof}
\end{proposition}

\begin{proposition}
    Let \(R\) be a Noetherian local domain of dimension \(1\) with maximal ideal
    \(\maxid\) and residual field \(k=R/\maxid\). The following are equivalent:
    \begin{enumerate}
        \item \(R\) is a discrete valuation ring,
        \item \(R\) is integrally closed,
        \item \(\maxid\) is principal,
        \item \(\dim_{k}\maxid/\maxid^{2}=1\),
        \item Every non-zero ideal is a power of \(\maxid\).
    \end{enumerate}
\end{proposition}

\begin{proposition}
    Let \(R\) be a Noetherian domain of dimension \(1\). Let \(0\neq I\ideal
    R\). Then \(I\) can be written in a unique way as the product of primary ideals
    with distinct radicals.

    \begin{proof}
        Because \(R\) is Noetherian \(I\) has a primary decomposition
        \(I=\bigcap_{i}q_{i}\). Each \(\sqrt{q_{i}}=\primeid_{i}\) is non-zero
        prime and must therefore be maximal because \(R\) has dimension \(1\) so
        they are pairwise coprime. This means the \(q_{i}\) are also coprime so
        \[
            \bigcap_{i}q_{i}=\prod_{i}q_{i}.
        \]

        For uniqueness we can obtain primary decomposition from product
        which means uniqueness.
    \end{proof}
\end{proposition}

\begin{theorem}
    Let \(R\) be a Noetherian domain of dimension \(1\). The following are
    equivalent:
    \begin{enumerate}
        \item \(R\) is integrally closed,
        \item Every primary ideal \(I\ideal R\) is a prime power,
        \item For all \(\maxid\ideal R\) maximal the local ring \(R_{\maxid}\)
              is a discrete valuation ring.
    \end{enumerate}

    \begin{proof}
        We've seen all implications.
    \end{proof}
\end{theorem}

\begin{definition}
    Rings of the above form are called Dedekind domains.
\end{definition}

\begin{example}
    The rings \(\Z\) and \(k[x]\) are Dedekind domains.
\end{example}

\begin{corollary}
    In a Dedekind domain every non-zero ideal is a unique product of primary
    ideals.
\end{corollary}

\section{Completions}
\begin{definition}
    A topological group is a topological space \(G\) with am abelian group
    operation \(G\times G\to G\) such that \(a\cdot-\), \(-\cdot a\), and
    \((-)^{-1}\) are continuous.
\end{definition}

\begin{lemma}
    Let \(G\) be a topological group. Then it is Hausdorff iff
    \(\set{0}\) is a closed subset.

    \begin{proof}
        Proof by wizardry.
    \end{proof}
\end{lemma}

\begin{lemma}
    Let \(G\) be a topological group and \(H=\bigcap\setwith{U\subseteq G}{0\in
        U}\). Then
    \begin{enumerate}
        \item \(H\) is a subgroup of \(G\),

              \begin{proof}
                  Use that \(+\) and \(-\) are continuous. Take \(x,y\in H\). Then we want
                  that \(x+y\in H\). We have the continuous \(G\times G\to U\) given by \(+\).
                  We want \((x,y)\in+^{-1}(U)\). There is a map
                  \((\Id,0):G\to G\times G\). Therefore \((id,0)^{-1}+^{-1}U\)
                  is an open neighbourhood of \(0\in G\) so it contains \(x\).
                  Repeat for \(y\).
              \end{proof}

        \item \(H\) is the closure of \(\set{0}\),

              \begin{proof}
                  \begin{align*}
                      x\in\overline{0} & \Leftrightarrow\forall U\ni x: 0\in U    \\
                                       & \Leftrightarrow\forall U\ni 0: 0\in -U+x \\
                                       & \Leftrightarrow\forall U\ni 0: x\in U    \\
                                       & x\in U.
                  \end{align*}
              \end{proof}

        \item \(G/H\) is Hausdorff,

              \begin{proof}
                  By part 2 we know \(\set{0}\) is closed in \(G/H\).
              \end{proof}

        \item \(G\) is Hausdorff iff \(H=\set{0}\).

              \begin{proof}
                  \(\Leftarrow\) Easy

                  \(\Rightarrow\) By 3
              \end{proof}
    \end{enumerate}
\end{lemma}

\subsection{Topologies generated by sequences of subgroups}
\begin{definition}
    Let \(G\) be a topological group. A fundamental system of open
    neighbourhoods of \(0\) is a sequence \((U_{n})_{n\in\omega}\) with
    \begin{enumerate}
        \item \(U_{n+1}\subseteq U_{n}\),
        \item \(0\in U_{n}\)
        \item \(U_{n}\) open
        \item If \(V\subseteq G\) open with \(0\in V\) then there is some
              \(n\) such that \(U_{n}\subseteq V\).
    \end{enumerate}
    We focus on systems such that \(U_n\) are subgroups.
\end{definition}

\begin{definition}
    Given a group and a decreasing sequence of subsets \((U_{n})_{n\in\omega}\)
    containing \(0\), equip \(G\) with the topology such that all \(U_{i}\) are
    open.
\end{definition}

\begin{lemma}
    In the above setting, suppose \(U_{n}\) are subgroups. Then this is a
    fundamental system of neighbourhoods of \(0\) and \(G/U_{n}\) has discrete
    topology.
\end{lemma}

\begin{example}
    Given a ring \(R\) then \(I^{n}\) is sequence of decreasing subgroups.
    Call generated topology the \(I\)-adic topology.
\end{example}

We want to define completion.

\begin{definition}
    An inverse system is a sequence of abelian groups \((A_{n})_{n\in\omega}\)
    and morphisms \(\vartheta_{n}:A_{n}\to A_{n-1}\). A coherent sequence is a
    sequence \((a_{n})_{n\in\omega}\in\prod_{n\in\omega}A_{i}\) such that
    \(\vartheta(a_{n})=a_{n-1}\). This is an abelian group by pointwise
    addition.

    Notation: \(\lim_{A_{n}}\).

    Projection maps are group morphisms. If each \(A_{n}\) are topological groups then we can equip \(\lim A_{n}\)
    with the coarsest topology making \(\pi_{i}:\lim A_{n}\to A_{i}\)
    continuous.
\end{definition}

\begin{definition}
    If \(G\) is a topological group with \((U_{n})_{n\in\omega}\) a fundamental
    system op open neighbourhoods with \(U_{n}\) subgroups. The completion of \(G\)
    is
    \[
        \widehat{G}=\lim\nicefrac{G}{U_{n}}
    \]
    where \(G/U_{n}\) has discrete topology (which is the quotient topology).
\end{definition}

\begin{remark}
    \begin{enumerate}
        \item This definition coincides with ``Cauchy sequences mod null
              sequences''.
        \item There is a continuous map \(\varphi:G\to \widehat{G}:(g\to
              U_{n})_{n\in\omega}\).
        \item If \(f:G\to H\) is a continuous maps to topological groups
              with fsno given by subgroups. Then there is a natural continuous map
              \(\psi:\widehat{G}\to\widehat{H}\).
    \end{enumerate}
\end{remark}

\begin{definition}
    The topological group \(G\) is complete if the natural \(G\to \widehat{G}\) is a
    homeomorphism.
\end{definition}

\begin{theorem}
    The group \(\widehat{G}\) is complete.
\end{theorem}

\begin{example}
    Let \(k\) be a field, \(G=k[x]\) and \(I=(x)\). Give \(G\) the \(I\)-adic
    topology. Then \(\widehat{G}=k[[x]]\).
\end{example}

If \(G\) is a topological group, \((G_{n})_{n}\) fsno.

\begin{enumerate}
    \item We will show that \(\widehat{G^{n}}\) is a subgroup of \(\widehat{G}\).
    \item \((\widehat{G_{n}})\) is a fsno in \(\widehat{G}\).
    \item The natural map \(\widehat{G}\to\widehat{\widehat{G}}\) is an iso of topological
          groups.
\end{enumerate}

\begin{definition}
    A morphism of inverse systems makes the following square commute:
    \[
        \begin{tikzcd}[ampersand replacement=\&]
            {A_{n}} \&\& {A_{n-1}} \\
            \\
            {B_{n}} \&\& {B_{n-1}}
            \arrow["{\vartheta_{n}}", from=1-1, to=1-3]
            \arrow["{f_{n}}"', from=1-1, to=3-1]
            \arrow[from=3-1, to=3-3]
            \arrow["{f_{n-1}}", from=1-3, to=3-3]
        \end{tikzcd}
    \]
\end{definition}

\begin{definition}
    A sequence of inverse systems
    \[
        0\to (A_{n})_{n}\to (B_{n})_{n}\to (C_{n})_{n}\to 0
    \]
    is exact if each
    \[
        0\to A_{n}\to B_{n}\to C_{n}\to 0
    \]
    is exact.
\end{definition}

\begin{proposition}
    Given an exact sequence of an inverse system as above, we get an exact
    sequence:
    \[
        0\to \lim(A_{n})_{n}\to \lim(B_{n})_{n}\to \lim(C_{n})_{n}.
    \]
    If \((A_{n})_{n}\) is surjective (\(\vartheta_{n}:A_{n}\to A_{n-1}\) is
    surjective) then the final map is surjective.

    \begin{proof}
        Define \(\tilde{A}=\prod_{n}A_{n}\) and \(d_{A}:\tilde{A}\to\tilde{A}\)
        with \(d((a_{n})_{n})=(a_{n}-\vartheta_{n+1}(a_{n+1}))\). Then
        \(\lim(A_{n})_{n}=\ker d_{A}\). Similarly for \((B_{n})_{n}\) and
        \((C_{n})_{n}\). We then get the commutative diagram and by snake lemma:

        \[\begin{tikzcd}[ampersand replacement=\&]
                0 \&\& {\ker d_{A}} \&\& {\ker d_{B}} \&\& {\ker d_{C}} \\
                \\
                0 \&\& {\tilde{A}} \&\& {\tilde{B}} \&\& {\tilde{C}} \&\& 0 \\
                \\
                0 \&\& {\tilde{A}} \&\& {\tilde{B}} \&\& {\tilde{C}} \&\& 0 \\
                \\
                \&\& {\coker d_{A}}
                \arrow[from=3-1, to=3-3]
                \arrow[from=3-3, to=3-5]
                \arrow[from=3-5, to=3-7]
                \arrow[from=3-7, to=3-9]
                \arrow[from=5-1, to=5-3]
                \arrow[from=5-3, to=5-5]
                \arrow[from=5-5, to=5-7]
                \arrow[from=5-7, to=5-9]
                \arrow["{d_{A}}"{description}, from=3-3, to=5-3]
                \arrow["{d_{B}}"{description}, from=3-5, to=5-5]
                \arrow["{d_{C}}"{description}, from=3-7, to=5-7]
                \arrow[from=1-3, to=3-3]
                \arrow[from=1-5, to=3-5]
                \arrow[from=1-7, to=3-7]
                \arrow[from=1-1, to=1-3]
                \arrow[from=1-3, to=1-5]
                \arrow[from=1-5, to=1-7]
                \arrow[from=5-3, to=7-3]
                \arrow[out=0, in=180, looseness=1, from=1-7, to=7-3]
            \end{tikzcd}\]

        If \(\coker d_{A}=0\) then the entire sequence is exact.

        Let \((a_{n})_{n}\in\tilde{A}\). Define each \(x_{n}\in A_{n}\)
        recursively such that \(x_{n}-\vartheta_{n+1}(x_{n+1})=a_{n}\) and
        \(x_{1}=0\). This is possible because \(\vartheta_{n+1}\) is surjective.
    \end{proof}
\end{proposition}

\begin{corollary}
    Let
    \[
        0\to G'\to G\stackrel{p}{\to} G''\to 0
    \]
    be an exact sequence of abelian groups. Give \(G\) the topology
    from a sequence of subgroups \((G_{n})_{n}\) which form a fnso.

    Let \(G'_{n}=G'\cap G_{n}\) and \(G''_{n}=p[G_{n}]\). Then the sequence
    \[
        0\to\widehat{G'}\to\widehat{G}\to\widehat{G''}\to 0
    \]
    is exact.

    \begin{proof}
        Apply 2.10 to the sequence of inverse systems given by
        \[
            0\to G'/G'_{n}\to G/G_{n}\to G''/G''_{n}\to 0.
        \]
        The left inverse system is surjective so we get the desired exact
        sequence.
    \end{proof}
\end{corollary}

\begin{corollary}
    The group \(\widehat{G_{n}}\) is a subgroup of \(\widehat{G}\) and
    \[
        \frac{\widehat{G}}{\widehat{G_{n}}}\cong\frac{G}{G_{n}}.
    \]

    \begin{proof}
        Apply the previous corollary to
        \[
            0\to G_{n}\to G\to G/G_{n}\to 0.
        \]
        Therefore we have exact
        \[
            0\to\widehat{G_{n}}\to\widehat{G}\to G/G_{n}\to 0.
        \]
    \end{proof}
\end{corollary}

\begin{proposition}
    The natural map \(\widehat{G}\to\widehat{\widehat{G}}\) is an isomorphism.

    \begin{proof}
        Take the inverse limit with the above isomorphisms. Therefore the two
        completions are isomorphic.
    \end{proof}
\end{proposition}

When do different fsno's give the same topology/completion?

\begin{lemma}
    Let \(G\) be an abelian group and two decreasing sequences of subgroups
    \((G_{n})_{n}\) and \((H_{n})_{n}\). Consider four conditions
    \begin{enumerate}
        \setcounter{enumi}{-1}
        \item There is some \(n_{0}\) such that for all \(n>n_{0}\)
              \[
                  H_{n_{0}+n}\subseteq G_{n}
              \]
              and
              \[
                  G_{n_{0}+n}\subseteq H_{n},
              \]
        \item For all \(n\) there is an \(m\) such that
              \[
                  H_{m}\subseteq G_{n}
              \]
              and
              \[
                  G_{m}\subseteq H_{n}.
              \]
        \item \((G_{n})_{n}\) and \((H_{n})_{n}\) induce the same topology,
        \item \((G_{n})_{n}\) and \((H_{n})_{n}\) induce the same completion.
    \end{enumerate}
    Then \(0\Rightarrow1\), \(1\Leftrightarrow2\) and \(2\Rightarrow3\).

    \begin{proof}
        \(0\Rightarrow1\) is trivial.

        \(1\Rightarrow 2\) The topology is given by the union of translates
        of\(G_{n}\) and \(H_{n}\).

        \(2\Rightarrow 1\) Use that \(H_{n}\) are fsno in the \(H\) topology on
        \(G\) and vice versa.

        \(1\Rightarrow3\) Let \((h_{n})_{n\in\omega}\)  be a coherent sequence
        in \((G/H_{n})_{n}\). Fox \(m\). We want to build alement \(g_{m}\in
        G/G_{n}\).

        There is an \(m'\) such that \(H_{m'}\subseteq G_{m}\). So we get
        \(f:G/H_{m'}\to G/G_{m}\). Define \(g_{m}=f(h_{m'})\).
    \end{proof}
\end{lemma}

\begin{example}
    Let \(R\) be a ring and \(I\ideal R\) an ideal. Let \(M'\subseteq M\) be two
    \(R\)-modules. Then there are two natural topologies on \(M'\).
    \begin{itemize}
        \item \(I\)-adic: \(M'_{n}=I^{n}M'\),
        \item restricted \(I\)-adic: \(M''_{n}=M'\cap I^{n}M\).
    \end{itemize}
\end{example}

\begin{definition}
    Let \(I\ideal R\) and \(M\in\modules_{R}\). A filtration is an infinite
    decreasing sequence of submodules
    \[
        M=M_{0}\supseteq M_{1}\supseteq M_{2}\supseteq\ldots.
    \]
    A filtration is an \(I\)-filtration if \(IM_{n}\subseteq M_{m+1}\).

    A stable \(I\)-filtration if \(IM_{n}=M_{n+1}\) for sufficiently large
    \(n\).
\end{definition}

\begin{lemma}
    Any two stable \(I\)-filtrations have bounded difference and so generate the
    same topology.

    \begin{proof}
        Left to the reader.
    \end{proof}
\end{lemma}

\begin{lemma}[Artin-Rees lemma]
    If \(R\) is a Noetherian ring, \(I\ideal R\), and \(M\) a fin. gen
    \(R\)-module. If \((M_{n})_{n}\) is a stable \(I\)-filtration and
    \(M'\subseteq M\) a submodule then \(((M'\cap M_{n}))_{n}\) is a stable
    \(I\)-filtration on \(M'\).
\end{lemma}

\begin{corollary}
    The above lemma applied to \(M_{n}=I^{n}M\).
\end{corollary}

\begin{corollary}
    Completion is exact. Let \(R\) be a Noetherian ring, \(I\ideal R\) and
    \[
        0\to M'\to M\to M''\to 0
    \]
    exact sequence of fin. gen. modules. Then the sequence
    \[
        0\to\widehat{M'}\to\widehat{M}\to\widehat{M''}\to 0
    \]
    is exact.

    \begin{proof}
        By theorems.
    \end{proof}
\end{corollary}

\begin{proposition}
    In general for a Noetherian ring \(R\) we have
    \begin{enumerate}
        \item \(\widehat{R}\) (as \(I\)-adic completion) is a flat \(R\)-module,
        \item If \(M\) is finitely generated then
              \(\widehat{R}\otimes_{R}M\cong\widehat{M}\).
    \end{enumerate}
\end{proposition}

\begin{theorem}[Krull's theorem]
    Let \(R\) be a Noetherian ring and \(I\ideal R\). If \(M\) is a fin. gen.
    \(R\)-module then
    \[
        \ker(M\to\widehat{M})=\bigcap_{n}I^{n}M=\setwith{x\in M}{\exists a\in 1+I:ax=0}.
    \]
\end{theorem}

\begin{proposition}
    Let \((R,\maxid)\) be Noetherian local. Then \(\widehat{R}\) is local
    with maximal ideal \(\maxid\).
\end{proposition}

\section{Graded Rings}
\begin{definition}
    A graded ring is a pair \((R,(R_{i})_{i\in\omega})\) such that
    \begin{enumerate}
        \item \(R_{n}\) is a subgroup of \(R^{+}\),
        \item \(R_{m}R_{n}\subseteq R_{m+n}\),
        \item \(\bigoplus_{n\in\omega}R_{n}\to R\) is an isomorphism of abelian
              groups.
    \end{enumerate}
\end{definition}

\begin{remark}
    \(R_{0}\) is a subring of \(R\) and \(R\) is an \(R_{0}\)-algebra. Furthermore
    each \(R_{i}\) is an \(R_{0}\)-module.
\end{remark}

\begin{definition}
    If \((R,(R_{n})_{n\in\omega})\) is a graded ring then a graded
    \((R,(R_{n})_{n\in\omega})\) module is a pair \((M,(M_{n})_{n\in\omega})\)
    where
    \begin{enumerate}
        \item \(M\) is an \(R\)-module,
        \item each \(M_{n}\) is a subgroup such that \(R_{n}M_{n'}\subseteq
              M_{n+n'}\)
        \item \(\bigoplus_{i\in\omega}M_{i}\to M\) is an isomorphism.
    \end{enumerate}
    A morphism of graded modules is a module morphism \(f:M\to N\)
    such that \(f(M_{n})\subseteq N_{n}\).
\end{definition}

\begin{example}
    Let \(M=(x^{2}+y^{2},y^{3})\ideal R=k[x,y]\). Define \(M_{n}=M\cap R_{n}\).

    The following is NOT an example: \(M=(x^{2}+y^{3})\ideal R\). Notice that
    the generator is not homogeneous. Taking \(M_{n}=M\cap R_{n}\). Then the
    direct sum property fails. This is because the map from the direct sum is
    not surjective. We have \(M_{2}=M_{1}=0\) and \(x^{2}+y^{3}\) cannot be
    written as the sum of homogeneous polynomials of degree \(\geq 3\).

    In general: an ideal \(I\ideal k[x_{1},\ldots,x_{n}]\) admits the structure
    of a graded module iff it can be generated by homogeneous polynomials.
\end{example}

\begin{proposition}
    For a graded ring \(R\) the following are equivalent:
    \begin{enumerate}
        \item \(R\) is Noetherian
        \item \(R_{0}\) is Noetherian and \(R\) is finitely generated as an
              \(R_{0}\)-algebra.
    \end{enumerate}

    \begin{proof}
        \(2\Rightarrow1\) By theorem from book.

        \(1\Rightarrow 2\) \(R^{+}=\bigoplus_{i>1}R_{n}\ideal R\) is an ideal of
        \(R\) and  \(R/R^{+}=R_{0}\) so \(R\) is Noetherian. The other part is
        omitted.
    \end{proof}
\end{proposition}

\begin{definition}[Blowup algebra]
    Let \(R\) be a ring, \(I\ideal R\) an ideal, \(M\) an \(R\)-module and
    \((M_{n})_{n\in\omega}\) an \(I\)-filtration.

    The blowup algebra and module are \(R^{*}\) and \(M^{*}\). Then
    \(R^{*}:=\bigoplus_{i\in\omega}I^{n}\). Similarly
    \(M^{*}=\bigoplus_{i\in\omega}M_{n}\).
\end{definition}

\begin{definition}[Associated graded ring]
    Let \(R\) be a ring, \(I\ideal R\) an ideal, \(M\) an \(R\)-module and
    \((M_{n})_{n\in\omega}\) an \(I\)-filtration.

    We define
    \begin{align*}
        G_{I}(R) & =\bigoplus_{n\in\omega}I^{n}/I^{n+1}, \\
        G_{I}(M) & =\bigoplus_{n\in\omega}M_{n}/M_{n+1}.
    \end{align*}
\end{definition}

\begin{example}
    \(R=\Z\), \(I=(p)\). Then \(G(R)\cong\F_{p}[t]\).
\end{example}

\begin{proposition}
    Let \(R\) be a Noetherian ring and \(I\ideal R\). Then
    \begin{enumerate}
        \item \(G_{U}(R)\) is Noetherian,
        \item \(G_{I}(R)=G_{\widehat{I}}(\widehat{R})\) where \(\widehat{R}\) is the
              \(I\)-adic completion.
    \end{enumerate}
\end{proposition}

\begin{proposition}
    If \(R\) is Noetherian and \(I\ideal R\), then \(\widehat{R}\) is
    Noetherian.
\end{proposition}

If \(R\) is a graded ring and \(R_{0}\) is a field, how does
\(\dim_{R_{0}}R_{n}\) vary with \(n\)?

\begin{example}
    Let \(R=k[x_{1},\ldots,x_{s}]\). Then
    \[
        \lambda(R_{n}):=\dim_{R_{0}}R_{n}=\frac{(s+n-1)!}{n!(s-1)!}.
    \]

    Generating function:
    \begin{align*}
        P(R,t) & =\sum_{n=0}^{\infty}\lambda(R_{n})t^{n} \\
               & =\frac{1}{(1-t)^{s}}.
    \end{align*}

    The function \(\N\to\Z\) given by \(n\mapsto \lambda(R_{n})\) is a
    polynomial in \(n\) with \(\Q\)-coefficients.
\end{example}

\begin{remark}
    Let \(R\) be a Noetherian graded ring (\(R_{0}\) Noetherian and \(R\) fin.
    gen as \(R_{0}\)-algebra). Let's say it's generated by
    \(x_{1},\ldots,x_{s}\). We can choose these to be homogeneous of degree
    \(k_{i}>0\).

    Let \(M\) be a fin. gen. graded \(R\)-module. Then each \(M_{n}\) is
    finitely generated as \(R_{0}\)-module.

    Let \(\lambda:\set{\textnormal{fin. gen. \(R_{0}\)-modules}}\to\Z\) be an
    additive function: given short exact sequence
    \[
        0\to M'\to M\to M''\to 0
    \]
    we have \(\lambda(M)=\lambda(M')+\lambda(M'')\).
\end{remark}

\begin{theorem}[Hilbert series]
    \(P(M,t)\) is a rational function of the form
    \[
        \frac{f(t)}{\prod_{i=1}^{s}(1-t^{k_{i}})}
    \]
    where \(f\in\Z[t]\).

    \begin{proof}
        Induction on \(s\).

        For \(s=0\) this is easy.

        Let \(s>0\) and assume the theorem is true for \(s-1\) generators.
        Then we have an exact sequence
        \[
            0\to K_{n}\to M_{n}\stackrel{\cdot x_{s}}{\to}M_{n+k_{s}}\to L_{n+k_{s}}\to 0.
        \]
        Let \(L=\bigoplus_{i\in\omega}K_{n}\) and
        \(L=\bigoplus_{i\in\omega}L_{n+k_{s}}\). Both are finitely generated
        \(R\)-modules killed by \(x_{s}\) so both are finitely generated
        \(R/(x_{s})\)-modules.

        Additivity of \(\lambda\) gives that
        \[
            \lambda(K_{n})-\lambda(M_{n})+\lambda(M_{n+k_{s}})-\lambda(L_{n+k_{s}})=0.
        \]
        Multiply this equality by \(t^{n+k_{s}}\) and sum over \(n\) to get
        \begin{align*}
            (1-t^{k_{s}})P(M,t) & =P(L,t)-t^{k_{s}}P(K,t)+g(t)
        \end{align*}
        with \(g\in\Z[t]\). Algebraic manipulation I don't want to \LaTeX{}
        gives the proof.
    \end{proof}
\end{theorem}

\begin{definition}
    We define \(d(M)\) as the order of the pole of \(P(M,t)\) at \(t=1\).
    Explicitly this means \(f(t)=(1-t)^{r}\tilde{f}(t)\) with
    \((1-t)\nmid\tilde{f}\). Then \(d(M)=s-r\).
\end{definition}

\begin{proof}[Hilbert polynomial of \(M\)]
    If all \(k_{i}=1\) then for \(n\) sufficiently large \(\lambda(M_{n})\) is a
    polynomial in \(n\) with \(\Q\)-coefficients and degree \(d(M)-1\).

    \begin{proof}
        Proof by authority.
    \end{proof}
\end{proof}

\begin{corollary}
    If \(x\in R_{k}\) for some \(k\geq 0\) is a non-zero divisor of \(M\) then
    \(d(M/xM)=d(M)-1\).
\end{corollary}

\section{Something}
Let \((R,\maxid)\) be a Noetherian local ring and \(\primeid\ideal R\) an
\(\maxid\)-primary ideal. \(M\) a finitely generated \(R\)-module, \(M_{n}\) a
\(\primeidd\)-stable filtration.

Associated graded ring \(\bigoplus_{n}\primeidd^{n}/\primeidd^{n+1}\) and
associated graded module \(\bigoplus_{n}M_{i}/M_{i+1}\).
\end{document}