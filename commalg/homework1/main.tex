\documentclass{article}

\usepackage[utf8]{inputenc}
\usepackage{enumerate}
\usepackage{amsthm, amssymb, mathtools, amsmath, bbm, mathrsfs, stmaryrd}
\usepackage[margin=1in]{geometry}
\usepackage[parfill]{parskip}
\usepackage[hidelinks]{hyperref}
\usepackage{quiver}
\usepackage{float}
\usepackage{cleveref}

\newcommand{\N}{\mathbb{N}}
\newcommand{\Z}{\mathbb{Z}}
\newcommand{\NZ}{\mathbb{N}_{0}}
\newcommand{\Q}{\mathbb{Q}}
\newcommand{\R}{\mathbb{R}}
\newcommand{\C}{\mathbb{C}}

\newcommand{\maxid}{\mathfrak{m}}
\newcommand{\primeid}{\mathfrak{p}}

\newcommand{\F}{\mathbb{F}}
\newcommand{\incl}{\imath}

\newcommand{\tuple}[2]{\left\langle#1\colon #2\right\rangle}

\DeclareMathOperator{\order}{orde}
\DeclareMathOperator{\Id}{Id}
\DeclareMathOperator{\im}{im}
\DeclareMathOperator{\ggd}{ggd}
\DeclareMathOperator{\kgv}{kgv}
\DeclareMathOperator{\degree}{gr}
\DeclareMathOperator{\coker}{coker}

\DeclareMathOperator{\gl}{GL}

\DeclareMathOperator{\Aut}{Aut}
\DeclareMathOperator{\Hom}{Hom}
\DeclareMathOperator{\End}{End}
\newcommand{\isom}{\overset{\sim}{\longrightarrow}}

\newcommand{\Grp}{{\bf Grp}}
\newcommand{\Ab}{{\bf Ab}}
\newcommand{\cring}{{\bf CRing}}
\DeclareMathOperator{\modules}{{\bf Mod}}
\newcommand{\catset}{{\bf Set}}
\newcommand{\cat}{\mathcal{C}}
\newcommand{\chains}{{\bf Ch}}
\newcommand{\homot}{{\bf Ho}}
\DeclareMathOperator{\objects}{Ob}
\newcommand{\gen}[1]{\left<#1\right>}
\DeclareMathOperator{\cone}{Cone}
\newcommand{\set}[1]{\left\{#1\right\}}
\newcommand{\setwith}[2]{\left\{#1:#2\right\}}
\DeclareMathOperator{\Ext}{Ext}
\DeclareMathOperator{\Nil}{Nil}
\DeclareMathOperator{\idem}{Idem}
\DeclareMathOperator{\rad}{Rad}
\DeclareMathOperator{\JRad}{JRad}

\newenvironment{question}[1][]{\begin{paragraph}{Question #1}}{\end{paragraph}}
\newenvironment{solution}{\begin{proof}[Solution]\renewcommand\qedsymbol{}}{\end{proof}}

\newtheorem{theorem}{Theorem}[section]
\newtheorem{lemma}[theorem]{Lemma}
\newtheorem{proposition}[theorem]{Proposition}


\theoremstyle{definition}
\newtheorem{definition}[theorem]{Definition}
\newtheorem{remark}[theorem]{Remark}
\newtheorem{example}[theorem]{Example}
\newtheorem{corollary}[theorem]{Corollary}

\title{Homework Commutative Algebra}
\author{Jonas van der Schaaf \and 12798797 \and Universiteit van Amsterdam \and jonas.vanderschaaf@student.uva.nl}
\date{September 26, 2022}

\begin{document}
\maketitle

\begin{question}[1]
    Let \(A\) be a local ring. Show that \(A\) has exactly two idempotent
    elements.

    \begin{proof}
        The ring \(A\) can clearly not be trivial, because the trivial ring has
        no maximal ideals. Therefore \(0\) and \(1\) are two non-trivial
        idempotents.

        Suppose that \(x\) is an idempotent not equal to \(0\) or \(1\). Then
        \(e^{2}=e\) so \(e^{2}-e=(e(e-1)=0)\). This means that \(e\) is a
        zero-divisor in the ring \(A\). Then the ideal \((e)\) is clearly not
        the whole ring \(A\), so it must be contained within the maximal ideal
        \(\maxid\) and \(e\in\maxid\), and similarly \(e-1\in\maxid\).

        Because \(\maxid\) is closed under addition \(1=e-(e-1)\in\maxid\), so
        \(\maxid=A\). This is clearly a contradiction so there is no such
        idempotent \(e\).
    \end{proof}
\end{question}

\begin{question}[2]
    Let \(A\) be the infinite product \(\prod_{i\in\N}\F_{2}\) with
    componentwise addition and multiplication.

    \begin{enumerate}[(a)]
        \item Show that every prime ideal \(\primeid\) is maximal.

              \begin{proof}
                  In \(A\), we have the property that for all \(x\in A\)
                  \(x(1-x)=0\). Therefore for any prime ideal \(\primeid\) we
                  have \(x(x-1)=0\) in \(A/\primeid\). Because this is an
                  integral domain this means \(x=0\) or \(x=1\) in
                  \(A/\primeid\). This means that \(A/\primeid\cong\F_{2}\)
                  which is a field, so \(\primeid\) is maximal.
              \end{proof}

        \item Let \(n\in\N\), and let \(\maxid_{n}\subset A\) be ideal
              consisting of those elements \((x_{i})_{i\in\N}\) with \(x_{n}=0\).
              Prove that \(\maxid_{n}\) is maximal.

              \begin{proof}
                  Take two \(x,y\in A\) with \(xy\in\maxid_{n}\), then
                  \((xy)_{n}=0\) so \(x_{n}=0\) or \(y_{n}=0\) because
                  \(\F_{2}\) is a field. Therefore \(x\in\maxid_{n}\) or
                  \(y\in\maxid_{n}\).
              \end{proof}

        \item Is every maximal ideal \(\maxid\) of \(A\) of the form
              \(\maxid_{n}\)?

              \begin{solution}
                  No it is not. The set \(I=\bigoplus_{i\in\N}\F_{2}\) is an
                  ideal of \(A\). This is because it is closed under addition,
                  contains \(0\), and is closed under multiplication because
                  multiplication cannot increase the amount of non-zero parts of
                  an element. The ideal \(I\) is not contained in \(\maxid_{n}\)
                  so it is contained in a different maximal
                  \(\maxid'\neq\maxid_{n}\) for all \(n\in\N\).
              \end{solution}
    \end{enumerate}
\end{question}

\begin{question}[3]
    Let \(A=\Q[[t]]\) be the ring of formal power series
    \(\sum_{n\in\N}a_{i}t^{i}\).
    \begin{enumerate}[a)]
        \item Show that \(A\) is a local ring.

              \begin{proof}
                  I will show that \((t)\) is the only maximal ideal by proving
                  \(1+(t)=A\). By proposition 1.6, this means \(A\) is local.
                  First of all the map
                  \(\pi:\Q[[t]]\to\Q:\sum_{i\in\N}a_{i}t^{i}\) is a surjective
                  map to the field \(\Q\) with kernel \((t)\), so \((t)\) is
                  maximal.

                  Now consider \(f=\sum_{i\in\N}a_{i}t^{i}\) with \(a_{0}=1\).
                  Then for any \(g=\sum_{i\in\N}b_{i}t^{i}\) we have
                  \((fg)_{k}=\sum_{i+j=k}a_{i}b_{j}\). If we define
                  \(b_{k}=-a_{0}^{-1}\sum_{i=1}^{n}a_{i}b_{n-i}=-\sum_{i=1}^{n}a_{i}b_{n-i}\)
                  we get \((fg)_{i}=\delta_{i0}\).
              \end{proof}

        \item Show that \(A\) is an integral domain.

              \begin{proof}
                  Let \(f,g\in A\) be non-zero elements with \(fg=0\). Then
                  there are some smallest \(i,j\) such that \(a_{i}\) and
                  \(b_{j}\) are both non-zero. Then
                  \((fg)_{i+j}=a_{i}b_{j}\neq0\) because \(\Q\) is a field.
                  Therefore \(fg\neq0\) which is a contradiction, so there are
                  no such \(f,g\) and \(A\) is an integral domain.
              \end{proof}

        \item Compute the nilradical and Jacobson radical of \(A\).

              \begin{solution}
                  The ring \(A\) has no zero-divisors, so \(\Nil(A)=(0)\).
                  Because \(A\) is local, it has the unique maximal ideal
                  \((t)\). This is therefore the Jacobson ideal.
              \end{solution}
    \end{enumerate}
\end{question}

\begin{question}[4]
    Find an integer \(n\) such that \(\Z/n\Z\) has a non-trivial unit,
    zero-divisor, nilpotent, and idempotent.

    \begin{solution}
        Consider \(n=12\) and let \(A=\Z/12\Z\). Then \(5^{2}=25=24+1=1\) in
        \(A\) so \(5\) is a unit. The element \(6\) is both nilpotent and a
        zero-divisor because \(6^{2}=36=12\cdot3=0\) in \(A\). The element \(4\) has
        \(4^{2}=16=12+4=4\) in \(A\).
    \end{solution}
\end{question}

\begin{question}[5]
    Determine all prime ideals of \(\Z[x]\). Compute the nilradical and Jacobson
    ideal of \(\Z[x]/(2x^{2})\).

    \begin{solution}
        For any irreducible polynomial \(f\), the ideal \((f)\) is clearly prime
        and for any prime number \(p\in\Z\) the ideal \((p)\) is also clearly
        prime. For any prime number \(p\) and irreducible polynomial \(f\), the
        ideal \((p,f)\) is also prime. There are no other prime ideals.

        To find the nilradical, we need to find all nilpotents. Because
        \(\Z[x]\) is a domain this means we need to find all elements
        \(f\in\Z[x]\) for which \(f^{n}\in (2x^{2})\). Obviously this includes
        the ideal \((2x^{2})\) itself. Another element in this ideal is \(2x\).
        Any other element must have the property that \(2x^{2}\mid f^{n}\) for
        some \(n\), however this means that \(2x\mid f^{n}\). Because \(2x\) is
        irreducible this means that \(2x\mid f\). Therefore the nilradical is
        given by \((2x)\).

        To compute the Jacobson radical. We can take the quotient
        \(\Z[x]/(2x^{2})/(x)/(2x^{2})\cong \Z[x]/(x)\cong\Z\) which has maximal
        ideals \((p)\) for \(p\in\Z\). The intersection of these ideals is the
        ideal \((x)\) so \(\JRad(\Z[x]/(2x^{2}))\subseteq(x)\). There is however
        a maximal ideal which doesn't contain \(x\) at all: consider the ideal
        \((2,x^{2}+x+1)\) which has its quotient given by
        \(\Z[x]/(2x^{2})/(2,x^{2}+x+1)/(2x^{2})\cong\F_{4}\). In particular the
        image of \(x\) under this quotient is the generator of \(\F_{4}^{*}\).

        This means that \((2x)\subseteq\JRad(\Z[x]/(2x^{2}))\subsetneq (x)\).
        This means that the Jacobson radical is \((2x)\).
    \end{solution}
\end{question}
\end{document}