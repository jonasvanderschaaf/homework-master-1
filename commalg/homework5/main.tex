\documentclass{article}

\usepackage[utf8]{inputenc}
\usepackage{enumerate}
\usepackage{scalerel}
\usepackage{amsthm, amssymb, mathtools, amsmath, bbm, mathrsfs, stmaryrd}
\usepackage[margin=1in]{geometry}
\usepackage[parfill]{parskip}
\usepackage[hidelinks]{hyperref}
\usepackage{quiver}
\usepackage{float}
\usepackage{nicefrac}

\newcommand{\N}{\mathbb{N}}
\newcommand{\Z}{\mathbb{Z}}
\newcommand{\NZ}{\mathbb{N}_{0}}
\newcommand{\Q}{\mathbb{Q}}
\newcommand{\R}{\mathbb{R}}
\newcommand{\C}{\mathbb{C}}

\newcommand{\F}{\mathbb{F}}
\newcommand{\incl}{\imath}

\newcommand{\tuple}[2]{\left\langle#1\colon #2\right\rangle}

\DeclareMathOperator{\order}{orde}
\DeclareMathOperator{\Id}{Id}
\DeclareMathOperator{\im}{im}
\DeclareMathOperator{\coker}{coker}
\DeclareMathOperator{\ggd}{ggd}
\DeclareMathOperator{\kgv}{kgv}
\DeclareMathOperator{\degree}{gr}

\DeclareMathOperator{\gl}{GL}

\DeclareMathOperator*{\bigplus}{\scalerel*{+}{\sum}}

\DeclareMathOperator{\Aut}{Aut}
\DeclareMathOperator{\End}{End}
\DeclareMathOperator{\Hom}{Hom}
\newcommand{\isom}{\overset{\sim}{\longrightarrow}}

\newcommand{\Grp}{{\bf Grp}}
\newcommand{\Ab}{{\bf Ab}}
\newcommand{\cring}{{\bf CRing}}
\DeclareMathOperator{\modules}{{\bf Mod}}
\newcommand{\catset}{{\bf Set}}
\newcommand{\cat}{\mathcal{C}}
\newcommand{\chains}{{\bf Ch}}
\newcommand{\homot}{{\bf Ho}}
\DeclareMathOperator{\objects}{Ob}
\newcommand{\gen}[1]{\left<#1\right>}
\DeclareMathOperator{\cone}{Cone}
\newcommand{\set}[1]{\left\{#1\right\}}
\newcommand{\setwith}[2]{\left\{#1:#2\right\}}
\DeclareMathOperator{\Ext}{Ext}
\DeclareMathOperator{\nil}{Nil}
\DeclareMathOperator{\idem}{Idem}
\DeclareMathOperator{\JRad}{JRad}
\DeclareMathOperator{\Nil}{Nil}
\DeclareMathOperator{\Idem}{Idem}
\DeclareMathOperator{\Rad}{Rad}
\DeclareMathOperator{\Ann}{Ann}
\DeclareMathOperator{\Mat}{Mat}
\DeclareMathOperator{\ch}{ch}
\DeclareMathOperator{\Ass}{Ass}

\newcommand{\maxid}{\mathfrak{m}}
\newcommand{\primeid}{\mathfrak{p}}
\newcommand{\primeeid}{\mathfrak{q}}
\newcommand{\ideal}{\triangleleft}

\newenvironment{solution}{\begin{proof}[Solution]\renewcommand\qedsymbol{}}{\end{proof}}

\newtheorem{theorem}{Theorem}[section]
\newtheorem{lemma}[theorem]{Lemma}
\newtheorem{proposition}[theorem]{Proposition}

\newtheorem{question}{Question}

\theoremstyle{definition}
\newtheorem{definition}[theorem]{Definition}
\newtheorem{remark}[theorem]{Remark}
\newtheorem{example}[theorem]{Example}
\newtheorem{corollary}[theorem]{Corollary}

\title{Homework Commutative Algebra}
\author{Jonas van der Schaaf}
\date{November 7, 2022}

\begin{document}
\maketitle

\begin{question}
    Let \(A=\Q[t]\). Let \(B\) be a ring such that \(A\subseteq B\subseteq
    Q(A)\).

    \begin{enumerate}[(a)]
        \item Prove there exists a multiplicatively closed set \(S\) such that
              \(B=S^{-1}A\).

              \begin{proof}
                  Define
                  \[
                      S=\Q[t]\cap B.
                  \]
                  This is multiplicatively closed because both \(\Q[t]\) and
                  \(B\) are multiplicatively closed subsets of \(Q(A)\) and
                  therefore their intersection is as well.

                  For \(\nicefrac{a}{b}\in S^{-1}A\) we must have that \(b\in
                  B^{*}\) and \(a\in A\) so \(\nicefrac{a}{b}\in B\). This means
                  \(S^{-1}A\subseteq B\).

                  Taking any \(\nicefrac{f}{g}\in B\) such that \(\gcd(a,b)=1\)
                  we have that the ideal \((f,g)\) is a principal ideal a \(h\)
                  such that \(h\mid f\) and \(h\mid g\). This means that \(h\)
                  must be a unit so there are \(a,b\) such that \(af+bg=1\).
                  Then
                  \begin{align*}
                      \frac{1}{g} & =\frac{af+bg}{g}      \\
                                  & =a\frac{f}{g}+b\in B.
                  \end{align*}
                  This means \(g\) is a unit in \(B\) so \(g\in S^{*}\) so
                  \(\nicefrac{f}{g}\in S^{-1}A\). This means \(S^{-1}A=B\).
              \end{proof}

        \item Determine all valuation rings \(B\) such that \(A\subseteq
              B\subseteq\Q(t)\).

              \begin{solution}
                  For this exercise we need to determine all multiplicative sets
                  \(S\) such that \(S^{-1}A\) is a valuation ring. We can assume
                  that \(\Q\subseteq S\) because these are already invertible
                  elements and this will change nothing about the ring.

                  The ring \(B\) is a valuation rings iff for all
                  \(\nicefrac{a}{b}\in Q(A)\) either \(\nicefrac{a}{b}\in B\) or
                  \(\nicefrac{b}{a}\in B\). By assuming that \(a,b\in\Q[t]\) are
                  coprime then this means that \(a\in S\) or \(b\in S\). In
                  particular this holds for irreducible polynomials. This means
                  that at most one irreducible polynomial up to multiplication
                  with a constant cannot be in the set \(S\). This means that
                  \(\Q[t]\setminus S\subseteq(f)\) for some irreducible
                  polynomial \(f\).

                  Conversely if \(\Q[t]\setminus S\subseteq(f)\) for some
                  irreducible \(f\) then \(B=S^{-1}\Q[t]\) is a valuation ring.
                  This is because for all \(\nicefrac{a}{b}\in\Q(t)\) with
                  \(a,b\) coprime we know \(f\) divides at most one of \(a,b\)
                  so either \(a\in S\) or \(b\in S\). This means that
                  \(\nicefrac{a}{b}\in B\) or \(\nicefrac{b}{a}\in B\)
                  so it is a valuation ring.

                  This means \(B=S^{-1}A\) is a valuation ring iff
                  \(\Q[t]\setminus S\subseteq (f)\) for some irreducible \(f\).
              \end{solution}

        \item Determine all Artin rings \(B\) such that \(A\subseteq
              B\subseteq\Q(t)\).

              \begin{solution}
                  Any Artin ring which is a domain must be a field. Any ring
                  \(A\subseteq B\subseteq\Q(t)\) is a subring of a domain, and
                  so must be a domain as well. If it is Artin it must be a
                  field. The smallest field which contains \(A\) is \(\Q(t)\),
                  so \(\Q(t)\subseteq B\subseteq\Q(t)\). This means \(B=\Q(t)\).
              \end{solution}
    \end{enumerate}
\end{question}

\begin{question}
    Let \(A=\Q[x,y]/(x^{2},xy^{2},y^{4})\). Determine the length of \(A\) as an
    \(A\)-module.

    \begin{solution}
        The ring \(A\) has length \(6\) as \(A\)-module. The chain of inclusions
        is given by
        \[
            (0)\subsetneq (\overline{xy})\subsetneq (\overline{x})\subsetneq (\overline{x},\overline{y}^{3})\subsetneq (\overline{x},\overline{y}^{2})\subsetneq (\overline{x},\overline{y})\subsetneq A.
        \]
        To prove this is a decompostion chain we will prove all quotients of
        this chain are simple.

        The module \((\overline{xy})\) is simple. All elements of the ideal are
        of the form \(a\overline{xy}\) for some \(a\in \Q\). This is easily seen
        from the fact that \(\overline{x}^{2}=0=\overline{xy^{2}}\). This means
        that any multiplication of \(\overline{xy}\) with a non-constant
        polynomial will give \(0\). Therefore it has dimension \(1\) as
        \(\Q\)-module so it must be simple.

        The quotient \((\overline{x})/(\overline{xy})\) only has elements of the
        form \(a\overline{x}\) for \(a\in\Q\). Any multiplication with
        \(\overline{x}\) is zero because \(\overline{x}^{2}=0\) and
        \(\overline{xy}\in(\overline{xy})\) so it is zero in the quotient.
        Therefore it has dimension \(1\) as \(\Q\)-module so
        \((\overline{x})/(\overline{xy})\) must be simple.

        The quotient \((\overline{x},\overline{y}^{3})/(\overline{x})\) is
        isomorphic to \((\overline{y^{3}})\) as \(A\)-module. Every product of
        \(\overline{y^{3}}\) with a non-constant polynomial is zero so all
        elements of this quotient are of the form \(a\overline{y^{3}}\) for
        \(a\in\Q\). By the same reasoning as before this means the quotient is
        simple.

        The quotient
        \((\overline{x},\overline{y^{2}})/(\overline{x},\overline{y^{3}})\) only
        has elements of the form \(a\overline{y^{2}}\) for \(a\in\Q\) This is
        because \(\overline{y^{3}}\in(\overline{x},\overline{y^{3}})\) so any
        product with \(\overline{y}\) is zero and \(\overline{xy^{2}}=0\). This
        means it is once again a simple module.

        The exact same reasoning as before holds for the next quotient. The
        quotient \((\overline{x},\overline{y})/(\overline{x},\overline{y^{2}})\)
        only has elements of the shape \(a\overline{y}\) for \(a\in\Q\). This
        means it is simple.

        The quotient \(A/(\overline{x},\overline{y})\) is isomorphic to \(\Q\)
        as \(A\)-module which is once again a \(1\)-dimensional \(\Q\) module
        so it is simple.

        Therefore the above chain consists of strict inclusions where the
        quotients are simple: it is a decomposition chain. Therefore the length
        of the module is \(6\).
    \end{solution}
\end{question}

\begin{question}
    Let \(A=\Q[x,y]/(y^{2}-x^{11})\).
    \begin{enumerate}[(a)]
        \item Determine the integral closure of \(A\).

              \begin{solution}
                  The ring \(A\) is isomorphic to \(\Q[x,x^{5}\sqrt{x}]\). To
                  see this consider the ring morphism from the polynomial ring
                  in two variables \(\varphi:\Q[x,y]\to\Q[x,x^{5}\sqrt{x}]\)
                  determined by \(\varphi(x)=x\) and
                  \(\varphi(y)=x^{5}\sqrt{x}\). This map is surjective because
                  \(\Q\), \(x\), and \(x^{5}\sqrt{x}\) are in the image of
                  \(\varphi\) and these generate the ring. On top of that the
                  kernel of the morphism is exactly \((y^{2}-x^{11})\).

                  The integral closure of \(\Q[x,x^{5}\sqrt{x}]\) is given by
                  \(\Q[\sqrt{x}]\). It's clear that
                  \(\sqrt{x}=\nicefrac{x^{5}\sqrt{x}}{x^{5}}\in\Q(x,x^{5}\sqrt{x})\)
                  and the monic polynomial \(t^{11}-x^{5}\sqrt{x}\) has
                  \(\sqrt{x}\) as root so \(\sqrt{x}\) is integral over
                  \(\Q[x,x^{5}\sqrt{x}]\).

                  To see that \(\Q[\sqrt{x}]\) is integrally closed, it suffices
                  to see it is isomorphic to a polynomial ring. This isomorphism
                  is given by \(\psi:\Q[t]\to\Q[\sqrt{x}]\) determined by
                  \(\psi(t)=\sqrt{x}\). This is a surjective map because its
                  generators are in the image. It also has trivial kernel
                  because \(\sqrt{x}\) is algebraically independent from \(\Q\).

                  This means that \(\Q[\sqrt[11]{y}]\) must be contained in the
                  integral closure of \(A\), and it is integrally closed so it
                  is the integral closure of \(A\).
              \end{solution}

        \item Determine a valuation ring of \(Q(A)\) that dominates
              \(A_{(x,y)}\).

              \begin{proof}
                  Because valuation rings are integrally closed, any valuation
                  ring dominating \(A\) must contain \(\Q[\sqrt{x}]\). The
                  localization \(\Q[\sqrt{x}]_{(\sqrt{x})}\) is a valuation
                  ring. The ideal \((\sqrt{x})\) is maximal so this is indeed a
                  localization. Now take any \(\nicefrac{f}{g}\in\Q(\sqrt{x})\)
                  such that \(f,g\) are coprime. Then \(\sqrt{x}\) divides at
                  most one of \(f,g\). This means that \(f\) or \(g\) is a unit
                  in the ring so \(\nicefrac{f}{g}\in\Q[\sqrt{x}]_{(\sqrt{x})}\)
                  or \(\nicefrac{g}{f}\in \Q[\sqrt{x}]_{(\sqrt{x})}\) and so it
                  is a valuation ring.

                  The ring \(\Q[x,x^{5}\sqrt{x}]_{(x,x^{5}\sqrt{x})}\) is a
                  subring of this ring as well. To see this take an
                  \(\nicefrac{a}{b}\) in the ring. Then
                  \(b\in\Q[x,x^{5}\sqrt{x}]\setminus(x,x^{5}\sqrt{x})\) so it
                  must be a polynomial with non-zero constant term. This means
                  it is also a unit in \(\Q[\sqrt{x}]_{(\sqrt{x})}\) so
                  \(\nicefrac{a}{b}\) is also in \(\Q[\sqrt{x}]_{(\sqrt{x})}\).
                  This means
                  \(\Q[x,x^{5}\sqrt{x}]_{(x,x^{5}\sqrt{x})}\subseteq\Q[\sqrt{x}]_{(\sqrt{x})}\).

                  The maximal ideal of this ring is \((\sqrt{x})\). Taking the
                  intersection of this ideal with the ring of interest
                  \(\Q[x,x^{5}\sqrt{x}]_{(x,x^{5}\sqrt{x})}\) clearly contains
                  \((x,x^{5}\sqrt{x})\) because \(x=\sqrt{x}^{2}\) and
                  \(x^{5}\sqrt{x}=\sqrt{x}^{11}\). There is no element of \(\Q\)
                  in \((\sqrt{x})\subseteq\Q[\sqrt{x}]_{(\sqrt{x})}\) and so
                  \(\Q\) has no common elements with
                  \((\sqrt{x})\cap\Q[x,x^{5}\sqrt{x}]_{(x,x^{5}\sqrt{x})}\).
                  This means that the intersection is not the entire ring and
                  contains the maximal ideal \((x,x^{5}\sqrt{x})\) so
                  \[
                      (\sqrt{x})\cap\Q[x,x^{5}\sqrt{x}]_{(x,x^{5}\sqrt{x})}=(x,x^{5}\sqrt{x}).
                  \]
                  This means that \(\Q[\sqrt{x}]_{(\sqrt{x})}\) dominates
                  \(\Q[x,x^{5}\sqrt{x}]_{(x,x^{5}\sqrt{x})}\). Applying the
                  previously found isomorphism gives that \(\Q[x,y]_{(x,y)}\) is
                  dominated by \(\Q[\sqrt[5]{y}]_{(\sqrt[5]{y})}\).
              \end{proof}
    \end{enumerate}
\end{question}
\end{document}