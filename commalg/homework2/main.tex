\documentclass{article}

\usepackage[utf8]{inputenc}
\usepackage{enumerate}
\usepackage{scalerel}
\usepackage{amsthm, amssymb, mathtools, amsmath, bbm, mathrsfs, stmaryrd}
\usepackage[margin=1in]{geometry}
\usepackage[parfill]{parskip}
\usepackage[hidelinks]{hyperref}
\usepackage{quiver}
\usepackage{float}

\newcommand{\N}{\mathbb{N}}
\newcommand{\Z}{\mathbb{Z}}
\newcommand{\NZ}{\mathbb{N}_{0}}
\newcommand{\Q}{\mathbb{Q}}
\newcommand{\R}{\mathbb{R}}
\newcommand{\C}{\mathbb{C}}

\newcommand{\F}{\mathbb{F}}
\newcommand{\incl}{\imath}

\newcommand{\tuple}[2]{\left\langle#1\colon #2\right\rangle}

\DeclareMathOperator{\order}{orde}
\DeclareMathOperator{\Id}{Id}
\DeclareMathOperator{\im}{im}
\DeclareMathOperator{\coker}{coker}
\DeclareMathOperator{\ggd}{ggd}
\DeclareMathOperator{\kgv}{kgv}
\DeclareMathOperator{\degree}{gr}

\DeclareMathOperator{\gl}{GL}

\DeclareMathOperator*{\bigplus}{\scalerel*{+}{\sum}}

\DeclareMathOperator{\Aut}{Aut}
\DeclareMathOperator{\End}{End}
\DeclareMathOperator{\Hom}{Hom}
\newcommand{\isom}{\overset{\sim}{\longrightarrow}}

\newcommand{\Grp}{{\bf Grp}}
\newcommand{\Ab}{{\bf Ab}}
\newcommand{\cring}{{\bf CRing}}
\DeclareMathOperator{\modules}{{\bf Mod}}
\newcommand{\catset}{{\bf Set}}
\newcommand{\cat}{\mathcal{C}}
\newcommand{\chains}{{\bf Ch}}
\newcommand{\homot}{{\bf Ho}}
\DeclareMathOperator{\objects}{Ob}
\newcommand{\gen}[1]{\left<#1\right>}
\DeclareMathOperator{\cone}{Cone}
\newcommand{\set}[1]{\left\{#1\right\}}
\newcommand{\setwith}[2]{\left\{#1:#2\right\}}
\DeclareMathOperator{\Ext}{Ext}
\DeclareMathOperator{\nil}{Nil}
\DeclareMathOperator{\idem}{Idem}
\DeclareMathOperator{\JRad}{JRad}
\DeclareMathOperator{\Nil}{Nil}
\DeclareMathOperator{\Idem}{Idem}
\DeclareMathOperator{\Rad}{Rad}
\DeclareMathOperator{\Ann}{Ann}
\DeclareMathOperator{\Mat}{Mat}
\DeclareMathOperator{\ch}{ch}

\newcommand{\maxid}{\mathfrak{m}}
\newcommand{\primeid}{\mathfrak{p}}
\newcommand{\ideal}{\triangleleft}

\newenvironment{question}[1][]{\begin{paragraph}{Question #1}}{\end{paragraph}}
\newenvironment{solution}{\begin{proof}[Solution]\renewcommand\qedsymbol{}}{\end{proof}}

\newtheorem{theorem}{Theorem}[section]
\newtheorem{lemma}[theorem]{Lemma}
\newtheorem{proposition}[theorem]{Proposition}


\theoremstyle{definition}
\newtheorem{definition}[theorem]{Definition}
\newtheorem{remark}[theorem]{Remark}
\newtheorem{example}[theorem]{Example}
\newtheorem{corollary}[theorem]{Corollary}

\title{Homework Commutative Algebra}
\author{Jonas van der Schaaf \and 12798797 \and Universiteit van Amsterdam \and jonas.vanderschaaf@student.uva.nl}
\date{September 26, 2022}

\begin{document}
\maketitle

\begin{question}[1]
    Let \(k\) be a field. Compute the nilradicals of the following rings:
    \begin{enumerate}[(a)]
        \item \(A=k[x,y]/(x^{3}, xy^{2})\)

              \begin{solution}
                  I will first prove that \(\Nil(k[x,y]/I)\) is given by the
                  polynomials in \(\sqrt{I}\). Then \(f\in\sqrt{I}\) iff
                  \(f^{n}\in I\) for some \(n\in\N_{>0}\) which is equivalent to
                  \(0=\overline{f^{n}}=\overline{f}^{n}\) which in turn is true
                  exactly iff \(f\in\Nil(k[x,y]/I)\). Therefore
                  \(\overline{f}\in\Nil(k[x,y]/I)\) iff \(f\in\sqrt{I}\).


                  It is clear that \((x)=(x,xy)\subseteq\sqrt{(x^{3},xy^{2})}\).
                  Now take any \(f\in\sqrt{x^{3},xy^{2}}\). Then
                  \(f^{n}=ax^{3}+bxy^{2}\). Because \(x\) is irreducible this
                  means \(x\mid f\) so \(f\in(x)\). This means that
                  \(\sqrt{I}=(x)\). Therefore also
                  \(\Nil(k[x,y]/(x^{3},xy^{2}))=(\overline{x})\).
              \end{solution}

        \item \(k[x,y,z]/(x^{2}+y^{2}+z^{2})\)

              \begin{solution}
                  For \(k\) a field of characteristic \(2\) we have
                  \((x+y+z)^{2}=x^{2}+y^{2}+z^{2}\). Therefore \(x+y+z\) is
                  nilpotent and in the nilradical. It is also a linear
                  polynomial and therefore irreducible. This means that for all
                  \(f\in k[x,y,z]\) with
                  \(f^{n}=a(x^{2}+y^{2}+z^{2})=a(x+y+z)^{2}\) we must have
                  \(x+y+z\mid f\). This means that
                  \(\Nil(k[x,y,z]/(x^{2}+y^{2}+z^{2}))=(x+y+z)\) if \(\ch k=2\).

                  If \(k\) has characteristic different from \(2\), then in
                  particular \(x^{2}+1\) has two distinct roots in
                  \(\overline{k}\): \(\pm i\). We will show
                  \(x^{2}+y^{2}+z^{2}\) is irreducible by applying the
                  Eisenstein criterion over the algebraic closure of \(k\). In
                  \(\overline{k}\) the polynomial \(x^{2}+y^{2}+z^{2}\) can be
                  written as \(z^{2}+(x+iy)(x-iy)\) where \(x+iy\neq x-iy\).
                  Each of these linear factors is irreducible so
                  \(x^{2}+y^{2}+z^{2}\) must be an irreducible element of
                  \(\overline{k}[x,y,z]\). In particular if
                  \(fg\in(x^{2}+y^{2}+z^{2})\) for \(f,g\in
                  k[x,y,z]\subseteq\overline{k}[x,y,z]\) then \(f\) or \(g\) is
                  a unit in \(\overline{k}[x,y,z]\) so an element of
                  \(\overline{k}\). We know \(f,g\in k[x,y,z]\) so it must also
                  be an element of \(k\). Therefore the polynomial is
                  irreducible.

                  Because \(x^{2}+y^{2}+z^{2}\) is irreducible
                  \(k[x,y,z]/(x^{2}+y^{2}+z^{2})\) is a domain so it has no
                  nilpotents and \(\Nil(k[x,y,z]/(x^{2}+y^{2}+z^{2}))=(0)\).
              \end{solution}
    \end{enumerate}
\end{question}

\begin{question}[2]
    Consider the ideal \(I=(x_{1}x_{3},x_{2}x_{3})\) of
    \(\Q[x_{1},x_{2},x_{2}]\). Show that \((I:(x_{1},x_{2},x_{3})^{m})=I\) for
    all \(m\geq 1\).

    \begin{proof}
        By exercise 1.12 we get
        \begin{align*}
            ((x_{1}x_{3},x_{2}x_{3}):(x_{1},x_{2},x_{3})) & =\bigcap_{1\leq i\leq 3}((x_{1}x_{3},x_{2}x_{3}):x_{i}) \\
                                                          & =(x_{3})\cap(x_{3})\cap(x_{1},x_{2})                    \\
                                                          & =(x_{3})\cdot(x_{1},x_{2})                              \\
                                                          & =(x_{1}x_{3},x_{2},x_{3}).
        \end{align*}
        This proves the case for \(n=1\).

        Suppose the statement is true for \(n\), then
        \begin{align*}
            ((x_{1}x_{3},x_{2}x_{3}):(x_{1},x_{2},x_{3})^{n+1}) & =((x_{1}x_{3},x_{2}x_{3}):(x_{1},x_{2},x_{3})^{n}(x_{1},x_{2},x_{3}))    \\
                                                                & =(((x_{1}x_{3},x_{2}x_{3}):(x_{1},x_{2},x_{3})^{n}):(x_{1},x_{2},x_{3})) \\
                                                                & =((x_{1}x_{3},x_{2},x_{3}):(x_{1},x_{2},x_{3}))                          \\
                                                                & =I.
        \end{align*}
    \end{proof}
\end{question}

\begin{question}[3]
    Let \(A\) be a ring. An \(A\)-module \(M\) is said to be simple if the only
    submodules are \(0\) and \(M\).

    \begin{enumerate}[(a)]
        \item Show that every simple \(A\)-module is of the form \(Am\) for some
              \(m\in M\).

              \begin{proof}
                  Take any \(m\in M\) non-zero then clearly \(Am\neq 0\).
                  Therefore \(Am=M\).
              \end{proof}

        \item Let \(M\) be a module and let \(m\in M\). Show that the set
              \(\Ann(M)\subseteq A\) is an ideal of \(A\) and that \(M\cong
              A/\Ann(M)\) as \(A\)-modules.

              \begin{proof}
                  To show \(\Ann(M)\) is an ideal we need to show that it is
                  closed under addition and multiplication with \(A\).

                  Take any \(m\in M\), \(x,y\in\Ann(m)\) and \(a,b\in A\). Then
                  \begin{align*}
                      (ax+by)m & =axm+bym \\
                               & =a0+b0   \\
                               & =0.
                  \end{align*}
                  Therefore \(\Ann(M)\) is an ideal.

                  The map \(\varphi:A\to Am:a\mapsto am\) is clearly surjective,
                  so \(A/\ker(\varphi)\cong Am\) by the first isomorphism
                  theorem. By definition \(\ker(\varphi)=\setwith{a\in
                      A}{am=0}=\Ann(m)\). Therefore \(A/\Ann(m)\cong Am\).
              \end{proof}

        \item Let \(k\) be a field and \(M\) a simple \(k[x]\)-module. Prove
              that \(M\cong k[x]/(f)\) for an irreducible polynomial \(f\).

              \begin{proof}
                  By part a and be we know that \(M=k[x]m\cong k[x]/\Ann(m)\)
                  for some \(m\in M\). Because \(k[x]\) is a PID we know that
                  \(\Ann(m)=(g)\) for some \(g\in k[x]\). Suppose \(g\) is not
                  irreducible take \(f\) to be an irreducible factor of \(g\).
                  Then \((f)/(g)\) is a non-zero strict submodule of
                  \(k[x]/(g)\). But \(k[x]/(g)\cong M\) which is simple so this
                  is a contradiction.
              \end{proof}

        \item Which of the following \(\Z\) modules are simple?
              \(\Z,\Z/6\Z,\Z/7\Z\).

              \begin{solution}
                  The modules \(\Z\) and \(\Z/6\Z\) are not simple because
                  \(\gen{2}\) is a non-zero strict submodule in both.

                  The module \(\Z/7\Z\) is a simple module because it is
                  generated by any single non-zero element. Therefore any
                  non-zero submodule must be the entire module.
              \end{solution}
    \end{enumerate}
\end{question}

\begin{question}[4]
    Let \(A\) be a ring, \(M\) a finitely generated \(A\) module, and
    \(\varphi:M\to M\) a surjective \(A\)-linear map. Show that \(\varphi\) is
    an isomorphism.

    \begin{proof}
        The module \(M\) is also an \(A[x]\) module with \(xm=\varphi(m)\). Then
        \(xM=M\) by surjectivity of \(\varphi\). Therefore \((x)M=M\). This
        means there is some \(y\in 1+(x)\) such that \(yM=0\).  This \(y\) in
        particular has \(y=1+sx\) for some \(s\in A[x]\). This means that the
        map \(f:A[x]\to\End(M)\) given by the module structure maps \(y\) to
        zero so \(0=f(1+sx)\). Therefore \(\Id + f(s)\varphi=0\), or
        equivalently \(f(-s)\) is the left inverse of \(\varphi\). This means
        that \(\varphi\) is injective, so it is an automorphism.
    \end{proof}
\end{question}

\begin{question}[5]
    Let \(k\) be a field and \(A=k[[x]]\) a power series ring over \(k\) in one
    variable. Let \(M=A\oplus A\) as module. Let
    \(a=(\sum_{i}a_{i}x^{i},\sum_{i}a_{i}'x^{i})\) and
    \((\sum_{i}b_{i}x^{i},\sum_{i}b_{i}'x^{i})\). Show \(a\) and \(b\) generate
    \(M\) if and only if the matrix
    \[
        \begin{pmatrix}
            a_{0} & a_{0}' \\
            b_{0} & b_{0}' \\
        \end{pmatrix}\in\Mat_{n\times n}(k)
    \]
    has non-zero determinant.

    \begin{proof}
        The quotient \(M/\maxid M\) for \(\maxid=(x)\ideal A\) is isomorphic to
        \(k^{2}\). By proposition 2.8 if it is finitely generated by
        \(\overline{a}=(a_{0},a'_{0}),\overline{b}=(b_{0},b'_{0})\) then \(a,b\)
        generate \(M\). By basic linear algebra this is equivalent to
        \[
            \det\begin{pmatrix}
                a_{0} & a'_{0} \\
                b_{0} & b'_{0}
            \end{pmatrix}\neq0.
        \]
        Therefore if the determinant is non-zero the elements \(a,b\) generate
        \(M\).

        Conversely if \(a,b\) generate \(M\) then they also generate
        \(M/\maxid M\) which means once again
        \[
            \det\begin{pmatrix}
                a_{0} & a'_{0} \\
                b_{0} & b'_{0}
            \end{pmatrix}\neq0.
        \]
        Therefore these statements are equivalent.
    \end{proof}
\end{question}
\end{document}