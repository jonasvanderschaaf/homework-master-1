\documentclass{article}

\usepackage[utf8]{inputenc}
\usepackage{enumerate}
\usepackage{scalerel}
\usepackage{amsthm, amssymb, mathtools, amsmath, bbm, mathrsfs, stmaryrd}
\usepackage[margin=1in]{geometry}
\usepackage[parfill]{parskip}
\usepackage[hidelinks]{hyperref}
\usepackage{quiver}
\usepackage{float}
\usepackage{nicefrac}

\newcommand{\N}{\mathbb{N}}
\newcommand{\Z}{\mathbb{Z}}
\newcommand{\NZ}{\mathbb{N}_{0}}
\newcommand{\Q}{\mathbb{Q}}
\newcommand{\R}{\mathbb{R}}
\newcommand{\C}{\mathbb{C}}

\newcommand{\F}{\mathbb{F}}
\newcommand{\incl}{\imath}

\newcommand{\tuple}[2]{\left\langle#1\colon #2\right\rangle}

\DeclareMathOperator{\order}{orde}
\DeclareMathOperator{\Id}{Id}
\DeclareMathOperator{\im}{im}
\DeclareMathOperator{\coker}{coker}
\DeclareMathOperator{\ggd}{ggd}
\DeclareMathOperator{\kgv}{kgv}
\DeclareMathOperator{\degree}{gr}

\DeclareMathOperator{\gl}{GL}

\DeclareMathOperator*{\bigplus}{\scalerel*{+}{\sum}}

\DeclareMathOperator{\Aut}{Aut}
\DeclareMathOperator{\End}{End}
\DeclareMathOperator{\Hom}{Hom}
\newcommand{\isom}{\overset{\sim}{\longrightarrow}}

\newcommand{\Grp}{{\bf Grp}}
\newcommand{\Ab}{{\bf Ab}}
\newcommand{\cring}{{\bf CRing}}
\DeclareMathOperator{\modules}{{\bf Mod}}
\newcommand{\catset}{{\bf Set}}
\newcommand{\cat}{\mathcal{C}}
\newcommand{\chains}{{\bf Ch}}
\newcommand{\homot}{{\bf Ho}}
\DeclareMathOperator{\objects}{Ob}
\newcommand{\gen}[1]{\left<#1\right>}
\DeclareMathOperator{\cone}{Cone}
\newcommand{\set}[1]{\left\{#1\right\}}
\newcommand{\setwith}[2]{\left\{#1:#2\right\}}
\DeclareMathOperator{\Ext}{Ext}
\DeclareMathOperator{\nil}{Nil}
\DeclareMathOperator{\idem}{Idem}
\DeclareMathOperator{\JRad}{JRad}
\DeclareMathOperator{\Nil}{Nil}
\DeclareMathOperator{\Idem}{Idem}
\DeclareMathOperator{\Rad}{Rad}
\DeclareMathOperator{\Ann}{Ann}
\DeclareMathOperator{\Mat}{Mat}
\DeclareMathOperator{\ch}{ch}
\DeclareMathOperator{\Ass}{Ass}

\newcommand{\maxid}{\mathfrak{m}}
\newcommand{\maxidd}{\mathfrak{n}}
\newcommand{\primeid}{\mathfrak{p}}
\newcommand{\primeeid}{\mathfrak{q}}
\newcommand{\ideal}{\triangleleft}
\newcommand{\powser}[2]{#1\llbracket#2\rrbracket}

\newenvironment{solution}{\begin{proof}[Solution]\renewcommand\qedsymbol{}}{\end{proof}}

\newtheorem{theorem}{Theorem}[section]
\newtheorem{lemma}[theorem]{Lemma}
\newtheorem{proposition}[theorem]{Proposition}

\newtheorem{question}{Question}

\theoremstyle{definition}
\newtheorem{definition}[theorem]{Definition}
\newtheorem{remark}[theorem]{Remark}
\newtheorem{example}[theorem]{Example}
\newtheorem{corollary}[theorem]{Corollary}

\title{Homework Commutative Algebra}
\author{Jonas van der Schaaf}
\date{November 7, 2022}

\begin{document}
\maketitle

\begin{question}
    Let \(R\) be a ring, \(\maxid\ideal R\) a maximal ideal, and \(n\) a
    positive integer. Show that \(R/\maxid^{n}\) is a local ring with maximal
    ideal \(\maxid\).

    \begin{proof}
        It's clear that \(\maxid R/\maxid^{n}\) is a maximal ideal in
        \(R/\maxid^{n}\) because by some isomorphism theorem we have
        \(R/\maxid^{n}/(\maxid/\maxid^{n})\cong R/\maxid\) which is a field.

        We will show this by induction. For \(n=1\) we have that
        \(R/\maxid\) is a field which is a local ring.

        Suppose the statement is true for \(n\) and let \(\maxid'\ideal
        R/\maxid^{n+1}\) be any maximal ideal. We have the projection morphism
        \(\pi:R/\maxid^{n}\twoheadrightarrow R/\maxid^{n+1}\) so we can contract
        \(\maxid'\) to an ideal \(\maxid'^{c}\) of \(R/\maxid^{n}\). Because
        \(\pi\) is a surjective map and \(\maxid'\) is in particular prime,
        \(\maxid'^{c}\) must also be prime and therefore contained in \(\maxid
        R/\maxid^{n}R\). Because \(\pi\) is surjective we also have
        \(\pi[\maxid'^{c}]=\maxid'^{ce}=\maxid'\) and because
        \(\maxid'^{c}\subseteq\maxid R/\maxid^{n}\) we must have
        \(\maxid'^{ce}\subseteq\pi[\maxid R/\maxid^{n}]\subseteq\maxid
        R/\maxid^{n+1}\). This means that \(\maxid'\) is contained within the
        maximal ideal \(\maxid\) and so the two are equal.
    \end{proof}
\end{question}

\begin{question}
    Let \(R\) be a Noetherian ring and \(\primeid\ideal R\) a prime ideal. Show
    that the following are equivalent:
    \begin{enumerate}
        \item \(\primeid\) is maximal,
        \item \(R/\primeid^{n}\) is an Artin local ring for all \(n>0\),
        \item \(R/\primeid^{n}\) is an Artin local ring for some \(n>0\).
    \end{enumerate}

    \begin{proof}
        \(1\Rightarrow2\) Let \(\primeid\) be a maximal ideal and take any
        \(n>0\). By question 1 \(R/\primeid^{n}\) is a local ring. It is also
        Noetherian because it is a quotient of a Noetherian ring. The power
        \((\primeid/\primeid^{n})^{n}=0\) so by theorem 8.6 we have that this
        ring is Artin local.

        \(2\Rightarrow3\) If the statement is true for all \(n>0\) then in
        particular it is true for some \(n>0\).

        \(3\Rightarrow1\) Suppose \(R/\primeid^{n}\) is Artin local for some
        \(n>0\). Then \(\primeid/\primeid^{n}\) is a prime ideal of \(R\)
        because the quotient is \(R/\primeid\) by some isomorphism theorem and
        this is a domain because \(\primeid\) is prime. This means that the
        ideal \(\primeid/\primeid^{n}\) must be maximal and its quotient a
        field, the quotient is isomorphic to \(R/\primeid\) so this must also be
        a field. This means \(\primeid\) is maximal.
    \end{proof}
\end{question}

\begin{question}
    There are natural ring homomorphisms \(\Z[x]\to\powser{\Z[x]}{y}\) and
    \(\powser{\Z}{y}\to\powser{\Z[x]}{y}\), inducing a ring homomorphism
    \[
        f:\Z[x]\otimes_{\Z}\powser{\Z}{y}\to\powser{\Z[x]}{y}.
    \]
    \begin{enumerate}[(a)]
        \item Is \(f\) injective?

              \begin{solution}
                  Yes this morphism is injective. This is too tiring to prove.
              \end{solution}

        \item Is \(f\) surjective?

              \begin{solution}
                  No \(f\) is not surjective. For example
                  \[
                      \sum_{i=0}^{\infty}x^{i}y^{i}\in\powser{\Z[x]}{y}
                  \]
                  is a formal power series because \(x^{i}\) is a polynomial in
                  \(\Z[x]\) for each \(i\). This element is not in the image of
                  the tensor product however. Any basis element \(p\otimes g\)
                  for \(p\in\Z[x]\) and \(g\in\powser{\Z}{y}\) is mapped onto
                  \begin{align*}
                      p\cdot g & =\sum_{i=0}^{\infty}pa_{i}y^{i} \\
                               & \in\powser{\Z[x]}{y}.
                  \end{align*}
                  This means that the coefficients of the above power series
                  only differ by some constant factor
                  \(\nicefrac{a_{i}}{a_{j}}\). In porticular all coefficients
                  are polynomials of the same degree.

                  This means that for a finite combination
                  \[
                      \sum_{i=1}^{n}p_{i}\otimes y_{i}
                  \]
                  the resulting coefficients also have constant degree. This
                  means the above element cannot be in the image.
              \end{solution}
    \end{enumerate}
\end{question}

\begin{question}
    Let \(R\) be a ring and \(\maxid\) a maximal ideal with localisation
    \(R_{\maxid}\), together with its natural map \(f:R\to R_{\maxid}\). We
    write \(\maxidd=\maxid R_{\maxid}\) for the maximal ideal of \(R_{\maxid}\).

    \begin{enumerate}[a]
        \item Show that the map \(f\) is a continuous map of topological rings,
              where \(R\) is equipped with the \(\maxid\)-adic topology and
              \(R_{\maxid}\) with the \(\maxidd\)-adic topology.


              \begin{proof}
                  We will show the inverse image of any basis element of
                  \(R_{\maxid}\) is open. This will give that the function is
                  continuous. The set cosets \(\setwith{x+\maxidd^{k}}{x\in
                      R_{\maxid},k\in\N}\) form a basis for the topology.

                  We will show that
                  \[
                      f^{-1}[x+\maxidd^{i}]=\bigcup_{y,f(y)\in x+\maxidd^{i}}y+\maxidd^{i}.
                  \]
                  The right set is the union of open sets and so the thing
                  is open which means the map is continuous.

                  The right to left inclusion goes as follows: we take any \(y\)
                  with \(f(y)\in x+\maxidd^{i}\). Then for \(z\in y+\maxid^{i}\)
                  we have that \(f(z)=\nicefrac{z}{1}=\nicefrac{y+a}{1}\) for
                  some \(a\in\maxidd^{i}\). This means that \(f(z)\in
                  y+\maxidd^{i}\) and so
                  \[
                      \bigcup_{y,f(y)\in x+\maxidd^{i}}y+\maxidd^{i}\subseteq f^{-1}[x+\maxidd^{i}]
                  \]
                  For the other inclusion take any \(y\in
                  f^{-1}[x+\maxidd^{i}]\). Then \(f(y)\in x+\maxidd^{i}\).
                  \(y+\maxid^{i}\subseteq\bigcup_{y,f(y)\in
                      x+\maxidd^{i}}y+\maxid^{i}\). This proves the other inclusion
                  and therefore equality.

                  As explained before this suffices to show continuity so \(f\)
                  is continuous.
              \end{proof}

        \item Show for all non-negative integers \(i\) that \(\maxid^{i}\otimes
              R_{\maxid}=\maxidd^{i}\).

              \begin{proof}
                  Using properties of localisation gives
                  \begin{align*}
                      \maxid^{i}\otimes R_{\maxid} & \cong(\maxid^{i})_{\maxid} \\
                                                   & =\maxid^{i}R_{\maxid}.
                  \end{align*}
                  Therefore showing the desired statement is equivalent to
                  showing
                  \[
                      \maxid^{i}R_{\maxid}=\maxidd^{i}=(\maxid R_{\maxid})^{i}.
                  \]
                  The right to left inclusion is immediate and so we will focus
                  on the left to right inclusion.

                  We will show this by showing the generators of
                  \(\maxid^{i}R_{\maxid}\) are in the left ideal. This is
                  sufficient. Take any \(\prod_{j=1}^{i}a_{i}\) for
                  \(a_{i}\in\maxid\). The elements of this form generate
                  \(\maxid^{i}\). For any \(\nicefrac{x}{y}\in R_{\maxid}\)
                  where
                  \[
                      b_{i}=\begin{cases}
                          y & i=1,    \\
                          1 & i\neq1.
                      \end{cases}
                  \]
                  This means that the ideals \(\maxid^{i}R_{\maxid}\) and
                  \(\maxidd^{i}\) are equal.
              \end{proof}

        \item Given a positive integer \(i\), show that the map
              \(R/\maxid^{i}\to R_{\maxid}/\maxidd^{i}\) induced by \(f\) is an
              isomorphism.

              \begin{proof}
                  We know that \(\maxidd^{i}=(\maxid^{i})_{\maxid}\) by the
                  previous part. This means that
                  \begin{align*}
                      R_{\maxid}/\maxidd^{i} & =R_{\maxid}/(\maxid^{i})_{\maxid} \\
                                             & \cong(R/\maxid^{i})_{\maxid}      \\
                                             & =R/\maxid^{i}
                  \end{align*}
                  because every element of \(R/\maxid^{i}\setminus \maxid
                  R/\maxid^{i}\) is a unit because the ring is local.

                  To show this is the induced map, we consider how this found
                  isomorphism works. Given some \(x\in R\) we have get the
                  following sequence from the above isomorphisms:
                  \[
                      R/\maxid^{i}\ni\overline{r}\mapsto \overline{\frac{r}{1}}\mapsto\frac{\overline{r}}{\overline{1}}=\overline{f(r)}\in R_{\maxid}/\maxidd^{i}.
                  \]
                  Therefore the isomorphism is the induced morphism.
              \end{proof}

        \item Show that the natural map
              \(\widehat{f}:\widehat{R}\to\widehat{R_{\maxid}}\) is an
              isomorphism of groups.

              \begin{proof}
                  For each \(i\in\Z_{>0}\) we have the following diagram:
                  \[
                      \begin{tikzcd}[ampersand replacement=\&]
                          \& R \\
                          {R/\maxid^{i}} \&\& {R/\maxid^{i+1}} \\
                          \\
                          {R_{\maxid}/\maxidd^{i}} \&\& {R_{\maxid}/\maxidd^{i+1}}
                          \arrow[from=1-2, to=2-1]
                          \arrow[from=2-3, to=2-1]
                          \arrow[from=2-1, to=4-1]
                          \arrow[from=1-2, to=4-1]
                          \arrow[from=2-3, to=4-3]
                          \arrow[from=4-3, to=4-1]
                          \arrow[from=1-2, to=2-3]
                          \arrow[from=1-2, to=4-3]
                      \end{tikzcd}
                  \]
                  Each of the triangles with apex \(R\) commutes. Because the
                  maps from \(R\) are also epimorphisms this means the square
                  square commutes as well.

                  This means we have an exact sequence of inverse systems
                  \[
                      0\to0\to\set{0}\to\set{R/\maxid^{i}}\to\set{R_{\maxid}/\maxidd^{i}\to 0}
                  \]
                  so we have an exact sequence
                  \[
                      0\to\lim R/\maxid_{i}\to R_{\maxid}/\maxidd^{i}\to 0
                  \]
                  so the two limits are isomorphic.
              \end{proof}
    \end{enumerate}
\end{question}
\end{document}