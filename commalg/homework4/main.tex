\documentclass{article}

\usepackage[utf8]{inputenc}
\usepackage{enumerate}
\usepackage{scalerel}
\usepackage{amsthm, amssymb, mathtools, amsmath, bbm, mathrsfs, stmaryrd}
\usepackage[margin=1in]{geometry}
\usepackage[parfill]{parskip}
\usepackage[hidelinks]{hyperref}
\usepackage{quiver}
\usepackage{float}
\usepackage{nicefrac}

\newcommand{\N}{\mathbb{N}}
\newcommand{\Z}{\mathbb{Z}}
\newcommand{\NZ}{\mathbb{N}_{0}}
\newcommand{\Q}{\mathbb{Q}}
\newcommand{\R}{\mathbb{R}}
\newcommand{\C}{\mathbb{C}}

\newcommand{\F}{\mathbb{F}}
\newcommand{\incl}{\imath}

\newcommand{\tuple}[2]{\left\langle#1\colon #2\right\rangle}

\DeclareMathOperator{\order}{orde}
\DeclareMathOperator{\Id}{Id}
\DeclareMathOperator{\im}{im}
\DeclareMathOperator{\coker}{coker}
\DeclareMathOperator{\ggd}{ggd}
\DeclareMathOperator{\kgv}{kgv}
\DeclareMathOperator{\degree}{gr}

\DeclareMathOperator{\gl}{GL}

\DeclareMathOperator*{\bigplus}{\scalerel*{+}{\sum}}

\DeclareMathOperator{\Aut}{Aut}
\DeclareMathOperator{\End}{End}
\DeclareMathOperator{\Hom}{Hom}
\newcommand{\isom}{\overset{\sim}{\longrightarrow}}

\newcommand{\Grp}{{\bf Grp}}
\newcommand{\Ab}{{\bf Ab}}
\newcommand{\cring}{{\bf CRing}}
\DeclareMathOperator{\modules}{{\bf Mod}}
\newcommand{\catset}{{\bf Set}}
\newcommand{\cat}{\mathcal{C}}
\newcommand{\chains}{{\bf Ch}}
\newcommand{\homot}{{\bf Ho}}
\DeclareMathOperator{\objects}{Ob}
\newcommand{\gen}[1]{\left<#1\right>}
\DeclareMathOperator{\cone}{Cone}
\newcommand{\set}[1]{\left\{#1\right\}}
\newcommand{\setwith}[2]{\left\{#1:#2\right\}}
\DeclareMathOperator{\Ext}{Ext}
\DeclareMathOperator{\nil}{Nil}
\DeclareMathOperator{\idem}{Idem}
\DeclareMathOperator{\JRad}{JRad}
\DeclareMathOperator{\Nil}{Nil}
\DeclareMathOperator{\Idem}{Idem}
\DeclareMathOperator{\Rad}{Rad}
\DeclareMathOperator{\Ann}{Ann}
\DeclareMathOperator{\Mat}{Mat}
\DeclareMathOperator{\ch}{ch}
\DeclareMathOperator{\Ass}{Ass}

\newcommand{\maxid}{\mathfrak{m}}
\newcommand{\primeid}{\mathfrak{p}}
\newcommand{\primeeid}{\mathfrak{q}}
\newcommand{\ideal}{\triangleleft}

\newenvironment{solution}{\begin{proof}[Solution]\renewcommand\qedsymbol{}}{\end{proof}}

\newtheorem{theorem}{Theorem}[section]
\newtheorem{lemma}[theorem]{Lemma}
\newtheorem{proposition}[theorem]{Proposition}

\newtheorem{question}{Question}

\theoremstyle{definition}
\newtheorem{definition}[theorem]{Definition}
\newtheorem{remark}[theorem]{Remark}
\newtheorem{example}[theorem]{Example}
\newtheorem{corollary}[theorem]{Corollary}

\title{Homework Commutative Algebra}
\author{Jonas van der Schaaf}
\date{November 7, 2022}

\begin{document}
\maketitle

\begin{question}
    Let \(k\) be a field. Compute the minimal primary decomposition and the
    embedded primes of the following ideals:
    \begin{enumerate}[(a)]
        \item \(I=(x^{2},xy,yz,xz)\ideal k[x,y,z]\)

              \begin{solution}
                  The primary decomposition of \(I\) is given by
                  \[
                      I=(x^{2},y,z)\cap(x,y)\cap(x,z).
                  \]
                  These ideals are primary. The first one because its radical is
                  maximal and the other ones because they are prime.

                  To see this intersection is an equality, notice that
                  \(I\subseteq(x^{2},y,z)\cap(x,y)\cap(x,z)\) because each ideal
                  contains \(I\). To see the other inclusion, first by the
                  modular law \((x^{2},y,z)\cap(x,y)=(x^{2},xy,xz,y)\). To see
                  \((x^{2},xy,xz,y)\cap(x,z)\subseteq I\) we need to consider
                  all elements in the intersection. These are exactly those
                  elements with \(ax^{2}+bxy+cxz+dy=ex+fz\) for \(a,b,c,d,e,f\in
                  k[x,y,z]\). Because no monomial with only a power of \(y\)
                  occurs on the right side of the equality we get
                  \(d=(d'x+d''z)\) for some \(d',d''\in k[x,y,z]\). This means
                  that \(ax^{2}+bxy+cxz+dy\in I\). Therefore the intersection
                  of primary ideals as given above is indeed a decomposition.

                  The associated ideals are given by \((x,y,z),(x,y),(x,z)\)
                  where \((x,y)\) and \((x,z)\) are embedded because they're
                  contained in the minimal associated ideal \((x,y,z)\).
              \end{solution}

        \item \(I=(x^{2}+1)\ideal k[x]\)

              \begin{solution}
                  If \(x^{2}+1\) is irreducible in \(k[x]\) then \(I\) is a
                  prime ideal so it is primary. Therefore it has a unique
                  associated ideal which must be minimal: there are no embedded
                  ideals.

                  Suppose \(x^{2}+1\) is not irreducible. For \(k\) of
                  characteristic \(2\) we have \(x^{2}+1=(x+1)^{2}\). Then
                  \(r(x^{2}+1)=(x+1)\) which is maximal. Therefore \((x^{2}+1)\)
                  is primary so its primary decomposition is given by just
                  itself. There are no embedded ideals because this is a
                  decomposition into one primary ideal.

                  If \(k\) has character different from \(2\) and the polynomial
                  is not irreducible then there is an \(i\in k\) such that
                  \((x+i)(x-i)=x^{2}+1\). The ideals \((x+i),(x-i)\) are coprime
                  because \(\nicefrac{i}{2}(x+i-(x-i))=1\) so \(1\in
                  (x+i)+(x-i)\). This means
                  \((x^{2}+1)=(x-i)\cdot(x+i)=(x+i)\cap(x-i)\). The latter is a
                  decomposition into maximal (and therefore primary) ideals.
                  From coprimality we conclude neither ideal contains the other
                  so the decomposition is minimal. The associated primes are
                  \((x+i)\) and \((x-i)\), both of which are minimal.
              \end{solution}
    \end{enumerate}
\end{question}

\begin{question}
    Let \(A\) be a ring.
    \begin{enumerate}[(a)]
        \item Let \(M,N\) be two \(A\)-modules. Show that \(\Ass(M\oplus
              N)=\Ass(M)\cup\Ass(N)\).

              \begin{proof}
                  Suppose \(\primeid\in\Ass(M)\cup\Ass(N)\). Without loss of
                  generality we can assume \(\primeid\in\Ass(M)\) so
                  \(\primeid=\Ann(m)\) for some \(m\in M\). Then \(\primeid\cdot
                  m\oplus 0=\primeid m\oplus0=0.\) so \(\primeid=\Ann(m\oplus
                  0)\) and therefore \(\primeid\in\Ass(M\oplus N)\).

                  If \(\primeid=\Ann(m\oplus n)\) is a prime ideal for \(m\in
                  M,n\in N\) then \(\primeid=\Ann(m)\cap\Ann(n)\). Then
                  \(\primeid=\Ann(m)\) or \(\primeid=\Ann(n)\) by proposition
                  1.11 of Atiyah and MacDonald.
              \end{proof}

        \item Let \(n\notin\set{0,1,-1}\). Compute \(\Ass_{\Z}(M)\) for
              \(M=\Z^{5}\oplus\Z/n\Z\). Justify your answer

              \begin{solution}
                  By the previous part we know
                  \(\Ass_{\Z}(M)=\Ass_{\Z}(\Z^{5})\cup\Ass_{\Z}(\Z/n\Z)\).
                  Because \(\Z\) is a torsion free module we have
                  \(\Ass_{\Z}(\Z^{5})=\varnothing\).

                  The torsion elements of \(\Z/n\Z\) are those elements
                  \(\overline{x}\) with \(\gcd(x,n)\neq 1\). An torsion \(n\)
                  element has a prime ideal \((p)\ideal\Z\) as annihilator
                  exactly iff \(p\) divides every annihilator of \(n\). This
                  means the torsion elements of \(\Z/n\Z\) with prime ideals as
                  annihilators are given by \(\nicefrac{n}{p}\) where \(p\) is a
                  prime dividing \(n\). Equivalently, the associated ideals of
                  \(\Z/n\Z\) are \((p)\) where \(p\mid n\). Therefore
                  \(\Ass(\Z^{5}\oplus\Z/n\Z)=\Ass_{\Z}(\Z/n\Z)=\setwith{(p)\ideal
                      \Z}{\textnormal{\(p\)  is prime.}}\).

              \end{solution}
    \end{enumerate}
\end{question}

\begin{question}
    Let \(A=\Q[x,y,z]/((x+1)^{2}-z(y+z^{2022})^{2})\).
    \begin{enumerate}[(a)]
        \item Prove that \(A\) is an integral domain.

              \begin{proof}
                  There is a morphism \(\varphi:\Q[x,y,z]\to\Q[x,y,z]\) given by
                  \[
                      \varphi(x)=x+1,\varphi(y)=y+z^{2022},\varphi(z)=z
                  \]
                  which is invertible by
                  \[
                      \varphi(x)=x-1,\varphi(y)=y-z^{2022},\varphi(z)=z
                  \]
                  and is therefore an isomorphism. This means that
                  \((x+1)^{2}-z(y-z^{2022})^{2}\) is irreducible in
                  \(\Q[x,y,z]\) iff \(x^{2}-zy^{2}\) is as well.

                  We can see that the latter satisfies the Eisenstein criterion
                  because \(z\mid zy^{2}\), \(z^{2}\nmid zy^{2}\), and \(z\nmid
                  x^{2}\). This means that \((x+1)^{2}-z(y+z^{2022})^{2}\) is
                  also irreducible and therefore
                  \(((x+1)^{2}-z(y+z^{2022})^{2})\) is a prime ideal and so it's
                  quotient is a domain.
              \end{proof}

        \item Prove that the localization \(A_{y+z^{2022}}\) is integrally
              closed.

              \begin{proof}
                  Applying the isomorphism from part a, this question equivalent
                  to showing that for \(A'=\Q[x,y,z]/(x^{2}-zy^{2})\) its
                  localization \(A'_{y}\) is integrally closed.

                  First notice that \(z=x^{2}/y^{2}\) in \(A'\) and therefore we
                  get the isomorphism \(A'\cong
                  \Q[x,y,\nicefrac{x^{2}}{y^{2}}]\). We easily see that
                  \(A'_{y}=\setwith{y^{n}}{n\in\N}^{-1}\Q[x,y,\nicefrac{x^{2}}{y^{2}}]=\setwith{y^{n}}{n\in\N}^{-1}\Q[x,y]\).

                  Because \(\Q[x,y]\) is a UFD it's integrally closed in
                  \(\Q(x,y)\). By proposition 5.12 this means \(S^{-1}\Q[x,y]\)
                  is integrally closed in \(S^{-1}\Q(x,y)\). So in particular
                  for \(S=\setwith{y^{n}}{n\in\N}\) we have
                  \(\Q[x,y,\nicefrac{1}{y}]\) being integrally closed in
                  \(S^{-1}\Q(x,y)=\Q(x,y)\). Therefore \(A_{y+z^{2022}}\) is
                  integrally closed.
              \end{proof}

        \item Compute the integral closure of \(A\).

              \begin{proof}
                  The integral closure \(\overline{A}\) of \(A\) is a subset of
                  \(A_{y-z^{2022}}\). To see this take any \(b\in\overline{A}\).
                  Then it is the root of a monic polynomial \(f\in A[x]\).
                  However because \(A[t]\subseteq A_{y+z^{2022}}[t]\) we get
                  that \(b\) is also integral over \(A_{y+z^{2022}}\), which is
                  integrally closed so \(b\in A_{y+z^{2022}}\). This gives
                  \(\overline{A}\subseteq A_{y+z^{2022}}\).

                  It's clear that \(\nicefrac{x}{y}\in\overline{A}\) because
                  \(t^{2}-\nicefrac{x}{y}\) has \(\nicefrac{x}{y}\) as a root
                  and is monic. We will show this is integrally closed.

                  Now let \(\nicefrac{f}{y^{m}}\in \Q[x,y,\nicefrac{1}{y}]\) be
                  integral over \(\Q[x,y,\nicefrac{x}y]\). We can assume
                  \(y\nmid f\). Then defining \(g=y^{m}\)
                  \[
                      \frac{f^{n}}{g^{n}}+a_{1}\frac{f^{n-1}}{g^{n-1}}+\ldots+a_{n}=0.
                  \]
                  where \(a_{i}\) are coefficients in
                  \(\Q[x,y,\nicefrac{x^{2}}{y^{2}}]\). Multiplying with
                  \(g^{n}\) gives
                  \[
                      f^{n}+a_{1}f^{n-1}g+\ldots+a_{n}g^{n}=0
                  \]
                  which with minimal algebraic manipulation gives
                  \[
                      -(a_{1}f^{n-1}y^{m}+\ldots+a_{n}y^{nm})=f^{n}\in\Q[x,y,\nicefrac{x}{y}].
                  \]
                  From this someone smarter than me could conclude that
                  \(x^{m}\mid f\) and thus that
                  \(\nicefrac{f}{g}\in\Q[x,y,\nicefrac{x}{y}]\) which would mean
                  it is integrally closed.
              \end{proof}
    \end{enumerate}
\end{question}
\end{document}