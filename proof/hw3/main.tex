\documentclass{article}

\usepackage[utf8]{inputenc}
\usepackage[english]{babel}
\usepackage[shortlabels]{enumitem}
\usepackage{amsthm, amssymb, mathtools, amsmath, bbm, mathrsfs,stmaryrd}
\usepackage[margin=3cm]{geometry}
\usepackage[parfill]{parskip}
\usepackage[hidelinks]{hyperref}
\usepackage{ebproof}
\usepackage{cleveref}

\usepackage{caption}
\usepackage{subcaption}
\usepackage{tikz}

\newcommand{\sequent}[2]{#1\Rightarrow#2}
\newcommand{\set}[1]{\left\{#1\right\}}
\newcommand{\setwith}[2]{\set{#1:#2}}

\theoremstyle{definition}
\newtheorem{question}{Exercise}
\newtheorem*{lemma}{Lemma}

\newenvironment{solution}{\begin{proof}[Solution]\renewcommand{\qedsymbol}{}}{\end{proof}}

\title{Homework Proof Theory}
\author{Jonas van der Schaaf}
\date{\today}

\begin{document}
\maketitle

\begin{question}
    Suppose \(\sequent{\Gamma}{\forall x:\varphi[a/x],\Delta}\) is provable in
    G3c with the cut-rule by a proof of depth \(n\) and cutrank \(m\) where
    \(a\) does not occur in \(\Gamma,\Delta\). Show that
    \(\sequent{\Gamma}{\varphi,\Delta}\) is provable by a proof of depth \(\leq
    n\) and cutrank \(\leq m\).

    \begin{proof}
        This will be a proof by induction of proof depth.

        If \(\sequent{\Gamma}{\forall x:\varphi[a/x]},\Delta\) is an axiom, then
        there are three options: either \(\top\in\Delta\), \(\bot\in\Gamma\) or
        there is an atomic formula in the intersection \(\Gamma\cap\Delta\). In
        the former two cases the sequent \(\sequent{\Gamma}{\varphi,\Delta}\) is
        also an axiom. In the latter case, because \(\forall x:\varphi[a/x]\) is
        not an atomic formula, the formula occurring on both sides of the
        sequent must be a different one than it. Therefore
        \(\sequent{\Gamma}{\varphi,\Delta}\) is also an axiom.

        Now for the induction step. We will not show the induction step for all
        rule applications. We will show it for one connective, two quantifiers
        and the cut rule. The induction step for other rules will be similar or
        easier.

        The connective rule we demonstrate the theorem for is \(\vee L\).
        Suppose \(\vdash\sequent{\psi\vee\chi,\Gamma}{\forall
            x:\varphi[a/x],\Delta}\) and the last rule in the proof was \(\vee L\)
        where \(\psi\vee\chi\) was principal. Then the proof must have ended
        with
        \[
            \begin{prooftree}
                \hypo{\sequent{\psi,\Gamma}{\forall x:\varphi[a/x],\Delta}}
                \hypo{\sequent{\chi,\Gamma}{\forall x:\varphi[a/x],\Delta}}
                \infer2[\(\vee L\)]{\sequent{\psi\vee\chi,\Gamma}{\forall x:\varphi[a/x],\Delta}}
            \end{prooftree}
        \]
        where the two sequents \(\sequent{\psi,\Gamma}{\forall
            x:\varphi[a/x],\Delta}\) are provable by a proof of depth \(n-1\)
        and cut rank \(\leq m\). Therefore by the induction hypothesis
        \(\sequent{\psi,\Gamma}{\varphi,\Delta}\) and
        \(\sequent{\psi,\Gamma}{\varphi,\Delta}\) must also be provable by a
        proof of depth \(n-1\) and cut-rank \(\leq m\). This means that we
        can combine these proofs to get a proof of depth \(n\) and cut rank
        \(\leq m\):
        \[
            \begin{prooftree}
                \hypo{\sequent{\psi,\Gamma}{\varphi,\Delta}}
                \hypo{\sequent{\chi,\Gamma}{\varphi,\Delta}}
                \infer2[\(\vee L\)]{\sequent{\psi\vee\chi,\Gamma}{\varphi,\Delta}}
            \end{prooftree}
        \]

        Suppose the last applied rule was the cut rule on some formula
        \(\vartheta\): either
        \[
            \begin{prooftree}
                \hypo{\sequent{\Gamma}{\forall x:\varphi[a/x],\Delta},\vartheta}
                \hypo{\sequent{\vartheta,\Gamma'}{\Delta'}}
                \infer2[Cut]{\sequent{\Gamma,\Gamma'}{\forall x:\varphi[a/x],\Delta,\Delta'}}
            \end{prooftree}
        \]
        or
        \[
            \begin{prooftree}
                \hypo{\sequent{\Gamma}{\Delta,\vartheta}}
                \hypo{\sequent{\vartheta,\Gamma'}{\forall x:\varphi[a/x],\Delta'}}
                \infer2[Cut]{\sequent{\Gamma,\Gamma'}{\forall x:\varphi[a/x],\Delta,\Delta'}}
            \end{prooftree}
        \]
        In either case we see that by the induction hypothesis
        \(\sequent{\Gamma}{\varphi,\Delta},\vartheta\) must have been provable
        in the former case by a proof of depth \(n-1\) and in the latter case
        the same holds for \(\sequent{\vartheta,\Gamma'}{\varphi,\Delta'}\).
        This means that either
        \[
            \begin{prooftree}
                \hypo{\sequent{\Gamma}{\varphi,\Delta},\vartheta}
                \hypo{\sequent{\vartheta,\Gamma'}{\Delta'}}
                \infer2[Cut]{\sequent{\Gamma,\Gamma'}{\varphi,\Delta,\Delta'}}
            \end{prooftree}
        \]
        or
        \[
            \begin{prooftree}
                \hypo{\sequent{\Gamma}{\Delta,\vartheta}}
                \hypo{\sequent{\vartheta,\Gamma'}{\forall x:\varphi[a/x],\Delta'}}
                \infer2[Cut]{\sequent{\Gamma,\Gamma'}{\forall x:\varphi[a/x],\Delta,\Delta'}}
            \end{prooftree}
        \]
        must be the end of a valid proof of depth \(n\) with the same cutrank as
        the original.

        If the last applied rule in the proof was \(\forall R\) then the proof
        must have ended as follows:
        \[
            \begin{prooftree}
                \hypo{\sequent{\Gamma}{\varphi,\Delta}}
                \infer1[\(\forall R\)]{\sequent{\Gamma}{\forall x\varphi[a/x],\Delta}}
            \end{prooftree}
        \]
        from which it is immediately clear that
        \(\sequent{\Gamma}{\varphi,\Delta}\) is provable with depth \(\leq n\)
        and cutrank \(\leq m\). Note that this will always work because \(a\)
        does not occur in \(\Gamma,\Delta\) by assumption.

        If the last step was \(\exists L\), then we have to be a bit more
        careful: suppose the proof ends in the following manner
        \[
            \begin{prooftree}
                \hypo{\sequent{\vartheta,\Gamma}{\forall x:\varphi[a/x],\Delta}}
                \infer1[\(\exists L\)]{\sequent{\exists y:\vartheta[b/y],\Gamma}{\forall x:\varphi[a/x],\Delta}}
            \end{prooftree}
        \]
        If \(b=a\) then the assumptions do not hold, so we cannot apply the
        induction hypothesis. We do know that \(a\) does not occur elsewhere, so
        using depth-preserving substitution we know that taking \(c\) to be a
        fresh variable
        \[
            \sequent{\vartheta[a/c],\Gamma}{\forall x:\varphi[a/x],\Delta}
        \]
        is provable with depth \(n-1\) and cutrank \(\leq m\). Then by
        assumption
        \[
            \sequent{\vartheta[a/c],\Gamma}{\varphi,\Delta}
        \]
        is provable with the same depth \(n-1\) and cutrank \(m\) and \(c\) does
        not occur in \(\Delta,\Gamma\). Therefore there is also a proof ending
        on
        \[
            \begin{prooftree}
                \hypo{\sequent{\vartheta[a/c],\Gamma}{\varphi,\Delta}}
                \infer1[\(\exists L\)]{\sequent{\exists y:\vartheta[a/c][c/y],\Gamma}{\varphi,\Delta}                }
            \end{prooftree}
        \]

        This shows the induction step for some important rule applications and
        will have to do.
    \end{proof}
\end{question}
\end{document}