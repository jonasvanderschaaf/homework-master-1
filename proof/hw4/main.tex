\documentclass{article}

\usepackage[utf8]{inputenc}
\usepackage[english]{babel}
\usepackage[shortlabels]{enumitem}
\usepackage{amsthm, amssymb, mathtools, amsmath, bbm, mathrsfs,stmaryrd}
\usepackage[margin=3cm]{geometry}
\usepackage[parfill]{parskip}
\usepackage[hidelinks]{hyperref}
\usepackage{ebproof}
\usepackage{cleveref}

\usepackage{caption}
\usepackage{subcaption}
\usepackage{tikz}

\newcommand{\sequent}[2]{#1\Rightarrow#2}
\newcommand{\set}[1]{\left\{#1\right\}}
\newcommand{\setwith}[2]{\set{#1:#2}}

\theoremstyle{definition}
\newtheorem{question}{Exercise}
\newtheorem*{lemma}{Lemma}

\newenvironment{solution}{\begin{proof}[Solution]\renewcommand{\qedsymbol}{}}{\end{proof}}

\title{Homework Proof Theory}
\author{Jonas van der Schaaf}
\date{\today}

\begin{document}
\maketitle

\begin{question}
    Suppose \(\Gamma\) consists of Harrop formulas and
    \(\sequent{\Gamma}{\varphi\vee\psi}\) is provable in G3i. Show that either
    \(\sequent{\Gamma}{\varphi}\) or \(\sequent{\Gamma}{\psi}\).

    \begin{proof}
        We show this by induction on proof depth.

        If \(\sequent{\Gamma}{\varphi\vee\psi}\) is an axiom, then either
        \(\bot\in\Gamma\), in which case \(\sequent{\Gamma}{\varphi}\) and
        \(\sequent{\Gamma}{\psi}\) are both axioms as well and therefore
        provable.

        It cannot be that \(\Gamma=\Gamma',\varphi\vee\psi\) because
        \(\varphi\vee\psi\) is not a Harrop formula.

        Now for the induction step suppose the theorem holds for sequents
        provable by proofs of depth \(n\) and let
        \(\sequent{\Gamma}{\varphi\vee\psi}\) be a sequent provable by a proof
        of depth \(n+1\). Then either \(\varphi\vee\psi\) was principal in the
        last step of this proof or it was not.

        If it was principal then the last step of the proof was \(\vee R\). The
        sequent above it in the proof was then \(\sequent{\Gamma}{\varphi}\) or
        \(\sequent{\Gamma}{\psi}\) so one of these sequents is also provable.

        If this formula was not principal, then because \(\Gamma\) is a set of
        Harrop formulas this must have been one of the following rule
        applications: \(\wedge L\), \(\forall L\), \(\to L\).

        If the rule was \(\wedge L\) and \(\theta\wedge\chi\) was principal and
        \(\Gamma=\theta\wedge\chi,\Gamma'\), then the proof must have ended as
        follows:
        \[
            \begin{prooftree}
                \hypo{\sequent{\theta,\chi,\Gamma'}{\varphi\vee\psi}}
                \infer1[\(\wedge L\)]{\sequent{\Gamma}{\varphi\vee\psi}}
            \end{prooftree}
        \]
        By the induction hypothesis then either
        \(\sequent{\theta,\chi,\Gamma'}{\psi}\) is provable or
        \(\sequent{\theta,\chi,\Gamma'}{\varphi}\) is. This means that with an
        application of \(\wedge L\) one of the two desired sequents is also
        provable proving the induction hypothesis.

        If the principal formula was \(\forall x.\varphi[a/x]\) and
        \(\Gamma=\forall x.\varphi[a/x],\Gamma'\) then the end of the proof must
        have looked like
        \[
            \begin{prooftree}
                \hypo{\sequent{\forall x.\varphi[a/x],\varphi[a/t],\Gamma'}{\varphi\vee\psi}}
                \infer1[\(\forall L\)]{\sequent{\forall x.\varphi[a/x]\Gamma}{\varphi\vee\psi}}
            \end{prooftree}
        \]
        The left hand side of this sequent is Harrop, so by the induction
        hypothesis \(\sequent{\forall
            x.\varphi[a/x],\varphi[a/t],\Gamma'}{\varphi}\) or \(\sequent{\forall
            x.\varphi[a/x],\varphi[a/t],\Gamma'}{\psi}\) is provable. Then applying
        \(\forall L\) to whichever of the previous sequent was provable shows
        one of the desired sequents is also provable.

        Now suppose the principal formula was an implication \(\theta\to\chi\)
        and \(\Gamma=\theta\to\chi,\Gamma'\), then the proof ended as
        \[
            \begin{prooftree}
                \hypo{\sequent{\theta\to\chi,\Gamma'}{\theta}}
                \hypo{\sequent{\chi,\Gamma'}{\varphi\vee\psi}}
                \infer2[\(\wedge L\)]{\sequent{\theta\to\chi,\Gamma'}{\varphi\vee\psi}}
            \end{prooftree}
        \]
        We know \(\chi\) is Harrop so \(\chi,\Gamma'\) is also Harrop
        and therefore by induction \(\sequent{\chi,\Gamma'}{\varphi}\)
        or \(\sequent{\chi,\Gamma'}{\psi}\) must be provable. This means that
        \[
            \begin{prooftree}
                \hypo{\sequent{\theta\to\chi,\Gamma'}{\theta}}
                \hypo{\sequent{\chi,\Gamma'}{\varphi}}
                \infer2[\(\wedge L\)]{\sequent{\theta\to\chi,\Gamma'}{\psi}}
            \end{prooftree}
        \]
        or
        \[
            \begin{prooftree}
                \hypo{\sequent{\theta\to\chi,\Gamma'}{\theta}}
                \hypo{\sequent{\chi,\Gamma'}{\varphi}}
                \infer2[\(\wedge L\)]{\sequent{\theta\to\chi,\Gamma'}{\varphi}}
            \end{prooftree}
        \]
        must be the end of a valid proof tree and so either of the desired
        sequents must be provable. aksbdja

        This proves the induction hypothesis and therefore the desired
        statement.
    \end{proof}
\end{question}
\end{document}