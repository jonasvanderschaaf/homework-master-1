\documentclass{article}

\usepackage[utf8]{inputenc}
\usepackage[english]{babel}
\usepackage[shortlabels]{enumitem}
\usepackage{amsthm, amssymb, mathtools, amsmath, bbm, mathrsfs,stmaryrd}
\usepackage[margin=3cm]{geometry}
\usepackage[parfill]{parskip}
\usepackage[hidelinks]{hyperref}
\usepackage{ebproof}
\usepackage{cleveref}

\usepackage{caption}
\usepackage{subcaption}
\usepackage{tikz}

\newcommand{\sequent}[2]{#1\Rightarrow#2}
\newcommand{\set}[1]{\left\{#1\right\}}
\newcommand{\setwith}[2]{\set{#1:#2}}

\theoremstyle{definition}
\newtheorem{question}{Exercise}
\newtheorem*{lemma}{Lemma}

\DeclareMathOperator{\ea}{EA}

\newenvironment{solution}{\begin{proof}[Solution]\renewcommand{\qedsymbol}{}}{\end{proof}}

\title{Homework Proof Theory}
\author{Jonas van der Schaaf}
\date{\today}

\begin{document}
\maketitle

\begin{question}
    Show that the function \(f(n)=\textnormal{the maximal prime number \(p\leq
        n\)}\) is elementary.

    \begin{proof}
        We demonstrate it is elementary by finding a \(\Delta_{0}\) formula
        which is true iff \(p\) is the maximal prime number below \(n\).

        The formula
        \[
            \varphi(n)=\forall m\leq n:\left(m\neq S(0)\wedge m\neq n\to\neg\exists k\leq n:k\cdot m=n\right)
        \]
        is true iff \(n\) is a prime number: for all \(m\leq n\) if \(m\) is not
        \(1\) or \(n\) then there is no \(k\leq n\) with \(m\cdot k=n\).

        Then
        \[
            \psi(n,m)=m\leq n\wedge\forall k\leq n:m\leq k\wedge m\neq k\to \neg\varphi(k)
        \]
        is true iff \(m\) is the largest prime number less than or equal to
        \(n\): it is true if \(m\leq n\) and for all numbers \(k\) between \(m\) and
        \(n\) the number \(k\) is not a prime number.

        This means there is an elementary function \(\chi_{\psi}(n,m)\) which
        corresponds exactly to the truth value of \(\psi\) so this function is
        elementary.
    \end{proof}
\end{question}

\begin{question}
    Prove in EA: \(\varphi(a,b)=a\leq b\vee b\leq a\).

    \begin{proof}
        We show that
        \begin{equation}
            \ea\vdash\forall n:n\leq b\vee b\leq n
        \end{equation}
        and then
        conclude by the inversion lemma that
        \begin{equation}
            \ea\vdash a\leq b\vee b\leq a
        \end{equation}

        We know that \(\varphi(0,b)\wedge\forall
        x:\varphi(x,b)\to\varphi(S(x),b)\) is an axiom of EA because \(\varphi\)
        is a \(\Delta_{0}\) formula. By exercise 2.3 this means that if
        \begin{equation}
            \ea\vdash\varphi(0,b)\wedge\forall x:(\varphi(x,b)\to\varphi(S(x),b)) \label{eq:antec-induc1}
        \end{equation}
        then \(\ea\vdash\forall x:\varphi(x,b)\). If we can prove that
        \(\ea\vdash\varphi(0,b)\) and \(\ea\vdash\forall
        x:(\varphi(x,b)\to\varphi(S(x),b))\) then by applying the \(\wedge R\)
        rule we conclude that \Cref{eq:antec-induc1} must hold as well. We know
        that \(\ea\vdash\varphi(0,b)\) by exercise 3.4. Now let us consider the
        other formula. We show that
        \begin{equation}
            \ea\vdash\varphi(a,b)\to\varphi(S(a),b)
        \end{equation}
        and by applying \(\forall R\) then we obtain the desired statement.

        We now prove this implication as follows: suppose
        \(\ea\vdash\varphi(a,b)\) i.e. \(\ea\vdash a\leq b\vee b\leq a\). This
        means there is a finite \(\Gamma_{0}\subseteq\ea\) such that
        \(\sequent{\Gamma_{0}}{a\leq b\vee b\leq a}\). By the inversion lemma
        this means that \(\sequent{\Gamma_{0}}{a\leq b,b\leq a}\). Using
        exercise 3.5 and multiple uses of the inversion lemma we know there is a
        finite \(\Gamma_{1}\subseteq\ea\) such that \(\sequent{\Gamma_{1},a\leq
            b}{S(a)\leq b,S(a)=b}\). Applying cut elimination and then inversion
        then gives
        \begin{equation}
            \vdash\sequent{\Gamma_{0},\Gamma_{1}}{b\leq a,S(a)\leq b,S(a)=b}. \label{eq:almostthere}
        \end{equation}
        By exercise 3.5 we know there is a finite \(\Gamma_{2}\subseteq\ea\)
        such that \(\vdash\sequent{\Gamma_{2}}{b\leq a\to b\leq S(a)}\) so by
        the inversion lemma we know that \(\vdash\sequent{\Gamma_{2},b\leq a}{b\leq
            S(a)}\) so applying cut elimination to the previous provable sequent
        and \Cref{eq:almostthere} we get
        \begin{equation}
            \vdash\sequent{\Gamma_{0},\Gamma_{1},\Gamma_{2}}{b\leq S(a),S(a)\leq b,S(a)=b}.
        \end{equation}
        We know that because \(=\) is defined to be a congruence relation
        \(S(a)=b\to S(a)\leq b\) is provable in \(\ea\) using some finite set of
        axioms \(\Gamma_{3}\subseteq\ea\):
        \(\vdash\sequent{\Gamma_{3}}{S(a)=b\to S(a)\leq b}\). Applying the
        inversion lemma and another cut elimination this gives that
        \begin{equation}
            \vdash\sequent{\Gamma_{0},\Gamma_{1},\Gamma_{2},\Gamma_{3}}{b\leq S(a),S(a)\leq b,S(a)\leq b}.
        \end{equation}
        Applying contraction and \(\vee R\), this gives the statement that:
        \[
            \vdash\sequent{\Gamma_{0},\Gamma_{1},\Gamma_{2},\Gamma_{3}}{b\leq S(a)\vee S(a)\leq b}.
        \]

        This proves that
        \[
            \vdash\sequent{\Gamma_{0},\Gamma_{1},\Gamma_{2},\Gamma_{3}}{(a\leq b\vee b\leq a)\to(S(a)\leq b\vee b\leq S(a))}.
        \]
        By the reasoning as given above this means that the desired statement is
        true.
    \end{proof}
\end{question}
\end{document}