\documentclass{article}

\usepackage[utf8]{inputenc}
\usepackage[english]{babel}
\usepackage[shortlabels]{enumitem}
\usepackage{amsthm, amssymb, mathtools, amsmath, bbm, mathrsfs,stmaryrd}
\usepackage[margin=3cm]{geometry}
\usepackage[parfill]{parskip}
\usepackage[hidelinks]{hyperref}
\usepackage{ebproof}
\usepackage{cleveref}

\usepackage{caption}
\usepackage{subcaption}
\usepackage{tikz}

\newcommand{\sequent}[2]{#1\Rightarrow#2}
\newcommand{\set}[1]{\left\{#1\right\}}
\newcommand{\setwith}[2]{\set{#1:#2}}

\theoremstyle{definition}
\newtheorem{question}{Exercise}
\newtheorem*{lemma}{Lemma}

\newenvironment{solution}{\begin{proof}[Solution]\renewcommand{\qedsymbol}{}}{\end{proof}}

\title{Homework Proof Theory}
\author{Jonas van der Schaaf}
\date{\today}

\begin{document}
\maketitle

\begin{question}
    Construct a derivation of \(\sequent{}{(p\to q)\to((\neg p\to q)\to q)}\).

    \begin{solution}
        A possible proof tree for the sequent is as follows:
        \begin{figure}[ht]
            \centering
            \begin{prooftree}
                \hypo{\sequent{q}{p,q}}

                \hypo{\sequent{p}{p,q}}
                \infer1[\(\neg L\)]{\sequent{}{\neg p,p,q}}
                \infer2[\(\to L\)]{\sequent{\neg p\to q}{p,q}}

                \hypo{\sequent{q,\neg p\to q}{q}}

                \infer2[\(\to L\)]{\sequent{p\to q,\neg p\to q}{q}}
                \infer1[\(\to R\)]{\sequent{p\to q}{(\neg p\to q)\to q}}
                \infer1[\(\to R\)]{\sequent{}{(p\to q)\to((\neg p\to q)\to q)}}
            \end{prooftree}
            \caption{The proof tree for the desired sequent.}
        \end{figure}
    \end{solution}
\end{question}

\begin{question}
    Show that the following sequent is unprovable: \(\sequent{}{(p\to q)\to\neg
        p}\).

    \begin{proof}
        We will demonstrate this sequent is unprovable by showing the sequent
        \[
            \sequent{\Gamma}{\Delta}:=\sequent{p,q,p\to q}{(p\to q)\to\neg p,\neg p}
        \]
        is unprovable first. Then because the weakening rule is admissible the
        original is also unprovable. Note that we have chosen this sequent to be
        saturated without adding more formulas than necessary. This will be
        important for our proof.

        If this sequent were provable, there would be three rules which could be
        applied resulting in the above sequent in a possible proof: \(\to L\),
        \(\to R\) and \(\neg R\).

        This sequent is not provable without using \(\to R\). If only \(\to L\)
        and \(\neg R\) were used in the proof then \(\sequent{p,q}{(p\to
            q)\to\neg p,p}\) and \(\sequent{p,q}{(p\to q)\to\neg p}\) would have to
        be axioms. The former is one, but the latter is not. Therefore \(\to R\)
        must have been applied to get to this sequent.

        We can also assume that before the application of \(\to R\), the formula
        \((p\to q)\to\neg p\) did not occur in the right sequent before the
        application of the rule. This is by exactly the same reasoning above:
        we know there is no way to get to derive the sequent from an axiom if
        it contains the formula.

        From this it is easy to see that in the hypothetical proof, there would
        be a sequent above \(\sequent{\Gamma}{\Delta}\) which is a
        ``subsequent'' of \(\sequent{p,q,p\to q}{\neg p}\). If we show this is
        not provable, then \(\sequent{\Gamma}{\Delta}\) is not either so the
        original sequent must be unprovable as well.

        But this sequent is unprovable, exactly by the same reasoning as before:
        applying both \(\neg p\) and \(\to L\) in either order shows that the
        sequent is only provable if \(\sequent{p,q}{}\) is an axiom. It is not
        so the original sequent is not provable.
    \end{proof}
\end{question}
\end{document}