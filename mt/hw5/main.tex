\documentclass{article}

\usepackage[utf8]{inputenc}
\usepackage{enumerate}
\usepackage{amsthm, amssymb, mathtools, amsmath, bbm, mathrsfs, stmaryrd, xcolor}
\usepackage{nicefrac}
\usepackage[margin=1in]{geometry}
\usepackage[parfill]{parskip}
\usepackage[hidelinks]{hyperref}
\usepackage{float}
\usepackage{cleveref}
\usepackage{svg}
\usepackage{tikz-cd}

\renewcommand{\qedsymbol}{\raisebox{-0.5cm}{\includesvg[width=0.5cm]{../../qedboy.svg}}}


\newcommand{\N}{\mathbb{N}}
\newcommand{\Z}{\mathbb{Z}}
\newcommand{\NZ}{\mathbb{N}_{0}}
\newcommand{\Q}{\mathbb{Q}}
\newcommand{\R}{\mathbb{R}}
\newcommand{\C}{\mathbb{C}}
\newcommand{\A}{\mathbb{A}}
\newcommand{\proj}{\mathbb{P}}
\newcommand{\sheaf}{\mathcal{O}}
\newcommand{\FF}{\mathcal{F}}
\newcommand{\G}{\mathcal{G}}

\newcommand{\zproj}{Z_{\textnormal{proj}}}

\newcommand{\maxid}{\mathfrak{m}}
\newcommand{\primeid}{\mathfrak{p}}
\newcommand{\primeidd}{\mathfrak{q}}

\newcommand{\F}{\mathbb{F}}
\newcommand{\incl}{\imath}

\newcommand{\tuple}[2]{\left\langle#1\colon #2\right\rangle}

\DeclareMathOperator{\order}{order}
\DeclareMathOperator{\Id}{Id}
\DeclareMathOperator{\im}{im}
\DeclareMathOperator{\ggd}{ggd}
\DeclareMathOperator{\kgv}{kgv}
\DeclareMathOperator{\degree}{gr}
\DeclareMathOperator{\coker}{coker}
\DeclareMathOperator{\matrices}{Mat}

\DeclareMathOperator{\gl}{GL}

\DeclareMathOperator{\Aut}{Aut}
\DeclareMathOperator{\Hom}{Hom}
\DeclareMathOperator{\End}{End}
\DeclareMathOperator{\colim}{colim}
\newcommand{\isom}{\overset{\sim}{\longrightarrow}}

\newcommand{\schemes}{{\bf Sch}}
\newcommand{\aff}{{\bf Aff}}
\newcommand{\Grp}{{\bf Grp}}
\newcommand{\Ab}{{\bf Ab}}
\newcommand{\cring}{{\bf CRing}}
\DeclareMathOperator{\modules}{{\bf Mod}}
\newcommand{\catset}{{\bf Set}}
\newcommand{\cat}{\mathcal{C}}
\newcommand{\chains}{{\bf Ch}}
\newcommand{\homot}{{\bf Ho}}
\DeclareMathOperator{\objects}{Ob}
\newcommand{\gen}[1]{\left<#1\right>}
\DeclareMathOperator{\cone}{Cone}
\newcommand{\set}[1]{\left\{#1\right\}}
\newcommand{\setwith}[2]{\left\{#1:#2\right\}}
\DeclareMathOperator{\Ext}{Ext}
\DeclareMathOperator{\Nil}{Nil}
\DeclareMathOperator{\idem}{Idem}
\DeclareMathOperator{\rad}{Rad}
\DeclareMathOperator{\divisor}{div}
\DeclareMathOperator{\Pic}{Pic}
\DeclareMathOperator{\spec}{Spec}
\newcommand{\ideal}{\triangleleft}
\newcommand{\lang}{\mathscr{L}}
\DeclareMathOperator{\diagram}{Diag}
\DeclareMathOperator{\theory}{Th}
\DeclareMathOperator{\type}{tp}

\newenvironment{solution}{\begin{proof}[Solution]\renewcommand\qedsymbol{}}{\end{proof}}

\newtheorem{lemma}{Lemma}
\newtheorem{proposition}{Proposition}

\theoremstyle{definition}
\newtheorem{question}{Exercise}
\newtheorem{definition}{Definition}

\title{Homework Model Theory}
\author{Jonas van der Schaaf}
\date{}

\begin{document}
\maketitle

\begin{question}\,
    \begin{enumerate}[(1)]
        \item Give a complete description of the models of \(T\).

              \begin{solution}
                  A model of \(T\) is a set \(X\) with a bijection without
                  cycles \(f\) and a unary relation \(P\) with the desired
                  properties. In particular this means that we have a group
                  \(G\) of bijections of \(X\) generated by \(f\). This group
                  \(G\) has a single generator and \(f\) is acyclic so
                  \(G\cong\Z\). Because \(f\) is acyclic for any \(x\in X\) the
                  element \(f\in G\) works on the orbit \(Gx\) in the same way
                  \(1\) works on \(\Z\): \(x\mapsto 1+x\).

                  Therefore \(X\) consists of chains isomorphic to \(\Z\) where
                  for each orbit \(O\) there is either a smallest \(x\in O\)
                  with \(\neg P(x)\) and \(P(y)\) for all \(y=f^{n}(x)\) and
                  \(n>0\) or \(P(x)\) is true for all or no elements of \(O\).
                  In particular there must be infinitely many chains of the
                  first kind.
              \end{solution}

              Now we prove a lemma:

              \begin{lemma}
                  The theory \(T\) is model complete.

                  \begin{proof}
                      First of all, all atomic formulas are of the form
                      \(f^{n}(x)=f^{m}(y)\) with \(n,m\in\N\) or \(P(f^{n}(x))\)
                      with \(n\in\N\).

                      For any model \(M\models T\) we have that \(\setwith{a\in
                          M}{M\models P(f^{n}(x))}\) is clearly infinite but it
                      cannot be cofinite. To see this notice that there are
                      infinitely many \(a\in M\) such that \(M\models\neg
                      P(a)\) but \(M\models P(f(a))\). Then \(M\models\neg
                      P(f(b))\) for \(b=f^{-n}(a)\) and there are infinitely
                      many such \(a\) by assumption.

                      For the second kind of atomic formula
                      \(f^{n}(x)=f^{m}(y)\) if \(x\) or \(y\) is fixed then
                      clearly only one element of \(M\) can satisfy the formula.
                      In particular this element must be in the same orbit under
                      the action of \(f\).

                      Any quantifier free formula therefore consists of boolean
                      combinations of sets of this type.

                      Now we prove that any embedding of models of \(T\) is
                      existential. This is sufficient for model completeness.

                      Let \(M\subseteq N\) be two models of \(T\) and
                      \(\varphi(\bar{x},\bar{y})\) a quantifier free formula
                      where \(\bar{x}\) are \(k\) free variables and \(\bar{y}\)
                      are \(l\) free variables. Take any \(\bar{m}\in M^{k}\)
                      such that the set
                      \[
                          X=\setwith{\bar{n}\in N^{l}}{N\models\varphi(\bar{m},\bar{n})}
                      \]
                      is non-empty. We claim \(X\cap M^{l}\neq\varnothing\). If
                      \(X\) is cofinite, it must intersect \(M\) so we are done.
                      If it is finite, then by the above reasoning, all tuples
                      in \(X\) must have only elements from the orbits of
                      \(\bar{m}\). This means that they are also in \(M\). Now
                      if it is infinite but not cofinite, then it must also
                      intersect \(M\) by the above remarks. Therefore there is
                      an \(\bar{m'}\in M^{l}\) such that
                      \(M\models\varphi(\bar{m},\bar{m'})\) showing this
                      embedding is existential.

                      Therefore \(T\) is model complete.
                  \end{proof}
              \end{lemma}

        \item Show that \(T\) has a prime model.

              \begin{proof}
                  We define \(M=\bigsqcup_{i\in\N}\Z\) where we write
                  \(n_{i}\in\Z_{i}\): the \(i\)th copy of \(\Z\). Then we define
                  \(P(n_{i})\) iff \(n>0\). We claim this is a prime model.

                  Let \(N\) be any model of \(T\) and \(X\subseteq N\) the
                  infinite set elements \(x\) with \(N\not\models P(x)\) and
                  \(N\models P(f(x))\). Then there is an injective map
                  \(g:\setwith{0_{i}}{i\in\N}\to X\) because \(X\) is infinite.
                  This induces an embedding \(M\to N\): every \(n_{i}\in M\) is
                  \(f^{n}(0_{i})\) and so we send \(n_{i}\) to
                  \(f^{n}(g(0_{i}))\). This is an embedding because \(g\) is
                  injective and orbits of group actions are disjoint.

                  The embedding is automatically elementary by the above lemma.
              \end{proof}

        \item Show that \(T\) is complete.

              \begin{proof}
                  We have shown that \(T\) has a prime model \(M\). Therefore
                  \(T=\theory(M)\) which is complete.
              \end{proof}

        \item Determine all complete \(1\)-types of \(T\), and indicate which of
              those are isolated.

              \begin{solution}
                  I claim that the complete types of \(T\) are characterized as
                  follows:
                  \[
                      p_{n}(x)=\set{\neg P(f^{n-1}(x)),P(f^{n}(x))},
                  \]
                  for \(n>0\) and
                  \[
                      p_{-n}(x)=\setwith{\forall y:f^{n}(y)=x\to P(y)\wedge f^{n+1}(y)=x\to\neg P(y)}{}
                  \]
                  for \(n\leq 0\) and
                  \[
                      p_{\infty}(x)=\setwith{\neg P(f^{i}(x))}{i\in\omega},
                  \]
                  and finally
                  \[
                      q(x)=\setwith{\forall y:f^{n}(y)=x\to P(y)}{n\in\omega}.
                  \]
                  These correspond to elements \(x\) such that \(P\) is true
                  \(n\) steps after \(x\), \(n\) steps before \(x\), nowhere in
                  the orbit of \(x\), and in the entire orbit of \(x\). For any
                  element of any model, it must satisfy some of these partial
                  types. We show these extend uniquely to a complete type.

                  Notice that every element of the prime model
                  \(\bigsqcup_{i\in\N}\Z\) satisfies some \(p_{n}(x)\) for
                  \(n\in\Z\) and that any two that satisfy the same of these
                  partial types can be mapped onto each other by an automorphism
                  of the model by simply permuting the orbits.

                  First suppose there is an \(a\in N\models T\) such that
                  \(N\models p_{n}(a)\) for some \(n\in\Z\). Then we define an
                  embedding \(M\to N\) such that \(a\) is in the image of the
                  embedding. This will uniquely fix the type because this is an
                  elementary embedding. Because in the orbit of \(a\) there is a
                  first element where \(P\) is true, we can map a single orbit
                  of \(\bigsqcup_{i}\Z\) onto this orbit and map the other
                  orbits as described in part b. This will then fully fix the
                  type of \(a\). Therefore \(p_{n}\) must extend to a unique
                  complete type for all \(n\in\Z\).

                  For \(p_{\infty}\) we consider the model
                  \(\Z\sqcup\bigsqcup_{i\in\N}\Z\) where \(P\) is nowhere true
                  on the extra orbit. If \(M\) is a model of \(T\) and \(a\in
                  M\) satisfies \(p_{\infty}(a)\) then it is in an orbit where
                  \(P\) is nowhere true. We can map the extra orbit onto the
                  orbit of \(a\), and embed the prime model into \(M\) in any
                  way. This is an elementary embedding and so \(a\) satisfies
                  the type of any element of the extra orbit. Therefore
                  \(p_{\infty}\) extends to a unique complete type.

                  Similarly for the case of an element \(a\in M\models T\) such
                  that \(a\) is in an orbit where \(P\) is true everywhere. We
                  add an orbit to the prime model where \(P\) is true everywhere
                  and make an embedding as in the case of \(p_{\infty}\).
              \end{solution}
    \end{enumerate}
\end{question}

\begin{question}
    Let \(T=\theory((\R,<,(q)_{q\in\Q}))\) be the theory of the reals with
    order, with a constant for each rational number.

    \begin{enumerate}[(1)]
        \item Show that for each real number \(r\in\R\), it realises a unique
              \(1\)-type.

              \begin{proof}
                  Take \(a<b\in\R\). Then there is a \(q\in\Q\) such that
                  \(x<q<y\). Then \(x<q\in\type(a)\), and \(\neg
                  x<q\in\type(b)\), so they satisfy different types.
              \end{proof}

        \item Show that \((\Q,<,(q)_{q\in\Q})\models T\), and that it is the
              prime model of this theory.

              \begin{proof}
                  We prove that \(\Q\to\R\) is an elementary embedding in DLO
                  without these constants. This will automatically yield that it
                  is an elementary embedding here as well because all elements
                  of \(\Q\) are constants of the language.

                  We use the Tarski-Vaught test to show the embedding is
                  elementary. Take \(\varphi(x,\bar{y})\) to be a formula of DLO
                  and \(\bar{q}\in\Q\). If there is an \(r\) such that
                  \(\R\models\varphi(r,\bar{q})\), then by homework 1 there is
                  an automorphism of \(\R\) fixing \(\bar{q}\) and sending \(r\)
                  to a rational number \(q'\). This automorphism preserves truth
                  of \(\varphi\) and so there is an element \(q'\in\Q\) such
                  that \(\R\models\varphi(q',\bar{q})\). This proves
                  elementarity of the embedding and thus \(\Q\models T\).

                  The theory \(T\) is clearly nice. We therefore show \(\Q\) is
                  prime by showing it is atomic. Let \(\type_{M}(\bar{m})\) be
                  the non-isolated \(n\)-type associated to
                  \(\bar{m}=m_{1},\ldots,m_{n}\in M\). Then one of the \(m_{i}\)
                  must not be in \(\Q\): if they were then for every formula
                  \(\varphi(\bar{x})\) we would either have
                  \(T\models\varphi(\bar{m})\) or
                  \(T\models\neg\varphi(\bar{m})\) by completeness thus making
                  the type of \(\bar{m}\) isolated. This means that \(q\neq
                  m_{i}\in\type_{M}(\bar{m})\) for all \(q\in\Q\). This clearly
                  means that \(\Q\) omits this type.

                  This means that \(\Q\) is a prime model.
              \end{proof}

        \item Describe the isolated \(1\)-types and non-isolated \(1\)-types of
              this theory realized in \(\R\). Are there \(1\)-types which are
              not realized in \(\R\).

              \begin{solution}
                  As shown in part \(2\), the type of any \(r\in\R\setminus\Q\)
                  must be non-isolated, while the type of \(q\in\Q\) is
                  isolated.

                  There are non-isolated \(1\)-types not realised in \(\R\): add a
                  fresh constant \(c\) to \(T\) and add axioms \(0<c\) and
                  \(c<q\) for all \(q>0\). This is satisfiable in a model \(M\)
                  because it is trivially finitely satisfiable. Then the type of
                  \(c\) in \(M\) contains \(0<c\) and \(c<q\) for all
                  \(0<q\in\Q\). This type is obviously not realised in \(\R\).
              \end{solution}
    \end{enumerate}
\end{question}

\begin{question}
    Let \(T=DAG\) be the theory of non-trivial divisible torsion-free abelian
    groups.

    \begin{enumerate}[(1)]
        \item Show that \(T\) has a prime model.

              \begin{proof}
                  In my previous homework I proved that \(T\) is model complete.
                  Therefore if we find a model which embeds into all others,
                  this is a prime model. The theory \(DAG\) has exactly
                  non-trivial \(\Q\)-vector spaces as models with \(\Q\)-linear
                  maps as morphisms.

                  A \(1\)-dimensional vector space embeds into any other and so
                  \(\Q\) is a prime model of \(DAG\).
              \end{proof}

        \item Determine all complete \(n\)-types over this theory, and determine
              which of those are isolated and which are non-isolated.

              \begin{solution}
                  I claim that the complete \(n\)-types can be characterized as
                  follows. Take \(\bar{x}\in M\) and consider
                  \(\type_{M}(\bar{x})\). We can disregard any \(x_{i}=0\)
                  because being \(0\) uniquely determines \(x_{i}\) already.
                  Then \(\bar{x}\) has a maximal linearly independent subset
                  \(x_{i_{1}},\ldots,x_{i_{m}}\). If we have \((q_{i})_{1\leq
                          i\leq m}\in\Q\) and \(q_{i}\neq 0\) for some \(i\) then
                  \[
                      M\models\varphi_{q_{1},\ldots,q_{n}}=``q_{1}x_{i_{1}}+q_{m}x_{i_{m}}\neq 0''.
                  \]
                  Where \(q_{i}x_{i}\) is an abuse of notation: for
                  \(q_{i}=\nicefrac{a}{b}\) we actually write \(\exists
                  y:y^{b}=x_{i}\) and then replace \(q_{i}x_{i}\) with
                  \(y^{a}\).

                  Then for each \(j\) there are unique
                  \(q_{1},\ldots,q_{i_{m}}\) such that
                  \[
                      M\models\varphi_{j}=x_{j}=q_{1}x_{i_{1}}+\ldots+q_{m}x_{i_{m}}.
                  \]
                  We claim that the set of formulas
                  \[
                      p(\bar{x})=\setwith{\varphi_{q_{1},\ldots,q_{n}}}{(q_{i})_{i\leq m}\neq(0)_{i\leq m}}
                      \cup\setwith{q_{j}}{j\leq n}
                  \]
                  completely determines the type of \(\bar{x}\). We show this as
                  follows: consider the sub-vector space \(M'\subseteq M\)
                  generated by \(x_{1},\ldots,x_{n}\). This embeds into \(M\),
                  and the embedding is elementary because \(DAG\) is model
                  complete. Therefore the type of \(\bar{x}\) is the same as the
                  type in \(M'\).

                  Conversely is for any model \(N\) and \(\bar{y}\in N\) we have
                  \(N\models p(\bar{y})\) then \(M'\) embeds into \(N\) mapping
                  \(\bar{x}\) into \(y\) and this is an elementary embedding
                  because \(DAG\) is model complete. This is simply because
                  vector spaces are nice objects. Therefore \(\bar{y}\) must
                  have the same type as the original \(\bar{x}\in M\).

                  Therefore this describes all complete \(n\)-types: given \(n\)
                  elements \(\bar{x}\subset M\) we have a maximal linearly
                  independent subset of \(\bar{x}\) of some size \(k\leq n\),
                  and all other vectors are linear combinations of these.

                  The isolated types are exactly those satisfied in the
                  \(1\)-dimensional space: these are exactly the types
                  determined uniquely by: \(x_{j}=q_{ji}x_{i}\) for some
                  \(q_{ji}\in\Q\) or \(x_{j}=0\). This is because in a
                  \(1\)-dimensional space any two non-zero elements are linearly
                  dependent.
              \end{solution}
    \end{enumerate}
\end{question}
\end{document}