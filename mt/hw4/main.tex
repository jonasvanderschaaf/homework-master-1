\documentclass{article}

\usepackage[utf8]{inputenc}
\usepackage{enumerate}
\usepackage{amsthm, amssymb, mathtools, amsmath, bbm, mathrsfs, stmaryrd, xcolor}
\usepackage{nicefrac}
\usepackage[margin=1in]{geometry}
\usepackage[parfill]{parskip}
\usepackage[hidelinks]{hyperref}
\usepackage{float}
\usepackage{cleveref}
\usepackage{svg}
\usepackage{tikz-cd}

\renewcommand{\qedsymbol}{\raisebox{-0.5cm}{\includesvg[width=0.5cm]{../../qedboy.svg}}}


\newcommand{\N}{\mathbb{N}}
\newcommand{\Z}{\mathbb{Z}}
\newcommand{\NZ}{\mathbb{N}_{0}}
\newcommand{\Q}{\mathbb{Q}}
\newcommand{\R}{\mathbb{R}}
\newcommand{\C}{\mathbb{C}}
\newcommand{\A}{\mathbb{A}}
\newcommand{\proj}{\mathbb{P}}
\newcommand{\sheaf}{\mathcal{O}}
\newcommand{\FF}{\mathcal{F}}
\newcommand{\G}{\mathcal{G}}

\newcommand{\zproj}{Z_{\textnormal{proj}}}

\newcommand{\maxid}{\mathfrak{m}}
\newcommand{\primeid}{\mathfrak{p}}
\newcommand{\primeidd}{\mathfrak{q}}

\newcommand{\F}{\mathbb{F}}
\newcommand{\incl}{\imath}

\newcommand{\tuple}[2]{\left\langle#1\colon #2\right\rangle}

\DeclareMathOperator{\order}{order}
\DeclareMathOperator{\Id}{Id}
\DeclareMathOperator{\im}{im}
\DeclareMathOperator{\ggd}{ggd}
\DeclareMathOperator{\kgv}{kgv}
\DeclareMathOperator{\degree}{gr}
\DeclareMathOperator{\coker}{coker}
\DeclareMathOperator{\matrices}{Mat}

\DeclareMathOperator{\gl}{GL}

\DeclareMathOperator{\Aut}{Aut}
\DeclareMathOperator{\Hom}{Hom}
\DeclareMathOperator{\End}{End}
\DeclareMathOperator{\colim}{colim}
\newcommand{\isom}{\overset{\sim}{\longrightarrow}}

\newcommand{\schemes}{{\bf Sch}}
\newcommand{\aff}{{\bf Aff}}
\newcommand{\Grp}{{\bf Grp}}
\newcommand{\Ab}{{\bf Ab}}
\newcommand{\cring}{{\bf CRing}}
\DeclareMathOperator{\modules}{{\bf Mod}}
\newcommand{\catset}{{\bf Set}}
\newcommand{\cat}{\mathcal{C}}
\newcommand{\chains}{{\bf Ch}}
\newcommand{\homot}{{\bf Ho}}
\DeclareMathOperator{\objects}{Ob}
\newcommand{\gen}[1]{\left<#1\right>}
\DeclareMathOperator{\cone}{Cone}
\newcommand{\set}[1]{\left\{#1\right\}}
\newcommand{\setwith}[2]{\left\{#1:#2\right\}}
\DeclareMathOperator{\Ext}{Ext}
\DeclareMathOperator{\Nil}{Nil}
\DeclareMathOperator{\idem}{Idem}
\DeclareMathOperator{\rad}{Rad}
\DeclareMathOperator{\divisor}{div}
\DeclareMathOperator{\Pic}{Pic}
\DeclareMathOperator{\spec}{Spec}
\newcommand{\ideal}{\triangleleft}
\newcommand{\lang}{\mathscr{L}}
\DeclareMathOperator{\diagram}{Diag}
\DeclareMathOperator{\theory}{Th}

\newenvironment{solution}{\begin{proof}[Solution]\renewcommand\qedsymbol{}}{\end{proof}}

\newtheorem{lemma}{Lemma}
\newtheorem{proposition}{Proposition}

\theoremstyle{definition}
\newtheorem{question}{Exercise}
\newtheorem{definition}{Definition}

\title{Homework Model Theory}
\author{Jonas van der Schaaf}
\date{}

\begin{document}
\maketitle

\begin{question}
    Let \(\lang=\set{s}\) be a language with unary function symbol \(s\). Let
    \(\N\) be the model of the natural numbers with the successor function.

    \begin{enumerate}[(1)]
        \item Show that the theory \(T\) of \((\N,s)\) does not have quantifier
              elimination.

              \begin{proof}
                  We show there are models of \(T\) \(M,N\) which have a common
                  substructure \(A\) and that there is a formula \(\varphi(x)\)
                  and \(a\in A\) such that \(M\models\varphi(a)\) but
                  \(N\nvDash\varphi(a)\). If \(\varphi\) were equivalent to a
                  quantifier free formula, then this could not occur because
                  quantifier free formulas are preserved under substructure and
                  extension of a structure. Therefore \(T\) does not have
                  quantifier elimination.

                  Take \(M,N,A=(\N,s)\) let \(f:A\to M\) be the identity
                  embedding and \(g:A\to N\) be the embedding uniquely
                  determined by \(g(0)=1\in N\). Take \(a=0\in A\) then
                  trivially
                  \[
                      M\nvDash\exists x:s(x)=a
                  \]
                  but
                  \[
                      N\models\exists x:s(x)=a.
                  \]
              \end{proof}

        \item Consider now the language \(\lang=\set{s,<,0}\). Show that the
              theory of \((\N,s,<,0)\) has quantifier elimination.

              \begin{proof}

              \end{proof}
    \end{enumerate}
\end{question}

\begin{question}
    Let \(T_{\forall}\) denote the set of universal consequences of \(T\).

    \begin{enumerate}[(1)]
        \item Show that if \(T\) is model complete, then whenever \(M\models
              T\), \(N\models T_{\forall}\) and \(f:M\to N\) is an embedding,
              then \(f\) is existentially closed.

              \begin{proof}
                  Let \(M\models T\), \(N\models T_{\forall}\) and \(f:M\to N\)
                  be an embedding. Define
                  \[
                      U=\setwith{\exists x\psi(x,\bar{m})}{\textnormal{\(\psi(x,\bar{m})\) and \(\bar{m}\) are elements of \(M\)}}.
                  \]
                  We claim \(T\cup U\) is satisfiable in a model \(M'\) into
                  which \(M\) can be embedded. This is because for any
                  \(\psi\in\diagram(M)\) the formula \(\exists x\psi\) is
                  equivalent to \(\psi\) because \(x\) does not occur in it
                  and so \(T\cup U\models\diagram(M)\).

                  Suppose it is not satisfiable, then there are finitely many
                  \(\varphi_{1},\ldots,\varphi_{n}\in U\) with
                  \(\varphi_{i}=\exists x\psi_{i}\) such that
                  \(T\models\neg\varphi_{i}\) for all \(1\leq i\leq n\) and
                  therefore
                  \[
                      T\models\bigvee_{1\leq i\leq n}\forall x\neg\psi_{i}.
                  \]
                  We know this formula implies \(\theta=\forall x\bigvee_{1\leq
                      i\leq n}\psi_{i}(x)\), which is universal so
                  \(N\models\theta\). This is a contradiction with the
                  assumption that \(\psi_{i}(x)\) is true for some \(x\in
                  N\).

                  Now we have a model \(M'\) of \(T\cup U\) into which \(M\) can
                  be embedded, so this embedding is elementary. This means
                  clearly that \(M\models U\) as well showing the original
                  embedding \(M\to N\) was existential.
              \end{proof}

        \item Show that if \(T\) is a theory such that
              \begin{itemize}
                  \item \(T\) is model complete;
                  \item \(T_{\forall}\) has the amalgamation property;
              \end{itemize}
              then \(T\) has quantifier elimination.

              \begin{proof}
                  Let \(A\) be an arbitrary substructure of two models
                  \(M,N\models T\) by maps \(f:A\to M\) and \(g:A\to N\). Then
                  by \L{}os-Tarski \(A\models T_{\forall}\) and so there is an
                  amalgamation \(F\) with existential embeddings \(h_{0}:M\to
                  F,h_{1}:N\to F\).

                  Then we have the following equivalence by existentiality and
                  because existential formulas are upward closed for all
                  \(\bar{a}\in A\) and quantifier free \(\varphi(\bar{x})\):
                  \begin{align*}
                      M\models\exists x\varphi(x,f(\bar{a})) & \Leftrightarrow F\models\exists x\varphi(x,(h_{0}\circ f)(\bar{a})) \\
                                                             & \Leftrightarrow F\models\exists x\varphi(x,(h_{1}\circ g)(\bar{a})) \\
                                                             & \Leftrightarrow N\models\exists x\varphi(x,g(\bar{a})).
                  \end{align*}
                  This is sufficient quantifier elimination as we have seen in
                  the lecture.
              \end{proof}
    \end{enumerate}
\end{question}

\begin{question}
    Let \(DAG\) be the theory of torsion-free divisible abelian groups.

    \begin{enumerate}[(1)]
        \item Prove that whenever \(T\) is a theory with algebraically prime
              models, such that whenever \(M,N\models T\) and \(M\leq N\), then
              \(M\) is existentially closed in \(N\), then \(T\) has quantifier
              elimination.

              \begin{proof}
                  Let \(M,N\) be two models of \(T\), \(A\) a common
                  substructure of \(M,N\) and \(F\) the structure given by
                  having algebraically prime models because \(A\models
                  T_{\forall}\) by \L{}os-Tarski. Then we have the following
                  diagram of embeddings
                  \[
                      \begin{tikzcd}[ampersand replacement=\&]
                          M \& F \& N \\
                          \\
                          \& A
                          \arrow["f", from=3-2, to=1-1, tail]
                          \arrow["g"', from=3-2, to=1-3, tail]
                          \arrow["\imath", from=3-2, to=1-2, tail]
                          \arrow["h_{0}"', from=1-2, to=1-1, dashed, tail]
                          \arrow["h_{1}", from=1-2, to=1-3, dashed, tail]
                      \end{tikzcd}
                  \]
                  where \(h_{0}\) and \(h_{1}\) are existential by assumption.

                  Then for \(\bar{a}\in A\) we have the following equivalence
                  because existential sentences are upwards closed and \(h_{0}\)
                  and \(h_{1}\) are existential:
                  \begin{align*}
                      M\models\exists x\varphi(x,f(\bar{a})) & \Leftrightarrow M\models\exists x\varphi(x,h_{0}\circ\imath(\bar{a})) \\
                                                             & \Leftrightarrow F\models\exists x\varphi(x,\imath(\bar{a}))           \\
                                                             & \Leftrightarrow N\models\exists x\varphi(x,h_{1}\circ\imath(\bar{a})) \\
                                                             & \Leftrightarrow N\models\exists x\varphi(x,g(\bar{a})).
                  \end{align*}
                  This is sufficient for quantifier elimination.
              \end{proof}

        \item Use the construction of the divisible hull to show that \(DAG\)
              has algebraically prime models.

              \begin{proof}
                  First we claim that \(T_{\forall}\) is the theory of torsion
                  free abelian groups. Every substructure of a torsion-free
                  divisible abelian group is a torsion-free abelian group.
                  Conversely using the divisible hull every torsion-free abelian
                  group can be embedded into a torsion-free divisible abelian
                  group. Therefore if \(T'\) is the theory of torsion free
                  abelian groups then \(T'\models T_{\forall}\) and
                  \(T_{\forall}\models T'\) so the two are equivalent.

                  Let \(G\) be a torsion-free abelian group with divisible hull
                  \(D(G)\). Then clearly we have an embedding \(f:G\to D(G)\)
                  given by \(f(x)=[(x,1)]\).

                  Now suppose there are a torsion-free abelian division group
                  \(G'\) and an embedding \(g:G\to G'\). We define \(h:D(G)\to
                  G'\) as \(h([(x,a)])=a^{-1}g(x)\). Clearly his has the
                  property that \(hf=g\) if it is well-defined. We show this
                  now: Suppose \((x,a)\sim (y,b)\), then \(ay=bx\) and so
                  \(ag(y)=bg(x)\). Because \(G'\) is a torsion-free abelian
                  division ring this means that
                  \(\nicefrac{g(x)}{a}=\nicefrac{g(y)}{b}\) showing this map is
                  well-defined. We show it is injective by showing its kernel is
                  trivial. Suppose \(h([(x,a)])=0\), then
                  \(\nicefrac{h([(x,1)])}{a}=0\) so
                  \(a\nicefrac{h([(x,1)])}{a}=g(x)=0\). This shows that
                  \(x\in\ker(g)=0\) because \(g\) is injective.
              \end{proof}

        \item Show that every embedding \(M\to N\) with \(M,N\models DAG\) is
              existential.

              \begin{proof}
                  We know torsion-free divisible abelian groups are
                  \(\kappa\)-categorical for \(\kappa>\aleph_{0}\). From this we
                  can easily see that any \(f:G\to G'\) with \(|G|\neq|G'|\) is
                  in fact elementary. We show this in two parts: first that any
                  embedding into a group of higher cardinality is elementary and
                  then for the same cardinality.

                  By Löwenheim-Skolem we know that there is an elementary
                  embedding \(g:G\to G'\), we claim there is an isomorphism
                  \(h:G'\to G'\) such that \(hg=f\). Isomorphisms are elementary
                  embeddings and composition of elementary maps is still
                  elementary. Let \(B\) be a basis of \(G\) as \(\Q\)-vector
                  space, then \(f[B]\) is a basis of \(f[G]\) because \(f\) is
                  injective. We can extend \(f[B]\) to a basis \(B'\) of \(G'\)
                  and similarly \(g[B]\) to a basis \(B''\). Then bijective map
                  \(B''\to B'\) such that \(g(b)\in g[B]\) is mapped onto
                  \(f(b)\) is an isomorphism of \(G'\) which has \(hg=f\)
                  because a vector space morphism is uniquely determined by its
                  action on a basis.

                  Now let \(G,G'\) be two countable torsion-free divisible
                  abelian groups and \(f:G\to G'\) an embedding. Then there is
                  an embedding \(g:G'\to G''\) with \(|G'|<|G''|\). Then both
                  \(g\) and \(gf\) are elementary by the previous remark. This
                  must mean that \(f\) is elementary as well:
                  \begin{align*}
                      G\models\varphi(\bar{a}) & \Leftrightarrow G''\models\varphi(gf(\bar{a})) \\
                                               & \Leftrightarrow G'\models\varphi(f(\bar{a})).
                  \end{align*}
                  This means all embeddings are in fact elementary.
              \end{proof}

        \item Conclude that \(DAG\) has quantifier elimination.

              \begin{proof}
                  We have now shown that \(T\) is a theory with algebraically
                  prime models and only existential embeddings. By part 1 it
                  therefore has quantifier elimination.
              \end{proof}
    \end{enumerate}
\end{question}
\end{document}