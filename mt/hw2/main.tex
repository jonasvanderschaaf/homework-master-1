\documentclass{article}

\usepackage[utf8]{inputenc}
\usepackage{enumerate}
\usepackage{amsthm, amssymb, mathtools, amsmath, bbm, mathrsfs, stmaryrd, xcolor}
\usepackage{nicefrac}
\usepackage[margin=1in]{geometry}
\usepackage[parfill]{parskip}
\usepackage[hidelinks]{hyperref}
\usepackage{float}
\usepackage{cleveref}
\usepackage{svg}
\usepackage{tikz-cd}

\usepackage{quiver}

\renewcommand{\qedsymbol}{\raisebox{-0.5cm}{\includesvg[width=0.5cm]{../../qedboy.svg}}}


\newcommand{\N}{\mathbb{N}}
\newcommand{\Z}{\mathbb{Z}}
\newcommand{\NZ}{\mathbb{N}_{0}}
\newcommand{\Q}{\mathbb{Q}}
\newcommand{\R}{\mathbb{R}}
\newcommand{\C}{\mathbb{C}}
\newcommand{\A}{\mathbb{A}}
\newcommand{\proj}{\mathbb{P}}
\newcommand{\sheaf}{\mathcal{O}}
\newcommand{\FF}{\mathcal{F}}
\newcommand{\G}{\mathcal{G}}

\newcommand{\zproj}{Z_{\textnormal{proj}}}

\newcommand{\maxid}{\mathfrak{m}}
\newcommand{\primeid}{\mathfrak{p}}
\newcommand{\primeidd}{\mathfrak{q}}

\newcommand{\F}{\mathbb{F}}
\newcommand{\incl}{\imath}

\newcommand{\tuple}[2]{\left\langle#1\colon #2\right\rangle}

\DeclareMathOperator{\order}{order}
\DeclareMathOperator{\Id}{Id}
\DeclareMathOperator{\im}{im}
\DeclareMathOperator{\ggd}{ggd}
\DeclareMathOperator{\kgv}{kgv}
\DeclareMathOperator{\degree}{gr}
\DeclareMathOperator{\coker}{coker}
\DeclareMathOperator{\matrices}{Mat}

\DeclareMathOperator{\gl}{GL}

\DeclareMathOperator{\Aut}{Aut}
\DeclareMathOperator{\Hom}{Hom}
\DeclareMathOperator{\End}{End}
\DeclareMathOperator{\colim}{colim}
\newcommand{\isom}{\overset{\sim}{\longrightarrow}}

\newcommand{\schemes}{{\bf Sch}}
\newcommand{\aff}{{\bf Aff}}
\newcommand{\Grp}{{\bf Grp}}
\newcommand{\Ab}{{\bf Ab}}
\newcommand{\cring}{{\bf CRing}}
\DeclareMathOperator{\modules}{{\bf Mod}}
\newcommand{\catset}{{\bf Set}}
\newcommand{\cat}{\mathcal{C}}
\newcommand{\chains}{{\bf Ch}}
\newcommand{\homot}{{\bf Ho}}
\DeclareMathOperator{\objects}{Ob}
\newcommand{\gen}[1]{\left<#1\right>}
\DeclareMathOperator{\cone}{Cone}
\newcommand{\set}[1]{\left\{#1\right\}}
\newcommand{\setwith}[2]{\left\{#1:#2\right\}}
\DeclareMathOperator{\Ext}{Ext}
\DeclareMathOperator{\Nil}{Nil}
\DeclareMathOperator{\idem}{Idem}
\DeclareMathOperator{\rad}{Rad}
\DeclareMathOperator{\divisor}{div}
\DeclareMathOperator{\Pic}{Pic}
\DeclareMathOperator{\spec}{Spec}
\newcommand{\ideal}{\triangleleft}

\newenvironment{solution}{\begin{proof}[Solution]\renewcommand\qedsymbol{}}{\end{proof}}

\newtheorem{lemma}{Lemma}
\newtheorem{proposition}{Proposition}

\theoremstyle{definition}
\newtheorem{question}{Exercise}
\newtheorem{definition}{Definition}

\title{Homework Model Theory}
\author{Jonas van der Schaaf}
\date{}

\begin{document}
\maketitle

\begin{question}
    Let \(L\) be a language with a single binary relation \(E\), and let \(T\)
    be the theory of an equivalence relation with infinitely many infinite
    cells, and without infinite cells.

    \begin{enumerate}[(a)]
        \item Write axioms for \(T\).

              \begin{solution}
                  We first take the axioms of an equivalence relation
                  \begin{gather*}
                      \forall x:E(x,x),\\
                      \forall x,y:E(x,y)\to E(y,x),\\
                      \forall x,y,z:E(x,y)\wedge E(y,z)\to E(x,z).
                  \end{gather*}
                  On top of that we define for all \(n\in\omega\)
                  \[
                      \varphi_{n}=\exists x_{1},\ldots,x_{n}:\neg E(x_{1},\ldots,x_{n})
                  \]
                  and
                  \[
                      \psi_{n}=\forall x\exists y_{1},\ldots,y_{n}:
                      \bigwedge_{1\leq i\neq j\leq n}y_{i}\neq y_{j}
                      \wedge\bigwedge_{1\leq i\leq n}E(x,y_{i}).
                  \]
                  Taking \(T=\bigcup_{n\in\omega}\set{\varphi_{n},\psi_{n}}\)
                  together with the equivalence relation axioms above defines
                  the desired theory.
              \end{solution}

        \item How many models of \(T\) are there up to isomorphism of
              cardinality \(\aleph_{0}\) and \(\aleph_{1}\)?

              \begin{solution}
                  First we concern ourselves with the models of cardinality
                  \(\aleph_{0}\). Each infinite equivalence class must have at
                  least \(\aleph_{0}\) elements and there are at least
                  \(\aleph_{0}\) many of them. There cannot be more equivalence
                  classes or larger equivalence classes because we assumed a
                  cardinality of \(\aleph_{0}\). This means that for any model
                  of \(T\) of countable cardinality there is a bijection between
                  the set of equivalence classes and a bijection between each
                  equivalence class, so the two models are isomorphic as the
                  relation \(E\) is the only structure preserved by an
                  isomorphism. This means there is exactly one model op to
                  isomorphism.

                  For models of cardinality \(\aleph_{1}\) the picture becomes a
                  bit more complex. If there are \(\kappa\) equivalence classes
                  of cardinality \(\aleph_{0}\) and \(\lambda\) of cardinality
                  \(\aleph_{1}\) the model has cardinality
                  \(\kappa\cdot\aleph_{0}+\lambda\cdot\aleph_{1}\) with
                  \(\kappa,\lambda\leq\aleph_{1}\) and \(\lambda\geq1\) or
                  \(\kappa=\aleph_{1}\). We compute the cardinality of the set
                  of such tuples \((\kappa,\lambda)\).

                  If \(\kappa=\aleph_{1}\) then \(\lambda\) can be anything
                  giving \(\aleph_{0}\) many of such tuples. If
                  \(\kappa<\aleph_{1}\), then \(1\leq\lambda\leq\aleph_{1}\) so
                  there are \(\aleph_{0}\cdot\aleph_{0}=\aleph_{0}\) many tuples
                  of this shape.

                  This means that in total there are
                  \(\aleph_{0}=\aleph_{0}+\aleph_{0}\) many models of \(T\) of
                  cardinality \(\aleph_{1}\) up to isomorphism.
              \end{solution}

        \item Show that \(T\cup DLO\) is not \(\omega\)-categorical.

              \begin{proof}
                  Consider the dense linear order \(\Q\) and define the
                  equivalence relations with the following equivalence classes:
                  \[
                      E=\setwith{(i,i+1]}{i\in\Z}
                  \]
                  and
                  \[
                      E'=\setwith{[i,i+1)}{i\in\Z}.
                  \]
                  Taking \(\mathbb{M}=(\Q,<,E)\) and \(\mathbb{M}'=(\Q,<,E')\)
                  we see easily that
                  \[
                      \mathbb{M}\models\forall x\exists y:E(x,y)\wedge\forall z:E(x,z)\to z=y\vee z<y
                  \]
                  because each \((i,i+1]\) has a supremum in the class. The
                  model \(\mathbb{M}'\) does not have this property so these
                  models are not isomorphic. Both have cardinality
                  \(\aleph_{0}\) so the theory is not
                  \(\omega\)-categorical\footnote{In fact this shows the theory
                      is not complete so it is not categorical for any cardinality.}.
              \end{proof}
    \end{enumerate}
\end{question}

\begin{question}
    Let \(L=\set{<}\) be the language of linear orders.

    \begin{enumerate}[(a)]
        \item Let \(T\) be an \(L\)-theory extending the theory of linear
              orders, and assume that \(T\) has an infinite model. Show that
              \(T\) has a model extending \((\Q,<)\).

              \begin{proof}
                  Let \(\Delta\) be the diagram of \((\Q,<)\). We claim that
                  \(T\cup\Delta\) is a satisfiable set of formulas. This means
                  there is a structure \((\mathbb{L},<)\) satisfying \(T\) into
                  which \(\Q\) can be embedded, showing the desired statement.
                  We write \(c_{q}\) for a constant in the language of
                  \(\Delta\) corresponding to \(q\in\Q\) and fix an infinite
                  model \((\mathbb{M},<)\) of \(T\).

                  By compactness it is sufficient to demonstrate any finite
                  \(\Gamma_{0}\subseteq T\cup\Delta\) is satisfiable. Let
                  \(\Gamma_{0}\) be such a finite set and
                  \(c_{q_{1}},\ldots,c_{q_{n}}\) be the constants occurring in
                  \(\Gamma_{0}\) labelled such that \(q_{i}<q_{j}\) for \(i<j\).
                  Pick any set \(M=\set{x_{1},\ldots,x_{n}}\subseteq\mathbb{M}\)
                  of cardinality \(n\) such that \(x_{i}<x_{j}\) iff \(i<j\) and
                  interpret the constant \(c_{i}=x_{i}\). It is clear that
                  \((\mathbb{M},<,x_{1},\ldots,x_{n})\models c_{i}<c_{j}\) iff
                  \(i<j\) and so \(\mathbb{M}\) satisfies all \emph{atomic}
                  formulas of \(\Delta\) containing the constants
                  \(c_{q_{1}},\ldots,c_{q_{n}}\). A simple induction argument
                  shows that this means it satisfies all quantifier-free
                  formulas with these constants, so in particular
                  \(\mathbb{M}\models\Gamma_{0}\).

                  Any arbitrary finite subset \(\Gamma_{0}\subseteq
                  T\cup\Delta\) is therefore satisfiable and so the whole set is
                  as well by compactness completing the proof.
              \end{proof}

        \item Show that the class \(\mathcal{K}\) of well-orders is not
              elementary.

              \begin{proof}
                  Suppose \(\mathcal{K}\) is elementary and has theory \(T\).
                  Then \(T\) extends the theory of linear orders so there is a
                  model \((\mathbb{L},<)\) of \(T\) extending \((\Q,<)\). This
                  means \(\mathbb{L}\) has a subset \(X\) isomorphic to \(\Q\).
                  The order on \(\Q\) has no least element so \(X\) does not
                  either, meaning \(\mathbb{L}\) is not well-ordered by \(<\)
                  contradicting the assumption. This means that \(\mathcal{K}\)
                  is not elementary.
              \end{proof}
    \end{enumerate}
\end{question}

\begin{question}
    Use Ramsey's theorem to derive this\footnote{Technically speaking not this
        actual statement, but the one stated in the question.} statement.

    \begin{proof}
        We consider the language with one binary relation \(L=\set{R}\). We let
        \(R\) be a symmetric relation: we consider the theory \(T=\set{\forall
            x,y:R(x,y)\leftrightarrow R(y,x)}\).

        Suppose the statement is not true: there is a \(k\) such that no finite
        set has a homogeneous set of size \(k\). This means that the formula
        \[
            \varphi=\neg\exists x_{1},\ldots,x_{k}:\bigwedge_{1\leq i\neq j\leq k}x_{i}\neq x_{j}
            \wedge\bigwedge_{1\leq i\neq j,n\neq m\leq k}R(x_{i},x_{j})\leftrightarrow R(x_{n},x_{m})
        \]
        expressing that there are no \(k\) elements of a model which form a
        homogeneous subset with respect to the colouring induced by the relation
        \(R\) is satisfiable in arbitrarily large finite models of \(T\). This
        means \(\varphi\) is satisfiable in an infinite model \((\mathbb{M},R)\)
        of \(T\) by compactness. Add infinitely many constants
        \((c_{i})_{i\in\N}\) and define \(T'=\setwith{c_{i}\neq c_{j}}{i\neq
            j}\): then \(T\cup T'\) is clearly finitely satisfiable so \(T\cup T'\)
        is as well, giving an infinite model of \(T\).

        By Ramsey's theorem \(\mathbb{M}\) has an infinite homogeneous subset,
        so \(\mathbb{M}\nVdash\varphi\): take any \(k\) elements in this
        homogeneous set as a counterexample.

        This is clearly a contradiction and therefore the desired statement is
        true for all \(k\in\omega\).
    \end{proof}
\end{question}
\end{document}