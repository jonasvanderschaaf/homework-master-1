\documentclass{article}

\usepackage[utf8]{inputenc}
\usepackage{enumerate}
\usepackage{amsthm, amssymb, mathtools, amsmath, bbm, mathrsfs, stmaryrd, xcolor}
\usepackage{nicefrac}
\usepackage[margin=1in]{geometry}
\usepackage[parfill]{parskip}
\usepackage[hidelinks]{hyperref}
\usepackage{float}
\usepackage{cleveref}
\usepackage{svg}
\usepackage{tikz-cd}

\renewcommand{\qedsymbol}{\raisebox{-0.5cm}{\includesvg[width=0.5cm]{../../qedboy.svg}}}


\newcommand{\N}{\mathbb{N}}
\newcommand{\Z}{\mathbb{Z}}
\newcommand{\NZ}{\mathbb{N}_{0}}
\newcommand{\Q}{\mathbb{Q}}
\newcommand{\R}{\mathbb{R}}
\newcommand{\C}{\mathbb{C}}
\newcommand{\A}{\mathbb{A}}
\newcommand{\proj}{\mathbb{P}}
\newcommand{\sheaf}{\mathcal{O}}
\newcommand{\FF}{\mathcal{F}}
\newcommand{\G}{\mathcal{G}}

\newcommand{\zproj}{Z_{\textnormal{proj}}}

\newcommand{\maxid}{\mathfrak{m}}
\newcommand{\primeid}{\mathfrak{p}}
\newcommand{\primeidd}{\mathfrak{q}}

\newcommand{\F}{\mathbb{F}}
\newcommand{\incl}{\imath}

\newcommand{\tuple}[2]{\left\langle#1\colon #2\right\rangle}

\DeclareMathOperator{\order}{order}
\DeclareMathOperator{\Id}{Id}
\DeclareMathOperator{\im}{im}
\DeclareMathOperator{\ggd}{ggd}
\DeclareMathOperator{\kgv}{kgv}
\DeclareMathOperator{\degree}{gr}
\DeclareMathOperator{\coker}{coker}
\DeclareMathOperator{\matrices}{Mat}

\DeclareMathOperator{\gl}{GL}

\DeclareMathOperator{\Aut}{Aut}
\DeclareMathOperator{\Hom}{Hom}
\DeclareMathOperator{\End}{End}
\DeclareMathOperator{\colim}{colim}
\newcommand{\isom}{\overset{\sim}{\longrightarrow}}

\newcommand{\schemes}{{\bf Sch}}
\newcommand{\aff}{{\bf Aff}}
\newcommand{\Grp}{{\bf Grp}}
\newcommand{\Ab}{{\bf Ab}}
\newcommand{\cring}{{\bf CRing}}
\DeclareMathOperator{\modules}{{\bf Mod}}
\newcommand{\catset}{{\bf Set}}
\newcommand{\cat}{\mathcal{C}}
\newcommand{\chains}{{\bf Ch}}
\newcommand{\homot}{{\bf Ho}}
\DeclareMathOperator{\objects}{Ob}
\newcommand{\gen}[1]{\left<#1\right>}
\DeclareMathOperator{\cone}{Cone}
\newcommand{\set}[1]{\left\{#1\right\}}
\newcommand{\setwith}[2]{\left\{#1:#2\right\}}
\DeclareMathOperator{\Ext}{Ext}
\DeclareMathOperator{\Nil}{Nil}
\DeclareMathOperator{\idem}{Idem}
\DeclareMathOperator{\rad}{Rad}
\DeclareMathOperator{\divisor}{div}
\DeclareMathOperator{\Pic}{Pic}
\DeclareMathOperator{\spec}{Spec}
\newcommand{\ideal}{\triangleleft}
\newcommand{\lang}{\mathscr{L}}

\newenvironment{solution}{\begin{proof}[Solution]\renewcommand\qedsymbol{}}{\end{proof}}

\newtheorem{lemma}{Lemma}
\newtheorem{proposition}{Proposition}

\theoremstyle{definition}
\newtheorem{question}{Exercise}
\newtheorem{definition}{Definition}

\title{Homework Model Theory}
\author{Jonas van der Schaaf}
\date{}

\begin{document}
\maketitle

\begin{question}
    Let \(T\) be an \(\mathscr{L}\)-theory. Show that:
    \begin{enumerate}[(a)]
        \item If \(T\) is a Skolem theory, then each formula
              \(\varphi(\bar{x})\in\lang\) there is a quantifier free
              \(\varphi^{*}(\bar{x})\in\lang\).

              \begin{proof}
                  We prove this by induction on the structure of \(\varphi\).

                  If \(\varphi\) is atomic \(\varphi^{*}=\varphi\) gives a
                  quantifier free formula trivially equivalent to \(\varphi\).

                  For any binary connective \(\circ\) and formula
                  \(\varphi=\psi\circ\chi\) we define
                  \(\varphi^{*}=\psi^{*}\circ\chi^{*}\) which is trivially
                  equivalent. We do a similar thing for negation:
                  \((\neg\psi)^{*}=\neg\psi^{*}\).

                  If \(\varphi(\bar{x})=\exists y\psi(\bar{x},y)\) we know that there
                  is a term \(t\) such that
                  \[
                      T\models\forall\bar{x}\exists y\psi(\bar{x},y)\to\psi(\bar{x},t(\bar{x}))
                  \]
                  because \(T\) is a Skolem theory. Then in particular \(\exists
                  y\psi(\bar{x},y)\) is equivalent to
                  \(\psi(\bar{x},t(\bar{x}))\), one implication is given by the
                  fact \(T\) is a Skolem theory and the other implication is
                  trivially true. This formula is now equivalent to
                  \(\psi^{*}(\bar{x},t(\bar{x}))\). Therefore we define
                  \(\varphi^{*}=\psi^{*}(\bar{x},t(\bar{x}))\).

                  If \(\varphi(\bar{x})=\forall y:\psi(\bar{x},y)\), then
                  \(\varphi\equiv\neg\exists y\neg\psi(\bar{x},y)\). We have
                  shown in the previous induction steps that
                  \(\varphi\) has an equivalent \(\varphi^{*}\) as well.
              \end{proof}

        \item If \(T\) is a Skolem theory, then every substructure is an
              elementary substructure.

              \begin{proof}
                  We prove this by showing any substructure satisfies Vaught's
                  test. Let \(M\subseteq N\) be a substructure and
                  \(\varphi(y)\in\lang_{M}\). Then there is a term \(t\) such
                  that \(\exists y\varphi(y)\) is equivalent to
                  \(\varphi^{*}(t)\) by the previous proof\footnote{We just
                      treat the constants from \(M\) as free variables and proceed
                      as in part \(a\).}. Then \(t\) only has constants
                  corresponding to elements of \(M\) so \(t\) corresponds to an
                  element \(m\) of \(M\) by simply evaluating the term \(t\) in
                  \(M\). Then \(N\models\exists y\varphi(y)\) iff
                  \(N\models\varphi^{*}(t)\) which is in turn equivalent to
                  \(N\models\varphi(m)\) because \(N\) agrees with \(M\) on the
                  new constants.
              \end{proof}

        \item Show that given an \(\lang\)-theory \(T\), there is an
              \(\lang^{+}\supseteq\lang\) and \(T^{+}\supseteq T\) an
              \(\lang^{+}\)-theory which is a Skolemisation of \(T\).

              \begin{proof}
                  We define the language \(\lang^{+}\) and theory \(T^{+}\)
                  iteratively. We define \(\lang_{0}=\lang\) and \(T_{0}=T\).

                  Given a language \(\lang_{n}\) and theory \(T_{n}\), for each
                  \(\varphi(\bar{x})\) we take a fresh symbol \(f_{\varphi}\)
                  and define it to be an \(m\)-ary function where \(m\) is the
                  size of \(\bar{x}\). Define
                  \[
                      \lang_{n+1}=\lang_{n}\cup\setwith{f_{\varphi}}{\textnormal{\(\varphi(\bar{x},y)\) is a formula of \(\lang\)}}
                  \]
                  and
                  \[
                      T_{n+1}=T_{n}\cup
                      \setwith{\forall\bar{x}\left(\exists y\varphi(\bar{x},y)\right)\to\varphi(\bar{x},f_{\varphi}(\bar{x}))}
                      {\textnormal{\(\varphi(\bar{x},y)\) is a formula of \(\lang\)}}.
                  \]
                  We then define \(\lang^{+}=\bigcup_{n}\lang_{n}\) and
                  \(T^{+}=\bigcup_{n}T_{n}\). For any formula \(\varphi\) of
                  \(\lang^{+}\) it is a formula of some \(\lang_{n}\) and
                  therefore the term \(f_{\varphi}(\bar{x})\) trivially has the
                  property that
                  \[
                      T^{+}\supseteq T_{n}\models\forall\bar{x}\left(\exists y\varphi(\bar{x},y)\right)\to\varphi(\bar{x},f_{\varphi}(\bar{x})).
                  \]

                  Now we prove every model of \(T\) can be extended to a model
                  of \(T^{+}\) by showing it can be extended to a model of each
                  \(T_{n}\) such that each extension agrees with the previous
                  one. This is immediately clear for \(T_{0}=T\).

                  Now suppose the statement is true for \(T_{n}\), we consider
                  \(T_{n+1}\) and a model \(M\) of \(T_{n}\). For an arbitrary
                  formula \(\varphi\) in the language \(\lang_{n}\) take
                  \(\bar{a}\in M\). Suppose there is a \(b\in M\) such that
                  \(M\models\varphi(\bar{a},b)\), then define
                  \(f_{\varphi}(\bar{a})=b\) else pick \(f_{\varphi}(\bar{a})\)
                  arbitrary in \(M\). Then \(M\models\forall\bar{x}\left(\exists
                  y\varphi(\bar{x},y)\right)\to\varphi(\bar{x},f_{\varphi}(\bar{x}))\).

                  This means that inductively we can define all
                  \(f_{\varphi}\in\lang^{+}\setminus\lang\) extending a model
                  \(M\) of \(T\) to a model of \(T^{+}\).
              \end{proof}
    \end{enumerate}
\end{question}

\begin{question}
    Let \(M\) be a model of \(PA\).

    \begin{enumerate}[(a)]
        \item Prove that if \(\varphi(\bar{x})\) is a bounded
              formula, and \(N\) is an initial segment, then for each
              \(\bar{a}\in N\)
              \[
                  M\models\varphi(\bar{a})\Longleftrightarrow N\models\varphi(\bar{a}).
              \]

              \begin{proof}
                  We prove the statement by induction on formulas. The base case
                  of the induction are atomic formulas. Because \(N\) is a
                  substructure of \(M\) the statement is true for these.

                  The boolean connective case is trivial, leaving the case of
                  quantifiers.

                  Suppose the statement holds for \(\psi(y,\bar{x})\), take any
                  term \(t(\bar{x})\) and consider \(\varphi(\bar{x})=\exists
                  y<t(\bar{x}):\psi(y,\bar{x})\). Take \(\bar{a}\in N\), then
                  \(t(\bar{a})\) are interpreted the same in both \(N\) and
                  \(M\) because \(N\) is a substructure. Then for all
                  \(b<t(\bar{a})\) we have
                  \[
                      N\models\psi(b,\bar{a})\Longleftrightarrow M\models\psi(b,\bar{a}).
                  \]
                  Because \(N\) is an initial segment there is such a \(b\) in
                  \(N\) iff it exists in \(M\) so
                  \[
                      N\models\exists y<t(\bar{a}):\psi(y,\bar{a})\Longleftrightarrow M\models\exists y<t(\bar{a}):\psi(y,\bar{a}).
                  \]

                  For the universal quantifier case: \(\forall
                  y<t(\bar{x}):\psi(y,\bar{x})\) notice that it is shorthand for
                  the following formula: \(\forall
                  y<t(\bar{x})\to\psi(y,\bar{x})\). This is equivalent to
                  \(\neg\exists y<t(\bar{x})\wedge\neg\psi(y\bar{x})\) which is
                  the full version of the shorthand \(\neg\exists
                  y<t(\bar{x}):\neg\psi(y,\bar{x})\). Using the above induction
                  steps we easily see that the induction hypothesis is true here
                  as well.
              \end{proof}

        \item Show that \(\Pi_{1}\)-formulas are downward preserved and
              \(\Sigma_{1}\)-formulas are upward preserved.

              \begin{proof}
                  We prove this by induction on the amount of quantifiers.

                  First for the \(\Pi_{1}\) case. Our induction hypothesis is as
                  follows: if \(\psi\) is a downward preserved formula,
                  \(\forall x\psi\) is as well. A \(\Delta_{0}\) formula is
                  downward preserved by the previous part.

                  Let \(N\) be an initial segment of \(M\) and
                  \(\psi(\bar{x},y)\) be a downward preserved formula. Take
                  \(a\in M\) and \(\bar{b}\in M\), then if
                  \(M\models\psi(\bar{b},a)\) we also have
                  \(N\models\psi(\bar{b},a)\) by the induction hypothesis. This
                  means that if \(M\models\forall y\psi(\bar{b},y)\) then
                  \(N\models\forall y\psi(\bar{b},y)\) so \(\forall
                  y\psi(\bar{x},y)\) is also downward closed.

                  Now for the \(\Sigma_{1}\) case. Our induction is similar to
                  before: if \(\psi\) is upward closed then \(\exists y\psi\) is
                  as well. is as well. The base case for \(\Delta_{0}\) formulas
                  follows once again from the previous part.

                  If \(\psi(\bar{x},y)\) is upward closed take \(\bar{b}\in N\)
                  and suppose there is an \(a\in N\) such that
                  \(N\models\psi(\bar{b},a)\). Then \(M\models\psi(\bar{b},a)\)
                  as well by the induction hypothesis. Therefore if
                  \(N\models\exists y\psi(\bar{b},y)\) then \(M\models\exists
                  y\psi(\bar{b},y)\).
              \end{proof}

        \item Show that every initial segment of \(M\) satisfies the induction
              schema for bounded induction.

              \begin{proof}
                  The axiom schema of induction for bounded formulas is the
                  formula
                  \[
                      \forall y(\forall z<y\varphi(\bar{x},z)\to\varphi(\bar{x},y))\to\forall y\varphi(\bar{x},y).
                  \]
                  for all bounded formulas \(\varphi(\bar{x},y)\). This
                  principle for a \(\Delta_{0}\) formula \(\varphi\) is
                  clearly\footnote{According to the proof theory syllabus.}
                  equivalent to the least element principle
                  \[
                      \exists y\neg\varphi(\bar{x},y)\to\exists y\neg\varphi(\bar{x},y)\wedge\forall z<y\varphi(\bar{x},z).
                  \]

                  Now we prove the principle of bounded induction for initial
                  segments. Suppose \(M\) is a model of \(PA\) and \(N\) is an
                  initial segment of \(M\) and bounded induction fails for a
                  formula \(\varphi(\bar{x},y)\). Then the antecedent of the
                  implication is true but the consequent is false: there is are
                  \(n,\bar{a}\in N\subseteq M\) such that
                  \(N\nVDash\varphi(\bar{a},n)\). Because \(\varphi\) is a
                  \(\Delta_{0}\) formula this means that
                  \(M\nVDash\varphi(\bar{a},n)\). By the least element principle
                  there is a least \(n'\leq n\) such that
                  \(M\nVDash\varphi(\bar{a},n')\) and
                  \(M\models\varphi(\bar{a},b)\) for all \(b<n'\), \(N\) is an
                  initial segment so \(n'\in N\) and therefore
                  \(N\nVDash\varphi(\bar{a},n')\) and
                  \(N\models\varphi(\bar{a},b)\) for all \(b<n'\). We assumed
                  the antecedent of the bounded induction schema was true so
                  this must mean that \(N\models\varphi(\bar{a},n')\) which is a
                  contradiction. Therefore \(N\) must satisfy the axiom schema
                  of bounded induction.
              \end{proof}

        \item Show that \(I\Delta_{0}\) does not prove \(\forall x\exists
              y\exp(x,y)\).

              \begin{proof}
                  We consider the set of formulas
                  \[
                      \Gamma=PA\cup\setwith{\varphi_{n}}{n\in\omega}
                  \]
                  with \(\varphi_{n}=\exists y:\exp(c,y)\wedge c^{k}<y\) where
                  \(c\) is a fresh constant added to the language. We first show
                  it is satisfiable using compactness. Consider \(\N\) as a
                  model of \(PA\) and consider a finite subset
                  \(\Gamma_{0}\subseteq\Gamma\). Let \(n\in\omega\) be the
                  highest number such that \(\varphi_{n}\in\Gamma_{0}\). Then
                  because \(\lim_{c\to\infty}\frac{c^{n}}{2^{c}}\to 0\) there
                  must be an \(x\in\N\) such that \(x^{n}<2^{x}\). We have
                  \(x^{m}<x^{n}\) for all \(m<n\) so interpreting \(c=x\) gives
                  that \(\N\models\Gamma_{0}\) which is therefore satisfiable.
                  This means that \(\Gamma\) is satisfiable as well in some
                  model \(M\).

                  Now consider the initial segment \(M\) generated by
                  \(\setwith{c^{n}}{n\in\omega}\). We claim \(2^{c}\notin M\).
                  If \(\forall x\exists y\exp(x,y)\) were \(I\Delta_{0}\)
                  provable then \(2^{c}\) would have been in \(M\). This is not
                  the case so it cannot be provable.

                  The initial segment generated by
                  \(\setwith{c^{n}}{n\in\omega}\) is the closure of this set
                  under \(S\) and taking all smaller elements. Suppose
                  \(2^{c}<S^{n}(c^{k})\) for some \(n,k\in\omega\), then it is
                  easy to see that \(c^{i+k}>S^{n}(c^{k})\) for some
                  \(i\in\omega\) because \(c\) must in particular be larger than
                  \(0\) and \(1\) so \(c^{i+k}\) is a strictly increasing
                  function in \(i\). This contradicts the assumption that
                  \(2^{c}>c^{k}\) for all \(k\in\omega\). Therefore
                  \(2^{c}\notin M\) so \(\forall x\exists y\exp(x,y)\) is not
                  provable in \(I\Delta_{0}\).
              \end{proof}
    \end{enumerate}
\end{question}

\begin{question}
    Let \(\lang\) be the signature of graphs.

    \begin{enumerate}[(a)]
        \item Write axioms for \(RG\) as first order theory.

              \begin{solution}
                  We define
                  \[
                      \varphi_{n}=\forall x_{0}\ldots\forall x_{n}\forall y_{0}\ldots\forall y_{n}
                      \bigwedge_{i\neq j\leq n}(x_{i}\neq x_{j}\wedge y_{i}\neq y_{j})
                      \wedge\bigwedge_{i,j\leq n}(x_{i}\neq y_{j})
                      \wedge\exists z\bigwedge_{i\leq n}Rzx_{i}\wedge\neg Rzy_{i}.
                  \]
                  This exactly states the property of random graphs for two
                  finite disjoint sets of points \(X,Y\) with
                  \(\left|X\right|=\left|Y\right|=n\).

                  Therefore the set
                  \[
                      RG=\setwith{\varphi_{n}}{n\in\omega}\cup\set{\forall x\forall y xRy\leftrightarrow yRx,\forall x\neg xRx}
                  \]
                  is a theory defining random graphs because edges in graphs are
                  symmetric and there are no self-edges.
              \end{solution}

        \item Show that the theory \(RG\) is \(\omega\)-categorical.

              \begin{proof}
                  We prove this using a back-and-forth argument. Let \(G,G'\) be
                  two random graphs of cardinality \(\aleph_{0}\). Fix
                  enumerations \(\gen{a_{n}}{n\in\omega}\) and
                  \(\gen{b_{n}}{n\in\omega}\) of \(G\) and \(G'\) respectively
                  and define \(G_{0}=\set{a_{0}}\) and \(G'_{0}=\set{b_{0}}\)
                  the obviously isomorphic one point graphs.

                  We inductively define \(G_{n+1}\) and \(G'_{n+1}\) iteratively
                  as isomorphic graphs adding new points in such a way that
                  every point will at some point be included in a graph
                  \(G_{n}\).

                  Suppose we have \(G_{n}\) and \(G_{n}'\) isomorphic subgraphs
                  of \(G\) and \(G'\) with witnessing isomorphism
                  \(f_{n}:G_{n}\to G_{n}'\). If \(n=2k\) is even, take
                  \(a_{k}\in G\). If \(a_{k}\in G_{n}\) then we do nothing. If
                  not, we claim there is a point \(v\in G'\) such that \(vRy\)
                  iff \(a_{n}Rx\) for all \(x\in G_{n}\) and \(y=f_{n}(x)\in
                  G_{n}'\). Define the two disjoint and finite sets
                  \(X=G_{n}\cap R[a_{n}]\) and \(Y=G_{n}\setminus X\). Now let
                  \(X'=f_{n}[X]\) and \(Y'=f_{n}[Y]\). If \(X'\) and \(Y'\) are
                  sets of different size, extend the smaller one with points
                  from \(G\setminus G_{n}'\). Then there is a \(v\in G\setminus
                  G_{n}'\) such that \(X'\subseteq R[v]\) and \(Y'\cap
                  R[v]=\varnothing\). Therefore
                  \(G_{n+1}=G_{n}\cup\set{a_{k}}\subseteq G\) and
                  \(G_{n+1}=G_{n}'\cup\set{v}\subseteq G'\) agree on all edges
                  and are therefore isomorphic through the isomorphism
                  \(f_{n+1}=f_{n}\cup\set{(a_{k},v)}\).

                  For the odd steps \(n=2k+1\) we do the same thing for
                  \(b_{k}\) switching the roles of \(G_{n}\) and \(G_{n}'\)
                  adding \((v,b_{k})\) to \(f_{n}\).

                  Now \(\bigcup_{n}G_{n}\cong\bigcup_{n}G_{n}'\) by the
                  isomorphism \(f=\bigcup_{n}f_{n}\). This is first of all a
                  total function because for \(a_{n}\in G\) in step \(2n\) the
                  vertex \(a_{n}\) is added to the domain of \(f\).

                  This map is injective because each \(f_{n}\) is injective and
                  is surjective because for each \(b_{n}\in G'\) we have that
                  \(f_{2n+1}\) has \(b_{n}\) in its image.
              \end{proof}
    \end{enumerate}
\end{question}
\end{document}