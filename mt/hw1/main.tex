\documentclass{article}

\usepackage[utf8]{inputenc}
\usepackage{enumerate}
\usepackage{amsthm, amssymb, mathtools, amsmath, bbm, mathrsfs, stmaryrd, xcolor}
\usepackage{nicefrac}
\usepackage[margin=1in]{geometry}
\usepackage[parfill]{parskip}
\usepackage[hidelinks]{hyperref}
\usepackage{float}
\usepackage{cleveref}
\usepackage{svg}
\usepackage{tikz-cd}

\usepackage{quiver}

\renewcommand{\qedsymbol}{\raisebox{-0.5cm}{\includesvg[width=0.5cm]{../../qedboy.svg}}}


\newcommand{\N}{\mathbb{N}}
\newcommand{\Z}{\mathbb{Z}}
\newcommand{\NZ}{\mathbb{N}_{0}}
\newcommand{\Q}{\mathbb{Q}}
\newcommand{\R}{\mathbb{R}}
\newcommand{\C}{\mathbb{C}}
\newcommand{\A}{\mathbb{A}}
\newcommand{\proj}{\mathbb{P}}
\newcommand{\sheaf}{\mathcal{O}}
\newcommand{\FF}{\mathcal{F}}
\newcommand{\G}{\mathcal{G}}

\newcommand{\zproj}{Z_{\textnormal{proj}}}

\newcommand{\maxid}{\mathfrak{m}}
\newcommand{\primeid}{\mathfrak{p}}
\newcommand{\primeidd}{\mathfrak{q}}

\newcommand{\F}{\mathbb{F}}
\newcommand{\incl}{\imath}

\newcommand{\tuple}[2]{\left\langle#1\colon #2\right\rangle}

\DeclareMathOperator{\order}{order}
\DeclareMathOperator{\Id}{Id}
\DeclareMathOperator{\im}{im}
\DeclareMathOperator{\ggd}{ggd}
\DeclareMathOperator{\kgv}{kgv}
\DeclareMathOperator{\degree}{gr}
\DeclareMathOperator{\coker}{coker}
\DeclareMathOperator{\matrices}{Mat}

\DeclareMathOperator{\gl}{GL}

\DeclareMathOperator{\Aut}{Aut}
\DeclareMathOperator{\Hom}{Hom}
\DeclareMathOperator{\End}{End}
\DeclareMathOperator{\colim}{colim}
\newcommand{\isom}{\overset{\sim}{\longrightarrow}}

\newcommand{\schemes}{{\bf Sch}}
\newcommand{\aff}{{\bf Aff}}
\newcommand{\Grp}{{\bf Grp}}
\newcommand{\Ab}{{\bf Ab}}
\newcommand{\cring}{{\bf CRing}}
\DeclareMathOperator{\modules}{{\bf Mod}}
\newcommand{\catset}{{\bf Set}}
\newcommand{\cat}{\mathcal{C}}
\newcommand{\chains}{{\bf Ch}}
\newcommand{\homot}{{\bf Ho}}
\DeclareMathOperator{\objects}{Ob}
\newcommand{\gen}[1]{\left<#1\right>}
\DeclareMathOperator{\cone}{Cone}
\newcommand{\set}[1]{\left\{#1\right\}}
\newcommand{\setwith}[2]{\left\{#1:#2\right\}}
\DeclareMathOperator{\Ext}{Ext}
\DeclareMathOperator{\Nil}{Nil}
\DeclareMathOperator{\idem}{Idem}
\DeclareMathOperator{\rad}{Rad}
\DeclareMathOperator{\divisor}{div}
\DeclareMathOperator{\Pic}{Pic}
\DeclareMathOperator{\spec}{Spec}
\newcommand{\ideal}{\triangleleft}

\newenvironment{solution}{\begin{proof}[Solution]\renewcommand\qedsymbol{}}{\end{proof}}

\newtheorem{lemma}{Lemma}
\newtheorem{proposition}{Proposition}

\theoremstyle{definition}
\newtheorem{question}{Exercise}
\newtheorem{definition}{Definition}

\title{Homework Model Theory}
\author{Jonas van der Schaaf}
\date{}

\begin{document}
\maketitle

\begin{question}
    Show the following:
    \begin{enumerate}[(1)]
        \item Show that \((\R,+,-,0)\) can be seen as a vector space over the
              field \(\Q\).

              \begin{proof}
                  We show this using two lemmas, note that a module is a vector
                  space over an arbitrary ring:

                  \begin{lemma}[Free module]
                      Let \(R\) be any ring, then \(R\) is both a left and
                      right\footnote{The left and right is important in case the
                          ring \(R\) is non-commutative.} \(R\)-module.

                      \begin{proof}
                          We know that \((R,+,-,0)\) is an abelian group for any
                          ring \(R\) by the axioms of a ring. The natural
                          multiplication of the ring is distributive over
                          addition both from the left and right side and is
                          associative. This shows all listed axioms.

                          Other necessary properties of modules are that
                          multiplication with \(1\) is the identity and
                          multiplication with \(0\) gives \(0\). These are both
                          properties of rings.
                      \end{proof}
                  \end{lemma}

                  \begin{lemma}[Restriction of scalars]
                      Let \(R,S\) be rings and \(M\) a left \(S\)-module. A
                      morphism of rings \(f:R\to S\) induces a left \(R\)-module
                      structure on \(M\).

                      \begin{proof}
                          For \(r\in R,m\in M\) we define \(rm=f(r)m\). It is
                          clear that \(M\) still has the structure of an abelian
                          group.

                          Because \(f\) preserves \(0,1\in R\), we know that
                          \(1m=f(1)m=m\) and similarly for \(0\) we have
                          \(0m=0\).

                          Because ring morphisms are additive we have
                          \[
                              (r+r')m=f(r+r')m=(f(r)+f(r'))m=rm+r'm.
                          \]
                          The other distributivity follows immediately from the
                          definition of the multiplication.

                          Associativity holds because
                          \(f(rr')m=(f(r)f(r'))m=f(r)(f(r')m)\) by
                          multiplicativity of ring morphisms.

                          This shows \(M\) is an \(R\)-module.
                      \end{proof}
                  \end{lemma}

                  Now \(\R\) is a \(\Q\)-module (i.e. vector space) by
                  restriction of scalars through the unique morphism \(\Q\to
                  \R\).
              \end{proof}

        \item Use Zorn's lemma to show that every vector space has a basis.

              \begin{proof}
                  We show that any module \(M\) over a division ring \(K\) has a
                  basis. This is a strictly stronger condition than every vector
                  space having a basis.

                  We show that the poset of sets of linearly independent subsets
                  of \(M\) has the property that every chain has an upper bound
                  and is non-empty. By Zorn's lemma this means it has a maximal
                  element. We will show this is a basis for \(M\).

                  Let \((\set{x_{\alpha,\beta}}_{\beta})_{\alpha}\) be a chain
                  of linearly independent elements of \(M\). Then
                  \(\bigcup_{\alpha}\set{x_{\alpha,\beta}}_{\beta}\) is a
                  linearly independent set. If it is not then there are finitely
                  many
                  \(x_{\alpha_{1},\beta_{1}},\ldots,x_{\alpha_{n},\beta_{n}}\)
                  in the union which are not linearly independent. Because there
                  are finitely many elements there is a maximal \(\alpha\) such
                  that \(\alpha=\alpha_{i}\) for some \(i\) and then
                  \[
                      x_{\alpha_{j},\beta_{j}}\in\bigcup_{\alpha\leq\alpha_{i}}\set{x_{\alpha,\beta}}_{\beta}=\set{x_{\alpha,\beta}}_{\beta}
                  \]
                  for all \(j\). This is impossible because then
                  \(\set{x_{\alpha,\beta}}_{\beta}\) is not a linearly
                  independent set of elements of \(M\).

                  This shows chains have upper bounds so there is a maximal set
                  of linearly independent elements. Let
                  \(B=\set{x_{\beta}}_{\beta}\) be such a set and suppose it
                  does not span \(M\). Take \(m\in M\) disjoint from the span of
                  \(B\). Then \(m\) is linearly independent from \(B\): if not
                  then \(am+\sum_{\beta}a_{\beta}x_{\beta}=0\) for some
                  \(a,a_{\beta}\in K\). This means that
                  \(m=\sum_{\beta}a^{-1}a_{\beta}x_{\beta}\) so \(m\) was in the
                  span of \(B\). This means that \(B\cup\set{m}\) is linearly
                  independent: a contradiction with the fact that \(B\) was
                  maximal. This means that \(m\) is in the span of \(B\) so
                  \(B\) is a basis.
              \end{proof}

        \item Let \(V\) be a vector space over a field \(K\), and let \(B\) be a
              basis. Show that if \(f:B\to B\) is a permutation, then \(f\)
              induces an automorphism \(\bar{f}:V\to V\).

              \begin{proof}
                  We define \(\overline{f}\) as follows on a basis
                  \(B=\set{x_{\alpha}}_{\alpha}\) for collections
                  \((a_{i})_{i}\) with finite support:
                  \[
                      \overline{f}\left(\sum_{i}a_{i}x_{i}\right)=\sum_{i}a_{i}f(x_{i}).
                  \]
                  Because \(B\) is a basis every element of \(V\) has a unique
                  representation as such a sum and therefore this
                  \(\overline{f}\) is well defined. This is a linear map because
                  for two collections \((a_{i})_{i},(b_{i})_{i}\) with finite
                  support we see that
                  \begin{align*}
                      \overline{f}\left(\sum_{i}a_{i}x_{i}+\sum_{i}b_{i}x_{i}\right) & =\overline{f}\left(\sum_{i}(a_{i}+b_{i})x_{i}\right)                                     \\
                                                                                     & =\sum_{i}(a_{i}+b_{i})f(x_{i})                                                           \\
                                                                                     & =\sum_{i}a_{i}f(x_{i})+\sum_{i}b_{i}f(x_{i})                                             \\
                                                                                     & =\overline{f}\left(\sum_{i}a_{i}x_{i}\right)+\overline{f}\left(\sum_{i}b_{i}x_{i}\right)
                  \end{align*}
                  showing that \(\overline{f}\) is linear.

                  The inverse permutation \(f^{-1}\) induces a map
                  \(\overline{f^{-1}}:V\to V\) which has that both
                  \(\overline{f^{-1}}\circ\overline{f}\) and
                  \(\overline{f}\circ\overline{f^{-1}}\) are the identity on
                  \(B\) and therefore on the entirety of \(V\) because \(V\) is
                  generated by \(B\). This shows that \(\overline{f}\) has an
                  inverse and is an isomorphism.
              \end{proof}

        \item Use the previous parts to show that the order \(\leq\) is not
              definable in \((\R,+,-,0)\).

              \begin{proof}
                  Suppose there is a formula \(\varphi\) such that
                  \(\R\models\varphi(x,y,a_{1},\ldots,a_{n})\) is a formula iff
                  \(x<y\). We know there is a basis \(\set{x_{\beta}}_{\beta}\)
                  of \(\R\) as vector space over \(\Q\). Therefore
                  \(a_{i}=\sum_{j}b_{i,j}x_{\beta_{i,j}}\) for some finite amount
                  of basis vectors.

                  Take a non-identity permutation \(f\) swapping two
                  \(x_{\beta},x_{\beta'}\) taken from the complement of
                  \(\set{x_{\beta_{i,j}}}_{i,j}\). This is clearly not an order
                  preserving function and we have
                  \begin{align*}
                      \R\models\varphi(x,y,a_{1},\ldots,a_{n}) & \Leftrightarrow\R\models\varphi(\overline{f}(x),\overline{f}(y),\overline{f}(a_{1}),\ldots,\overline{f}(a_{n})) \\
                                                               & \Leftrightarrow\R\models\varphi(\overline{f}(x),\overline{f}(y),a_{1},\ldots,a_{n}).
                  \end{align*}
                  This is because
                  \begin{align*}
                      \overline{f}(a_{i}) & =\overline{f}\left(\sum_{j}b_{i,j}x_{\beta_{i,j}}\right) \\
                                          & =\sum_{j}b_{i,j}f(x_{\beta_{i,j}})                       \\
                                          & =\sum_{j}b_{i,j}x_{\beta_{i,j}}                          \\
                                          & =a_{i}.
                  \end{align*}
                  Because \(f\) is not order preserving \(\varphi\) can
                  therefore express the order \(<\).
              \end{proof}
    \end{enumerate}
\end{question}

\begin{question}
    Let \(R\) be an integral domain.

    \begin{enumerate}[(a)]
        \item Show that there exists a unique field with a long universal
              property.

              \begin{proof}
                  This is actually a much more general concept called
                  localisations. Unlike my proof above defining free modules and
                  restriction of scalars this is actually more work so I'd
                  rather not prove the full universal properties of those so I
                  will just define the fraction field.

                  Given an integral domain \(R\), consider \(R\times
                  (R\setminus\set{0})\) with an ``addition'' given by
                  \((a,b)+(c,d)=(ad+bc,bd)\) and ``multiplication'' given by
                  \((a,b)(c,d)=(ac,bd)\). Then define an equivalence relation
                  \((a,b)\sim(a',b')\) iff \(ab'=a'b\). Symmetry and reflexivity
                  are trivial. For transitivity suppose
                  \((a,b)\sim(c,d)\sim(e,f)\). Then \(d(af-eb)=bcf-bcf=0\). We
                  know \(d\neq 0\) so \(af-eb=0\) and \((a,b)\sim(e,f)\).

                  The quotient of the product by this relation is compatible
                  with addition and multiplication: if \((a,b)\sim(a',b')\) and
                  \((c,d)\sim (c',d')\) then \((a,b)+(c,d)=(ad+bc,bd)\),
                  \((a',b')+(c',d')=(a'd'+b'c',b'd')\) and
                  \begin{align*}
                      (ad+bc)b'd' & =adb'd'+bcb'd'  \\
                                  & =a'bdd'+bc'b'd  \\
                                  & =(a'd'+b'c')bd.
                  \end{align*}
                  Compatibility with multiplication is readily verified as well.

                  We have that \((0,1)\) is the additive unit element and
                  \((1,1)\) the multiplicative one and the quotient of \(R\times
                  R\setminus\set{0}\) satisfies all properties of a ring:
                  multiplication and addition are associative and distributive.
                  We will suggestively write \((a,b)=\frac{a}{b}\) from now on.
                  It is readily checked that for \(\frac{a}{b}\neq0\) that
                  \(\frac{b}{a}\) is the inverse:
                  \((a,b)(b,a)=(ab,ab)\sim(1,1)\). This shows that any non-zero
                  element is a unit. The ring is commutative because any
                  integral domain \(R\) is.

                  We also have an injective \(\varepsilon:R\to Q(R)\) with
                  \(\varepsilon(a)=\frac{a}{1}\).

                  Now for the universal property let \(f:R\to K\) be any
                  injective map into a field \(K\). Define
                  \(\tilde{f}(\nicefrac{a}{b})=f(a)f(b)^{-1}\). We know
                  \(f(b)\neq 0\) because \(f\) is injective so it is a unit.
                  This map is clearly multiplicative and a bit of algebraic
                  manipulation gives that this map is also additive and
                  preserves \(0\) and \(1\).

                  This map \(\tilde{f}\) has
                  \begin{align*}
                      \tilde{f}\circ\varepsilon(a) & =\tilde{f}\left(\frac{a}{1}\right) \\
                                                   & =f(a)f(1)^{-1}                     \\
                                                   & =f(a).
                  \end{align*}
                  So \(\varepsilon\circ\tilde{f}=f\).

                  The map \(\tilde{f}\) is unique with the property that
                  \(\tilde{f}(\nicefrac{a}{1})=f(a)\) because this property
                  fixes the map \(\tilde{f}\): a ring morphism must preserve
                  inverses and thus \(\tilde{f}\) is unique.

                  We show \(Q(R)\) is unique up to isomorphism. Suppose there is
                  another field \(K\) with this property with injective
                  \(\varepsilon':R\to K\). Then there are maps
                  \(\tilde{\varepsilon'}:Q(R)\to K\) and
                  \(\tilde{\varepsilon}:K\to Q(R)\) such that the universal
                  property holds for both. We have that
                  \(\tilde{\varepsilon}\circ\tilde{\varepsilon'}\circ\varepsilon:R\to
                  Q(R)\) induces a map \(\varphi:Q(R)\to Q(R)\) such that
                  \(\varphi\circ\varepsilon=\varepsilon\). This map is unique
                  and both \(\Id\) and
                  \(\tilde{\varepsilon}\circ\tilde{\varepsilon'}\) have this
                  property so both are equal. By the same reasoning the opposite
                  composition is also the identity so \(\tilde{\varepsilon}\) is
                  an isomorphism showing \(K\cong Q(R)\). This shows \(R\) is
                  unique up to isomorphism.
              \end{proof}

        \item Describe explicitly the interpretation of a fraction field
              \(Q(R)\) in the ring \(R\).

              \begin{solution}
                  Both structures are structures with the ring signature
                  \((+,\cdot,-,0,1)\). We give a \(2\)-dimensional
                  interpretation with soft inequality as described in the notes.

                  Now we define the following interpretation exactly as given
                  above in the previous part:
                  \begin{align*}
                      D_{I}((x_{1},x_{2}))                              & \Leftrightarrow \neg x_{2}=0,                                       \\
                      (x_{1},x_{2})=_{I}(y_{1},y_{2})                   & \Leftrightarrow x_{1}\cdot y_{2}=x_{2}\cdot y_{1},                  \\
                      (x_{1},x_{2})+_{I}(y_{1},y_{2})=(z_{1},z_{2})     & \Leftrightarrow z_{1}=x_{1}y_{2}+y_{1}x_{2}\wedge z_{2}=x_{2}y_{2}, \\
                      (x_{1},x_{2})\cdot_{I}(y_{1},y_{2})=(z_{1},z_{2}) & \Leftrightarrow z_{1}=x_{1}y_{1}\wedge z_{2}=x_{2}y_{2},            \\
                      0_{I}((x_{1},x_{2}))                              & \Leftrightarrow x_{1}=0,                                            \\
                      1_{I}((x_{1},x_{2}))                              & \Leftrightarrow x_{1}=x_{2}.
                  \end{align*}
                  This is exactly the definition of the fraction field as given
                  in the previous part.
              \end{solution}
    \end{enumerate}
\end{question}

\begin{question}
    Consider the linear order of the rationals \((\Q,<)\).

    \begin{enumerate}[(a)]
        \item Suppose that \(a,b\in\Q\) are elements \(a<b\). Show that if
              \(x,y\in(a,b)\) then there is an automorphism \(f:\Q\to\Q\) such that
              \(f(x)=y\) and for each \(z\notin(a,b)\), \(f(z)=z\).

              \begin{proof}
                  We define
                  \[
                      f(z)=\begin{cases}
                          \frac{y-a}{x-a}(z-a)+a & a\leq z<x     \\
                          \frac{y-b}{x-b}(z-b)+b & x\leq z\leq b \\
                          z                      & z\notin(a,b)
                      \end{cases}
                  \]
                  We have \(f(a)=a,f(b)=b,f(x)=y\), \(f\) is piecewise linear
                  monotone strictly increasing and continuous on the entirety of
                  \(\Q\) meaning it is an order preserving bijection and
                  therefore an automorphism with the desired property.
              \end{proof}

        \item Use part 1 to show that a subset \(S\subseteq\Q\) is definable
              with paramaters if and only if it is a finite union of the
              following forms, for \(a,b\in\Q\):
              \begin{enumerate}[1.]
                  \item \((a,b)\)
                  \item \(\set{a}\)
                  \item \((-\infty,a)\)
                  \item \((a,\infty)\)
              \end{enumerate}

              \begin{proof}
                  Each of these is definable separately:
                  \begin{enumerate}[1.]
                      \item \((a,b)=\setwith{z\in\Q}{a<z<b}\)
                      \item \(\set{a}=\setwith{z\in\Q}{z=a}\)
                      \item \((-\infty,a)=\setwith{z\in\Q}{z<a}\)
                      \item \((a,\infty)=\setwith{z\in\Q}{a<z}\)
                  \end{enumerate}
                  so a finite union is a finite conjunction of these kinds of
                  formulas and is therefore definable.

                  Conversely suppose \(S\subseteq\Q\) is definable by
                  \(a_{1}<\ldots<a_{n}\in\Q\) and a formula
                  \(\varphi(x,a_{1},\ldots,a_{n})\). We show that for all
                  \(a_{i}<x,y<a_{i+1}\), \(x\in S\) iff \(y\in S\).

                  If \(x\in S\), then
                  \(\Q\models\varphi(x,a_{1},\ldots,a_{n})\). By part a there is
                  an automorphism of \(\Q\) with \(f(x)=y\) and
                  \(f|_{\Q\setminus(a_{i},a_{i+1})}=\Id\). This means that
                  \(\Q\models\varphi(f(x),f(a_{1}),\ldots,f(a_{n}))=\varphi(y,a_{1},\ldots,a_{n})\)
                  as well so \(y\in S\). By symmetry the other implication holds
                  as well.

                  Similarly if \(x,y<a_{1}\) then \(x\in S\) iff \(y\in S\). If
                  \(x\in S\) take any \(b<x,y\) then there is an automorphism
                  \(f\) of \(\Q\) with \(f(x)=y\) and \(f|_{(b,a_{1})}=\Id\).
                  This means by the same reasoning as before that because
                  \(\Q\models\varphi(x,a_{1},\ldots,a_{n})\) we also have
                  \(\Q\models\varphi(f(x),f(a_{1}),\ldots,f(a_{n}))=\varphi(y,a_{1},\ldots,a_{n})\)
                  so \(y\in S\). The opposite implication holds by symmetry.

                  By similar reasoning for \(x,y>a_{n}\) we have \(x\in S\) iff
                  \(y\in S\).

                  This means that \(S\) is a union of a subset of
                  \[
                      \set{(-\infty,a_{1}),(a_{n},\infty)}\cup\setwith{(a_{i},a_{i+1})}{1\leq i<n}\cup\setwith{a_{i}}{1\leq i\leq n}.
                  \]
                  This means it is of the desired form because the above set is
                  finite.
              \end{proof}
    \end{enumerate}
\end{question}
\end{document}