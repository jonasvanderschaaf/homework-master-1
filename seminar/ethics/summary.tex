\documentclass{article}

\usepackage[margin=1in]{geometry}

\usepackage[parfill]{parskip}

\title{Ethics Seminar}
\author{Jonas van der Schaaf}
\date{May 12th 2023}

\begin{document}
\maketitle

\section*{The four cases}
\subsection*{Mathematical testimony in court}
Multiple people have been convicted in court based on fallacious mathematical
testimony. People such as Lucia de Berk and Sally Clark have been sent to jail
because of a misunderstanding of statistics combined with malicious intent from
prosecutors and police.

The main question with regard to this case is in which manner mathematicians
should testify in court. What are rules and guidelines one should follow when
giving mathematical testimony? Our main conclusions were that mathematical
testimony should not just be given by one person: mathematics can be very
authoritative and a mistake by a single person can heavily sway opinions. In the
context of statistical arguments (and other arguments as well, but specifically
statistical ones), one should be mindful of the assumptions of thos arguments.

\subsection*{Ethical use of mathematical skills}
Many companies and government agencies hire mathematicians to solve complex
issues. Institutes such as the NSA employ many mathematicians, but working for
intelligence agencies comes with many ethical drawbacks. Should mathematicians
contribute to practices such as spying on citizens?

I would argue that working for such an organization is unethical and you should
not do it. Of course this is a position that might not be practical if you're in
need of a job, but if possible working on unethical projects should be avoided.

\subsection*{Mathematical misuse}
Many developments in society have been the direct consequence of advancements in
science and mathematics: the invention of computers immediately comes to mind,
but many explicitly negative ones as well: from atomic bombs to mathematical
models the misuse of which contributed to the 2008 stock market crash.

In my opinion a mathematician should always be weary of where their research can
be (mis)used, but this is in general not easy to see. Research is never a
straight line and finding an unfortunate truth is sometimes inevitable. I don't
think mathematicians should be responsible for obvious misuse of their work or
unforeseen consequences.

\subsection*{Systemic unethical activity}
All humans hold implicit biases and unfortunate opinions. These influence our
work and the things we create and if our work influences society then our biases
will influence these things as well. How does a mathematician make sure their
biases do not unduly influence their work?

As a mathematician working on algorithms influencing society in any way, a
mathematician should always be on the lookout for bias and explicitly test for
it. On top of that, independent third parties should also look for all of these
issues. Furthermore the outcome of algorithms detecting issues should never be
taken at face value and criticized.

\section*{Short reflection}

I believe that as a mathematician you have a moral obligation to consider the
consequences of your work. How this works in practice is obviously different for
different disciplines in math: statistics and financial mathematics have more
concrete applications than homotopy type theory. This does not mean that the
more theoretically minded mathematician should not be mindful of their research:
prime number factorization will probably not be the result of research into the
stock market.

I will probably never work for a large company the sole purpose of which is
profit. As someone with a double degree in mathematics and computer science, I
could probably work for many corporations which only detract from society such
as Google or Facebook and organizations such as intelligence agencies. I will
never do this.

As a main ethical guideline for mathematicians would be to consider the effects
of your work on the world and those around you. Ethics is a murky and difficult
subjects without concrete answers, therefore the best you can do is act
according to your own moral compass.

\section*{Short questions}
\begin{enumerate}
    \item I did because attendance is mandatory.
    \item I knew about most cases, except the mathematical misuse one.
    \item No
    \item No, I took philosophy in high school for 6 years. I was already pretty
          comfortable with ethics.
    \item Not really, I was familiar with most problems presented in the
          seminar and not a lot of new ideas were given.
    \item This format works pretty well. I do not think it should take up a much
          larger portion of the seminar.
\end{enumerate}
\end{document}