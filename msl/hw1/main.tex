\documentclass{article}

\usepackage[utf8]{inputenc}
\usepackage{enumerate}
\usepackage{amsthm, amssymb, mathtools, amsmath, bbm, mathrsfs, stmaryrd}
\usepackage[margin=1in]{geometry}
\usepackage[parfill]{parskip}
\usepackage[hidelinks]{hyperref}
\usepackage{float}
\usepackage{cleveref}
\usepackage{svg}

\newcommand{\N}{\mathbb{N}}
\newcommand{\Z}{\mathbb{Z}}
\newcommand{\NZ}{\mathbb{N}_{0}}
\newcommand{\Q}{\mathbb{Q}}
\newcommand{\R}{\mathbb{R}}
\newcommand{\C}{\mathbb{C}}

\newcommand{\F}{\mathbb{F}}
\newcommand{\incl}{\imath}

\newcommand{\tuple}[2]{\left\langle#1\colon #2\right\rangle}
\newcommand{\powset}{\mathcal{P}}
\DeclareMathOperator{\atoms}{At}
\newcommand{\set}[1]{\left\{#1\right\}}
\newcommand{\setwith}[2]{\set{#1:#2}}

\renewcommand{\qedsymbol}{\raisebox{-0.5cm}{\includesvg[width=0.5cm]{../../qedboy.svg}}}

\DeclareMathOperator{\order}{orde}
\DeclareMathOperator{\Id}{Id}
\DeclareMathOperator{\im}{im}
\DeclareMathOperator{\ggd}{ggd}
\DeclareMathOperator{\kgv}{kgv}
\DeclareMathOperator{\degree}{gr}
\DeclareMathOperator{\coker}{coker}

\DeclareMathOperator{\gl}{GL}

\DeclareMathOperator{\Aut}{Aut}
\DeclareMathOperator{\Hom}{Hom}
\DeclareMathOperator{\End}{End}
\DeclareMathOperator{\Up}{Up}

\newenvironment{solution}{\begin{proof}[Solution]\renewcommand\qedsymbol{}}{\end{proof}}

\newtheorem{theorem}{Theorem}[section]
\newtheorem{lemma}[theorem]{Lemma}
\newtheorem{proposition}[theorem]{Proposition}


\theoremstyle{definition}
\newtheorem{question}{Exercise}
\newtheorem{definition}[theorem]{Definition}
\newtheorem{remark}[theorem]{Remark}
\newtheorem{example}[theorem]{Example}
\newtheorem{corollary}[theorem]{Corollary}

\title{Homework: Mathematical Structures in Logic}
\author{Jonas van der Schaaf}
\date{}

\begin{document}
\maketitle

\begin{question}
    Let \(P\) be a poset and \(\Up(P)\) be the sets of all up-sets of \(P\).

    \begin{enumerate}[a)]
        \item Show that \((\Up(P),\subseteq)\) is a distributive lattice.

              \begin{proof}
                  We know that \(\Up(P)\) is a subposet of \(\powset(P)\) and
                  \(\powset(P)\) has \(\cup\) as join and \(\cap\) as meet. The
                  intersection and union of two upsets is also an upset and so
                  these must be the meet and join in \(\Up(P)\). The inclusion
                  \(\Up(P)\rightarrowtail\powset(P)\) is therefore an injective
                  morphism of lattices.

                  Distributivity then follows immediately from distributivity
                  in \(\powset(P)\).
                  \qedhere
              \end{proof}

        \item Characterize the join irreducible elements of \(\Up(P)\) for a
              finite \(P\).

              \begin{solution}
                  Let \(X\subseteq P\) be join irreducible, we claim that
                  \(X\) is join irreducible iff it is of the form
                  \[
                      \uparrow x=\setwith{y\in P}{x\leq y}.
                  \]

                  Suppose \(X\) is join irreducible, consider the finite
                  collection of \(\leq\) minimal elements of \(X\):
                  \(x_{1},\ldots,x_{n}\). Then clearly \(X=\bigcup_{i}\uparrow
                  x_{i}\). If \(n>1\) none of the upsets \(\uparrow x_{i}\) are
                  equal to \(X\) because no \(\uparrow x_{i}\) contains
                  \(x_{j}\) for \(j\neq i\). Therefore \(n=1\) and \(X=\uparrow
                  x_{1}\).

                  Conversely if \(\uparrow x=X\cup Y\) for upward closed \(X,Y\)
                  then \(x\in X\) or \(x\in Y\). Assume without loss of
                  generality \(x\in X\). Then because \(X\) is upward-closed we
                  have \(\uparrow x\subseteq X\subseteq \uparrow x\). This
                  proves \(X=\uparrow x\) so \(\uparrow x\) is join-irreducible.
              \end{solution}
    \end{enumerate}
\end{question}

\begin{question}
    Let \(B\) be a boolean algebra and \(a,b\in B\). Show that
    \begin{enumerate}[a)]
        \item \(\neg 0=1\) and \(\neg 1=0\).

              \begin{proof}
                  We know that \(1=\neg 0\vee 0\) and \(0=(0\wedge\neg 0)\) so
                  \begin{align*}
                      1 & =1\wedge(\neg 0\vee 0)           & 0 & =0\vee(\neg 1\wedge 1)        \\
                        & =(1\wedge\neg 0)\vee (1\wedge 0) &   & =(0\vee\neg 1)\wedge(0\vee 1) \\
                        & =\neg 0\vee 0                    &   & =\neg 1\wedge 1               \\
                        & =\neg 0                          &   & =\neg 1.
                  \end{align*}
                  Alternatively \(0\) and \(1\) are dual to each other and the
                  dual preserves complements so we could have made do with one
                  of these two calculations.
              \end{proof}

        \item \(\neg\neg a=a\).

              \begin{proof}
                  We know that the complement in a boolean algebra is unique.
                  Therefore if we show that \(a\) is the complement of \(\neg
                  a\), then \(\neg\neg a=a\).

                  The element \(a\) is trivially the complement of \(\neg a\)
                  because \(a\vee\neg a=1\) and \(a\wedge\neg a=0\) by
                  assumption.
              \end{proof}

        \item \(\neg(a\vee b)=\neg a\wedge\neg b\) and \(\neg(a\wedge b)=\neg
              a\vee \neg b\).

              \begin{proof}
                  To prove this we show that \(\neg a\wedge\neg b\) is the
                  complement of \(a\vee b\).

                  Firstly
                  \begin{align*}
                      (\neg a\wedge\neg b)\wedge(a\vee b) & =(\neg a\wedge(a\vee b))\wedge(\neg b\wedge (a\vee b))                               \\
                                                          & =((\neg a\wedge a)\vee (\neg a\wedge b))\wedge((\neg b\wedge a)\vee(\neg b\wedge b)) \\
                                                          & =(0\vee(\neg a\wedge b))\wedge((\neg b\wedge a)\vee 0)                               \\
                                                          & =\neg a\wedge b\wedge\neg b\wedge a                                                  \\
                                                          & =0
                  \end{align*}
                  and secondly
                  \begin{align*}
                      (\neg a\wedge\neg b)\vee(a\vee b) & =(\neg a\vee(a\vee b))\wedge(\neg b\vee (a\vee b)) \\
                                                        & =(1\vee b)\wedge(1\vee a)                          \\
                                                        & =1\wedge 1                                         \\
                                                        & =1.
                  \end{align*}
                  This proves both properties of the complement and therefore
                  the first equality.

                  We derive the second equality using the first and the previous
                  part:
                  \begin{align*}
                      \neg(\neg a\vee\neg b) & =\neg\neg a\wedge\neg\neg b \\
                                             & =a\wedge b.
                  \end{align*}
                  This proves \(a\wedge b=\neg(\neg a\vee\neg b)\) so
                  \begin{align*}
                      \neg(a\wedge b) & =\neg\neg(\neg a\vee\neg b) \\
                                      & =\neg a\vee\neg b.
                  \end{align*}
              \end{proof}

        \item \(a\wedge\neg b=0\) iff \(a\leq b\).

              \begin{proof}
                  Making efficient use of whitespace, on the left side we assume
                  \(a\wedge\neg b=0\) and show that \(b=a\vee b\) which means
                  \(a\leq b\). On the right hand we assume \(a=a\wedge b\) and
                  demonstrate \(a\wedge\neg b=0\):
                  \begin{align*}
                      b & =0\vee b                       & (a\wedge\neg b) & =(a\wedge b\wedge\neg b) \\
                        & =(a\wedge\neg b)\vee b         &                 & =a\wedge 0               \\
                        & =(a\vee b)\wedge(\neg b\vee b) &                 & =0.                      \\
                        & =a\vee b.
                  \end{align*}

              \end{proof}
    \end{enumerate}
\end{question}

\begin{question}
    Let \(B\) be a finite boolean algebra.

    \begin{enumerate}[a)]
        \item Prove that if \(a\in B\) is an atom, then \(a\) is
              join-irreducible.

              \begin{proof}
                  I am proving the original question, that atoms are
                  join-irreducible. This is equivalent to the new question, but
                  I had already written this before the question was changed.

                  Suppose \(a=x\vee y\). Then
                  \begin{align*}
                      x\wedge a & =x\wedge(x\vee y)  \\
                                & =x\vee (x\wedge y) \\
                                & =x
                  \end{align*}
                  and the same holds for \(y\) by symmetry so \(x,y\leq a\).
                  Therefore \(x,y\in\set{0,a}\). Both can't be zero because then
                  we have \(a=x\vee y=0\vee 0=0\). Therefore at least one of
                  \(x,y\) is equal to \(a\).
              \end{proof}

        \item Show that the map \(\eta:B\to\powset(\atoms(B))\) preserves meets
              and joins.

              \begin{proof}
                  Take arbitrary \(x,y\in B\). Then if \(a\in\atoms(B)\) has the
                  properties that \(a\wedge x=a,a\wedge y=a\) then
                  \[
                      a\wedge(x\wedge y)=a\wedge y=a.
                  \]
                  This proves \(\eta(x)\cap\eta(y)\subseteq\eta(x\wedge y)\).

                  Conversely take an atom \(a\in\eta(x\wedge y)\). Then
                  \(a\wedge(x\wedge y)=a\). However we also have
                  \[
                      a=a\wedge(x\wedge y)\leq x\wedge y\leq x,y
                  \]
                  so \(a\leq x\) and \(a\leq y\) so \(a\in\eta(x)\cap\eta(y)\).
                  This proves meets are preserved under \(\eta\).

                  We do approximately the same thing for joins: suppose
                  \(a\in\atoms(B)\) is less than \(x\) or \(y\) (without loss of
                  generality we assume \(a\leq x\) and use commutativity of
                  \(\vee\)). Then
                  \[
                      a\vee(x\vee y)=(a\vee x)\vee y=x\vee y
                  \]
                  so \(a\leq x\vee y\). This proves
                  \(\eta(x)\cup\eta(y)\subseteq\eta(x\vee y)\).

                  Conversely suppose \(a\leq (x\vee y)\) or equivalently
                  \(a\in\eta(x\vee y)\). Then
                  \begin{align*}
                      a & =a\wedge(x\vee y)            \\
                        & =(a\wedge x)\vee(a\wedge y).
                  \end{align*}
                  Because \(a\) is join-irreducible this means \(a=a\wedge x\)
                  or \(a=a\wedge y\), proving \(a\leq x\) or \(a\leq y\) so
                  \(a\in\eta(x)\cup\eta(y)\).
              \end{proof}
    \end{enumerate}
\end{question}

\begin{question}
    \begin{enumerate}[a)]
        \item Show that the lattice of finite and cofinite subsets of \(\N\)
              forms a boolean algebra, which is not complete.

              \begin{proof}
                  We first show that this set is closed under union and
                  intersection. These must then be the meet and join of the
                  lattice respectively, because it is a subposet of the poset
                  \(\powset(\N)\) where these are the meet and join. This will
                  also immediately give distributivity.

                  For any two finite sets \(x,y\), the intersection and union
                  are clearly finite. For one cofinite and one finite set,
                  the intersection is finite and the union is cofinite so
                  both union and intersection are elements of the poset.

                  If \(x,y\) are both cofinite their union is clearly cofinite.
                  By De Morgan laws for the intersection we get \(\neg(x\cap
                  y)=\neg x\cup\neg y\) which is the union of two finite things
                  so the complement of the intersection is finite. Therefore
                  \(x\cap y\) is also cofinite.

                  This proves this is a distributive lattice. It is bounded
                  because \(\varnothing\) and \(\N\) are finite and cofinite
                  respectively so the lattice is bounded. The complement of a
                  finite set is by definition cofinite and vice versa, so this
                  lattice has complements as well. Therefore it is a boolean
                  algebra.

                  To see the lattice is not complete consider the set
                  \[
                      \setwith{\setwith{2m}{m\in \N,m\leq n}}{n\in\N}
                  \]
                  of elements of the lattice. This set does not have a join:
                  because the union of these sets is infinite, the meet must be
                  cofinite and therefore contain infinitely many odd numbers as
                  well. However, for any such set we can always remove one more
                  odd number to get a smaller upper bound. Therefore there is no
                  least upper bound and the algebra is not complete.
              \end{proof}

        \item Show that the lattice of finite subsets with \(\N\) forms a
              complete distributive lattive.

              \begin{proof}
                  This set is clearly closed under intersection and union. Both
                  sets are finite both operations give back a finite set. If one
                  of the sets is \(\N\) then intersection gives the other and
                  union gives \(\N\). Therefore these are the meet and join in
                  the poset. Distributivity follows immediately and the poset is
                  bounded by \(\varnothing\) and \(\N\) above.

                  Therefore this is a distributive bounded lattice.

                  Now we show it is complete. Let \(X\) be a set of finite
                  subsets of \(\N\). If \(\bigcup X\) is finite, this is the
                  join. If it is infinite the join is \(\N\) as this is the only
                  infinite set. Having joins automatically means there are meets
                  as well, so this lattice is complete.
              \end{proof}
    \end{enumerate}
\end{question}
\end{document}