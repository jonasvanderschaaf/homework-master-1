\documentclass{article}

\usepackage[utf8]{inputenc}
\usepackage{enumerate}
\usepackage{amsthm, amssymb, mathtools, amsmath, bbm, mathrsfs, stmaryrd}
\usepackage[margin=1in]{geometry}
\usepackage[parfill]{parskip}
\usepackage[hidelinks]{hyperref}
\usepackage{float}
\usepackage{cleveref}
\usepackage{svg}
\usepackage{tikz-cd}

\newcommand{\N}{\mathbb{N}}
\newcommand{\Z}{\mathbb{Z}}
\newcommand{\NZ}{\mathbb{N}_{0}}
\newcommand{\Q}{\mathbb{Q}}
\newcommand{\R}{\mathbb{R}}
\newcommand{\C}{\mathbb{C}}

\newcommand{\F}{\mathbb{F}}
\newcommand{\incl}{\imath}

\newcommand{\tuple}[2]{\left\langle#1\colon #2\right\rangle}
\newcommand{\powset}{\mathcal{P}}
\DeclareMathOperator{\atoms}{At}
\newcommand{\set}[1]{\left\{#1\right\}}
\newcommand{\setwith}[2]{\set{#1:#2}}

\renewcommand{\qedsymbol}{\raisebox{-0.5cm}{\includesvg[width=0.5cm]{../../qedboy.svg}}}

\DeclareMathOperator{\order}{orde}
\DeclareMathOperator{\Id}{Id}
\DeclareMathOperator{\im}{im}
\DeclareMathOperator{\ggd}{ggd}
\DeclareMathOperator{\kgv}{kgv}
\DeclareMathOperator{\degree}{gr}
\DeclareMathOperator{\coker}{coker}

\DeclareMathOperator{\gl}{GL}

\DeclareMathOperator{\Aut}{Aut}
\DeclareMathOperator{\Hom}{Hom}
\DeclareMathOperator{\End}{End}
\DeclareMathOperator{\Up}{Up}
\DeclareMathOperator{\fin}{Fin}
\DeclareMathOperator{\cofin}{Cofin}
\DeclareMathOperator{\fincofin}{FinCofin}
\DeclareMathOperator{\spec}{Spec}
\DeclareMathOperator{\iso}{Iso}
\DeclareMathOperator{\filters}{Fil}
\DeclareMathOperator{\closed}{Cl}
\DeclareMathOperator{\inter}{Int}

\newenvironment{solution}{\begin{proof}[Solution]\renewcommand\qedsymbol{}}{\end{proof}}

\newtheorem{theorem}{Theorem}
\newtheorem{lemma}[theorem]{Lemma}
\newtheorem{proposition}[theorem]{Proposition}


\theoremstyle{definition}
\newtheorem{question}{Exercise}
\newtheorem{definition}[theorem]{Definition}
\newtheorem{remark}[theorem]{Remark}
\newtheorem{example}[theorem]{Example}
\newtheorem{corollary}[theorem]{Corollary}

\title{Homework: Mathematical Structures in Logic}
\author{Jonas van der Schaaf}
\date{}

\begin{document}
\maketitle

\begin{question}
    Let \(X\) be a Priestley space.

    \begin{enumerate}[(a)]
        \item Show that \(\downarrow\set{y}\) and \(\uparrow\set{y}\) are closed
              for each \(y\in X\).

              \begin{proof}
                  Take \(x\notin\uparrow\set{y}\). We show there is an open
                  up-set neighbourhood of \(x\) disjoint from
                  \(\uparrow\set{y}\). We know  \(y\nleq x\) so there is an
                  clopen up-set neighbourhood \(U_{x}\) of \(y\) not containing
                  \(x\). The complement \(X\setminus U_{x}\) is then a clopen
                  downset containing \(x\) and not \(y\), which means it cannot
                  contain elements of \(\uparrow\set{y}\) either. Therefore it
                  is the neighbourhood of \(x\) disjoint from
                  \(\uparrow\set{y}\).

                  Similarly take \(x\notin\downarrow{y}\). Then \(x\nleq y\) so
                  there is a clopen up-set \(U_{x}\) neighbourhood of \(x\) such
                  that \(y\notin U_{x}\). This set has to be disjoint from
                  \(\downarrow\set{y}\) because it is upward closed. Therefore
                  each point of \(X\setminus\uparrow\set{y}\) has an open
                  neighbourhood disjoint from \(\uparrow\set{y}\) and so this is
                  open. Its complement \(\uparrow\set{y}\) must then be closed
              \end{proof}

        \item Show that, if \(Y\subseteq X\) is closed then \(\uparrow Y\) and
              \(\downarrow Y\) are closed.

              \begin{proof}
                  Take \(x\notin\uparrow Y\), we show it has an open
                  neighbourhood of \(x\) disjoint from \(\uparrow Y\). This
                  means that the set \(\uparrow Y\) is closed. We know that
                  \(y\not\leq x\) for all \(y\in Y\), so there is a clopen upset
                  \(U_{x,y}\) with \(y\in _{x,y}\) and \(x\notin U_{x,y}\). The
                  set \(\bigcup_{y\in Y}U_{x,y}\) is then an upset and open
                  cover of \(Y\) which is compact. Therefore there is a finite
                  subset \(Z\subseteq Y\) such that \(Y\subseteq\bigcup_{y\in
                      Z}U_{x,y}\) and this union is clopen. Because this union is
                  still an upset we also have \(\uparrow Y\subseteq\bigcup_{y\in
                      Z}U_{x,y}\). The complement \(X\setminus\bigcup_{y\in
                      Z}U_{x,y}\) is then an open neighbourhood of \(x\) disjoint
                  from \(\uparrow Y\) because \(x\notin U_{x,y}\) for all \(y\in
                  Z\). This means \(\uparrow Y\) is closed.

                  Similarly take \(x\notin\downarrow Y\), then for all \(y\in
                  Y\) we have \(x\not\leq y\) so there is a clopen downset
                  \(U_{x,y}\) such that \(y\in U_{x,y}\) and \(x\notin U_{x,y}\)
                  (it is the complement of the clopen upset given by the
                  Priestley axiom). The union \(\bigcup_{y\in Y}U_{x,y}\) covers
                  \(Y\) so there is a finite \(Z\subseteq Y\) such that the
                  clopen downset \(\bigcup_{y\in Z}U_{x,y}\) covers \(Y\) and
                  therefore also \(\downarrow Y\). We know \(x\notin U_{x,y}\)
                  for all \(y\in Z\) so \(X\setminus\bigcup_{y\in Z}U_{x,y}\) is
                  a clopen neighbourhood of \(x\) disjoint from \(\downarrow Y\)
                  showing that set is closed as well.
              \end{proof}
    \end{enumerate}
\end{question}

\begin{question}
    I'm not going to write this question down.

    \begin{enumerate}[(a)]
        \item Show that this space is a Priestley space.

              \begin{proof}
                  Let \(x,y\) be two elements with \(x\nleq y\). Then \(x=0\) or
                  \(y=0\) and the other element cannot be \(\infty\) because
                  \(0\) is the only element that is incomparable with another
                  element.

                  If \(x=0\) then \(\set{0}\) is an up-set containing \(0\) but
                  not any \(y\in\N_{>0}\).

                  If \(y=0\) then \(\N_{>0}\) is an up-set containing
                  \(x\) but not \(y\).

                  This proves this is a Priestley space.
              \end{proof}

        \item Show that it is not an Esakia space.

              \begin{proof}
                  Consider the clopen set \(\set{0}\). Its downset is
                  \(\set{0,\infty}\) which is not clopen because the clopens
                  containing \(\infty\) are cofinite. Therefore this is not an
                  Esakia space.
              \end{proof}

        \item Draw the distributive lattice dual to this space.

              \begin{solution}
                  The distributive lattice dual to this space is the lattice of
                  clopen up-sets. The clopen up-sets are exactly the following:
                  The whole set \(\uparrow\infty\), \(\set{0}\), \(\uparrow n\)
                  and \(\uparrow n\cup\set{0}\) for \(n>0\), and vacuously the
                  empty set \(\varnothing\).

                  These are contained in each other by inclusion as shown in the
                  following diagram:
                  \[
                      \begin{tikzcd}[ampersand replacement=\&]
                          \& \uparrow\infty \\
                          \& \uparrow3\cup\uparrow0 \\
                          \& \uparrow2\cup\uparrow0 \& \uparrow3 \\
                          \& \uparrow1\cup\uparrow0 \& \uparrow2 \\
                          \uparrow0 \&\& \uparrow1 \\
                          \& \varnothing
                          \arrow[from=4-2, to=3-2]
                          \arrow[from=3-2, to=2-2]
                          \arrow[dotted, no head, from=2-2, to=1-2]
                          \arrow[from=6-2, to=4-2]
                          \arrow[from=6-2, to=5-1]
                          \arrow[from=5-1, to=4-2]
                          \arrow[from=6-2, to=5-3]
                          \arrow[from=5-3, to=4-2]
                          \arrow[from=5-3, to=4-3]
                          \arrow[from=4-3, to=3-2]
                          \arrow[from=4-3, to=3-3]
                          \arrow[from=3-3, to=2-2]
                          \arrow[dotted, no head, from=3-3, to=1-2]
                      \end{tikzcd}
                  \]
              \end{solution}
    \end{enumerate}
\end{question}

\begin{question}\,
    \begin{enumerate}[(a)]
        \item Let \((X,\leq)\) be an Esakia space. Show that for any up-set
              \(U\subseteq X\) we have that its closure \(\closed(U)\) is also
              an up-set.

              \begin{proof}
                  Let \(U\) be an up-set with closure \(\closed(U)\), take any
                  limit point \(x\in\closed(U)\) of \(U\) and consider \(y\geq
                  x\). We show that this must also be a limit point of \(U\).
                  Suppose it is not: i.e. there is some clopen neighbourhood
                  \(V\) of \(y\) such that \(U\cap V=\varnothing\). Then the
                  downset \(\downarrow V\) must also be disjoint from \(U\)
                  because \(U\) is an upset. However because \(x\leq y\) we must
                  have that \(x\in \downarrow V\) which is clopen by the
                  defining property of Esakia spaces. Therefore \(x\) was not a
                  limit point in the first place because \(\downarrow V\) is an
                  open neighbourhood of \(x\) disjoint from \(U\). This is a
                  contradiction so \(y\) is a limit point.
              \end{proof}

        \item Show that this is not in general true in Priestley spaces.

              \begin{solution}
                  Consider the set \(\N\setminus\set{0}\) as subset of the
                  Priestley space from question 2. This is an upset. Its closure
                  is the set \((\N\setminus\set{0})\cup\set{\infty}\) because
                  all opens containing \(\infty\) from \(\N\setminus\set{0}\)
                  contain a cofinite subset of \(\N\) and so must intersect with
                  this set and so \(\infty\) is a limit point of the set. The
                  element \(0\) is not a limit point because \(\set{0}\) is an
                  open neighbourhood disjoint from the set.

                  This closure is however not an upset because \(\infty\leq 0\)
                  and \(0\) is not in the set.
              \end{solution}
    \end{enumerate}
\end{question}

\begin{question}
    Show a Priestley space has minimal and maximal elements.

    \begin{proof}
        We demonstrate this using Priestley duality and Zorn's lemma.

        We show that the poset of prime ideal of any distributive lattice has
        maximal/minimal elements by showing any increasing/decreasing chain has
        an upper/lower bound in the poset of primes. By Zorn's lemma, this
        immediately gives maximal and minimal elements.

        Given an increasing sequence of filter \((F_{i})_{i\in\N}\) the union
        \(F=\bigcup_{i}F_{i}\) is also a prime filter. It is clear that it is
        still upward closed. It is closed under meets because for \(x,y\in F\),
        \(x,y\in F_{i}\) for some \(i\) so \(x\wedge y\in F_{i}\subseteq F\).
        Also if \(x\vee y\in F\), then \(x\vee y\in F_{i}\) for some \(i\) so
        \(x\in F_{i}\subseteq F\) or \(y\in F_{i}\subseteq F\). This proves
        \(F\) is a prime filter and thus every chain has an upper bound. By
        Zorn's lemma then the set of prime filters has a maximal element.

        By Priestley duality, this maximal element in the poset of prime filters
        corresponds to a maximal element in the poset \(\leq\).

        Similarly for a descending chain of prime filters \((F_{i})_{i\in\N}\)
        the intersection is still a prime filter and a lower bound of the chain.
        It is clear that \(\bigcap_{i}F_{i}\) is still upwards closed. It is
        closed under intersection because for \(x,y\in F\), \(x,y\in F_{i}\) for
        all \(i\). This means \(x\wedge y\in F\) so \(F\) is still a filter.

        Similarly if \(x\vee y\in F\) for some \(x,y\) then at least one of
        \(x\) or \(y\) must be contained in all \(F_{i}\). This is because
        \(x\in F_{i}\) or \(y\in F_{i}\) for each \(i\) (though a priori) fixed
        when alternating \(i\). However suppose there are \(i<j\) such that
        \(y\notin\in F_{i}\) and \(x\notin F_{j}\), then \(x\notin F_{i}\)
        either so \(F_{i}\) is not a prime filter. This shows that \(x\in F\)
        or \(y\in F\) so it is a prime filter.

        By Zorn's lemma (applied to the opposite poset), this means the poset
        has minimal elements as well so by Priestley duality the poset of a
        Priestley space does as well.
    \end{proof}
\end{question}
\end{document}