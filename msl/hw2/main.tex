\documentclass{article}

\usepackage[utf8]{inputenc}
\usepackage{enumerate}
\usepackage{amsthm, amssymb, mathtools, amsmath, bbm, mathrsfs, stmaryrd}
\usepackage[margin=1in]{geometry}
\usepackage[parfill]{parskip}
\usepackage[hidelinks]{hyperref}
\usepackage{float}
\usepackage{cleveref}
\usepackage{svg}

\newcommand{\N}{\mathbb{N}}
\newcommand{\Z}{\mathbb{Z}}
\newcommand{\NZ}{\mathbb{N}_{0}}
\newcommand{\Q}{\mathbb{Q}}
\newcommand{\R}{\mathbb{R}}
\newcommand{\C}{\mathbb{C}}

\newcommand{\F}{\mathbb{F}}
\newcommand{\incl}{\imath}

\newcommand{\tuple}[2]{\left\langle#1\colon #2\right\rangle}
\newcommand{\powset}{\mathcal{P}}
\DeclareMathOperator{\atoms}{At}
\newcommand{\set}[1]{\left\{#1\right\}}
\newcommand{\setwith}[2]{\set{#1:#2}}

\renewcommand{\qedsymbol}{\raisebox{-0.5cm}{\includesvg[width=0.5cm]{../../qedboy.svg}}}

\DeclareMathOperator{\order}{orde}
\DeclareMathOperator{\Id}{Id}
\DeclareMathOperator{\im}{im}
\DeclareMathOperator{\ggd}{ggd}
\DeclareMathOperator{\kgv}{kgv}
\DeclareMathOperator{\degree}{gr}
\DeclareMathOperator{\coker}{coker}

\DeclareMathOperator{\gl}{GL}

\DeclareMathOperator{\Aut}{Aut}
\DeclareMathOperator{\Hom}{Hom}
\DeclareMathOperator{\End}{End}
\DeclareMathOperator{\Up}{Up}
\DeclareMathOperator{\fin}{Fin}
\DeclareMathOperator{\cofin}{Cofin}
\DeclareMathOperator{\fincofin}{FinCofin}
\DeclareMathOperator{\spec}{Spec}
\DeclareMathOperator{\iso}{Iso}
\DeclareMathOperator{\filters}{Fil}
\DeclareMathOperator{\closed}{Cl}

\newenvironment{solution}{\begin{proof}[Solution]\renewcommand\qedsymbol{}}{\end{proof}}

\newtheorem{theorem}{Theorem}
\newtheorem{lemma}[theorem]{Lemma}
\newtheorem{proposition}[theorem]{Proposition}


\theoremstyle{definition}
\newtheorem{question}{Exercise}
\newtheorem{definition}[theorem]{Definition}
\newtheorem{remark}[theorem]{Remark}
\newtheorem{example}[theorem]{Example}
\newtheorem{corollary}[theorem]{Corollary}

\title{Homework: Mathematical Structures in Logic}
\author{Jonas van der Schaaf}
\date{}

\begin{document}
\maketitle

\begin{question}
    Let \(A\) be a boolean algebra. A filter \(F\) of the form \(\uparrow a\)
    for some \(a\in A\) is called a principal filter. Let \(\fincofin(\N)\) be
    the boolean algebra of all finite and cofinite subsets of \(\N\).

    \begin{enumerate}[a)]
        \item Characterize all principal ultrafilters in \(\fincofin(\N)\).

              We first prove a lemma:

              \begin{lemma}
                  Join-irreducible elements are atoms.

                  \begin{proof}
                      Let \(a\) be join-irreducible and \(b\leq a\). Then
                      \(b\vee(a\wedge\neg b)=a\). Therefore \(a=b\) or
                      \(a\wedge\neg b=a\). Because \(b\leq a\) we know
                      \(a\wedge\neg b=0\). Therefore \(b=a\).
                  \end{proof}
              \end{lemma}

              \begin{solution}
                  A principal filter \(\uparrow a\) is an ultrafilter iff \(a\)
                  is an atom.

                  Suppose \(\uparrow a\) is prime. We show \(a\) is
                  join-irreducible.

                  Then for \(b,c\in A\) with \(b\vee c=a\) we must have
                  \(b\in\uparrow a\) or \(c\in\uparrow a\) or equivalently
                  \(a\leq b\) or \(a\leq c\). We have \(b,c\leq a\) by
                  assumption so \(a=b\) or \(a=c\). Therefore \(a\) is
                  join-irreducible and so an atom.

                  If \(a\) is an atom take \(b,c\in A\) such that \(b\vee
                  c\in\uparrow a\). Then \(b\vee c\geq a\). Because \(a\) is an
                  atom it is join-prime so \(b\geq a\) or \(c\geq a\). Therefore
                  \(b\in\uparrow a\) or \(c\in\uparrow a\) so it is a prime
                  ideal.

                  The atoms are clearly the singleton sets, so these give the
                  prime filters.
              \end{solution}

        \item Show that there is a unique non-principal ultrafilter in
              \(\fincofin(\N)\).

              \begin{proof}
                  First we prove that any prime filter containing a finite set
                  must contain an atom. If \(x\in F\) is a finite set in a prime
                  filter \(F\), then \(x=\bigwedge_{a\in x}\set{a}\). Therefore
                  \(\set{a}\in F\) for some \(a\in x\). Singleton sets are atoms
                  so \(F\) contains an atom and therefore
                  \(\uparrow\set{a}\subseteq F\). The former is a maximal filter
                  so \(F=\uparrow\set{a}\).

                  Consider the set of cofinal sets of \(\N\): \(\cofin(\N)\).
                  They form a proper filter: there are no two cofinal sets with
                  intersection \(0\) and the intersection of cofinite sets is
                  cofinite. This must be a prime filter because it contains
                  \(x\) or \(\neg x\) for each \(x\in\fincofin(\N)\).

                  This is clearly non-principle because we have an infinitely
                  descending chain in the algebra containing any arbitrarily
                  chosen cofinal set. It is also the only non-principal prime
                  filter because any principal prime filter must be generated by
                  an atom by the proof of part \(a\).
              \end{proof}
    \end{enumerate}
\end{question}

\begin{question}
    Let \(A\) be a finite boolean algebra. Show that \(A\) has a least
    non-unital filter iff \(A\) is isomorphic to the 2 element boolean algebra.

    \begin{proof}
        I accidentally did the proof for any algebra \(A\), not just finite
        ones.

        It is clear that the 2 element boolean ring has one non-unital filter:
        \(\set{0,1}\). This must be the least element.

        Suppose a boolean algebra has such a least filter \(F\). Take any
        element \(x\in F\). If there is a \(y\) with \(x<y<1\) then \(\uparrow
        y\) is a strict subfilter of \(F\). This cannot happen so \(x\) must be
        a co-atom. Now \(\neg x\) is contained in a proper filter \(F'\) by the
        prime filter theorem (there is a filter containing \(x\) not containing
        \(0\)) and \(F\subseteq F'\) by assumption. But then \(x,\neg x\in F'\)
        so \(1\in F'\). This is a contradiction with the assumption that \(F'\)
        is proper so \(F'\) cannot exist.
    \end{proof}
\end{question}

\begin{question}
    Recall that in a topological space \(X\) a point \(x\in X\) is isolated
    if \(\set{x}\) is open.

    \begin{enumerate}[a)]
        \item Show that there is a 1-1 correspondence between the atoms of a BA
              \(A\) and the isolated points of its dual space \(\spec(A)\).

              \begin{proof}
                  Let \(a\) be an atom. Then \(\uparrow a\) is a maximal filter
                  by the reasoning in question 1a. All prime filters containing
                  \(a\) contain this maximal filter, so they must be \(\uparrow
                  a\). Therefore \(\varphi(a)=\set{\uparrow a}\) which is open
                  in \(\spec(A)\).

                  Conversely if \(F\) is an isolated point of \(\spec(A)\), then
                  \(\set{F}\) must be a basis element: it is the union of basis
                  elements and a singleton set so one of these basis elements is
                  \(\set{F}\). Therefore \(F=\uparrow a\) for some \(a\in A\).
                  This \(a\) must have been an atom by the reasoning in question
                  1a. This \(a\) is unique because two different atoms clearly
                  generate different prime ideals.

                  This demonstrates this is a bijection between isolated points
                  and atoms.
              \end{proof}

        \item Let \(\iso_{X}\) be the set of isolated points of a space \(X\).
              Show that if \(A\) is atomic, then \(\iso_{X}\) is dense in
              \(\spec(A)\).

              \begin{proof}
                  We demonstrate that for each \(b\in A\), there is an atom
                  \(a\) such that \(\uparrow a\) is contained in the
                  neighbourhood \(\varphi(b)\subseteq\spec(A)\). This is
                  equivalent to \(\iso_{X}\) being dense because
                  \(\iso_{X}=\setwith{\uparrow a}{a\in\atoms(A)}\) by part a.

                  We know there is an atom \(a\) with \(a\leq b\). Then
                  \(b\in\uparrow a\) so \(\uparrow a\in\varphi(b)\). This proves
                  our claim.
              \end{proof}
    \end{enumerate}
\end{question}

\begin{question}
    Consider the map \(\xi:\filters(B)\to\closed(X_{B})\) defined by
    \[
        .\xi(F)=\bigcap\setwith{\varphi(a)}{a\in F}.
    \]
    Show that \(\xi\) is an order-reversing bijection between \(\filters(B)\)
    and \(\closed(X_{B})\).

    \begin{proof}
        This intersection \(\xi(F)\) is the set of prime filters containing
        \(F\).

        We first show this is a bijection. Clearly these intersections are
        closed since the intersection of closed sets is closed. Therefore this
        map is a well-defined function into the closed sets of \(X_{B}\).

        Take two different filters \(F,F'\) and assume without loss of
        generality there is some \(x\in F\) and \(x\notin F'\). We know there is
        a prime filter \(P\) extending \(F'\) disjoint from \(\downarrow x\) by
        the prime filter theorem. Therefore \(P\in\xi(F')\) but
        \(P\notin\xi(F)\). This proves \(\xi(F)\neq\xi(F')\).

        For surjectivity let \(Y\subseteq X\) be a closed set, this must be the
        intersection of (infinitely many) clopens \(\varphi(\neg a_{i})\) with
        \(a_{i}\) elements of \(B\) and \(i\in I\) for some index set \(I\):
        \(A=\bigcap_{i\in I}\varphi(\neg a_{i})\). This is because the
        complement is open so a union of these basis elements:
        \[
            X_{B}\setminus A=\bigcup_{i\in I}\varphi(a_{i})
        \]
        so by De-Morgan's laws and because \(\varphi\) is a morphism
        \begin{align*}
            A & =X\setminus\bigcup_{i\in I}\varphi(a_{i}) \\
              & =\bigcap_{i\in I}\varphi(\neg a_{i}).
        \end{align*}
        Rewriting this we have
        \begin{align*}
            A & =\bigcap_{i\in I}\varphi(\neg a_{i})                          \\
              & =\setwith{x\in\spec B}{\forall i\in I:\neg a_{i}\in x}        \\
              & =\setwith{x\in\spec B}{\tuple{\neg a_{i}}{i\in I}\subseteq x} \\
              & =\varphi(\tuple{\neg a_{i}}{i\in I})
        \end{align*}
        where \(\tuple{\neg a_{i}}{i\in I}\) is the filter generated by the
        \(\neg a_{i}\).

        To show this is order-reversing take two filters \(F\subseteq F'\). If
        \(x\in X_{B}\) contains \(F'\) then it contains \(F\) as well. Therefore
        if \(x\in \xi(F')\) then \(x\in\xi(F)\), or equivalently
        \(\xi(F')\subseteq\xi(F)\). This proves the map is order-reversing.

        Fun fact: This is actually a special case of the correspondence between
        the radical ideals of a ring and its spectrum in algebraic geometry.
        Because all ideals of boolean rings are radical (all elements are
        idempotent) this means the correspondence is a bijection.
    \end{proof}
\end{question}
\end{document}