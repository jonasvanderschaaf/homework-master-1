\documentclass{article}

\usepackage[utf8]{inputenc}
\usepackage{enumerate}
\usepackage{amsthm, amssymb, mathtools, amsmath, bbm, mathrsfs, stmaryrd}
\usepackage[margin=1in]{geometry}
\usepackage[parfill]{parskip}
\usepackage[hidelinks]{hyperref}
\usepackage{float}
\usepackage{cleveref}
\usepackage{svg}
\usepackage{tikz-cd}

\newcommand{\N}{\mathbb{N}}
\newcommand{\Z}{\mathbb{Z}}
\newcommand{\NZ}{\mathbb{N}_{0}}
\newcommand{\Q}{\mathbb{Q}}
\newcommand{\R}{\mathbb{R}}
\newcommand{\C}{\mathbb{C}}

\newcommand{\F}{\mathbb{F}}
\newcommand{\incl}{\imath}

\newcommand{\tuple}[2]{\left\langle#1\colon #2\right\rangle}
\newcommand{\powset}{\mathcal{P}}
\DeclareMathOperator{\atoms}{At}
\newcommand{\set}[1]{\left\{#1\right\}}
\newcommand{\setwith}[2]{\set{#1:#2}}

\renewcommand{\qedsymbol}{\raisebox{-0.5cm}{\includesvg[width=0.5cm]{../../qedboy.svg}}}

\DeclareMathOperator{\order}{orde}
\DeclareMathOperator{\Id}{Id}
\DeclareMathOperator{\im}{im}
\DeclareMathOperator{\ggd}{ggd}
\DeclareMathOperator{\kgv}{kgv}
\DeclareMathOperator{\degree}{gr}
\DeclareMathOperator{\coker}{coker}

\DeclareMathOperator{\gl}{GL}

\DeclareMathOperator{\Aut}{Aut}
\DeclareMathOperator{\Hom}{Hom}
\DeclareMathOperator{\End}{End}
\DeclareMathOperator{\Up}{Up}
\DeclareMathOperator{\fin}{Fin}
\DeclareMathOperator{\cofin}{Cofin}
\DeclareMathOperator{\fincofin}{FinCofin}
\DeclareMathOperator{\spec}{Spec}
\DeclareMathOperator{\iso}{Iso}
\DeclareMathOperator{\filters}{Fil}
\DeclareMathOperator{\closed}{Cl}
\DeclareMathOperator{\inter}{Int}
\DeclareMathOperator{\var}{Var}

\newenvironment{solution}{\begin{proof}[Solution]\renewcommand\qedsymbol{}}{\end{proof}}

\newtheorem{theorem}{Theorem}
\newtheorem{lemma}[theorem]{Lemma}
\newtheorem{proposition}[theorem]{Proposition}


\theoremstyle{definition}
\newtheorem{question}{Exercise}
\newtheorem{definition}[theorem]{Definition}
\newtheorem{remark}[theorem]{Remark}
\newtheorem{example}[theorem]{Example}
\newtheorem{corollary}[theorem]{Corollary}

\title{Homework: Mathematical Structures in Logic}
\author{Jonas van der Schaaf}
\date{}

\begin{document}
\maketitle

\begin{question}
    Show that for every non-trivial variety \(V\) of Heyting algebras, if
    \(V\neq \var(2)\) then \(\var(3)\subseteq V\).

    \begin{proof}
        Let \(A\) be a subdirect irreducible Heyting algebra different from
        \(2\) and the trivial one with \(x\) the unique element directly below
        \(1\). Then the set \(\set{0,x,1}\) is a subalgebra of \(A\) isomorphic
        to \(3\):

        In \(A\) we have \(0\to x=x\to 1=1\) and \(x\to 0=1\to x=0\), we also
        have \(1\to0=0\) and \(0\to1=1\). This set is clearly closed under joins
        and meets as well.

        This means that \(3\in S(V)\) for any variety \(V\) of Heyting algebras
        different from \(\var(2)\) and \(\var(1)\) so \(\var(3)\subseteq V\).
    \end{proof}
\end{question}

\begin{question}
    Characterise all non-trivial subdirectly irreducible algebras in \(\var(A)\)
    where \(A\) is the Heyting algebra dual to the Esakia space
    \[
        \begin{tikzcd}[ampersand replacement=\&]
            \bullet \&\& \bullet \\
            \bullet \&\& \bullet \\
            \& \bullet
            \arrow[from=2-1, to=1-1]
            \arrow[from=2-3, to=1-3]
            \arrow[from=2-1, to=1-3]
            \arrow[from=2-3, to=1-1]
            \arrow[from=3-2, to=2-1]
            \arrow[from=3-2, to=2-3]
        \end{tikzcd}
    \]

    \begin{proof}
        We find all p-morphic images of this space using reductions. The spaces
        with a minimal non-minimal element are the subdirectly irreducible ones.

        We cannot apply \(\alpha\)-reductions to this space because no point has
        a unique immediate successor.

        The two maximal elements see the same elements (none), and the middle
        two do as well (the maximal elements). These can therefore be identified
        by \(\beta\)-reduction giving two new spaces:
        \[
            \begin{tikzcd}[ampersand replacement=\&]
                \bullet \&\& \bullet \&\&\& \bullet \\
                \& \bullet \&\&\& \bullet \&\& \bullet \\
                \& \bullet \&\&\&\& \bullet
                \arrow[from=3-2, to=2-2]
                \arrow[from=2-2, to=1-1]
                \arrow[from=2-2, to=1-3]
                \arrow[from=2-5, to=1-6]
                \arrow[from=2-7, to=1-6]
                \arrow[from=3-6, to=2-5]
                \arrow[from=3-6, to=2-7]
            \end{tikzcd}
        \]
        To the left space we can either apply an \(\alpha\)-reduction to the
        bottom two points, or a \(\beta\)-reduction to the maximal points
        to get the following two Esakia spaces:
        \[
            \begin{tikzcd}[ampersand replacement=\&]
                \bullet \&\& \bullet \&\& \bullet \\
                \& \bullet \&\&\& \bullet \\
                \&\&\&\& \bullet
                \arrow[from=2-2, to=1-1]
                \arrow[from=2-2, to=1-3]
                \arrow[from=3-5, to=2-5]
                \arrow[from=2-5, to=1-5]
            \end{tikzcd}
        \]
        Applying any reduction to either of these two gives the space dual to
        the boolean algebra \(2\).

        Applying a reduction to the right space, we get the space dual to the
        Heyting algebra \(3\), which we already found earlier.

        The subdirect irreducible Heyting algebras correspond to those with a
        minimal non-minimal element, so these are the spaces
        \[
            \begin{tikzcd}[ampersand replacement=\&]
                \bullet \&\& \bullet \&\& \bullet \\
                \& \bullet \&\&\& \bullet \&\& \bullet \\
                \& \bullet \&\&\& \bullet \&\& \bullet
                \arrow[from=2-2, to=1-1]
                \arrow[from=2-2, to=1-3]
                \arrow[from=3-2, to=2-2]
                \arrow[from=2-5, to=1-5]
                \arrow[from=3-5, to=2-5]
                \arrow[from=3-7, to=2-7]
            \end{tikzcd}
        \]
        Note that each space is a reduct of the one to the left of it,
        so we get a linear lattice of subvarieties given by
        \[
            \var(2)\subseteq\var(3)\subseteq\var(C)
        \]
        where \(C\) is the dual Heyting algebra of the leftmost Esakia space,
        which is
        \[
            C=\begin{tikzcd}[ampersand replacement=\&]
                \& \bullet \\
                \& \bullet \\
                \bullet \&\& \bullet \\
                \& \bullet
                \arrow[from=4-2, to=3-1]
                \arrow[from=4-2, to=3-3]
                \arrow[from=3-3, to=2-2]
                \arrow[from=3-1, to=2-2]
                \arrow[from=2-2, to=1-2]
            \end{tikzcd}
        \]
    \end{proof}
\end{question}

\begin{question}
    Let \(A\) be a Heyting algebra and consider an element \(a\in A\).

    \begin{enumerate}[(a)]
        \item Show that \(a\) is join prime iff \(\varphi(a)\) is rooted.

              \begin{proof}
                  Let \(a\) be join prime, we claim \(\uparrow a\) is a prime
                  filter. It is clearly a filter, and if \(b\vee c\in\uparrow
                  a\) then \(b\vee c\geq a\) so \(b\geq a\) or \(c\geq a\) and
                  therefore \(b\in\uparrow a\) or \(c\in\uparrow a\).

                  This means there is a least prime filter containing \(a\):
                  this upset \(\uparrow a\). This is because any prime filter
                  containing \(a\) must contain \(\uparrow a\).

                  Conversely if \(\varphi(a)\) is rooted let \(F\) be the
                  minimum prime of \(\varphi(a)\). We claim this is the filter
                  \(\uparrow a\). It is clear that \(\uparrow a\subseteq F\), so
                  suppose there is an \(x\in F\setminus\uparrow a\), then the
                  ideal \(\downarrow x\) and \(\uparrow a\) are disjoint. By the
                  prime filter theorem we can then find a prime filter \(F'\)
                  containing \(a\) not containing \(x\). This contradicts
                  minimality of \(F\). This means \(\uparrow a\) is a prime
                  filter. Therefore if \(b\vee c\geq a\), then \(b\in\uparrow
                  a\) or \(c\in\uparrow b\) or equivalently \(b\geq a\) or
                  \(c\geq a\).
              \end{proof}

        \item Show that \(a\) is completely join prime iff \(\varphi(a)\) is
              strongly rooted.

              \begin{proof}
                  Suppose \(a\) is completely join prime. Then we know in
                  particular that \(\varphi(a)\) is rooted with root \(\uparrow
                  a\). I did not figure out this implication.

                  Conversely suppose \(\varphi(a)\) is strongly rooted. Then it
                  has root \(\uparrow a\). We show this is an isolated point.
                  Suppose \(a\leq\bigvee_{i}b_{i}\) for some collection
                  \((b_{i})_{i}\). Then
                  \[
                      \varphi(a)\subseteq\varphi(\bigvee_{i}b_{i})=\overline{\bigcup_{i}\varphi(b_{i})}.
                  \]
                  This means that all neighbourhoods of any point of
                  \(\varphi(a)\) intersects \(\bigcup_{i}\varphi(b_{i})\). In
                  particular the open \(\set{\uparrow a}\) has this property.
                  This means \(\uparrow a\in\bigcup_{i}\varphi(b_{i})\) so
                  \(\uparrow a\in\varphi(b_{i})\) for some \(i\). equivalently
                  this means \(a\leq b_{i}\) for some \(i\).
              \end{proof}
    \end{enumerate}
\end{question}

\begin{question}
    Let \((X,\leq)\) be a Priestley space such that \(\leq\) is linear.

    \begin{enumerate}[(a)]
        \item Show that \((X,\leq)\) is an Esakia space.

              \begin{proof}
                  We show that the dual distributive algebra has implication,
                  and thus that it is a Heyting algebra. This means the original
                  space was an Esakia space.

                  The upsets of \(X\) must be linearly ordered: for \(U\neq U'\)
                  without loss of generality we can assume \(x\in U\setminus
                  U'\) so \(y\nleq x\) for \(y\in U\) so \(x\leq y\). This
                  proves \(U'\subseteq U\) so arbitrary open upsets must be
                  comparable.

                  This means that the corresponding distributive lattice of
                  clopen upsets is just a linear subposet of this larger linear
                  poset. We have seen that lattices like this are Heyting
                  algebras with implication \(a\to b =1\) if \(a\leq b\) and
                  \(a\to b=b\) if \(b\leq a\). This means the original space
                  must have been an Esakia space: its dual is a Heyting algebra.
              \end{proof}

        \item Show that \((X,\leq)\) is complete as a lattice.

              \begin{proof}
                  We use duality to show this must be complete. An
                  ascending/descending chain of prime ideals
                  \((F_{i})_{i\in\N}\) must have a least/greatest upper/lower
                  bound.

                  Given an increasing sequence of filters \((F_{i})_{i\in\N}\)
                  the union \(F=\bigcup_{i}F_{i}\) is also a prime filter. It is
                  clear that it is still upward closed. It is closed under meets
                  because for \(x,y\in F\), \(x,y\in F_{i}\) for some \(i\) so
                  \(x\wedge y\in F_{i}\subseteq F\). Also if \(x\vee y\in F\),
                  then \(x\vee y\in F_{i}\) for some \(i\) so \(x\in
                  F_{i}\subseteq F\) or \(y\in F_{i}\subseteq F\). This proves
                  \(F\) is a prime filter and thus every chain has an upper
                  bound. This must also be the least upper bound because any
                  upper bound must contain all \((F_{i})_{i\in\N}\) and
                  therefore also their union.

                  Similarly for a descending chain of prime filters
                  \((F_{i})_{i\in\N}\) the intersection is still a prime filter
                  and a lower bound of the chain. It is clear that
                  \(\bigcap_{i}F_{i}\) is still upwards closed. It is closed
                  under intersection because for \(x,y\in F\), \(x,y\in F_{i}\)
                  for all \(i\). This means \(x\wedge y\in F\) so \(F\) is still
                  a filter.

                  Similarly if \(x\vee y\in F\) for some \(x,y\) then at least
                  one of \(x\) or \(y\) must be contained in all \(F_{i}\). This
                  is because \(x\in F_{i}\) or \(y\in F_{i}\) for each \(i\)
                  (though a priori) fixed when alternating \(i\). However
                  suppose there are \(i<j\) such that \(y\notin\in F_{i}\) and
                  \(x\notin F_{j}\), then \(x\notin F_{i}\) either so \(F_{i}\)
                  is not a prime filter. This shows that \(x\in F\) or \(y\in
                  F\) so it is a prime filter. This is also the largest lower
                  bound because any prime filter contained in these filters
                  must be contained in their intersection.

                  This shows that the lattice of prime ideals of the dual
                  Priestley/Esakia space is complete. These lattices are
                  isomorphic so this lattice is complete as well.
              \end{proof}
    \end{enumerate}
\end{question}
\end{document}