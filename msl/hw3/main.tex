\documentclass{article}

\usepackage[utf8]{inputenc}
\usepackage{enumerate}
\usepackage{amsthm, amssymb, mathtools, amsmath, bbm, mathrsfs, stmaryrd}
\usepackage[margin=1in]{geometry}
\usepackage[parfill]{parskip}
\usepackage[hidelinks]{hyperref}
\usepackage{float}
\usepackage{cleveref}
\usepackage{svg}
\usepackage{tikz-cd}

\newcommand{\N}{\mathbb{N}}
\newcommand{\Z}{\mathbb{Z}}
\newcommand{\NZ}{\mathbb{N}_{0}}
\newcommand{\Q}{\mathbb{Q}}
\newcommand{\R}{\mathbb{R}}
\newcommand{\C}{\mathbb{C}}

\newcommand{\F}{\mathbb{F}}
\newcommand{\incl}{\imath}

\newcommand{\tuple}[2]{\left\langle#1\colon #2\right\rangle}
\newcommand{\powset}{\mathcal{P}}
\DeclareMathOperator{\atoms}{At}
\newcommand{\set}[1]{\left\{#1\right\}}
\newcommand{\setwith}[2]{\set{#1:#2}}

\renewcommand{\qedsymbol}{\raisebox{-0.5cm}{\includesvg[width=0.5cm]{../../qedboy.svg}}}

\DeclareMathOperator{\order}{orde}
\DeclareMathOperator{\Id}{Id}
\DeclareMathOperator{\im}{im}
\DeclareMathOperator{\ggd}{ggd}
\DeclareMathOperator{\kgv}{kgv}
\DeclareMathOperator{\degree}{gr}
\DeclareMathOperator{\coker}{coker}

\DeclareMathOperator{\gl}{GL}

\DeclareMathOperator{\Aut}{Aut}
\DeclareMathOperator{\Hom}{Hom}
\DeclareMathOperator{\End}{End}
\DeclareMathOperator{\Up}{Up}
\DeclareMathOperator{\fin}{Fin}
\DeclareMathOperator{\cofin}{Cofin}
\DeclareMathOperator{\fincofin}{FinCofin}
\DeclareMathOperator{\spec}{Spec}
\DeclareMathOperator{\iso}{Iso}
\DeclareMathOperator{\filters}{Fil}
\DeclareMathOperator{\closed}{Cl}
\DeclareMathOperator{\inter}{Int}

\newenvironment{solution}{\begin{proof}[Solution]\renewcommand\qedsymbol{}}{\end{proof}}

\newtheorem{theorem}{Theorem}
\newtheorem{lemma}[theorem]{Lemma}
\newtheorem{proposition}[theorem]{Proposition}


\theoremstyle{definition}
\newtheorem{question}{Exercise}
\newtheorem{definition}[theorem]{Definition}
\newtheorem{remark}[theorem]{Remark}
\newtheorem{example}[theorem]{Example}
\newtheorem{corollary}[theorem]{Corollary}

\title{Homework: Mathematical Structures in Logic}
\author{Jonas van der Schaaf}
\date{}

\begin{document}
\maketitle

\begin{question}
    We abbreviate \(a\to 0\) as \(\neg a\). Show that in every Heyting algebra:
    \begin{enumerate}[(a)]
        \item \(a\wedge\neg a=0\) but not necessarily \(a\vee\neg a=1\),

              \begin{proof}
                  We have \(a\wedge(a\to 0)=a\wedge0=0\) by the axioms of Heyting
                  algebras.

                  The lattice
                  \[
                      \begin{tikzcd}[ampersand replacement=\&]
                          1 \\
                          a \\
                          0
                          \arrow[from=3-1, to=2-1]
                          \arrow[from=2-1, to=1-1]
                      \end{tikzcd}
                  \]
                  is a Heyting algebra with
                  \begin{align*}
                      a\to 0 & =\bigvee\setwith{c}{c\wedge a\leq 0} \\
                             & =\bigvee\set{0}                      \\
                             & =0,
                  \end{align*}
                  therefore \(a\vee\neg a=a\).
              \end{proof}

        \item \(a\leq b\) iff \(a\to b=1\).

              \begin{proof}
                  If \(a\leq b\) for \(a,b\in H\) for some Heyting algebra \(H\), then
                  for all \(c\) we have \(a\wedge c\leq a\leq b\) so \(a\to b=\bigvee
                  H=1\).

                  Conversely if \(a\to b=1\) then
                  \begin{align*}
                      a\wedge b & =a\wedge(a\to b) \\
                                & =a\wedge1        \\
                                & =a.
                  \end{align*}
              \end{proof}

        \item \(a\leq\neg\neg a\).

              \begin{proof}
                  We know \(a\wedge(a\to 0)=0\) so
                  \(a\leq\bigvee\setwith{c}{c\wedge(a\to 0)\leq0}=\neg\neg a\).
              \end{proof}

        \item \(\neg a\wedge\neg b=\neg(a\vee b)\).

              \begin{proof}
                  We know that \((a\vee b)\to 0\leq a\to 0\) because
                  \begin{align*}
                      a\wedge(a\vee b\to 0) & =a\wedge (a\vee b)\wedge(a\vee b\to 0) \\
                                            & =a\wedge 0                             \\
                                            & =0.
                  \end{align*}
                  By symmetry the same argument holds for \(b\to 0\). This means
                  that \(\neg(a\vee b)\leq\neg a\vee\neg b\).

                  We prove the other inequality similarly:
                  \begin{align*}
                      (a\to 0)\wedge(b\to 0)\wedge(a\vee b) & =(a\wedge(a\to 0)\wedge(b\to 0))\vee(b\wedge(a\to 0)\wedge(b\to0)) \\
                                                            & =0\vee0                                                            \\
                                                            & =0.
                  \end{align*}
                  This proves the other inequality and therefore equality.
              \end{proof}

        \item \(\neg a\vee\neg b\leq\neg(a\wedge b)\) but not necessarily
              \(\neg(a\wedge b)\leq\neg a\vee\neg b\).

              \begin{proof}
                  We know this inequality is true iff \((\neg a\vee\neg
                  b)\wedge(a\wedge b)=0\). Using distributivity we see this
                  expression is equal to
                  \begin{align*}
                      (\neg a\vee\neg b)\wedge(a\wedge b) & =(\neg a\wedge(a\wedge b))\vee(\neg b\wedge(a\wedge b))            \\
                                                          & =((a\wedge\neg a)\wedge\neg b)\wedge((\neg b\wedge b)\wedge\neg a) \\
                                                          & =0\wedge0                                                          \\
                                                          & =0.
                  \end{align*}
                  A counterexample for the other implication is
                  \[
                      \begin{tikzcd}[ampersand replacement=\&]
                          \& 1 \\
                          \& c \\
                          a \&\& b \\
                          \& 0
                          \arrow[from=2-2, to=1-2]
                          \arrow[from=3-1, to=2-2]
                          \arrow[from=3-3, to=2-2]
                          \arrow[from=4-2, to=3-1]
                          \arrow[from=4-2, to=3-3]
                      \end{tikzcd}
                  \]
                  This is a distributive lattice because neither
                  \(\mathbb{M}_{5}\) nor \(\mathbb{D}_{3}\) can be embedded in
                  it. It is finite and satisfies the infinite distributivity law
                  trivially. However \(\neg(a\wedge b)=1\) but
                  \(\neg a=b\) and \(\neg b=a\) so \(\neg a\vee\neg b=c\).
              \end{proof}

        \item \(a\to(b\to c)=(a\wedge b)\to c\).

              \begin{proof}
                  We know
                  \begin{align*}
                      a\to(b\to c)\wedge(a\wedge b) & =(a\to(b\to c)\wedge a)\wedge b \\
                                                    & =a\wedge (b\to c)\wedge b       \\
                                                    & =a\wedge b\wedge c              \\
                                                    & \leq c.
                  \end{align*}
                  Therefore \(a\to(b\to c)\leq(a\wedge b)\to c\).

                  We know the other inequality holds iff \(a\wedge((a\wedge
                  b)\to c)\leq b\to c\). In turn this inequality holds iff
                  \(b\wedge a\wedge((a\wedge b)\to c)\leq c\). This is clearly
                  true because
                  \[
                      (a\wedge b)\wedge((a\wedge b)\to c)=(a\wedge b)\wedge c\leq c.
                  \]
              \end{proof}
    \end{enumerate}
\end{question}

\begin{question}
    Let \(B\) be a boolean algebra and \(X\) its dual space. Show that:
    \begin{enumerate}[a)]
        \item \(B\) is atomic iff the set of isolated points of \(X\) is dense.

              \begin{proof}
                  We know from the previous homework that the left to right
                  implication holds.

                  There is a bijection \(\varphi\) from the ideals of \(B\) to
                  the opens of \(X\) given by \(\varphi(I)=\setwith{F\in X}{\neg
                      I\not\subseteq F}\) similar to the map from filters to closed
                  sets from the previous homework. Take a \(b\neq 0\). Then
                  \(\varphi(b)\subseteq X\) is an open set, so there is an
                  isolated point \(\uparrow a\in\varphi(b)\) where \(a\) is an
                  atom of \(B\).

                  We then have \(\uparrow a\in\varphi(B)\) iff \(b\in\uparrow
                  a\) which is equivalent to \(a\leq b\). This gives the desired
                  atom.

                  Therefore \(B\) is atomic.
              \end{proof}

        \item \(B\) is countable iff \(X\) is second countable.

              \begin{proof}
                  If \(B\) is countable then \(\varphi[B]\) is also countable
                  and a basis for \(X\). Therefore the dual space of a countable
                  BA is second countable.

                  Conversely suppose that \(X\) is second countable. Then the
                  basis of clopens \(\varphi[B]\) has a countable subset of
                  clopens \(\varphi[Y]\) for some countable \(Y\subseteq B\)
                  that is also a basis for the topology: every \(\varphi(b)\) is
                  a union of \(\varphi(y_{i})\) with \(\set{y_{i}}_{i}\subseteq
                  Y\). We can assume this is a finite union because
                  \(\varphi(b)\) is closed and \(X\) is compact so
                  \(\varphi(b)\) is also compact. This means any element of
                  \(B\) is a finite union of elements of the countable set
                  \(Y\), so \(B\) is countable as well.
              \end{proof}
    \end{enumerate}
\end{question}

\begin{question}
    Let \(A\) be a BA and \(X\) its Stone dual space. Let \(S,T\subseteq A\) be
    such that \(\bigvee S\) and \(\bigwedge T\) exist. Show that
    \begin{enumerate}[a)]
        \item \(\varphi(\bigvee S)=\overline{\bigcup\varphi[S]}\).

              \begin{proof}
                  It is clear that \(\varphi(s)\subseteq\varphi(\bigvee S)\) for
                  all \(s\in S\), so \(\bigcup\varphi[S]\subseteq\varphi(\bigvee
                  S)\). Because \(\varphi(\bigvee S)\) is closed, the closure
                  \(\overline{\bigcup\varphi[S]}\) must be contained in
                  \(\varphi(\bigvee S)\). This gives
                  \(\overline{\bigcup\varphi[S]}\subseteq\varphi(\bigvee S)\).

                  For the converse take \(F\in\varphi(\bigvee S)\). We show this
                  must be a limit point of \(\bigcup\varphi[S]\). Suppose there
                  is an \(x\in A\) such that \(F\in\varphi(x)\) and
                  \(x\cap\bigcup\varphi[S]=\varnothing\). Then
                  \(\bigcup\varphi[S]\subseteq\varphi(\neg x)\), so \(s\leq\neg
                  x\) for all \(s\in S\) which means that \(\bigvee S\leq\neg
                  x\) because it is the smallest upper bound. Because
                  \(F\in\varphi(\bigvee S)\) this means \(\neg x\in F\), but
                  \(x\in F\) as well: a contradiction. Therefore no such \(x\)
                  can exist and \(F\) must be a limit point.
              \end{proof}

        \item \(\varphi(\bigwedge T)=\inter(\bigcap\varphi[T])\).

              \begin{proof}
                  Take \(F\in\inter(\bigcap\varphi[T])\). Then there is an open
                  neighbourhood \(\varphi(x)\) of \(F\) such that
                  \(F\in\varphi(x)\subseteq\inter(\bigcap\varphi[T])\). Then
                  \(\varphi(x)\subseteq\varphi(t)\) for all \(t\in T\) so
                  \(x\leq t\) for all \(t\in T\) thus by the universal property
                  \(x\leq\bigwedge T\). This means \(\bigwedge T\in F\) as well,
                  therefore
                  \(\inter(\bigcap\varphi[T])\subseteq\varphi(\bigwedge T)\).

                  To prove the converse implication take \(F\in\varphi(\bigwedge
                  T)\). Then \(t\in F\) for all \(t\in T\) so
                  \(F\in\bigcap\varphi[T]\). Then \(\varphi(\bigwedge T)\) is an
                  open set contained in this intersection and therefore also its
                  interior.
              \end{proof}
    \end{enumerate}
\end{question}

\begin{question}
    \begin{enumerate}[a)]
        \item Show that a boolean algebra \(A\) is complete if and only if the
              dual Stone space \(X\) is extremally disconnected.

              \begin{proof}
                  If \(B\) is a complete boolean algebra and \(Y\subseteq X\) is
                  an open set, then it is the union of
                  \((\varphi(x_{i}))_{x_{i}\in I}\) for some index set \(I\).
                  The closure of this set is then
                  \(\varphi(\bigvee\setwith{x_{i}}{i\in I})\) which is clopen.

                  Conversely if for each \(U=\bigcup_{x_{i}}\varphi(x_{i})\)
                  \(\overline{U}\) is clopen. Then there is some \(x\in A\) such
                  that \(\varphi(x)=\overline{U}\). This is the smallest clopen
                  such that \(\bigcup_{i}\varphi(x_{i})\) is a subset of it, so
                  this must be the limit \(\bigvee_{i}x_{i}\) because
                  \(\varphi\) is order-preserving.

                  We have shown that a lattice with joins has meets as well,
                  so it is therefore a complete lattice.
              \end{proof}

        \item Show that every boolean algebra is isomorphic to a product of a
              complete atomic boolean algebra and a complete atomless boolean
              algebra.

              \begin{proof}
                  Let \(B\) be a BA with dual Stone space \(X\).

                  The set of isolated points \(\iso(X)\) is an open subset so
                  its closure is a clopen subset of \(X\). This means we can
                  write
                  \[
                      X=\overline{\iso(X)}\sqcup X\setminus\overline{\iso(X)}.
                  \]
                  The space \(\overline{\iso(X)}\) is clearly a Stone space
                  where the isolated points are dense and so corresponds to an
                  atomic boolean algebra \(A\). Conversely the algebra \(A'\)
                  corresponding to \(X\setminus\overline{\iso(X)}\) must be
                  atomless because it has no isolated points. Both corresponding
                  algebras are complete because being extremally connected is
                  inherited by subspaces.

                  Therefore \(A\times A'\) is isomorphic to the original boolean
                  \(B\) giving the desired product.
              \end{proof}
    \end{enumerate}
\end{question}
\end{document}