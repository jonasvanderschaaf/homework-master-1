\documentclass{article}

\usepackage[utf8]{inputenc}
\usepackage{enumerate}
\usepackage{amsthm, amssymb, mathtools, amsmath, bbm, mathrsfs, stmaryrd}
\usepackage[margin=1in]{geometry}
\usepackage[parfill]{parskip}
\usepackage[hidelinks]{hyperref}
\usepackage{float}
\usepackage{cleveref}
\usepackage{svg}
\usepackage{tikz-cd}

\newcommand{\N}{\mathbb{N}}
\newcommand{\Z}{\mathbb{Z}}
\newcommand{\NZ}{\mathbb{N}_{0}}
\newcommand{\Q}{\mathbb{Q}}
\newcommand{\R}{\mathbb{R}}
\newcommand{\C}{\mathbb{C}}

\newcommand{\F}{\mathbb{F}}
\newcommand{\incl}{\imath}

\newcommand{\tuple}[2]{\left\langle#1\colon #2\right\rangle}
\newcommand{\powset}{\mathcal{P}}
\DeclareMathOperator{\atoms}{At}
\newcommand{\set}[1]{\left\{#1\right\}}
\newcommand{\setwith}[2]{\set{#1:#2}}

\renewcommand{\qedsymbol}{\raisebox{-0.5cm}{\includesvg[width=0.5cm]{../../qedboy.svg}}}

\DeclareMathOperator{\order}{orde}
\DeclareMathOperator{\Id}{Id}
\DeclareMathOperator{\im}{im}
\DeclareMathOperator{\ggd}{ggd}
\DeclareMathOperator{\kgv}{kgv}
\DeclareMathOperator{\degree}{gr}
\DeclareMathOperator{\coker}{coker}

\DeclareMathOperator{\gl}{GL}

\DeclareMathOperator{\Aut}{Aut}
\DeclareMathOperator{\Hom}{Hom}
\DeclareMathOperator{\End}{End}
\DeclareMathOperator{\Up}{Up}
\DeclareMathOperator{\fin}{Fin}
\DeclareMathOperator{\cofin}{Cofin}
\DeclareMathOperator{\fincofin}{FinCofin}
\DeclareMathOperator{\spec}{Spec}
\DeclareMathOperator{\iso}{Iso}
\DeclareMathOperator{\filters}{Fil}
\DeclareMathOperator{\closed}{Cl}
\DeclareMathOperator{\inter}{Int}
\DeclareMathOperator{\var}{Var}
\DeclareMathOperator{\prop}{Prop}
\DeclareMathOperator{\formulas}{Form}
\DeclareMathOperator{\subform}{Sub}

\newenvironment{solution}{\begin{proof}[Solution]\renewcommand\qedsymbol{}}{\end{proof}}

\newtheorem{theorem}{Theorem}
\newtheorem{lemma}[theorem]{Lemma}
\newtheorem{proposition}[theorem]{Proposition}


\theoremstyle{definition}
\newtheorem{question}{Exercise}
\newtheorem{definition}[theorem]{Definition}
\newtheorem{remark}[theorem]{Remark}
\newtheorem{example}[theorem]{Example}
\newtheorem{corollary}[theorem]{Corollary}

\title{Homework: Mathematical Structures in Logic}
\author{Jonas van der Schaaf}
\date{}

\begin{document}
\maketitle

\begin{question}
    Give full details of the algebraic proof of the finite model property of HAs
    (IPC) using \(\to\)-free reducts of Heyting algebras.

    \begin{proof}
        Let \(\mathbb{A}\) be a Heyting algebra such that
        \(\mathbb{A}\nvDash\varphi(x_{1},\ldots,x_{n})\approx\top\).
        where \(x_{1},\ldots,x_{n}\) are all free variables in \(\varphi\).

        Then there is a valuation \(v:\prop\to\mathbb{A}\) such that
        \(\widehat{v}(\varphi)\neq 1\) where \(\widehat{v}\) is the unique
        homomorphism extending \(v\) to the set of formulas \(\formulas\). Then
        consider the sublattice \(\mathbb{A}'\subseteq\mathbb{A}\) generated by
        \(\setwith{\varphi(\psi)}{\psi\in\subform(\varphi)}\). This is in
        particular a finite distributive sublattice and therefore a Heyting
        algebra with implication given by the expression \(x\Rightarrow
        y=\bigvee\setwith{a\in\mathbb{A}'}{a\wedge x\leq y}\).

        We claim that for \(x,y\in\mathbb{A}\) if \(x\to y\in\mathbb{A}'\) then
        \(x\Rightarrow y=x\to y\). We know that
        \[
            x\Rightarrow y=\bigvee\setwith{a\in\mathbb{A}'}{a\wedge x\leq y}\leq\bigvee\setwith{a\in\mathbb{A}}{a\wedge x\leq y}=x\to y
        \]
        and that \(x\wedge(x\to y)=x\wedge y\leq y\) so \(x\to y\leq
        x\Rightarrow y\). This proves the desired equality.

        A simple induction argument thus shows that \(\widehat{v}\) and
        \(\widehat{v'}\)
        agree on \(\subform(\varphi)\) because
        \(\widehat{v}(\varphi)\subseteq\mathbb{A}'\).

        This means in particular that
        \(\widehat{v'}(\varphi)=\widehat{v}(\varphi)\neq 1\in\mathbb{A'}\)
        and therefore \(\mathbb{A}'\) refutes \(\varphi\approx1\).
    \end{proof}
\end{question}

\begin{question}
    \begin{enumerate}[a)]
        \item Show that every stable variety has the FMP.

              \begin{proof}
                  Let \(V\) be a stable variety. Then the subdirect irreducible
                  algebras \(V_{SI}\) have the same theory as \(V\) because they
                  generate \(V\). In particular if there is an
                  \(\mathbb{A}\nvDash\varphi\approx\top\), then there is a
                  subdirect irreducible \(\mathbb{A}'\) which also refutes the
                  formula. Let \(v\) be the valuation such that
                  \(\widehat{v}(\varphi)\neq 1\), then by the reasoning of the
                  previous exercise the finite sublattice \(\mathbb{B}'\)
                  generated by
                  \(\setwith{\widehat{v}(\psi)}{\psi\in\subform(\varphi)}\) also
                  refutes the formula \(\varphi\approx\top\).

                  By stability \(\mathbb{B}'\in V\) and \(\mathbb{B}'\) refutes
                  \(\varphi\approx\top\) by assumption so \(V\) has the FMP.
              \end{proof}

        \item Give an example of a variety \(V\) of Heyting algebras which is
              not stable.

              \begin{solution}
                  Consider the variety of boolean algebras \(\var(2)\) and
                  consider the powerset \(\powset(\set{a,b})\). It has the
                  Heyting algebra \(3\) as sublattice, but this is not an
                  algebra in the variety.
              \end{solution}
    \end{enumerate}
\end{question}

\begin{question}
    Let \(LC=IPC+\varphi\to\psi\vee\psi\to\varphi\).

    \begin{enumerate}[a)]
        \item Show that the subdirect irreducible \(LC\)-algebras are chains
              with a second greatest element.

              \begin{proof}
                  We know that the subdirect irreducible \(LC\)-algebras have a
                  second greatest element. We show these are all chains.

                  Suppose \(x\) is the second greatest element and \(a,b\) are
                  incomparable. Then \(a\to b\neq 1\) and \(b\to a\neq 1\) and
                  therefore both are at most \(x\). Their join is therefore also
                  less than or equal to \(x\), which is different from \(1\)
                  contradicting the assumption that \(a\to b\vee b\to a=1\).
              \end{proof}

        \item Show that \(V_{LC}\) is not finitely generated.

              \begin{proof}
                  If \(V_{LC}\) were finitely generated all subdirect
                  irreducible algebras would be in \(HSP_{U}(\mathbb{A})\) for
                  some finite algebra \(\mathbb{A}\). We know
                  \(P_{U}(\mathbb{A})=\set{\mathbb{A}}\) so we get that every
                  subdirectly irreducible algebra is the homomorphic image of a
                  subalgebra of \(\mathbb{A}\) and therefore in particular
                  bounded in cardinality by the cardinality of \(\mathbb{A}\).

                  We know however that there are chains of unbounded size with a
                  second largest element, take the ordinal number
                  \(\aleph_{\alpha}+2\) for any ordinal \(\alpha\). These are
                  all algebras in this variety because for any two \(x,y\) in a
                  chain \(x\to y=1\) or \(y\to x=1\).
              \end{proof}

        \item Show that \(V_{LC}\) has the finite model property.

              \begin{proof}
                  The variety \(V_{LC}\) is stable: for any subdirect
                  irreducible algebra we must have that this is a chain. A
                  sublattice of a chain is also a chain and chains are algebras
                  in this variety.
              \end{proof}

        \item Characterize the lattice of subvarieties of all \(LC\)-algebras.

              \begin{solution}
                  First we show that any infinite chain has all finite chains as
                  subalgebras. This is easy to see because for an infinite chain
                  lattice \(\mathbb{A}\) simply taking subset \(\set{0,1}\cup
                  X\subseteq\mathbb{A}\) will be a subalgebra and chain of size
                  \(|A|+2\). The chain \(1\) is part of every variety.

                  This shows that any variety containing an infinite subdirect
                  irreducible algebra contains all finite chains. Any variety
                  containing a finite chain also contains all smaller finite
                  chains by the same reasoning.

                  We also see that for \(\var(n)\) where \(n\) is an
                  \(n\)-element chain, the subdirect irreducible algebras all
                  have smaller cardinality and so \(\var(n)\) has the chains of
                  size \(\leq n\) as subdirect irreducible algebras.

                  This means that the bottom of the lattice of subvarieties is a
                  chain \((\var(n))_{n\in\omega\setminus\set{0}}\). I don't know
                  what this lattice looks like elsewhere.
              \end{solution}
    \end{enumerate}
\end{question}
\end{document}